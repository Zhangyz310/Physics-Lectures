\documentclass[12pt,a4paper]{article}
\usepackage{ctex,hyperref}% 输出汉字
\usepackage{times}% 英文使用Times New Roman
\usepackage[utf8]{inputenc}
%usepackage[ngerman]{babel}
\usepackage[T1]{fontenc}
\usepackage{amssymb}
\usepackage{epsfig}
\usepackage{array}
\usepackage{CJKutf8}
\usepackage{amsmath}
\usepackage{amsfonts}
\usepackage{amssymb}
\usepackage{graphicx}
\usepackage{lmodern}
\usepackage{physics}
\usepackage[left=1.3cm,right=1.3cm,top=2.2cm,bottom=2.2cm]{geometry}
\usepackage{siunitx}
\usepackage{fancyhdr}
\usepackage{enumerate}
\usepackage{mhchem}
\usepackage{mathtools}
\usepackage{graphicx}
\usepackage{float}
\usepackage{caption}
\usepackage{ccaption}
\usepackage{bbm}
\usepackage{mathrsfs}
\usepackage{mdframed}
\mdfsetup{linewidth=1pt,linecolor=blue}
\mdfsetup{backgroundcolor=gray!02}
\mdfsetup{shadow=true}
\makeatletter
\newcommand{\rmnum}[1]{\romannumeral #1}
\newcommand{\Rmnum}[1]{\expandafter\@slowromancap\romannumeral #1@}
\makeatother

\usepackage{tikz}%封面
\usetikzlibrary{positioning}
\definecolor{innerblue}{RGB}{54,109,252}
\definecolor{middleblue}{RGB}{153,192,252}
\definecolor{outerblue}{RGB}{93,150,250}
\definecolor{white}{RGB}{245,245,245}
\tikzset{pics/.cd,
	triple circle/.style={code={
			\fill[outerblue] (4,0) circle (4);
			\fill[middleblue] (3.1,0) circle (3.1);
			\fill[innerblue] (2.2,0) circle (2.2);
}}}

%sisetup{locale=DE}
%sisetup{per-mode = symbol-or-fraction}
%DeclareSIUnit\year{a}
%DeclareSIUnit\clight{c}

\allowdisplaybreaks
\everymath{\displaystyle}

\begin{document}
	
	\thispagestyle{empty}
	\begin{tikzpicture}[overlay,remember picture]
		\fill[white] (current page.north west) rectangle (current page.south east); 
		\draw[blue!60] ([xshift=-1cm]current page.north) 
		-- ++ (-60:9) pic[rotate=90-60,scale=0.7] {triple circle};
		\draw[blue!60] (current page.north west) 
		-- ++ (-40:20) pic[rotate=90-40,scale=0.3] {triple circle};
		\draw[blue!60] ([yshift=-11cm]current page.north east) 
		-- ++ (-110:15) pic[rotate=90-110,scale=0.8] {triple circle};
		\node[above right=12cm and 4cm of current page.south west,font=\Huge\bfseries,blue!90] (H)
		{\textbf{\kaishu 电动力学}};
		\node[below=4mm of H.south west,anchor=north west,font=\sffamily]{Yaoze Zhang};
	\end{tikzpicture}
	\clearpage
	
	\newpage
	\thispagestyle{empty}
	\null
	\begin{center}
		\includegraphics[width=0.58\linewidth]{C:/Users/16965/OneDrive/Pictures/Electrondynamics/James-Clerk-Maxwell-1831-1879.jpg}
	\end{center}
	\begin{center}
		\par{\centering \large James Clerk Maxwell\quad1831-1879}
	\end{center}
	
	\vspace{0.5cm}
	
	\begin{huge}
		\begin{gather*}
			\mathfrak{\Large In \, templo \, magnae}\\\mathfrak{veritatis \, sophistarum \, locus \, non \, est}
		\end{gather*}
	\end{huge}

	
	\newpage
	
	\pagenumbering{roman}
	% 生成目录
	\tableofcontents
	% 插入新页
	\newpage
	% 设置页码格式是阿拉伯数字
	\pagenumbering{arabic}
	
	\numberwithin{equation}{subsection}
	\numberwithin{figure}{subsection}
	
	\section{\text{\kaishu 麦克斯韦方程组}}
	
	\pagestyle{fancy}
	\lhead{Lecture Notes}
	\chead{Electrodynamics}
	\rhead{\kaishu Chapter 1 麦克斯韦方程组}
	
	\par{\kaishu 本章将从电场、磁场最基本的性质(通过实验得到)出发,尽可能全面且迅速地由这些定律推导出麦克斯韦方程组——用于描述经典电动力学体系的完备方程。本章不会涉及较为繁琐的电场和磁场计算,因为目的是推导一般规律,在推出麦克斯韦方程组后,将以此为出发点再去解决具体的电磁问题。}
	
	\subsection{\text{\kaishu 静电现象的基本理论描述}}
	
	\subsubsection{\text{\kaishu 库伦定律}(Coulomb's Law)}
	
	\par{\kaishu 1785年,库伦做了大量的实验,总结发现:对于真空中两个点电荷(指自身尺度大小远小于相互作用间距的两个电荷,不考虑其尺度),我们设两个点电荷分别带电量为:\(q_1\)、\(q_2\),单位为 \(\mathrm{C}\)(库伦)电荷 \(q_{1}\) 对电荷 \(q_{2}\) 的作用力为:} 
	
	\vspace{-3mm}
	
	\begin{equation}
		\textbf{F}_{1\to2} = \frac{1}{4\pi\varepsilon_{0}}\frac{q_{1}q_{2}}{r_{21}^{2}}\hat{r}_{21} = \frac{1}{4\pi\varepsilon_{0}}\frac{q_{1}q_{2}}{r_{21}^{3}}\textbf{r}_{21}
	\end{equation}
	
	\vspace{3mm}
	
	\par{\kaishu 其中:\(\varepsilon_{0}\approx 8.85\times10^{-12}\,\mathrm{C^{2}/N\cdot m^2}\) 为真空介电常数,\(\textbf{r}_{21} \equiv \textbf{r}_2-\textbf{r}_1\) 为点电荷 \(q_1\) 指向点电荷 \(q_2\) 的位置矢量,式 (1.1.1) 是通过大量实验结论总结出来的数学表达式。}
	
	\begin{center}
		\includegraphics[width=0.6\linewidth]{C:/Users/16965/OneDrive/Pictures/Electrondynamics/Electrodynamics fig 1-1-1.pdf}
		\captionof{figure}{\text{\kaishu 库伦定律示意图}}\label{fig:1-1-1}
	\end{center}
	
	\par{\kaishu 我们仔细观察库伦定律的表达式,可以发现:库伦作用力满足平方反比定律,产生径向作用,并仅可以直接适用于点电荷之间的作用。事实上,这个表达式蕴含着丰富的物理涵义:
		
		\vspace{-3mm}
		
		\begin{enumerate}
			
			\item 牛顿第三定律(因为 \(\textbf{r}_{12} = \textbf{r}_1-\textbf{r}_2 = -\textbf{r}_{21}\),我们进一步得到:\(\textbf{F}_{1\to2} = -\textbf{F}_{2\to1}\))
			
			\vspace{-3mm}
			
			\item 可以提供向心力(因为库伦作用力只有径向作用,波尔的经典氢原子模型是一个很好的例子)
			
			\vspace{-3mm}
			
			\item 同性相斥,异性相吸(假设 \(q_1q_2<0\),我们发现:\(\textbf{F}_{12}\) 的方向变为 \(q_1\) 指向 \(q_2\);同理 \(\textbf{F}_{21}\) 的方向变为 \(q_2\) 指向 \(q_1\),两个电荷向着两者连线的中心靠近)
			
	\end{enumerate}}
	
	\subsubsection{\text{\kaishu 叠加原理}}
	
	\par{\kaishu 现在在真空中,我们考虑共有 \(N\)(\(N>2\))个电荷组成电荷系,第 \(i\) 个电荷的受力为 \(\textbf{F}_i\),此电荷的位置矢量满足下式:}
	
	\vspace{-3mm}
	
	\begin{equation}
		\textbf{r}_i = \textbf{r}_{j}+\textbf{r}_{ji} \qquad\Rightarrow\qquad \textbf{r}_{ji} = \textbf{r}_i-\textbf{r}_j \nonumber
	\end{equation}
	
	\begin{center}
		\includegraphics[width=0.6\linewidth]{C:/Users/16965/OneDrive/Pictures/Electrondynamics/Electrodynamics fig 1-1-2.pdf}
		\captionof{figure}{\text{\kaishu 点电荷叠加原理示意图}}\label{fig:1-1-2}
	\end{center}
	
	\par{\kaishu 根据库伦定律,写出第 \(j\) 个电荷对第 \(i\) 个电荷的作用力(\(j\ne i\)),并对 \(j\) 求和:}
	
	\vspace{-3mm}
	
	\begin{equation}
		\textbf{F}_{ji} = \frac{1}{4\pi\varepsilon_{0}}\frac{q_{j}q_{i}}{r_{ji}^{3}}\textbf{r}_{ji} \qquad \textbf{F}_i = \sum_{j}\textbf{F}_{ji} \qquad\Rightarrow\qquad \textbf{F}_{i} = \sum_{j}\frac{1}{4\pi\varepsilon_{0}}\frac{q_{j}q_{i}}{r_{ji}^{3}}\textbf{r}_{ji} \nonumber
	\end{equation}
	
	\vspace{3mm}
	
	\par{\kaishu 上面我们的讨论就是叠加原理,利用更加规范化的语言表述如下:}
	
	\begin{center}
		\begin{mdframed}[frametitle={\text{\kaishu 叠加原理}}, frametitlerule=true, frametitlebackgroundcolor=blue!05]
			\par{\kaishu 体系中同时存在多个电荷时,某一特定电荷所受的静电力为其他所有电荷\textbf{独立}施于其上的作用力的线性叠加,其数学表达式如下:}
			
			\vspace{-3mm}
			
			\begin{equation}
				\textbf{F}_i = \frac{1}{4\pi\varepsilon_0}\sum_{j\ne i}^{N}\frac{q_jq_i}{r_{ji}^3}\textbf{r}_{ji}
			\end{equation}
			
		\end{mdframed}
	\end{center}
	
	\vspace{5mm}
	
	\par{\kaishu 叠加原理的核心在于:\textbf{电荷之间的相互作用为两体相互作用,与第三者是否存在,其大小以及正负都没有关系。}值得注意的是,叠加原理是人们从大量实验结果中总结的规律,并不能根据其他规律独立推导出来。}
	
	\par{\kaishu 有了叠加原理,我们可以计算连续带电体之间的相互作用力。(用微元分解的思想)}
	
	\par{\kaishu 考虑一个连续带电体对于处于 \(\textbf{r}\) 带电量为 \(q\) 的点电荷的静电作用,将连续带点体划分为许多微元(这些微元均可以视为点电荷),其中一个为处于 \(\textbf{r}_{2} = \textbf{r}'\) 带电量为 \(q_2 = \rho(\textbf{r}')\mathrm{d}\tau'\),其中 \(\rho(\textbf{r}') = \left.\frac{\Delta q}{\Delta \tau'}\right|_{\Delta\tau'\to 0} = \frac{\mathrm{d}q}{\mathrm{d}\tau'} \) 为电荷密度,而 \(\mathrm{d}\tau'\) 为此微元的体积。令 \(\textbf{R}\equiv \textbf{r}-\textbf{r}'\),考虑微元对电荷的作用力:}
	
	\begin{center}
		\includegraphics[width=0.65\linewidth]{C:/Users/16965/OneDrive/Pictures/Electrondynamics/Electrodynamics fig 1-1-3.pdf}
		\captionof{figure}{\text{\kaishu 连续体对点电荷的静电作用}}\label{fig:1-1-3}
	\end{center}
	
	\vspace{-3mm}
	
	\begin{equation}
		\mathrm{d}\textbf{F} = \frac{1}{4\pi\varepsilon_0}\frac{q\rho(\textbf{r}')\mathrm{d}\tau'}{R^3}\textbf{R}\nonumber
	\end{equation}
	
	\par{\kaishu 对连续体体积微元积分得到:}
	
	\vspace{-3mm}
	
	\begin{equation}
		\textbf{F} = \frac{1}{4\pi\varepsilon_0}\int_{\tau}\frac{q\rho(\textbf{r}')\mathrm{d}\tau'}{R^3}\textbf{R}
	\end{equation}
	
	\vspace{3mm}
	
	\par{\kaishu 这就是连续带电体对点电荷的作用力,进一步推广,我们把点电荷替换为连续带电体。我们假设两个连续带电体的电量分布为 \(\rho_{1}(\textbf{r})\)、\(\rho_{2}(\textbf{r})\),带电体1受到带电体2的作用力:}
	
	\vspace{-3mm}
	
	\begin{equation}
		\textbf{F}_{12} = \frac{1}{4\pi\varepsilon_0}\iint\frac{\rho_{1}(\textbf{r})\rho_{2}(\textbf{r})\mathrm{d}\tau\mathrm{d}\tau'}{R^3}\textbf{R}
	\end{equation}
	
	\vspace{3mm}
	
	\par{\kaishu \textcolor{blue}{注意:上述矢量积分一般较为复杂,我们多数时候使用数值方法进行计算。}}
	
	\subsubsection{\kaishu 电场}
	
	\par{\kaishu 根据库伦定律,我们知道:对于一个电荷 \(q\),其所受的力与其本身的电量成正比,这启发我们定义一个新的物理量,与这个电荷本身的电量无关:}
	
	\vspace{-3mm}
	
	\begin{equation}
		\textbf{E}(\textbf{r}) = \frac{\textbf{F}(\textbf{r})}{q}
	\end{equation}
	
	\vspace{3mm}
	
	\par{\kaishu 这个物理量与放置在此处的电荷(我们称之为探测电荷)没有关系,只与空间其他电荷在这个地方产生的效果有关系。这个物理量被称为\textbf{电场}。电场的引入,不仅帮助我们更加容易地计算静电力,更重要的是:电场给予了我们一个静电相互作用的新图像。}
	
	\tikzstyle{arrow} = [thick,->,>=stealth]
	\begin{center}
		\begin{tikzpicture}[
			squarednode/.style={rectangle, rounded corners, draw=blue, fill=blue!2, thick, minimum size=10mm},
			]
			
			\node[squarednode](maintopic){\textbf{\kaishu 电场}};
			\node[squarednode](leftsquare)[left=of maintopic]{\textbf{\kaishu 电荷\(q_1\)}};
			\node[squarednode](rightsquare)[right=of maintopic]{\textbf{\kaishu 电荷\(q_2\)}};
			
			%Lines
			\draw [arrow] (leftsquare.east) -- (maintopic.west);
			\draw [arrow] (maintopic.east) -- (rightsquare.west);
			
		\end{tikzpicture}
	\end{center}
	
	\par{\kaishu 从这个新图像看出,电荷之间的作用力并不是一种“超距作用”,作用力通过“电场”这一媒介从电荷之间“传递”。在开始的静电部分看不出这种变化。但是,我们可以考虑对研究的源电荷施加一个扰动(快速移动一段距离,或者快速改变电量),根据“超距作用”的图像,我们知道:探测电荷会立刻对这个扰动做出响应;然而真实情况并非如此,探测电荷经过一段时间后才能感知到扰动,这与我们的新图像一致。}
	
	\par{\kaishu 根据电场的定义,我们可以轻松知道点电荷\(Q\)产生的电场为(\(\textbf{R}\) 为点电荷指向探测电荷所在位置的位置矢量):}
	
	\begin{equation}
		\textbf{E} = \frac{1}{4\pi\varepsilon_0}\frac{Q}{R^3}\textbf{R} 
	\end{equation}
	
	\vspace{3mm}
	
	\par{\kaishu 同理,我们也可以得到多个点电荷产生的电场和连续带电体产生的电场:}
	
	\begin{equation}
		\textbf{E} = \frac{1}{4\pi\varepsilon_0}\sum_{j}^{N}\frac{q_j}{R_j^3}\textbf{R}_j \qquad \textbf{E} = \frac{1}{4\pi\varepsilon_0}\int_\tau\frac{\rho(\textbf{r}')\mathrm{d}\tau'}{R^{3}}\textbf{R}
	\end{equation}
	
	\vspace{3mm}
	
	\par{\kaishu 其中:\(R_j\) 是第 \(j\) 个电荷指向探测电荷所在位置的位置矢量;\(\textbf{R}\equiv \textbf{r}-\textbf{r}'\) 和 \(\rho(\textbf{r}')\) 与式(1.1.3)的物理意义相同。}
	
	\par{\kaishu 现在,我们不满足于仅仅计算连续体在空间内产生的电场,我们需要找到更加深层、普适的规律,因此,我们要去考察电场在空间中的变化(即考察电场的梯度);任意闭合曲面内电场的通量(即考察电场的散度);任意一条闭合曲线中电场的环量(即考察电场的旋度)。}
	
	\subsubsection{\kaishu 高斯定理(Gauss's Law)}
	
	\par{\kaishu 首先我们考虑最简单的情况,一个电量为 \(Q\) 的点电荷,以及包含这个点电荷的闭合曲面 \(S\)。我们令 \(\textbf{r}\) 为 \(Q\) 到闭合曲面 \(S\) 表面上一点的位矢,\(\textbf{n}\) 是闭合曲面这一点对应的切平面的单位法向量;\(\mathrm{d}s\) 是闭合曲面 \(S\) 表面的微元,并且我们设电场 \(\textbf{E}\) 与矢量 \(\textbf{n}\) 之间的夹角为 \(\theta\)(示意图见图1.1.4)。根据点电荷产生的电场,我们计算 \(\textbf{E}\) 与 \(\textbf{n}\) 的标量积(点乘):}
	
	\vspace{-3mm}
	
	\begin{equation}
		\textbf{E}\cdot\textbf{n} = \frac{Q}{4\pi\varepsilon_0}\frac{\cos\theta}{r^2}\qquad \text{\kaishu 两边同乘}\mathrm{d}s \qquad \left(\textbf{E}\cdot\textbf{n}\right)\mathrm{d}s = \frac{Q}{4\pi\varepsilon_0}\frac{\cos\theta}{r^2}\mathrm{d}s \nonumber
	\end{equation}
	
	\vspace{3mm}
	
	\par{\kaishu 我们可以将 \(\cos\theta\,\mathrm{d}s\) 改写为 \(r^{2}\mathrm{d}\Omega\),\(\mathrm{d}\Omega\) 是立体角微元,则上式改写成:}
	
	\vspace{-3mm}
	
	\begin{equation}
		\left(\textbf{E}\cdot\textbf{n}\right)\mathrm{d}s = \frac{Q}{4\pi\varepsilon_0}\mathrm{d}\Omega
	\end{equation}
	
	\vspace{3mm}
	
	\par{\kaishu 我们继续对整个闭合曲面 \(S\) 进行积分,等式变为:}
	
	\begin{center}
		\includegraphics[width=0.6\linewidth]{C:/Users/16965/OneDrive/Pictures/Electrondynamics/Electrodynamics fig 1-1-4.pdf}
		\captionof{figure}{\text{\kaishu 高斯定律示意图,其中 \(\mathrm{d}\Omega\) 表示立体角微元}}\label{fig:1-1-4}
	\end{center}
	
	\begin{equation}
		\oint_{S}\textbf{E}\cdot\textbf{n}\,\mathrm{d}s = \oint_{S}\frac{Q}{4\pi\varepsilon_0}\mathrm{d}\Omega = \begin{cases}
			\frac{Q}{\varepsilon_0}&\text{\kaishu 点电荷在曲面内}\\
			0 &\text{\kaishu 点电荷在曲面外}
		\end{cases}
	\end{equation}
	
	\vspace{3mm}
	
	\par{\kaishu \textcolor{blue}{注意:等式左侧的积分也可以写成 \(\oint_{S}\textbf{E}\cdot\mathrm{d}\textbf{S}\),只是写法不同,物理意义没有区别。}}
	
	\vspace{3mm}
	
	\par{\kaishu 上面得到的结果是单个点电荷的高斯定理,对于多个点电荷的情况,我们只需要将 \(Q\) 替换为 \(\sum_{j}Q_{j}\),求和只针对在闭合曲面内的点电荷,此时的等式(1.1.9)写成:}
	
	\begin{equation}
		\oint_{S}\textbf{E}\cdot\textbf{n}\,\mathrm{d}s = \frac{1}{\varepsilon_0}\sum_{j}Q_{j}
	\end{equation}
	
	\begin{center}
		\includegraphics[width=0.9\linewidth]{C:/Users/16965/OneDrive/Pictures/Electrondynamics/Electrodynamics fig 1-1-5.pdf}
		\captionof{figure}{\text{\kaishu 点电荷在闭合曲面\(S\)内/外,\(\mathrm{d}\Omega\)沿全曲面的积分结果为\(4\pi\)/0,因此只考虑闭合曲面内的电荷}}\label{fig:1-1-5}
	\end{center}
	
	\par{\kaishu 对于连续带电体,假设电荷密度 \(\rho(\textbf{r})\),我们将高斯定理写成:}
	
	\begin{equation}
		\oint_{S}\textbf{E}\cdot\textbf{n}\,\mathrm{d}s = \frac{1}{\varepsilon_0}\int_{V}\rho(\textbf{r})\mathrm{d}V
	\end{equation}
	
	\vspace{3mm}
	
	\par{\kaishu 其中:\(V\) 是闭合曲面 \(S\) 所围成的区域的体积,\(\mathrm{d}V\) 表示体积微元,可以写成 \(\mathrm{d}V = \mathrm{d}x \,\mathrm{d}y \,\mathrm{d}z\)。}
	
	\par{\kaishu 方程(1.1.11)是静电学的基本方程之一,我们可以注意到这个积分等式取决于以下几点:
		
		\vspace{-3mm}
		
		\begin{enumerate}
			
			\item 电荷之间作用力的平方反比定律(即库伦定律)
			
			\vspace{-3mm}
			
			\item 静电力的中心性质(类似有心力场)
			
			\vspace{-3mm}
			
			\item 不同电荷作用的线性叠加(即叠加原理)
			
		\end{enumerate}
		
		\vspace{-3mm}
		
		其实数学上也有矢量场的高斯定理(\(\mathbb{R}^3\) 上满足),数学语言表述如下:}
	
	\vspace{3mm}
	
	\begin{center}
		\begin{mdframed}[frametitle={\text{\kaishu 高斯定理(Gauss's Law)/散度定理}}, frametitlerule=true, frametitlebackgroundcolor=blue!05]
			\par{\kaishu 假设 \(V\) 是一个简单的固定区域,\(S\) 是 \(V\) 的闭边界对应的闭合曲面,给定正方向(指向曲面外侧),设 \(\textbf{A}\) 为一个向量场,其分量函数在 \(V\) 的开区域上具有连续偏导数,则:}
			
			\vspace{-3mm}
			
			\begin{equation}
				\iint_{S}\textbf{A}\cdot\mathrm{d}\textbf{S} = \iiint_{V}\nabla\cdot\textbf{A}\,\mathrm{d}V
			\end{equation}
			
			\vspace{3mm}
			
			\par{\kaishu 其中:\(\nabla\cdot\textbf{V}\) 也可以表述为 \(\mathrm{div}\textbf{V}\),两者均表示向量场 \(\textbf{V}\) 的散度;\(\textbf{A}\cdot\mathrm{d}\textbf{S}\) 也可以表述为 \(\textbf{A}\cdot\textbf{n}\,\mathrm{d}s\),为了便于表达,我们统一写成 \(\textbf{A}\cdot\mathrm{d}\textbf{S}\) 的形式。}
			
		\end{mdframed}
	\end{center}
	
	\vspace{3mm}
	
	\par{\kaishu 我们可以利用数学的散度定理,假定 \(\textbf{E}\) 在闭合曲面 \(S\) 围成区域 \(V\) 对应的开区域上具有连续偏导数,代入散度定理的公式:}
	
	\vspace{-3mm}
	
	\begin{subequations}
		\begin{equation}
			\oint_{S}\textbf{E}\cdot\textbf{n}\,\mathrm{d}s = \int_V\nabla\cdot \textbf{E}\,\mathrm{d}V = \frac{1}{\varepsilon_0}\int_{V}\rho(\textbf{r})\,\mathrm{d}V \nonumber
		\end{equation}
		\begin{equation}
			\int_{V}\left(\nabla\cdot\textbf{E}-\frac{1}{\varepsilon_0}\rho(\textbf{r})\right)\mathrm{d}V = 0 \qquad\Rightarrow\qquad \nabla\cdot\textbf{E}-\frac{1}{\varepsilon_0}\rho(\textbf{r}) = 0 \nonumber
		\end{equation}
	\end{subequations}
	
	\vspace{3mm}
	
	\par{\kaishu 根据上面并不复杂的推导,我们得到了高斯定理的微分形式(其中 \(\rho(\textbf{r})\) 为电荷密度分布):}
	
	\begin{equation}
		\nabla\cdot\textbf{E} = \frac{\rho(\textbf{r})}{\varepsilon_0}
	\end{equation}
	
	\begin{center}
		\begin{mdframed}[shadow=false]
			
			\par{\kaishu \textbf{补充:}高斯定理的微分形式推导中,我们只讨论了连续带电体的情况,并未讨论点电荷,实际上,我们可以通过特殊函数,将点电荷看作一种特殊的连续带电体。}
			
			\par{\kaishu 定义 \(\delta\) 函数(Dirac's Function),这是一种广义函数,满足如下性质:}
			
			\begin{equation}
				\delta(x) = \begin{cases}
					+\infty & x = 0 \\
					0 & x \ne 0 
				\end{cases} \qquad \int_{-\infty}^{+\infty}\delta(x)\mathrm{d}x = 1 \nonumber
			\end{equation}
			
			\vspace{2mm}
			
			\par{\kaishu 点电荷在空间中就是一个点,我们可以猜想,是否可以利用 \(\delta\) 函数来表示点电荷?}
			
		\end{mdframed}	
		
		\begin{mdframed}[shadow=false]
			
			\par{\kaishu 不过在此之前,我们可以先尝试对点电荷使用微分形式的高斯定理,考虑电量为 \(Q\) 的点电荷在空间产生的电场,\(\textbf{R}\) 为点电荷指向探测电荷位置的位矢:}
			
			\begin{equation}
				\textbf{E} = \frac{1}{4\pi\varepsilon_0}\frac{Q}{R^3}\textbf{R}\nonumber
			\end{equation}
			
			\par{\kaishu 对电场求梯度,我们得到:}
			
			\begin{equation}
				\nabla\cdot\textbf{E} = \frac{Q}{4\pi\varepsilon_0}\nabla\cdot\left(\frac{\textbf{R}}{R^3}\right) = \frac{Q}{4\pi\varepsilon_0 R^2}\frac{\partial}{\partial R}\left(\frac{R^2}{\left|R^{2}\right|}\right)\nonumber
			\end{equation}
			
			\vspace{3mm}
			
			\par{\kaishu 当 \(R\ne0\) 时,显然这个微分式为 0,但是 \(R\to0\) 时,我们不能简单地考虑这个微分式,尝试从散度的定义考虑(\(\partial V\) 表示体积 \(V\) 对应的闭合曲面边界):}
			
			\begin{equation}
				\nabla\cdot\textbf{E} = \lim_{V\to0}\frac{1}{|V|}\oint_{\partial V}\textbf{E}\cdot\textbf{n}\,\mathrm{d}s = \lim_{V\to0}\frac{1}{|V|}\frac{Q}{4\pi\varepsilon_0}\oint_{\partial V}\frac{\textbf{R}}{R^3}\cdot\textbf{n}\,\mathrm{d}s \nonumber
			\end{equation}
			
			\vspace{3mm}
			
			\par{\kaishu 根据我们在式(8)中得到的结论,有:\(\oint_{\partial V}\frac{\textbf{R}}{R^3}\cdot\textbf{n}\,\mathrm{d}s = 4\pi\),代入上面的 \(\nabla\cdot\textbf{E}\),我们得到:}
			
			\begin{equation}
				\nabla\cdot\textbf{E} = \lim_{V\to0}\frac{1}{|V|}\frac{Q}{\varepsilon_0} \nonumber
			\end{equation}
			
			\par{\kaishu 从上式我们看出 \(\lim_{V\to0}\frac{1}{|V|} = +\infty\),因此我们可以引入 \(\delta\) 函数,点电荷的电荷分布写为 \(Q\delta(\textbf{r})\),严格来讲,我们定义的 \(\delta\) 函数是一维的,而此处的 \(\delta\) 函数是三维的,应该写成 \(\delta(\textbf{r}) = \delta(x)\delta(y)\delta(z)\)。此时我们终于可以写出点电荷的微分形式高斯定理:}
			
			\begin{equation}
				\nabla\cdot\textbf{E} = \frac{\rho(\textbf{r})}{\varepsilon_0} = \frac{Q\delta(\textbf{r})}{\varepsilon_0}
			\end{equation}
			
			\vspace{3mm}
			
			\par{\kaishu 不仅如此,我们还得到了 \(\delta\) 函数的另一个非常有用的性质:}
			
			\begin{equation}
				4\pi\delta(\textbf{r}) = \nabla\cdot\left(\frac{\textbf{R}}{R^3}\right) =- \nabla^{2}\left(\frac{1}{R}\right)
			\end{equation}
			
		\end{mdframed}
	\end{center}
	
	\subsubsection{\kaishu 静电场环路定理}
	
	\par{\kaishu 现在我们将目光集中在电场的旋度 \(\nabla\times\textbf{E}\) 上,直接通过旋度的定义式求解比较繁琐。我们可以先求解静电场中最简单的模型(点电荷的旋度)。考虑电荷量为 \(Q\) 的点电荷,在光滑的闭合曲线 \(L\) 上做曲线积分:}
	
	\begin{equation}
		\oint_{L}\textbf{E}\cdot\mathrm{d}\textbf{r} = \frac{Q}{4\pi\varepsilon_0}\oint_{L}\frac{\textbf{R}}{R^3}\cdot\mathrm{d}\textbf{r} = -\left.\frac{Q}{4\pi\varepsilon_0 R}\right|_{|\textbf{R}_0|}^{|\textbf{R}_0|} = 0 
	\end{equation}
	
	\vspace{3mm}
	
	\par{\kaishu 对于任意分布,我们有:}
	
	\begin{equation}
		\oint_{L}\textbf{E}\cdot\mathrm{d}\textbf{r} = \oint_{L}\sum_{j}\textbf{E}_{j}\cdot\mathrm{d}\textbf{r} = \sum_{j}\oint_{L}\textbf{E}_{j}\cdot\mathrm{d}\textbf{r} = 0
	\end{equation}
	
	\begin{center}
		\includegraphics[width=0.5\linewidth]{C:/Users/16965/OneDrive/Pictures/Electrondynamics/Electrodynamics fig 1-1-6.pdf}
		\captionof{figure}{\text{\kaishu 点电荷在闭合曲线\(L\)内,对这个环路进行积分,得到式(1.1.18)}}\label{fig:1-1-6}
	\end{center}
	
	\par{\kaishu 曲线积分的计算相比旋度更加简单和直观,因此,我们可以转换思路,用斯托克斯定理将旋度的问题转换为曲线积分,通过研究电场的曲线积分求解 \(\nabla\times\textbf{E}\) 的问题。下面我们先给出 \(\mathbb{R}^3\) 上的斯托克斯定理:} 
	
	\begin{center}
		\begin{mdframed}[frametitle={\text{\kaishu 斯托克斯定理(Stokes' Theorem)}}, frametitlerule=true, frametitlebackgroundcolor=blue!05]
			\par{\kaishu 设光滑曲面 \(S\) 的边界曲线 \(L\) 是按段光滑的连续曲线,设向量场 \(\textbf{A}\) 的分量函数在 \(S\)(连同 \(L\))上连续且具有一阶连续的偏导数,则向量场满足:}
			
			\begin{equation}
				\iint_{S}\nabla\times\textbf{A}\cdot\mathrm{d}\textbf{S} = \oint_{L}\textbf{A}\cdot\mathrm{d}\textbf{r}
			\end{equation}
			
			\vspace{3mm}
			
			\par{\kaishu 其中:\(\textbf{A}\cdot\mathrm{d}\textbf{r} = A_x \,\mathrm{d}x+A_{y}\,\mathrm{d}y+A_z \,\mathrm{d}z\),\(A_x\)、\(A_y\)、\(A_z\) 分别表示 \(\textbf{A}\) 在 \(x\)、\(y\)、\(z\) 三个方向的分量。且曲面 \(S\) 与边界曲线 \(L\) 的侧由右手法则确定。}
			
		\end{mdframed}
	\end{center}
	
	\vspace{3mm}
	
	\par{\kaishu 式(1.1.17)成立的前提是:电场的平方反比定律。此时我们再使用斯托克斯定理,有:}
	
	\begin{equation}
		\oint_{L}\textbf{E}\cdot\mathrm{d}\textbf{r} = \iint_{S}\nabla\times\textbf{E}\cdot\mathrm{d}\textbf{S} = 0 \nonumber
	\end{equation}
	
	\vspace{3mm}
	
	\par{\kaishu 我们考虑的闭合曲线 \(L\) 是任意的,所以我们最后得到:}
	
	\vspace{-3mm}
	
	\begin{equation}
		\nabla\times\textbf{E} = 0
	\end{equation}
	
	\vspace{3mm}
	
	\par{\kaishu 现在我们已经“证明”了静电场是没有旋度的,而且对于任意的闭合曲线 \(L\),有:\(\oint_{L}\textbf{E}\cdot\mathrm{d}\textbf{r}=0\)。由于静电场无旋的特性,我们可以定义一个新的物理量——\textbf{电势}。下面是数学上的引入:}
	
	\par{\kaishu 考虑一个性质良好的标量场 \(\varphi\),我们先取其梯度,再取旋度,即 \(\nabla\times\nabla\varphi\):}
	
	\vspace{-3mm}
	
	\begin{equation}
		\nabla\times\nabla\varphi = \epsilon^{ijk}\hat{e}_{i}\partial_{j}\left(\nabla\varphi\right)_{k} = \epsilon^{ijk}\hat{e}_{i}\partial_{j}\partial_{k}\varphi
	\end{equation}
	
	\vspace{3mm}
	
	\par{\kaishu 其中的角标均采用爱因斯坦求和规则,由于 \(\partial_{j}\partial_{k}\varphi = \partial_{k}\partial_{j}\varphi\),而 \(\epsilon^{ijk}\) 交换指标是反对称的,求和以后抵消为 \(0\),因此我们得到结论:\(\nabla\times\nabla\varphi = 0\)。对比上面的电场,我们猜想,是不是可以定义一个标量场 \(\varphi\),满足 \(\textbf{E} = \nabla\varphi\),实际上,这里的标量场就是我们想要的电势,只不过电势的定义式为:}
	
	\begin{equation}
		\textbf{E} = -\nabla\varphi
	\end{equation}
	
	\begin{center}
		\includegraphics[width=0.5\linewidth]{C:/Users/16965/OneDrive/Pictures/Electrondynamics/Electrodynamics fig 1-1-7.pdf}
		\captionof{figure}{\text{\kaishu 在闭合曲线\(L\)上取\(A\)、\(B\)两点,将曲线分段}}\label{fig:1-1-7}
	\end{center}
	
	\par{\kaishu 电势的定义与我们的猜想差了一个负号,为什么?我们还需要考虑 \(\oint_{L}\textbf{E}\cdot\mathrm{d}\textbf{r}\) 的条件:在闭合曲线 \(L\) 上取定 \(A\)、\(B\) 两个不同的点,将曲线积分化为如下形式:}
	
	\begin{equation}
		\oint_{L}\textbf{E}\cdot\mathrm{d}\textbf{r} = \left(\int_{A}^{B}\textbf{E}\cdot\mathrm{d}\textbf{r}\right)_{L_{1}}+\left(\int_{B}^{A}\textbf{E}\cdot\mathrm{d}\textbf{r}\right)_{L_2} = 0
	\end{equation}
	
	\vspace{2mm}
	
	\par{\kaishu 其中:\(\int_{A}^{B}\textbf{E}\cdot\mathrm{d}\textbf{r}\) 意为曲线积分的起点为 \(A\),终点为 \(B\),从上面的积分中,我们知道:}
	
	\[
	\left(\int_{A}^{B}\textbf{E}\cdot\mathrm{d}\textbf{r}\right)_{L_{1}} = \left(\int_{A}^{B}\textbf{E}\cdot\mathrm{d}\textbf{r}\right)_{L_2}
	\]
	
	\vspace{2mm}
	
	\par{\kaishu 我们可以发现,在 \(\mathbb{R}^3\) 上,电场 \(\textbf{E}\) 的曲线积分与路径的选取无关,只与选取路径的\textbf{起始点}和\textbf{终结点}有关,即对于任意以 \(A\) 为起点,\(B\) 为终点的路径 \(l\),电场 \(\textbf{E}\) 满足:}
	
	\begin{equation}
		\oint_{l}\textbf{E}\cdot\mathrm{d}\textbf{r} = \text{constant} = \varphi_{B}-\varphi_{A}
	\end{equation}
	
	\vspace{2mm}
	
	\par{\kaishu 我们可以将式(1.1.24)与动能定理进行类比(电场与作用力的量纲只差一个电荷),将类比的“电场动能”代入上式,可以写出:}
	
	\begin{equation}
		\oint_{l}\textbf{E}\cdot\mathrm{d}\textbf{r} = E_{k_B}-E_{k_A} = \varphi_{B}-\varphi_{A}\nonumber
	\end{equation}
	
	\vspace{3mm}
	
	\par{\kaishu 我们还可以发现,\(\varphi_{A}\) 类似于经典力学中的保守力势能,上式需要满足能量守恒,为了方便,做变换 \(\varphi\to-\varphi\),经过移项,上式变为:}
	
	\begin{equation}
		E_{k_B}-E_{k_A} = -\varphi_{B}+\varphi_{A} \qquad\Rightarrow\qquad E_{k_B}+\varphi_{B} = E_{k_A}+\varphi_{A}
	\end{equation}
	
	\vspace{3mm}
	
	\par{\kaishu 式(1.1.25)更加清晰地表达了静电场中能量守恒的物理本质,因此人们将电势定义为 \(\textbf{E} = -\nabla\varphi\)。}
	
	\vspace{2mm}
	
	\par{\kaishu 现在我们要求解电势的表达式,首先关注电场的表达式:\(\textbf{E}(\textbf{r}) = \frac{1}{4\pi\varepsilon_0}\int_{\tau}\frac{\rho(\textbf{r})\mathrm{d}\tau'}{r^3}\textbf{r}\),我们利用下面的两个恒等式(证明见附录):}
	
	\begin{equation}
		\nabla|\textbf{r}| = \frac{\textbf{r}}{|\textbf{r}|} \qquad \nabla\left(\frac{1}{|\textbf{r}|}\right) = -\frac{\textbf{r}}{|\textbf{r}|^3} \nonumber
	\end{equation}
	
	\par{\kaishu 因此对于电场,我们有:}
	
	\begin{equation}
		\textbf{E}(\textbf{r}) = \frac{1}{4\pi\varepsilon_0}\int_{\tau}\frac{\rho(\textbf{r})\mathrm{d}\tau'}{r^3}\textbf{r} = -\frac{1}{4\pi\varepsilon_0}\nabla\int_{\tau}\frac{\rho(\textbf{r})\mathrm{d}\tau'}{r}
	\end{equation}
	
	\vspace{3mm}
	
	\par{\kaishu 我们将式(1.1.26)与电势的定义 \(\textbf{E} = -\nabla\varphi\) 对比,得到电势的表达式:}
	
	\begin{equation}
		\varphi = \frac{1}{4\pi\varepsilon_0}\int_{\tau}\frac{\rho(\textbf{r})\mathrm{d}\tau'}{r}
	\end{equation}
	
	\vspace{3mm}
	
	\par{\kaishu 在上述的讨论中,我们得到了电势的微分表达式和积分表达式,经过讨论我们发现:静电场的“无旋”性;“保守”性;可以定义标量势这三个特性是相互关联的,物理本质均来源于静电场是向心力场(仍然是平方反比律)。}
	
	\par{\kaishu 现在,\(\nabla\cdot\textbf{E} = \frac{\rho}{\varepsilon_0}\) 和 \(\nabla\times\textbf{E} = 0\) 帮助我们建立了完整的静电场物理图像:电场是有“源”场(从正电荷发散出来,在负电荷上汇聚);电场是无旋场(电场有头有尾,有始有终)。两个简洁的表达式蕴含丰富的物理涵义,囊括了众多复杂的静电现象,这是优美的理论。}
	
	\subsubsection{\kaishu 电偶极子}
	
	\par{\kaishu 电偶极子是静电学里,除了点电荷以外最简单的物理模型,我们研究电偶极子的原因如下:电偶极子是中性物质对于外加电场的最低阶响应,当一个中性物体受到外界电场的影响,电场将物体中的正、负电荷拉开一段间距。为了描述物质对电场的这种响应,人们定义\textbf{电偶极子}——两个相距很近的、带有等量正负电荷组成的体系。}
	
	\begin{center}
		\begin{mdframed}[frametitle={\text{\kaishu 电偶极矩}}, frametitlerule=true, frametitlebackgroundcolor=blue!05]
			\par{\kaishu 空间中存在电荷量均为 \(q\) 的正、负电荷,间距为 \(l\),这个系统组成电偶极子,其电偶极矩为:}
			
			\begin{equation}
				\textbf{p} = q\textbf{l}
			\end{equation}
			
			\vspace{3mm}
			
			\par{\kaishu 其中间距 \(\textbf{l}\) 的方向是由负电荷指向正电荷,而且数值较小。}
			
		\end{mdframed}
	\end{center}
	
	\vspace{3mm}
	
	\par{\kaishu 电偶极矩的定义看起来有点突兀,下面我们先计算两个相距为 \(l\) 点电荷的电场。现在我们可以直接通过电势 \(\varphi\) 来计算电场(电势为标量,计算更加简单),空间中的电势为:}
	
	\begin{equation}
		\varphi = \frac{1}{4\pi\varepsilon_0}\left(\frac{q}{r_{+}}-\frac{q}{r_{-}}\right)
	\end{equation}
	
	\begin{center}
		\begin{minipage}[t]{0.4\textwidth}
			\centering
			\includegraphics[width=7cm]{C:/Users/16965/OneDrive/Pictures/Electrondynamics/Electrodynamics fig 1-1-8.pdf}
			\captionof{figure}{\text{\kaishu 电偶极子示意图}}\label{fig:1-1-8}
		\end{minipage}
		\begin{minipage}[t]{0.5\textwidth}
			\centering
			\includegraphics[width=8cm]{C:/Users/16965/OneDrive/Pictures/Electrondynamics/Electrodynamics fig 1-1-9.pdf}
			\captionof{figure}{\text{\kaishu 电偶极子在空间中一点的电势以及电场示意}}\label{fig:1-1-9}
		\end{minipage}
	\end{center}
	
	\par{\kaishu 其中:\(r_+\)、\(r_-\) 分别为正、负电荷到观察点的距离,\(r_- = r+\frac{l}{2}\)、\(r_+ = r-\frac{l}{2}\),将电偶极子连线的中点作为坐标原点,\(r\) 为坐标原点到观察点的距离,同时电偶极矩的方向沿 \(z\) 方向:}
	
	\begin{equation}
		\varphi = \frac{1}{4\pi\varepsilon_0}\left(\frac{q}{\left|\textbf{r}-\frac{1}{2}\textbf{l}\right|}-\frac{q}{\left|\textbf{r}+\frac{1}{2}\textbf{l}\right|}\right)
	\end{equation}
	
	\vspace{3mm}
	
	\par{\kaishu 我们通过余弦定理,将分母转化为:}
	
	\begin{equation}
		\begin{cases}
			\frac{1}{\left|\textbf{r}-\frac{1}{2}\textbf{l}\right|} = \frac{1}{\sqrt{r^2+\frac{l^2}{4}-rl\cos\theta}} = \frac{1}{r}\left(1+\frac{l^2}{4r^2}-\frac{l}{r}\cos\theta\right)^{-\frac{1}{2}} \\ \\
			\frac{1}{\left|\textbf{r}+\frac{1}{2}\textbf{l}\right|} = \frac{1}{\sqrt{r^2+\frac{l^2}{4}+rl\cos\theta}} = \frac{1}{r}\left(1+\frac{l^2}{4r^2}+\frac{l}{r}\cos\theta\right)^{-\frac{1}{2}}
		\end{cases}\nonumber
	\end{equation}
	
	\vspace{3mm}
	
	\par{\kaishu 上面的算法非常精确,但是过于繁琐了,现在我们考虑简化的条件——远场条件:\(\frac{l}{r} \ll 1\),那我们可以利用泰勒展开(Taylor expansion)将上式写成:}
	
	\begin{equation}
		\left(1+\frac{l^2}{4r^2}\pm\frac{l}{r}\cos\theta\right)^{-\frac{1}{2}} = 1 \mp \frac{l}{2r}\cos\theta-\frac{l^2}{8r^2} + \cdots = 1 \mp \frac{l}{2r}\cos\theta + \mathcal{O}\left(\frac{l}{r}\right) \nonumber
	\end{equation}
	
	\vspace{3mm}
	
	\par{\kaishu 略去二次及更高项,我们可以将得到的结果代入式(30),最终计算出:}
	
	\begin{equation}
		\varphi = \frac{q}{4\pi\varepsilon_0r}\left(1+\frac{l}{2r}\cos\theta-1+\frac{l}{2r}\cos\theta\right) = \frac{ql\cos\theta}{4\pi\varepsilon_0r^2} = \frac{\textbf{p}\cdot\textbf{r}}{4\pi\varepsilon_0r^3}
	\end{equation}
	
	\vspace{3mm}
	
	\par{\kaishu 我们现在利用 \(\textbf{E} = -\nabla\varphi\) 计算电场,注意现在我们仍然需要远场条件。}
	
	\begin{equation}
		\textbf{E} = -\nabla\varphi = -\frac{1}{4\pi\varepsilon_0}\nabla\left(\frac{\textbf{p}\cdot\textbf{r}}{r^3}\right)
	\end{equation}
	
	\vspace{3mm}
	
	\par{\kaishu 现在我们需要使用公式 \(\nabla\left(\textbf{A}\cdot\textbf{B}\right) = \left(\textbf{A}\cdot\nabla\right)\textbf{B}+\left(\textbf{B}\cdot\nabla\right)\textbf{A}+\textbf{B}\times\left(\nabla\times\textbf{A}\right)+\textbf{A}\times\left(\nabla\times\textbf{B}\right)\) 但是此时 \(\textbf{p}\) 是常矢量,我们可以简化上面的计算:}
	
	\begin{equation}
		\nabla\left(\frac{\textbf{p}\cdot\textbf{r}}{r^3}\right) = \left(\textbf{p}\cdot\nabla\right)\frac{\textbf{r}}{r^3}+\textbf{p}\times\left(\nabla\times\frac{\textbf{r}}{r^3}\right)\nonumber
	\end{equation}
	
	\vspace{3mm}
	
	\par{\kaishu 其中 \(\nabla\times\frac{\textbf{r}}{r^3} = 0\),现在计算上式的第一项:}
	
	\vspace{-5mm}
	
	\begin{align*}
		\left(\textbf{p}\cdot\nabla\right)\frac{\textbf{r}}{r^3} =  p_{x_{i}}\textbf{r}\frac{\partial}{\partial x_i}\left(\frac{1}{r^3}\right)+p_{x_i}\frac{1}{r^3}\frac{\partial}{\partial x_i}\textbf{r} = -\frac{3\textbf{r}\left(\textbf{p}\cdot\textbf{r}\right)-\textbf{p}r^2}{r^5}
	\end{align*}
	
	\vspace{3mm}
	
	\par{\kaishu 将上式代入式(1.1.32),那么我们就可以得到电偶极子的电场:}
	
	\begin{equation}
		\textbf{E} = \frac{1}{4\pi\varepsilon_0}\frac{\left[3\hat{r}\left(\textbf{p}\cdot\hat{r}\right)-\textbf{p}\right]}{r^3}
	\end{equation}
	
	\vspace{3mm}
	
	\par{\kaishu 其中,\(\hat{r}\) 表示 \(\textbf{r}\) 方向的单位向量。我们注意到:电偶极子的电场和电势相较于点电荷的电场和电势在远场衰减更快。(衰减较快的意思是,分母上位置矢量的次数更高)根据图像我们发现:电偶极子的对称性低于点电荷,高于其他更复杂的模型。我们从球坐标分量 \(\left(r \,,\theta\,,\phi\right)\) 中可以更清晰地看到这一特性,假设电偶极矩 \(\textbf{p} = p\hat{z}\):}
	
	\begin{equation}
		E_{r} = \textbf{E}\cdot\hat{r} = \frac{p\cos\theta}{2\pi\varepsilon_0 r^3} \qquad E_{\theta} = \textbf{E}\cdot\hat{\theta} = \frac{p\sin\theta}{4\pi\varepsilon_0r^3} \qquad E_{\phi} = \textbf{E}\cdot\hat{\phi} = 0
	\end{equation}
	
	\vspace{3mm}
	
	\par{\kaishu 上面最后的 \(\phi\) 分量为0,除了直接计算出来,还可以认为电偶极子的电场关于 \(\phi\) 分量球对称。}
	
	\subsection{\kaishu 静磁现象的基本理论描述}
	
	\subsubsection{\kaishu 电流(磁的来源)}
	
	\par{\kaishu 1820年,丹麦物理学家奥斯特发现:电流也可以和磁铁一样产生类似的磁场,并且电流也可以像小磁针一样感受到磁场的作用力。人们开始基于(容易理解并且控制)的电流定量研究磁现象。在下面的内容中,我们将会发现:磁现象在很多方面与电现象有很直观的对应关系。}
	
	\par{\kaishu 我们先从电流开始这一部分。\textbf{电流}——即电荷在空间中的流动,电流定量的定义如下:}
	
	\begin{center}
		\begin{mdframed}[frametitle={\text{\kaishu 电流}}, frametitlerule=true, frametitlebackgroundcolor=blue!05]
			\par{\kaishu 单位时间内垂直穿过某一特定截面的电荷量,用 \(I\) 表示:}
			\begin{equation}
				I = \left.\frac{\Delta Q}{\Delta t}\right|_{S}
			\end{equation}
			\par{\kaishu 其中,\(S\) 表示我们选定的截面;\(I\) 描述的是电荷流动的积分(总效果)。}
		\end{mdframed}
	\end{center}
	
	\vspace{3mm}
	
	\par{\kaishu 现在我们要观察电荷流动的局域(local)情况,那我们可以定义电流密度:\textbf{电流密度} \(\vec{j}\) ——在某一个观察点附近单位面积单位时间通过的电荷量,数学表达式为:}
	
	\vspace{-2mm}
	
	\[
	j = \frac{\Delta q}{\Delta t\Delta S}
	\]
	
	\vspace{3mm}
	
	\par{\kaishu 再进一步假设:该点附近电荷密度为 \(\rho\),电荷的运动速度为 \(v\),可以推知(如图1.2.1所示):}
	
	\begin{equation}
		j = \frac{\rho\Delta V}{\Delta t\Delta S} = \frac{\rho v\Delta t\Delta S}{\Delta t\Delta S} = \rho v
	\end{equation}
	
	\vspace{3mm}
	
	\par{\kaishu 电流密度 \(j\) 和电流 \(I\) 之间的关系为 \(I = \int_{S}j \,\mathrm{d}s\)。我们进一步考虑电流密度的矢量性,推广以上的结果,我们可以写出矢量形式的电流密度:}
	
	\vspace{-3mm}
	
	\begin{equation}
		\vec{j} = \rho\left(\textbf{r}\right)\vec{v}\left(\textbf{r}\right)
	\end{equation}
	
	\vspace{1mm}
	
	\par{\kaishu 式(1.2.3)中 \(\vec{v}\) 代表在 \(\textbf{r}\) 处电荷的平均速度,\(\rho\) 为电荷密度,考虑电流密度的矢量性以后,其与电流 \(I\) 之间可以写成更加一般的形式:}
	
	\vspace{-3mm}
	
	\begin{equation}
		I = \int_{S}\vec{j}\cdot\mathrm{d}\textbf{S}
	\end{equation}	
	
	\vspace{3mm}
	
	\par{\kaishu 事实上,式(1.2.4)的形式是合理的,因为只有投影到 \(\mathrm{d}\textbf{S}\) 方向上的电流密度才能“通过”这块面积,而与其平行的分量对通过此截面积的总电流 \(I\) 没有贡献。另外,积分后得到的正负也有明确的物理意义:大于0,意味着电流为正向(单位时间内有净的正电荷通过该表面);小于0,则意味着电流为负方向(单位时间内有净的负电荷通过该表面,或者正电荷反向通过)。}
	
	\begin{center}
		\includegraphics[width=0.65\linewidth]{C:/Users/16965/OneDrive/Pictures/Electrondynamics/Electrodynamics fig 1-2-1.pdf}
		\captionof{figure}{\text{\kaishu 电荷在体积为 \(v\Delta t\Delta S\) 圆柱内以速度 \(\vec{v}\) 运动}}\label{fig:1-2-1}
	\end{center}
	
	\par{\kaishu \textbf{电荷守恒:}实验标明电荷是守恒的,电荷并不能产生或消灭(至少在经典电动力学体系里,我们认为如此)。在空间内任意取一个封闭曲面 \(S\),单位时间内穿过曲面流出的电荷量为(封闭曲面的法向方向定义为垂直于曲面向外):\(\oint_{S}\vec{j}\cdot\mathrm{d}\textbf{S}\),流出的电荷量应该等于封闭曲面 \(S\) 内部区域总电荷在单位时间内的减少量:\(-\frac{\mathrm{d}}{\mathrm{d}t}\int_{V}\rho\,\mathrm{d}\tau\),\(V\) 是 \(S\) 包围的体积。写出方程:}
	
	\begin{equation}
		\oint_{S}\vec{j}\cdot\mathrm{d}\textbf{S}+\frac{\mathrm{d}}{\mathrm{d}t}\int_{V}\rho\,\mathrm{d}\tau = 0
	\end{equation}
	
	\begin{center}
		\includegraphics[width=0.7\linewidth]{C:/Users/16965/OneDrive/Pictures/Electrondynamics/Electrodynamics fig 1-2-2.pdf}
		\captionof{figure}{\text{\kaishu 电荷守恒示意图}}\label{fig:1-2-2}
	\end{center}
	
	\par{\kaishu 根据高斯定理,我们可以将等式左边第一项化简为:}
	
	\vspace{-3mm}
	
	\[
	\oint_{S}\vec{j}\cdot\mathrm{d}\textbf{S} = \int_{V}\nabla\cdot\vec{j}\,\mathrm{d}\tau
	\]
	
	\vspace{3mm}
	
	\par{\kaishu 将此结果代入式(1.2.5)中,我们可以得到:}
	
	\begin{equation}
		\int_{V}\left(\nabla\cdot\vec{j}+\frac{\partial\rho}{\partial t}\right)\mathrm{d}\tau = 0
	\end{equation}
	
	\vspace{3mm}
	
	\par{\kaishu 由于 \(S\) 是任意选取的,所以被积函数恒为 0,即:}
	
	\begin{equation}
		\nabla\cdot\vec{j}+\frac{\partial\rho}{\partial t} = 0
	\end{equation}
	
	\vspace{3mm}
	
	\par{\kaishu 式(1.2.7)是电荷守恒定律的数学表达式,也称为连续性方程。}
	
	\par{\kaishu \textcolor{blue}{注意:1. 所有的“流密度”的微观形式都是“密度乘速度”,如粒子流密度,能流密度;它们的物理意义均是单位时间内单位面积通过的“粒子”数。守恒律的普遍形式可以写成}
		
		\vspace{-2mm}
		
		\textcolor{blue}{\begin{equation}
				\text{流密度的散度}+\text{数密度的变化率} = 0 \nonumber
	\end{equation}}}
	
	\vspace{-3mm}
	
	\par{\kaishu \textcolor{blue}{2. 在稳定电流情况下,由于 \(\frac{\partial\rho}{\partial t} = 0\),所以有 \(\nabla\cdot\vec{j} = 0\),电流密度的散度为0。这一点从几何方面看,意味着电流在空间任何一点均没有源头,这表示稳恒电流下电流线是闭合无源的。非稳恒时的电流汇聚/发散总是伴随着电荷在局域中的积累,亦即 \(\frac{\partial\rho}{\partial t}\) 项。}}
	
	\subsubsection{\kaishu 安培定律(Ampère's Law)}
	
	\par{\kaishu 与库伦定律不同,我们考虑的是基本单位电流元(而非电荷元)之间的作用力,电流元定义为 \(\vec{j}\,\mathrm{d}\tau\)(与 \(\rho\,\mathrm{d}\tau\) 地位相仿)。安培定律也可以被称为静磁学的“库仑定律”。考虑真空中存在两个电流元 \(\vec{j}_1\,\mathrm{d}\tau_1\)、\(\vec{j}_2\,\mathrm{d}\tau_2\),电流元2对电流元1的作用力为:}
	
	\begin{equation}
		\mathrm{d}\textbf{F}_{12} = \frac{\mu_0}{4\pi}\frac{\vec{j}_1\,\mathrm{d}\tau_1\times\left(\vec{j}_2\,\mathrm{d}\tau_2\times\textbf{R}_{12}\right)}{|\textbf{R}_{12}|^{3}}
	\end{equation}
	
	\vspace{3mm}
	
	\par{\kaishu 其中:\(\textbf{R}_{12} = \textbf{r}_1-\textbf{r}_2\) 为2指向1的位置矢量,我们可以将安培定律与库伦定律比较:
		
		\vspace{-3mm}
		
		\begin{enumerate}
			
			\item 电流元之间的相互作用(与点电荷之间的库伦相互作用相同)也满足平方反比定律。
			
			\vspace{-3mm}
			
			\item 电流元之间的相互作用不是向心力,静磁场的部分性质与电场不同。
			
			\vspace{-11mm}
			
			\textcolor{blue}{\item 电流元之间的相互作用不满足牛顿第三定律,即:\(\mathrm{d}\textbf{F}_{12} \ne -\mathrm{d}\textbf{F}_{21}\)} 
			
	\end{enumerate}}
	
	\vspace{-2mm}
	
	\par{\kaishu 为什么这里不满足牛顿第三定律?对这个问题简单的解释是:实际上并不存在稳定的电流元,实验测量的仅是闭合回路的情况,而两个闭合回路的情况确实满足牛顿第三定律。}
	
	\vspace{2mm}
	
	\begin{center}
		\begin{mdframed}[shadow=false]
			
			\par{\kaishu \textbf{补充:}考虑两个闭合载流线圈,线圈2对1的作用力为:}
			
			\vspace{-3mm}
			
			\begin{subequations}
				\begin{equation}
					\textbf{F}_{12} = \frac{\mu_0I_1I_2}{4\pi}\oint_{l_1}\oint_{l_2}\frac{\mathrm{d}\vec{l}_1\times\left(\mathrm{d}\vec{l}_2\times\textbf{R}_{12}\right)}{R_{12}^{3}}
				\end{equation}
				\begin{equation}
					\textbf{A}\times\left(\textbf{B}\times\textbf{C}\right) = \textbf{B}\left(\textbf{C}\cdot\textbf{A}\right)-\textbf{C}\left(\textbf{A}\cdot\textbf{B}\right)\qquad\text{\kaishu 利用矢量公式}
				\end{equation}
			\end{subequations}
			
			\vspace{-5mm}
			
			\begin{align*}
				\textbf{F}_{12} &= \frac{\mu_0I_1I_2}{4\pi}\oint_{l_1}\oint_{l_2}\frac{\mathrm{d}\vec{l}_2\left(\mathrm{d}\vec{l}_{1}\cdot\textbf{R}_{12}\right)-\textbf{R}_{12}\left(\mathrm{d}\vec{l}_1\cdot\mathrm{d}\vec{l}_2\right)}{R_{12}^{3}} \\
				&= -\frac{\mu_0I_1I_2}{4\pi}\oint_{l_2}\mathrm{d}\vec{l}_2\oint_{l_1}\mathrm{d}\vec{l}_1\cdot\nabla\frac{1}{R_{12}}-\frac{\mu_0I_1I_2}{4\pi}\oint_{l_1}\oint_{l_2}\frac{\textbf{R}_{12}\left(\mathrm{d}\vec{l}_1\cdot\mathrm{d}\vec{l}_2\right)}{R_{12}^3}
			\end{align*}
			
		\end{mdframed}
	\end{center}
	
	\begin{center}	
		\begin{mdframed}[shadow=false]
			
			\begin{center}
				\includegraphics[width=0.55\linewidth]{C:/Users/16965/OneDrive/Pictures/Electrondynamics/Electrodynamics fig 1-2-3.pdf}
				\captionof{figure}{\text{\kaishu 两个闭合载流线圈相互作用示意图}}\label{fig:1-2-3}
			\end{center}
			
			\vspace{5mm}
			
			\par{\kaishu 其中:\(\frac{\mu_0I_1I_2}{4\pi}\oint_{l_2}\mathrm{d}\vec{l}_2\oint_{l_1}\mathrm{d}\vec{l}_1\cdot\nabla\frac{1}{R_{12}} = 0\),同理我们也可以写出:\(\frac{\mu_0I_1I_2}{4\pi}\oint_{l_1}\mathrm{d}\vec{l}_1\oint_{l_2}\mathrm{d}\vec{l}_2\cdot\nabla\left(-\frac{1}{R_{21}}\right) = 0\)。又有:\(R_{21} = R_{12}\) 因此我们将 \(\textbf{F}_{12}\) 改写成:}
			
			\vspace{-5mm}
			
			\begin{align}
				\textbf{F}_{12} &= \frac{\mu_0I_1I_2}{4\pi}\oint_{l_1}\mathrm{d}\vec{l}_1\oint_{l_2}\mathrm{d}\vec{l}_2\cdot\nabla\frac{1}{R_{21}}+\frac{\mu_0I_1I_2}{4\pi}\oint_{l_1}\oint_{l_2}\frac{\textbf{R}_{21}\left(\mathrm{d}\vec{l}_1\cdot\mathrm{d}\vec{l}_2\right)}{R_{12}^3}\nonumber \\
				&= -\frac{\mu_0I_1I_2}{4\pi}\oint_{l_2}\oint_{l_1}\frac{\mathrm{d}\vec{l}_{2}\times\left(\mathrm{d}\vec{l}_1\times\textbf{R}_{21}\right)}{R_{21}^3} = -\textbf{F}_{21}
			\end{align}
			
			\vspace{3mm}
			
			\par{\kaishu 我们在数学上也证明了闭合回路之间的磁相互作用力满足牛顿第三定律。}
			
			\vspace{3mm}
			
		\end{mdframed}
	\end{center}
	
	\vspace{2mm}
	
	\par{\kaishu 也许你会进一步提问:如果我让一个电荷做匀速运动(\(v\ll c\)),这样我们就得到了一个电流元 \(\vec{j} = q\textbf{v}\delta\left(\textbf{r}-vt\hat{x}\right)\),现在我考虑两个这样匀速运动的电荷,它们之间的磁相互作用如何解释?}
	
	\par{\kaishu 这并不是一个简单的问题,如果仅从静磁场的角度,那这两个电荷之间的磁相互作用不一定满足牛顿第三定律,比如两者夹角为 \(\frac{\pi}{2}\) 时,两个电荷受到的磁相互作用。实际上,用静磁场研究运动的电荷并不能得到正确的结果,这个问题会帮助我们得到新的物理,在之后的章节会有体现。}
	
	\subsubsection{\kaishu 磁场}
	
	\par{\kaishu 我们观察式(1.2.8),类比电场的定义,通过安培力定义磁场微元:}
	
	\vspace{-3mm}
	
	\begin{equation}
		\mathrm{d}\textbf{F}_{12} = \vec{j}_{1}\mathrm{d}\tau_1\times\mathrm{d}\textbf{B}\left(\textbf{R}_{12}\right) 
		\qquad \text{\kaishu 其中:}\mathrm{d}\textbf{B}\left(\textbf{R}_{12}\right) = \frac{\mu_0}{4\pi}\frac{\vec{j}_2\mathrm{d}\tau_2\times\textbf{R}_{12}}{R_{12}^3}
	\end{equation}
	
	\par{\kaishu \(\mathrm{d}\textbf{B}(\textbf{R}_{12})\) 是电流元 \(\vec{j}_{2}\mathrm{d}\tau_2\) 在 \(\textbf{R}_{12}\) 处产生的磁场微元。\textcolor{blue}{注意:磁场和电场均满足叠加原理(见1.1.2)。}考虑一个任意的电流密度分布 \(\vec{j}(\textbf{r}')\),其在真空中 \(\textbf{r}\) 处产生的磁场 \(\textbf{B}(\textbf{r})\) 可以写成:}
	
	\vspace{-3mm}
	
	\begin{equation}
		\textbf{B}(\textbf{r}) = \frac{\mu_0}{4\pi}\int\frac{\vec{j}(\textbf{r}')\mathrm{d}\tau'\times\textbf{r}}{r^3}
	\end{equation}
	
	\vspace{3mm}
	
	\par{\kaishu 函数\(\textbf{B}(\textbf{r})\)称为\textbf{磁感应强度}(出于历史原因不称其为磁场强度),式(1.2.12)被称为毕奥-萨法尔定律。}
	
	\par{\kaishu 考虑以速度 \(\textbf{v}\) 运动的电荷 \(q\) 产生的电流密度 \(\vec{j} = q\textbf{v}\delta\left(\textbf{r}-vt\hat{x}\right)\)(\(v\ll c\)),其在磁场 \(\textbf{B}(\textbf{r})\) 中所受的磁场作用为(一般考虑磁场会给出磁感应强度):}
	
	\begin{equation}
		\textbf{F} = \int_{\tau}q\delta\left(\textbf{r}-vt\hat{x}\right)\mathrm{d}\tau\,\textbf{v}\times\textbf{B} = q\textbf{v}\times\textbf{B}
	\end{equation}
	
	\vspace{3mm}
	
	\par{\kaishu 如果我们在这个空间中加上电场 \(\textbf{E}\),则电荷 \(q\) 的受力为:}
	
	\vspace{-3mm}
	
	\begin{equation}
		\textbf{F} = q\left(\textbf{E}+\textbf{v}\times\textbf{B}\right)
	\end{equation}
	
	\vspace{3mm}
	
	\par{\kaishu 式(1.2.14)就是描述带电粒子在空间中既有电场又有磁场时的受力——洛伦兹力(Lorentz)。}
	
	\subsubsection{\kaishu 磁场的散度与旋度}
	
	\par{\kaishu 定义了磁感应强度 \(\textbf{B}(\textbf{r})\) 以后,我们需要研究其散度和旋度,以便得到磁场更多的物理性质。为了使我们的计算更加简便,改写磁场(磁感应强度)的表达式:}
	
	\vspace{-3mm}
	
	\begin{equation}
		\textbf{B}(\textbf{r}) = \frac{\mu_0}{4\pi}\int\vec{j}(\textbf{r}')\times\frac{\textbf{R}}{R^3}\mathrm{d}\tau' = \frac{\mu_0}{4\pi}\int\left(\nabla\frac{1}{R}\right)\times\vec{j}(\textbf{r}')\mathrm{d}\tau'
	\end{equation}
	
	\vspace{3mm}
	
	\par{\kaishu 现在微分算符 \(\nabla\) 在积分内部,我们可以将其挪到积分外面,这样就可以将 \(\textbf{B}\) 写成一个势场的全微分。同时我们注意到微分算符 \(\nabla\) 仅对 \(\textbf{R}\) 作用,不作用于 \(\textbf{r}'\),则有 \(\nabla\times\vec{j}(\textbf{r}') \equiv 0\);\(\vec{j}(\textbf{r}')\) 对 \(\nabla\) 来说就是一个常矢量。利用公式 \(\nabla\times\left(\textbf{a}\varphi(\textbf{r})\right) = \nabla\varphi(\textbf{r})\times\textbf{a}\)(\(\textbf{a}\) 为常矢量),磁场改写为:}
	
	\begin{equation}
		\textbf{B}(\textbf{r}) = \frac{\mu_0}{4\pi}\int\left(\nabla\frac{1}{R}\right)\times\vec{j}(\textbf{r}')\mathrm{d}\tau' = \nabla\times\left[\frac{\mu_0}{4\pi}\int\frac{\vec{j}(\textbf{r}')}{R}\mathrm{d}\tau'\right] = \nabla\times\textbf{A}(\textbf{r})
	\end{equation}
	
	\vspace{3mm}
	
	\par{\kaishu 通过式(1.2.16),我们定义了磁场的矢量势,其表达式为:}
	
	\begin{equation}
		\textbf{A}(\textbf{r}) = \frac{\mu_0}{4\pi}\int\frac{\vec{j}(\textbf{r}')}{R}\mathrm{d}\tau'
	\end{equation}
	
	\vspace{3mm}
	
	\par{\kaishu 现在我们对 \(\textbf{B}(\textbf{r})\) 求散度:}
	
	\vspace{-3mm}
	
	\begin{equation}
		\nabla\cdot\textbf{B} = \nabla\cdot\left(\nabla\times\textbf{A}\right)
	\end{equation}
	
	\vspace{1mm}
	
	\par{\kaishu \textcolor{blue}{注意:上式仅仅表明了磁场 \(\textbf{B}\) 与磁矢势 \(A\) 的数学关系,我们现在并不知道(或者说假设我们不知道)\(\nabla\cdot\textbf{B}\) 的具体值,用这个公式计算并不直观方便。}我们在此给出 \(\nabla\cdot\textbf{B}\) 较为详细的计算:}
	
	\begin{equation}
		\nabla\cdot\textbf{B} = \frac{\mu_0}{4\pi}\int\nabla\cdot\left[\vec{j}(\textbf{r}')\times\frac{\textbf{R}}{R^3}\right]\mathrm{d}\tau'
	\end{equation}
	
	\vspace{3mm}
	
	\par{\kaishu 利用矢量公式:\(\nabla\cdot\left(\textbf{A}\times\textbf{B}\right) = \textbf{B}\cdot\left(\nabla\times\textbf{A}\right)-\textbf{A}\cdot\left(\nabla\times\textbf{B}\right)\),上式积分内改写为:}
	
	\vspace{-3mm}
	
	\begin{align*}
		\nabla\cdot\left[\vec{j}(\textbf{r}')\times\frac{\textbf{R}}{R^3}\right] = \frac{\textbf{R}}{R^3}\cdot\left(\nabla\times\vec{j}(\textbf{r}')\right) - \vec{j}(\textbf{r}')\cdot\left(\nabla\times\frac{\textbf{R}}{R^3}\right) = 0
	\end{align*}
	
	\par{\kaishu 因为 \(\vec{j}(\textbf{r'})\) 与 \(\textbf{r}\) 无关,且易证明 \(\nabla\times\textbf{R} \equiv 0\),至此,我们计算得到:\(\nabla\cdot\textbf{B}(\textbf{r}) = 0\)。实际上,如果对矢量分析较为熟悉,可以发现对任意矢量 \(\textbf{C}\),其满足恒等式:\(\nabla\cdot\left(\nabla\times\textbf{C}\right) = 0\)。}
	
	\newpage
	
	\par{\kaishu 尽管我们研究的主要是稳恒电流,但是此处的推导没有依赖如下假设:电流不依赖于时间。换言之,如果随时间变化的电流 \(\vec{j}(t',\,\textbf{r}')\) 产生的磁场也可以被毕奥-萨法尔定律所描述,那么此时磁场的高斯定理也成立。在后面我们推广麦克斯韦方程组时这条性质有重要作用。}
	
	\par{\kaishu 下面我们尝试求 \(\textbf{B}(\textbf{r})\) 的旋度,利用 \(\textbf{B} = \nabla\times\textbf{A}\),我们有:}
	
	\vspace{-3mm}
	
	\begin{equation}
		\nabla\times\textbf{B} = \nabla\times\left(\nabla\times\textbf{A}\right) = \nabla\left(\nabla\cdot\textbf{A}\right)-\nabla^2\textbf{A}
	\end{equation}
	
	\vspace{3mm}
	
	\par{\kaishu 现在我们先考虑 \(\nabla\cdot\textbf{A}\):}
	
	\vspace{-3mm}
	
	\begin{equation}
		\nabla\cdot\textbf{A} = \nabla\cdot\frac{\mu_0}{4\pi}\int\frac{\vec{j}(\textbf{r}')}{R}\mathrm{d}\tau' = \frac{\mu_0}{4\pi}\int\left(\nabla\frac{1}{R}\right)\cdot\vec{j}(\textbf{r}')\mathrm{d}\tau'\nonumber
	\end{equation}
	
	\vspace{3mm}
	
	\par{\kaishu 这里我们需要利用恒等式 \(\nabla\frac{1}{R} = -\nabla'\frac{1}{R}\),其中:\(\textbf{R} = \textbf{r}-\textbf{r}'\)。\textcolor{blue}{注意:\(\nabla\) 和 \(\nabla'\) 不同,前者作用于变量 \(\textbf{r}\);后者作用于变量 \(\textbf{r}'\),另外,梯度运算作用后得到的是矢量,前者有 \(\textbf{R} = \textbf{r}-\textbf{r}'\);后者 \(-\textbf{R} = \textbf{r}'-\textbf{r}\)。}现在我们将 \(\nabla\cdot\textbf{A}\) 改写成全微分形式:}
	
	\begin{equation}
		\nabla\cdot\textbf{A} = -\frac{\mu_0}{4\pi}\int\left(\nabla'\frac{1}{R}\right)\cdot\vec{j}(\textbf{r}')\mathrm{d}\tau' = -\frac{\mu_0}{4\pi}\int\left[\nabla'\left(\frac{\vec{j}(\textbf{r}')}{R}\right)-\frac{1}{R}\left(\nabla'\cdot\vec{j}(\textbf{r}')\right)\right]\mathrm{d}\tau'
	\end{equation}
	
	\vspace{3mm}
	
	\par{\kaishu 同时我们注意到在稳恒电流的条件下,根据电流连续性方程有:\(\nabla'\cdot\vec{j}(\textbf{r}') = - \frac{\mathrm{d}\rho}{\mathrm{d}t} = 0\),故:}
	
	\begin{equation}
		\nabla\cdot\textbf{A} = -\frac{\mu_0}{4\pi}\int_{\tau'}\nabla'\cdot\left(\frac{\vec{j}(\textbf{r'})}{R}\right)\mathrm{d}\tau' = -\frac{\mu_0}{4\pi}\left.\oint\frac{\vec{j}(\textbf{r}')}{R}\mathrm{d}\textbf{S}'\,\right|_{r\to\infty} = 0
	\end{equation}
	
	\vspace{3mm}
	
	\par{\kaishu 这里我们考虑电流分布在有限区域,无穷远边界处的电流密度一定为0。\textcolor{blue}{注意:\(\nabla\cdot\textbf{A} = 0\) 只对稳恒电流产生的静磁场适用,不能随意推广到一般情况。}}
	
	\par{\kaishu 下面考虑 \(\nabla^2\textbf{A}\):}
	
	\vspace{-3mm}
	
	\begin{equation}
		\nabla^2\textbf{A} = \frac{\mu_0}{4\pi}\int\nabla^2\frac{\vec{j}(\textbf{r}')}{R}\mathrm{d}\tau' = \frac{\mu_0}{4\pi}\int\vec{j}(\textbf{r}')\left(\nabla^2\frac{1}{R}\right)\mathrm{d}\tau'
	\end{equation}
	
	\vspace{3mm}
	
	\par{\kaishu 利用下面的恒等式:}
	
	\vspace{-3mm}
	
	\begin{equation}
		4\pi\delta(\textbf{R}) = \nabla\cdot\left(\frac{\textbf{R}}{R^3}\right) = -\nabla\cdot\left(\nabla\frac{1}{R}\right) = -\nabla^2\frac{1}{R}\nonumber
	\end{equation}
	
	\vspace{3mm}
	
	\par{\kaishu 将得到的结果代入 \(\nabla^2\textbf{A}\),我们得到:}
	
	\vspace{-3mm}
	
	\begin{equation}
		\nabla^2\textbf{A} = -\mu_0\int\vec{j}(\textbf{r}')\delta(\textbf{R})\mathrm{d}\tau' = -\mu_0\vec{j}(\textbf{r})
	\end{equation}
	
	\vspace{2mm}
	
	\par{\kaishu 上式中我们用到了 \(\delta(\textbf{r})\) 的性质:\(\int\delta(\textbf{r}-\textbf{r}')f(\textbf{r}) = f(\textbf{r}')\),将得到的结果汇总:}
	
	\begin{equation}
		\nabla\times\textbf{B} = \nabla\times\left(\nabla\times\textbf{A}\right)-\nabla^2\textbf{A} = \mu_0\vec{j}(\textbf{r})
	\end{equation}
	
	\vspace{3mm}
	
	\par{\kaishu 上式即为 \(\textbf{B}(\textbf{r})\) 的旋度,磁场的旋度不为0,意味着场在“打转”,绕闭合回路一周的积分结果必然不为0,这表明磁场是有旋场,且不是保守场。}
	
	\newpage
	
	\par{\kaishu 式(1.2.25)称为安培环路定理的微分形式,对应的积分形式为:}
	
	\vspace{-3mm}
	
	\begin{equation}
		\oint_{\Gamma}\textbf{B}\cdot\mathrm{d}\vec{l} = \mu_0 I
	\end{equation}
	
	\vspace{2mm}
	
	\par{\kaishu 其中 \(I\) 表示闭合曲线 \(\Gamma\) 内穿过的总电流,具体证明过程如下。}
	
	\vspace{2mm}
	\begin{center}
		\begin{mdframed}[shadow=false]
			
			\par{\kaishu \textbf{补充:}基于我们已经证明的安培环路定理微分形式推导积分形式。
				
				\vspace{3mm}
				
				考虑一个二维光滑曲面 \(S\),边界为 \(\Gamma\),\(\nabla\times\textbf{B} = \mu_0\vec{j}(\textbf{r})\) 两边对 \(S\) 积分:}
			
			\begin{equation}
				\iint_{S}\nabla\times\textbf{B}\cdot\mathrm{d}\textbf{S} = \iint_{S}\mu_0\vec{j}(\textbf{r})\cdot\mathrm{d}\textbf{S}
			\end{equation}
			
			\vspace{3mm}
			
			\par{\kaishu 对于等式左边,我们使用斯托克斯定理得到:}
			
			\begin{equation}
				\iint_{S}\nabla\times\textbf{B}\cdot\mathrm{d}\textbf{S} = \oint_{\Gamma}\textbf{B}\cdot\mathrm{d}\vec{l}
			\end{equation}
			
			\vspace{3mm}
			
			\par{\kaishu 对于等式右边,我们根据电流密度的定义得到:}
			
			\begin{equation}
				\iint_{S}\mu_0\vec{j}(\textbf{r})\cdot\mathrm{d}\textbf{S} = \mu_0\iint_{S}\vec{j}(\textbf{r})\cdot\mathrm{d}\textbf{S} = \mu_0I
			\end{equation}
			
			\vspace{3mm}
			
			\par{\kaishu 其中 \(I\) 表示流经光滑曲面 \(S\) 内的总电流,这样我们就得到了安培环路定理的积分形式:}
			
			\begin{equation}
				\oint_{\Gamma}\textbf{B}\cdot\mathrm{d}\vec{l} = \mu_0I
			\end{equation}
			
			\vspace{3mm}
			
		\end{mdframed}
	\end{center}
	
	\subsubsection{\kaishu 磁偶极子}
	
	\par{\kaishu 我们对一个物质施加外部电场,物质内部的正负电荷受到电场作用,被拉开形成电偶极子。如果我们对物质施加外部磁场,物质对外加磁场的最低级响应是什么?}
	
	\par{\kaishu 考虑一个无磁性物质,其中的电子作随机运动,不显示宏观电流。现在我们考虑电子在磁场中的运动,由洛伦兹力 \(e\textbf{v}\times\textbf{B}\),电子作回旋运动,形成多个分子环流。我们可以从中抽象出重要的物理模型——\textbf{环形稳恒电流},在静磁学中我们称之为\textbf{磁偶极子}。下面我们给出磁偶极子的定义:}
	
	\begin{center}
		\begin{mdframed}[frametitle={\text{\kaishu 磁偶极矩}}, frametitlerule=true, frametitlebackgroundcolor=blue!05]
			\par{\kaishu 在经典电动力学中,磁偶极子为:带有恒定电流的闭合回路(一般较小),假定这个闭合回路的面积为 \(S\),定义这个磁偶极子的\textbf{磁偶极矩}为:}
			
			\vspace{-3mm}
			
			\begin{equation}
				\textbf{m} = I\textbf{S}
			\end{equation}
			
			\vspace{3mm}
			
			\par{\kaishu 其中 \(I\) 是这个回路中包含的电流,\(\textbf{S} = S\hat{n}\),是面积元矢量,大小为 \(S\),方向如下图所示。}
			
		\end{mdframed}
	\end{center}
	
	\newpage
	
	\begin{center}
		\includegraphics[width=0.6\linewidth]{C:/Users/16965/OneDrive/Pictures/Electrondynamics/Electrodynamics fig 1-2-4.pdf}
		\captionof{figure}{\text{\kaishu 磁偶极子示意图,图中磁偶极矩的方向与电流方向满足右手法则}}\label{fig:1-2-4}
	\end{center}
	
	\par{\kaishu 直接计算磁偶极子的磁感应强度 \(\textbf{B}\) 比较繁琐,类比电偶极子,我们可以先计算磁偶极子的磁矢势 \(\textbf{A}\),再通过 \(\textbf{B} = \nabla\times\textbf{A}\) 来计算 \(\textbf{B}\)。}
	
	\par{\kaishu \textcolor{blue}{注意:虽然我们在图中展示的是一个圆环回路的磁偶极子,但是在下面的推导中,并没有假定闭合回路的形状,因此我们得到的结论对任何符合定义的磁偶极子都是成立的。}}
	
	\par{\kaishu 首先写出磁矢势:}
	
	\vspace{-3mm}
	
	\begin{equation}
		\textbf{A}(\textbf{r}) = \frac{\mu_0}{4\pi}\int\frac{\vec{j}(\textbf{r}')}{R}\mathrm{d}\tau' \qquad \text{\kaishu 其中:} \qquad R = |\textbf{R}| = |\textbf{r}-\textbf{r'}|
	\end{equation}
	
	\vspace{3mm}
	
	\par{\kaishu \(\textbf{r}'\) 表示磁偶极子闭合环路上一点到坐标原点(我们取在磁偶极子内部)的距离,\(\textbf{r}\) 表示我们考察的点到坐标原点的距离。根据余弦定理,我们有:}
	
	\vspace{-3mm}
	
	\begin{equation}
		|\textbf{r}-\textbf{r}'| = \left|\sqrt{r^2+r'^2-2rr'\cos\theta}\right|\nonumber
	\end{equation}
	
	\vspace{1mm}
	
	\par{\kaishu \(\theta\) 为 \(\textbf{r}\) 和 \(\textbf{r'}\) 之间形成的夹角,这样算的磁矢势同样很精确,但是数值计算较为繁琐。现在我们考虑远场条件:\(r\gg r'\),上式采用近似:\(\frac{1}{|\textbf{r}-\textbf{r}'|}\approx \frac{1}{|\textbf{r}|}\),代入磁矢势:}
	
	\begin{equation}
		\textbf{A}(\textbf{r}) = \frac{\mu_0}{4\pi}\int\frac{\vec{j}(\textbf{r}')}{r}\mathrm{d}\tau' = \frac{\mu_0}{4\pi}\int\frac{\vec{j}(\textbf{r}')}{r}\mathrm{d}S\cdot\mathrm{d}l'
	\end{equation}
	
	\vspace{3mm}
	
	\par{\kaishu 我们将体积微元 \(\mathrm{d}\tau'\) 分解成面积微元 \(\mathrm{d}S\) 和线微元 \(\mathrm{d}l'\) 的乘积,考虑电流密度的定义,有:}
	
	\begin{equation}
		\textbf{A}(\textbf{r}) = \frac{\mu_0}{4\pi}\int\frac{I(\textbf{r}')}{r}\mathrm{d}l'
	\end{equation}
	
	\vspace{3mm}
	
	\par{\kaishu 这里我们“巧妙”使用斯托克斯定理,将电流移出积分(积分此时针对环路微元):}
	
	\begin{equation}
		\textbf{A}(\textbf{r}) = -\frac{\mu_0}{4\pi}I(\textbf{r}')\int_{S'}\mathrm{d}\textbf{S}'\times\nabla\left(\frac{1}{r}\right)
	\end{equation}
	
	\newpage
	
	\par{\kaishu 注意到 \(\int_{S'}\mathrm{d}\textbf{S}' = S'\hat{n}\),为磁偶极子环路围成的面积,并根据磁偶极矩的定义,\(\textbf{m} = I(\textbf{r})S'\hat{n}\),我们将上式化为:}
	
	\vspace{-3mm}
	
	\begin{equation}
		\textbf{A}(\textbf{r}) = -\frac{\mu_0}{4\pi}\textbf{m}\times\nabla\left(\frac{1}{r}\right) = \frac{\mu_0}{4\pi}\frac{\textbf{m}\times\textbf{r}}{r^3}
	\end{equation}
	
	\vspace{3mm}
	
	\par{\kaishu 至此,我们求出了磁偶极子在远场条件下的磁矢势。回忆我们在1.1.6中计算得到的电偶极子在远场的电势:
		\(\varphi = \frac{1}{4\pi\varepsilon_0}\frac{\textbf{p}\cdot\textbf{r}}{r^3}\)。不难发现,静电场与静磁场的偶极子有极高的相似性,在本节末尾我们会列表总结这一点。}
	
	\par{\kaishu 下面我们继续求解磁偶极子在远场处的磁感应强度 \(\textbf{B}(\textbf{r})\):}
	
	\begin{equation}
		\textbf{B}(\textbf{r}) = \nabla\times\textbf{A}(\textbf{r}) = \nabla\times\left(\frac{\mu_0}{4\pi}\frac{\textbf{m}\times\textbf{r}}{r^3}\right)
	\end{equation}
	
	\vspace{3mm}
	
	\par{\kaishu 利用矢量分析公式:\(\nabla\times\left(\textbf{A}\times\textbf{B}\right) = \left(\textbf{A}\cdot\nabla\right)\textbf{B}-\textbf{B}\left(\nabla\cdot\textbf{A}\right)\),磁偶极矩 \(\textbf{m}\) 为常矢量,有:}
	
	\vspace{-6mm}
	
	\begin{align}
		\textbf{B}(\textbf{r}) &= \frac{\mu_0}{4\pi}\nabla\times\left(\textbf{m}\times\frac{\textbf{r}}{r^3}\right) = \frac{\mu_0}{4\pi}\left[\left(\textbf{m}\cdot\nabla\right)\frac{\textbf{r}}{r^3}-\frac{\textbf{r}}{r^3}\left(\nabla\cdot\textbf{m}\right)\right]\nonumber \\
		&= \frac{\mu_0}{4\pi}\left(m\cdot\nabla\right)\frac{\textbf{r}}{r^3} = \frac{\mu_0}{4\pi}\left[m_{x_{i}}\frac{\partial}{\partial x_{i}}\left(\frac{1}{r^3}\right)+m_{x_{i}}\frac{1}{r^3}\frac{\partial\textbf{r}}{\partial x_{i}}\right]\nonumber \\
		&= \frac{\mu_0}{4\pi}\frac{3\textbf{r}(\textbf{m}\cdot\textbf{r})-\textbf{m}r^2}{r^5} = \frac{\mu_0}{4\pi}\frac{\left[3\hat{r}\left(\textbf{m}\cdot\hat{r}\right)-\textbf{m}\right]}{r^3}
	\end{align}
	
	\vspace{3mm}
	
	\par{\kaishu 至此,我们计算得到了磁偶极子在远场条件下的磁感应强度 \(\textbf{B}(\textbf{r})\),不难发现,其与电偶极子在远场条件下的电场强度 \(\textbf{E}(\textbf{r})\) 高度相似。我们将静电场和静磁场的基本性质总结在下表中:}
	
	\begin{table}[h]
		\centering
		\label{tab-1-2-1}
		\setlength{\arrayrulewidth}{0.3mm}
		\renewcommand\arraystretch{1.9}
		\begin{tabular}[c]{|c|c|c|}
			\hline
			& \textbf{\kaishu 静电场} & \textbf{\kaishu 静磁场} \\
			\hline
			\textbf{\kaishu 场源} & \(\rho(\textbf{r}')\) & \(\vec{j}(\textbf{r}')\) \\
			\hline
			\textbf{\kaishu 场的强度} & \(\textbf{E}(\textbf{r}) = \frac{1}{4\pi\varepsilon_0}\int_{\tau}\frac{\rho(\textbf{r}')\textbf{r}}{r^3}\mathrm{d}\tau\) & \(\textbf{B}(\textbf{r}) = \frac{\mu_0}{4\pi}\int\frac{\vec{j}(\textbf{r}')\times\textbf{r}}{r^3}\mathrm{d}\tau\)\\
			\hline
			\textbf{\kaishu 场的散度} & \(\nabla\cdot\textbf{E}(\textbf{r}') = \frac{\rho}{\varepsilon_0}\) & $\nabla\cdot\textbf{B}(\textbf{r}) = 0$ \\
			\hline
			\textbf{\kaishu 场的旋度} & \(\nabla\times\textbf{E}(\textbf{r}) = 0\) & \(\nabla\times\textbf{B}(\textbf{r}) = \mu_0\vec{j}(\textbf{r})\) \\
			\hline
			\textbf{\kaishu 场的势} & \(\textbf{E}(\textbf{r}) = -\nabla\varphi;\quad\varphi = \frac{1}{4\pi\varepsilon_0}\int_{\tau}\frac{\rho(\textbf{r}')}{r}\mathrm{d}\tau\) & \(\textbf{B}(\textbf{r}) = \nabla\times\textbf{A};\quad \textbf{A} = \frac{\mu_0}{4\pi}\int\frac{\vec{j}(\textbf{r}')}{r}\mathrm{d}\tau'\)\\
			\hline
			\textbf{\kaishu 偶极子势} & \(\varphi = \frac{1}{4\pi\varepsilon_0}\frac{\textbf{p}\cdot\textbf{r}}{r^3}\) & \(\textbf{A} = \frac{\mu_0}{4\pi}\frac{\textbf{m}\times\textbf{r}}{r^3}\)\\
			\hline
			\textbf{\kaishu 偶极子场} & \(\textbf{E}_{p} = \frac{1}{4\pi\varepsilon_0}\frac{\left[3\hat{r}\left(\textbf{p}\cdot\hat{r}\right)-\textbf{p}\right]}{r^3}\) & \(\textbf{B}_m = \frac{\mu_0}{4\pi}\frac{\left[3\hat{r}\left(\textbf{m}\cdot\hat{r}\right)-\textbf{m}\right]}{r^3}\)\\
			\hline
		\end{tabular}
		\caption{\kaishu 静电场静磁场对照表}
	\end{table}
	
	\subsection{\kaishu 麦克斯韦方程组}
	
	\subsubsection{\kaishu 电磁感应定律}
	
	\par{\kaishu 自从奥斯特发现电流可以产生磁场之后,人们一直在探寻是否存在相反的效应,磁场能不能产生电流。答案是可以的:法拉第(Faraday)在总结大量实验结论的基础上,成功描述了磁场产生电流的规律,即\textbf{电磁感应定律}。对于一个闭合线圈,法拉第通过实验总结的规律如下:
		
		\vspace{-3mm}
		
		\begin{enumerate}
			
			\item 只有当通过一个线圈的\textbf{磁通量}产生变化时,线圈中才会产生感应电流。
			
			\vspace{-3mm}
			
			\item 电流的大小与线圈材料的电导率程正比(同样形状的线圈,其材料的导电性越好,电流越大)。
			
	\end{enumerate}}
	
	\vspace{-3mm}
	
	\par{\kaishu 我们在前文中并没有给出磁通量的定义,要理解磁通量,我们先给出数学上的通量定义:}
	
	\begin{center}
		\begin{mdframed}[frametitle={\text{\kaishu 通量}}, frametitlerule=true, frametitlebackgroundcolor=blue!05]
			\par{\kaishu 在 \(\mathbb{R}^3\) 空间中给定一个向量场 \(\textbf{F}\) 和一个简单有向曲面 \(\Sigma\),定义向量场 \(\textbf{F}\) 通过曲面 \(\Sigma\) 的通量 \(\Phi_F\) 为:曲面每一点 \(x\) 上的场向量 \(\textbf{F}(x)\) 在曲面法向上分量的积分:}
			
			\vspace{-3mm}
			
			\begin{equation}
				\Phi_F(\Sigma) = \iint\limits_{\Sigma}\textbf{F}\cdot\textbf{n}\,\mathrm{d}S
			\end{equation}
			
			\par{\kaishu 其中 \(\mathrm{d}S\) 是积分的面积元,\(\textbf{n}\)是曲面 \(\Sigma\) 在点 \(\left(x \,,y \,, z\right)\)处的单位法向量。\textcolor{blue}{注意:如果曲面是封闭的,例如球面,那么通常约定法向量是从曲面内指向曲面外的,所以这时候的通量是描述曲面上的场向量朝外的程度。}}
			
		\end{mdframed}
	\end{center}
	
	\vspace{3mm}
	
	\par{\kaishu 我们将上面通量的向量场 \(\textbf{F}\) 固定为 \(\textbf{B}(\textbf{r})\),此时我们得到的通量就是磁通量。实际上,我们可以将任意一个具有物理意义的向量场 \(\textbf{A}(\textbf{r})\) 代入,从而得到不同物理意义的通量。}
	
	\par{\kaishu 考虑物理空间中简单有向曲面 \(S\),磁通量的表达式写成:}
	
	\vspace{-3mm}
	
	\begin{equation}
		\Phi = \iint\limits_{S}\textbf{B}(\textbf{r})\cdot\mathrm{d}\textbf{S}
	\end{equation}
	
	\par{\kaishu 法拉第通过大量的\textbf{实验}总结出线圈的感应电动势的定量表达式:}
	
	\vspace{-3mm}
	
	\begin{equation}
		\mathcal{E} = \left|\frac{\mathrm{d}\Phi}{\mathrm{d}t}\right| = \left|\frac{\mathrm{d}}{\mathrm{d}t}\int\limits_{S}\textbf{B}(\textbf{r})\cdot\mathrm{d}\textbf{S}\right|
	\end{equation}
	
	\vspace{3mm}
	
	\par{\kaishu 遗憾的是,法拉第并没有给出感应电动势(电流)的方向,这一工作是由楞次(Lenz)完成的。用物理的语言表述楞次总结的定律:感生电流(来自于感生电动势)的产生是用来抵消磁通量的改变的。\textcolor{blue}{实际上,1834年,楞次发表楞次定律时无磁通量这一概念(磁通量概念是韦伯提出来的),因此定律不可能具有今天的表述形式。楞次是在综合法拉第电磁感应原理(发电机原理)和安培力原理的基础上,以“电动机发电机原理”的形式提出这个定律的。}}
	
	\par{\kaishu 总结法拉第和楞次的贡献,电磁感应定律的积分形式如下:}
	
	\vspace{-3mm}
	
	\begin{equation}
		\mathcal{E} = -	\frac{\mathrm{d}}{\mathrm{d}t}\int\limits_{S}\textbf{B}(\textbf{r})\cdot\mathrm{d}\textbf{S}
	\end{equation}
	
	\par{\kaishu 从磁通量的定义可以看出:磁通量的改变可以从以下两种机制中产生:}
	
	\newpage
	
	\par{\kaishu\begin{enumerate}
			
			\item 磁场本身发生变化(\(\textbf{B}(\textbf{r})\) 变化,称之为感生电动势)。
			
			\vspace{-3mm}
			
			\item 回路相对磁场发生变化(\(S\) 变化,称之为动生电动势)。
			
	\end{enumerate}}
	
	\vspace{-3mm}
	
	\par{\kaishu 电动势的定义为:外力(不考虑电荷之间的作用)将单位电量的电荷在闭合环路上驱动一周所提供的能量。\textcolor{blue}{在实际应用中,电动势只存在于局域空间:如电池内部。因此,电磁学中将电动势定义为:单位电量电荷从电池负极移动到正极时外力所做的功,这里给出的是更一般的定义。}}
	
	\par{\kaishu 我们可以写出电动势的定量表达式:}
	
	\vspace{-3mm}
	
	\begin{equation}
		\mathcal{E} = \frac{\Delta W}{q} = \oint_{L}\frac{\textbf{F}_{k}\cdot\mathrm{d}\vec{l}}{q}
	\end{equation}
	
	\vspace{3mm}
	
	\par{\kaishu 其中:\(\textbf{F}_k\) 为电荷受到的力,定义 \(\textbf{E}_k = \frac{\textbf{F}_k}{q}\),\textbf{与电场量纲相同}(实际上,电荷受到这个未知场的作用和等价的电场没有区别,因此不妨将其认为是“非静电来源的电场”),我们将上式改写成:}
	
	\vspace{-1mm}
	
	\begin{equation}
		\mathcal{E} = \oint_{L}\textbf{E}_k\cdot\mathrm{d}\vec{l}
	\end{equation}
	
	\vspace{3mm}
	
	\par{\kaishu 我们回到电磁感应定律,对于 \(\textbf{B}\) 变化导致的磁通量的变化,感应电动势满足如下等式:}
	
	\begin{equation}
		\mathcal{E} = \oint_{L}\textbf{E}_k\cdot\mathrm{d}\vec{l} = -\frac{\mathrm{d}}{\mathrm{d}t}\int_{S}\textbf{B}(\textbf{r})\cdot\mathrm{d}\textbf{S}
	\end{equation}
	
	\vspace{3mm}
	
	\par{\kaishu 等式右边的积分是关于面积元的,根据我们的假设,环路围成的面积不随时间改变,只有\(\textbf{B}\)随时间改变,因此我们可以交换积分和求导顺序,并且将导数改写成偏导数(因为实际上\(\textbf{B} = \textbf{B}(\textbf{r},\, t)\))。}
	
	\vspace{-2mm}
	
	\begin{equation}
		\oint_{L}\textbf{E}_k\cdot\mathrm{d}\vec{l} = -\int_{S}\frac{\partial\textbf{B}}{\partial t}\cdot\mathrm{d}\textbf{S}
	\end{equation}
	
	\vspace{3mm}
	
	\par{\kaishu 对等式左边利用 \(\mathbb{R}^3\) 上的斯托克斯定理,有:}
	
	\vspace{-2mm}
	
	\begin{equation}
		\nonumber\oint_{L}\textbf{E}_k\cdot\mathrm{d}\vec{l} = \int_{S}\left(\nabla\times\textbf{E}_k\right)\cdot\mathrm{d}\textbf{S}
	\end{equation}
	
	\par{\kaishu 将此结果代入上式,我们得到:}
	
	\vspace{-5mm}
	
	\begin{align}
		&\int_{S}\left(\nabla\times\textbf{E}_k+\frac{\partial\textbf{B}}{\partial t}\right)\cdot\mathrm{d}\textbf{S} = 0 \\
		&\Rightarrow\quad \nabla\times\textbf{E}_k = -\frac{\partial\textbf{B}}{\partial t}
	\end{align}
	
	\vspace{1mm}
	
	\par{\kaishu 上式就是电磁感应定律的微分表达式,其中 \(\textbf{E}_k\) 和 \(\textbf{B}\) 必须通过同一个参考系定义。}
	
	\par{\kaishu 对于回路相对磁场发生变化的情况,物理讨论比较复杂:我们考虑此时线圈在运动,线圈内电荷受到的驱动“电场”(即\(\textbf{E}_k\))所在的参考系相对于实验室参考系在运动。因此我们定义实验室参考系的驱动“电场”为\(\textbf{E}'_k\),\(\textbf{E}'_k\) 与 \(\textbf{B}\) 的关系是相对运动参考系中电磁场的关系,也无法在静电静磁学中解决,这个问题也将带来更深刻的物理。}
	
	\par{\kaishu 实际上,式(1.3.10)相比于法拉第电磁感应定律,蕴含了更丰富的物理内涵。空间磁场发生变化时:}
	
	\vspace{-3mm}
	
	\par{\kaishu\begin{enumerate}
			
			\item 空间中激发出涡旋电场,由变化的磁场产生。
			
			\vspace{-3mm}
			
			\item 涡旋电场的存在并不依赖于线圈,观察线圈中产生的电流只是我们探测这种涡旋电场的一个手段;即使空间没有线圈,涡旋电场依然存在。
			
			\vspace{-3mm}
			
			\item \(\textbf{E}_k\) 像静电场 \(\textbf{E}_s\) 一样对电荷产生驱动力,\(\textbf{F} = q\left(\textbf{E}_k+\textbf{E}_s\right)\),对于电荷,其感知的是总电场 \(\textbf{E} = \textbf{E}_k+\textbf{E}_s\),并不区分电场来源。
			
			\vspace{-3mm}
			
	\end{enumerate}}
	
	\subsubsection{\kaishu 真空中麦克斯韦方程组(Maxwell's equations)}
	
	\par{\kaishu 前面的章节中,我们给出了电磁现象的基本实验定律,可以总结为以下三个部分:}
	
	\vspace{-3mm}
	
	\begin{subequations}
		\begin{equation}
			\nabla\cdot\textbf{E}_s = \frac{\rho}{\varepsilon_{0}}\qquad\nabla\times\textbf{E}_s=0 \qquad \text{\kaishu 静止电荷}
		\end{equation}
		\begin{equation}
			\nabla\cdot\textbf{B}_s = 0 \qquad \nabla\times\textbf{B}_s = \mu_0\vec{j} \qquad \text{\kaishu 稳恒电流}
		\end{equation}
		\begin{equation}
			\nabla\times\textbf{E}_K = -\frac{\partial\textbf{B}}{\partial t} \qquad \text{\kaishu 变化磁场}
		\end{equation}
	\end{subequations}
	
	\vspace{3mm}
	
	\par{\kaishu 考虑一般的情况,所有描述电磁场的量,以及激发它们的源 \(\rho,\,\vec{j}\),都可能伴随时间变化,此时如何通过数学规律描述它们的物理行为?麦克斯韦(Maxwell)通过高超的数学技巧,在实验定律的基础上,总结出了麦克斯韦方程组。}
	
	\par{\kaishu \textbf{第一条方程}:我们从库仑定律导出 \(\nabla\cdot\textbf{E}_s = \frac{\rho}{\varepsilon_0}\),直观物理图像是:单位电荷激发 \(\frac{1}{\varepsilon_0}\) 电场。一般地,电场可以随时间变化,可能有两种起源:}
	
	\vspace{-3mm}
	
	\par{\kaishu\begin{enumerate}
			
			\item 源电荷运动;
			
			\vspace{-3mm}
			
			\item 总电场中含有感应电场 \(\textbf{E}_k\)。
			
	\end{enumerate}}
	
	\vspace{-3mm}
	
	\par{\kaishu 对于1的情况,图像上看,电荷运动是其发出的电力线跟随它运动,我们在 \(t\) 时刻 \(r\) 位置处做一个无限小的闭合空间 \(\mathrm{d}\Sigma\),可以预期此时刻电场线通过表面的总数 \(n\) 仍然与其中的电量成正比。对于2的情况,\(\textbf{E}_k\) 与静磁场非常相似,是围绕磁场变化产生的涡旋电场。从图像上看,这种电场场线是连续无源的,可以预期 \(\nabla\cdot\textbf{E}_k = 0\)。综上考虑,对于空间总电场,麦克斯韦认为其满足高斯定律:}
	
	\begin{equation}
		\nabla\cdot\textbf{E}(\textbf{r},t) = \nabla\cdot\textbf{E}_s+\nabla\cdot\textbf{E}_k = \frac{\rho(\textbf{r},t)}{\varepsilon_0}
	\end{equation}
	
	\vspace{3mm}
	
	\par{\kaishu \textcolor{blue}{注意:直接从电磁感应定律的原始形式是无法推出感应电场是无源场的结论的。\(\nabla\cdot\textbf{E}_k = 0\) 应当是麦克斯韦作出的合理推广。}}
	
	\par{\kaishu \textbf{第二条方程:}在普遍情况下,考虑总电场,而且静电场没有旋度,因此总电场旋度为:}
	
	\begin{equation}
		\nabla\times\textbf{E} = \nabla\times\textbf{E}_s+\nabla\times\textbf{E}_k = -\frac{\partial \textbf{B}}{\partial t}
	\end{equation}
	
	\vspace{3mm}
	
	\par{\kaishu \textbf{第三条方程:}\(\nabla\cdot\textbf{B}(\textbf{r},t) = 0\) 是针对静磁场推导得到的,不难注意到在推导的过程中没有假设电流必须稳恒,即 \(\nabla\cdot\textbf{j}(\textbf{r}) = 0\)。从物理图像上,散度可以刻画场的源。目前自然界不存在磁荷,即使物理量随时间变化,磁场线仍然始终连续(不存在奇点/源)。综上,在一般的情况下,我们可以大胆推论磁场的高斯定律仍成立:}
	
	\vspace{-3mm}
	
	\begin{equation}
		\nabla\cdot\textbf{B}(\textbf{r},t) = 0
	\end{equation}
	
	\vspace{3mm}
	
	\par{\kaishu 但是这个推广不能与前面两条推广后的方程产生矛盾,考虑(1.3.13)并两边取散度:}
	
	\begin{equation}
		\nabla\cdot\left(\nabla\times\textbf{E}\right) = -\nabla\cdot\frac{\partial\textbf{B}}{\partial t}
	\end{equation}
	
	\vspace{3mm}
	
	\par{\kaishu 我们知道 \(\nabla\cdot\left(\nabla\times\textbf{E}\right) = 0\),且 \(\nabla\) 与时间微分可以交换顺序,有:}
	
	\begin{equation}
		\nabla\cdot\frac{\partial\textbf{B}}{\partial t} = \frac{\partial}{\partial t}\nabla\cdot\textbf{B} = 0\quad\Rightarrow\quad\nabla\cdot\textbf{B} = \text{const}\nonumber
	\end{equation}
	
	\vspace{3mm}
	
	\par{\kaishu 我们得到的 \(\nabla\cdot\textbf{B} = \text{const}\) 常数取决于初始时刻的值。假设变化的电磁场是由初始静态的电磁场演化而来的,上式的常数可以取为0,因为我们已知静态磁场满足高斯定律。因此我们对 \(\nabla\cdot\textbf{B} = 0\) 的推广是合理的。}
	
	\par{\kaishu \textbf{第四条方程:}在推导环路定理 \(\nabla\times\textbf{B} = \mu_0\vec{j}\) 时,我们用到了稳恒电流的条件 \(\nabla\cdot\vec{j}_s = 0\)。然而 \(\nabla\cdot\vec{j} = 0\) 在非稳恒电流条件下并不成立,因为根据电荷守恒定律得到:}
	
	\begin{equation}
		\nabla\cdot\vec{j} = -\frac{\partial\rho}{\partial t}\ne0\nonumber
	\end{equation}
	
	\vspace{3mm}
	
	\par{\kaishu 因此在最一般的情况下 \(\nabla\times\textbf{B} = \mu_0\vec{j}\) 不再成立。下面我们证明这一点:对其两边同取散度:}
	
	\begin{equation}
		\nabla\cdot\left(\nabla\times\textbf{B}\right) = \mu_0\nabla\cdot\vec{j}
	\end{equation}
	
	\vspace{3mm}
	
	\par{\kaishu 根据电荷守恒定律,等式右侧显然不是0,但是根据矢量场 \(\textbf{B}\) 的性质:\(\nabla\cdot\left(\nabla\times\textbf{B}\right) = 0\)。我们发现:静态的环路定理与电荷守恒定律相悖。电荷守恒定律是更基本的定律(因其通过实验总结而来),我们需要修正安培环路定律。}
	
	\par{\kaishu 如果在一般情况下,我们假设环路定理具有如下形式:}
	
	\vspace{-2mm}
	
	\begin{equation}
		\nabla\times\textbf{B} = \mu_0\textbf{G}
	\end{equation}
	
	\vspace{3mm}
	
	\par{\kaishu 针对矢量 \(\textbf{G}\) 有如下要求:}
	
	\vspace{-3mm}
	
	\par{\kaishu \begin{enumerate}
			
			\item 矢量 \(\textbf{G}\) 具有电流的量纲,静态时退化为 \(\vec{j}\);
			
			\vspace{-3mm}
			
			\item  上式(1.3.17)应与电荷守恒定律相协调。
			
	\end{enumerate}}
	
	\vspace{-3mm}
	
	\par{\kaishu 现在我们要寻找 \(\textbf{G}\) 可能的性质,对式(1.3.17)取散度有:}
	
	\vspace{-2mm}
	
	\begin{equation}
		\nabla\cdot\left(\nabla\times\textbf{B}\right) = \mu_0\nabla\cdot\textbf{G} = 0
	\end{equation}
	
	\vspace{3mm}
	
	\par{\kaishu 因此我们得到对 \(\textbf{G}\) 的限制条件为:\(\nabla\cdot\textbf{G} = 0\),可以从电荷守恒定律的整体出发(即式(1.2.7)),考虑第一条方程,有 \(\rho = \varepsilon_{0}\nabla\cdot\textbf{E}\),代入电荷守恒定律得:}
	
	\begin{equation}
		\nabla\cdot\vec{j}+\frac{\partial\rho}{\partial t} = \nabla\cdot\vec{j}+\frac{\partial}{\partial t}\,\varepsilon_{0}\nabla\cdot\textbf{E} =\nabla\cdot\left(\vec{j}+\varepsilon_{0}\frac{\partial\textbf{E}}{\partial t}\right) = 0
	\end{equation}
	
	\vspace{3mm}
	
	\par{\kaishu 因此我们对 \(\textbf{G}\) 有一个合理的猜测:\(\textbf{G} = \vec{j}+\varepsilon_{0}\frac{\partial\textbf{E}}{\partial t}\),静态时恢复到 \(\textbf{G} = \vec{j}\),所以我们可以对安培定律作出合理的推广:}
	
	\vspace{-3mm}
	
	\begin{equation}
		\nabla\times\textbf{B} = \mu_0\vec{j}+\mu_0\,\varepsilon_{0}\frac{\partial\textbf{E}}{\partial t}
	\end{equation}
	
	\vspace{3mm}
	
	\par{\kaishu 式(1.3.20)中的 \(\varepsilon_{0}\frac{\partial\textbf{E}}{\partial t}\) 称为位移电流(实际上是电流密度),与传导电流量纲相同,但却不是真正的传导电流。便于理解,我们可以将位移电流看作电流线的延续,如图1.3.1所示。}
	
	\vspace{3mm}
	
	\begin{center}
		\includegraphics[width=0.45\linewidth]{C:/Users/16965/OneDrive/Pictures/Electrondynamics/Electrodynamics fig 1-3-1.pdf}
		\captionof{figure}{\text{\kaishu 位移电流的物理模型,有限长度的圆柱形金属线,电流驱动导致电荷在底面积累}}\label{fig:1-3-1}
	\end{center}
	
	\par{\kaishu 考虑一段有限长度的金属棒,其中激发出电流,因为电流在边界处不能连续,则必然在金属棒的两侧积累电荷,同时导致电流本身的变化。电荷的积累会在空间产生变化的电场,而其恰好是补偿真实的传导电流的变化的。\textcolor{blue}{从物理上讲,电流非稳恒的特性与电荷密度对时间的导数 \(\dot{\rho}\),通过电流守恒定律 \(\nabla\cdot\vec{j}+\dot{\rho} = 0\) 相互关联,而后者又反应到空间电场的变化。}}
	
	\par{\kaishu 因此我们可以总结出真空中电磁现象满足的普遍规律(麦克斯韦方程组):}
	
	\begin{equation}
		\begin{cases}
			\nabla\cdot\textbf{E} &= \frac{\rho}{\varepsilon_{0}}\\
			\nabla\times\textbf{E} &= -\frac{\partial\textbf{B}}{\partial t}\\
			\nabla\cdot\textbf{B} &= 0\\
			\nabla\times\textbf{B} &= \mu_0\vec{j}+\mu_0\varepsilon_{0}\frac{\partial\textbf{E}}{\partial t}
		\end{cases}
		\quad\xrightarrow[\vec{j} = 0,\, \rho = 0  ]{\text{\kaishu 无源}}\qquad
		\begin{cases}
			\nabla\cdot\textbf{E} &= 0\\
			\nabla\times\textbf{E} &= -\frac{\partial\textbf{B}}{\partial t}\\
			\nabla\cdot\textbf{B} &= 0\\
			\nabla\times\textbf{B} &= \mu_0\varepsilon_{0}\frac{\partial\textbf{E}}{\partial t}
		\end{cases}
	\end{equation}
	
	\par{\kaishu 真空中麦克斯韦方程组给我们展示了电磁场的图像(按照从上到下的顺序,我们将方程分别命名为方程 \(1\sim4\)):}
	
	\vspace{-3mm}
	
	\par{\kaishu \begin{enumerate}
			
			\item 电荷/电流会产生电场/磁场(方程1、方程3)
			
			\vspace{-3mm}
			
			\item 电场和磁场可以相互转化(方程2、方程4)
			
			\vspace{-3mm}
			
			\item 处于电磁场中的电荷/电流受到场的作用(洛伦兹力)\(\textbf{F} = q \,\textbf{E}+q \,\textbf{v}\times\textbf{B}\)
			
	\end{enumerate}}
	
	\vspace{-3mm}
	
	\par{\kaishu 从麦克斯韦方程组中,我们可以看出:电场和磁场有近乎完美的对称性(电场和磁场交换后接近等价),尤其是在无源的情况下。但是 \(\nabla\times\textbf{E}\) 和 \(\nabla\times\textbf{B}\) 有符号和系数的差异,系数的差异源于历史误会(在特殊情况中将不再有差异),而符号的差异给我们“带来”了电磁波。}
	
	\subsection{\kaishu 介质中的麦克斯韦方程组}
	
	\subsubsection{\kaishu 介质的极化与磁化}
	
	\par{\kaishu 电磁场在不同性质的介质中行为不同,从而使我们的世界色彩斑斓。要理解电磁场在介质中的行为,核心问题是理解电磁场与物质相互作用的过程。我们先给出对电磁场简单但是重要的认识:\textcolor{blue}{电磁场的来源和作用对象是电荷和电流,只要空间的给定区域存在电荷/电流(这里不考虑其起源),电荷/电流就可以产生电磁场,而其本身也会受到电磁场的作用。}}
	
	\par{\kaishu 我们先通过简单的物理分析大致描绘介质的极化与磁化:考虑一块放置于外部恒定场(电场/磁场)中的介质,电介质中的正负电荷受到电场作用被拉开,产生电偶极子,这就是极化过程;磁介质中运动的电子在外部磁场的作用下产生多个分子环流,即磁偶极子,且磁偶极子有序排列,这就是磁化过程。极化和磁化分别如图1.4.1和图1.4.2所示:}
	
	\begin{center}
		\includegraphics[width=0.5\linewidth]{C:/Users/16965/OneDrive/Pictures/Electrondynamics/Electrodynamics fig 1-4-1.pdf}
		\captionof{figure}{\text{\kaishu 电介质极化过程示意图,表示了一对正负电荷受外部电场作用形成电偶极子}}\label{fig:1-4-1}
	\end{center}
	
	\begin{center}
		\includegraphics[width=0.5\linewidth]{C:/Users/16965/OneDrive/Pictures/Electrondynamics/Electrodynamics fig 1-4-2.pdf}
		\captionof{figure}{\text{\kaishu 磁介质磁化过程示意图,表示了电子 \(e^-\) 受外部磁场作用形成磁偶极子}}\label{fig:1-4-2}
	\end{center}
	
	\par{\kaishu 极化和磁化存在另外的可能性。有些介质体系中的构成单元本身带有固有的电偶极矩/磁偶极矩,但是整个介质并不显示宏观极性/磁性(典型物质:水,顺磁介质)。因为没有外加电/磁场时,这些电/磁偶极子排列非常杂乱。然而,当我们对其施加电场或磁场时,这些原本杂乱排列的电/磁偶极子沿着电/磁场排列,产生宏观电/磁矩。}
	
	\par{\kaishu 我们定义极化强度和磁化强度两个\textbf{宏观量}。}
	
	\begin{center}
		\begin{mdframed}[frametitle={\text{\kaishu 极化强度和磁化强度}}, frametitlerule=true, frametitlebackgroundcolor=blue!05]
			\par{\kaishu 取一个\textbf{宏观小微观大}(意思是宏观上视作微元,微观上包含大量粒子)的区域,计算单位体积内的电偶极矩和磁偶极矩(假设有\(i\)个)的矢量和为:}
			
			\vspace{-2mm}
			
			\begin{equation}
				\textbf{P} = \frac{\sum_{i}\textbf{p}_i}{\Delta V}\qquad\textbf{M} = \frac{\sum_{i}\textbf{m}_i}{\Delta V}
			\end{equation}
			
			\vspace{3mm}
			
			\par{\kaishu 局域电磁场强度越大,极化和磁化强度也越大。}
			
		\end{mdframed}
	\end{center}
	
	\vspace{3mm}
	
	\par{\kaishu 对于外场产生偶极子引起的极化和磁化现象,容易知道:局域电磁场强度越大,极化和磁化强度也越大。对于外场促使物质的无序偶极子有序排列引起的极化和磁化现象,考虑外场效应(外界电/磁场强度越大,电/偶极子越趋向于平行排列),和温度效应(温度越高,电/偶极子越趋向于无规排列)之间的竞争,当局域电磁场强度大时,外场效应越明显,因此极化和磁化强度越大。}
	
	\par{\kaishu 在电磁场不太强的情形下(称之为低场近似),实验观察到,很多材料对电磁场线性响应。因此我们可以写出极化强度和磁化强度与外场的关系:}
	
	\vspace{-2mm}
	
	\begin{equation}
		\textbf{P}(\textbf{r},t) = \varepsilon_0\chi_{e}\textbf{E}(\textbf{r},t) \qquad \textbf{M}(\textbf{r},t) = \frac{1}{\mu_0}\frac{\chi_{m}}{1+\chi_{m}}\textbf{B}(\textbf{r},t)
	\end{equation}
	
	\vspace{3mm}
	
	\par{\kaishu 其中,\(\chi_e\) 为极化率,\(\chi_{m}\) 为磁化率。磁化强度的计算有点复杂,这个问题在我们引入磁场强度以后会给出合理的解答。\textcolor{blue}{注意:计算 \(\chi_{e}\) ,\(\chi_{m}\) 要用到其他理论(量子力学甚至更多),不能由经典电动力学理论本身的体系给出。}}
	
	\begin{center}
		\begin{mdframed}[shadow=false]
			
			\par{\kaishu\textbf{补充:}对介质极化和磁化的理解}
			
			\par{\kaishu 我们对研究的介质施加外部电/磁场(这里的外场由远处的源电荷/电流产生),这块电/磁介质在外场的作用下被极化/磁化,产生电/磁偶极子。电/磁偶极子的产生使介质中不呈现电/磁中性,从而产生了束缚于电/磁介质的极化电荷/磁化电流(简称束缚电荷/电流)。}
			
			\par{\kaishu 但是,根据我们对电磁场性质的认识,这些束缚电荷/电流也会在空间产生局域的电/磁场,我们称这种场为极化/磁化场。这种情况下,空间任意一点的真实总电/磁场,与我们最初“源”产生的场不同。对介质内部,空间内一点的场是“电荷/电流源”和束缚电荷/电流产生的场的矢量和。不能只考虑原来的外场。}
			
			\par{\kaishu 如果我们按照上面的物理分析来研究极化和磁化,会遇到问题:\textcolor{blue}{介质被极化/磁化的过程,和介质极化/磁化后对空间总场的修正,两个因素纠缠在一起,如何描述这种情况。}实际上,这两个因素的纠缠正是解决问题的关键:我们必须使上述两个因素达到平衡,达到平衡后的电磁场才是空间中真实存在的电磁场。}
			
		\end{mdframed}
	\end{center}
	
	\begin{center}
		\begin{mdframed}[shadow=false]
			
			\vspace{3mm}
			
			\begin{center}
				\tikzstyle{arrow} = [thick,->,>=stealth]
				\begin{tikzpicture}[
					squarednode/.style={rectangle, rounded corners, draw=blue, fill=blue!2, thick, minimum size=10mm},
					]
					
					\node[squarednode](maintopic){\textbf{\kaishu 源场:}\textbf{E},\,\textbf{B}};
					\node[squarednode, node distance=3cm](leftsquare)[left=of maintopic]{\(\boldsymbol{\rho}_f\),\,\(\vec{\textbf{j}}_f\)};
					\node[squarednode, node distance=3cm](rightsquare)[right=of maintopic]{\textbf{\kaishu 极化和磁化:}\textbf{P},\,\textbf{M}};
					\node[squarednode, node distance=2cm](belowmain)[below=of maintopic]{\textbf{\kaishu 总场:}\textbf{E'},\,\textbf{B'}};
					\node[squarednode, node distance=2.63cm](belowright)[right=of belowmain]{\textbf{\kaishu 束缚电荷/电流:}\(\boldsymbol{\rho}'\),\,\(\vec{\textbf{j}'}\)};
					
					\draw [arrow] (leftsquare.east) -- node[anchor=south]{\textbf{\kaishu 产生}}(maintopic.west);
					\draw [arrow] (maintopic.east) -- node[anchor=south]{\textbf{\kaishu 介质中}}(rightsquare.west);
					\draw [arrow] (rightsquare.south) -- node[anchor=west]{\textbf{\kaishu 产生}}(belowright.north);
					\draw [arrow] (belowright.west) -- node[anchor=south]{\textbf{\kaishu 极化/磁化场}}(belowmain.east);
					\draw [arrow] (belowmain.north) -- node[anchor=east]{\textbf{\kaishu 总场修正}}(maintopic.south);
					
				\end{tikzpicture}
			\end{center}
			
			\vspace{3mm}
			
			\par{\kaishu 这个过程用示意图表示如上,我们的极化和磁化强度正是出于这种目的定义的。理解了上图中所有的过程,我们才能描述介质中电磁场的完整图景。}
			
		\end{mdframed}
	\end{center}
	
	\subsubsection{\kaishu 极化电荷和磁化电流}
	
	\par{\kaishu \textbf{A. 极化:}由于极化现象,介质中的正负电荷发生了相对位移(如图1.4.1所示),导致每一个局域的正负电荷不能完全抵消,这样就呈现出束缚在介质中的宏观电荷。这种电荷称为\textbf{极化电荷}。下面我们尝试定量计算束缚电荷密度。}
	
	\par{\kaishu 假设空间的极化强度分布为 \(\textbf{P}(\textbf{r})\),我们取一块区域 \(\tau\),有向边界为 \(\textbf{S}\),我们现在计算 \(\tau\) 中包含的总极化电荷 \(Q_P\)。在我们的假设中,完全处于区域内部或者完全处于区域外部的偶极子对 \(Q_P\) 没有贡献(见补充讨论)。}
	
	\par{\kaishu 考虑偶极子穿过一块微元边界(记为 \(\mathrm{d}\textbf{S}\))对 \(Q_P\) 的贡献。\textcolor{blue}{假设 \(\mathrm{d}\textbf{S}\) 宏观上足够小,使得此区域内偶极子方向近似一致;微观上足够大,使得其中包含数量远大于1的偶极子。}每个偶极子的正负电荷间距为\(\vec{l}\),在体积 \(\vec{l}\cdot\mathrm{d}\textbf{S}\) 内的每个偶极子的正电荷穿过 \(\tau\) 界面,从而在 \(\tau\) 内部留下一个负的静电荷 \(q\),如图1.4.3所示。}
	
	\begin{center}
		\includegraphics[width=0.65\linewidth]{C:/Users/16965/OneDrive/Pictures/Electrondynamics/Electrodynamics fig 1-4-3.pdf}
		\captionof{figure}{\text{\kaishu 介质边界处偶极子示意图,柱形体积微元 \(\mathrm{d}V = \vec{l}\cdot\mathrm{d}\textbf{S}\),图中曲线标 \(\tau\) 的一侧为介质内部。}}\label{fig:1-4-3}
	\end{center}
	
	\par{\kaishu 假设此处介质的偶极子数量密度为 \(n\),这些偶极子对区域内的电荷数为 \(\mathrm{d}Q_P = -qn\vec{l}\cdot\mathrm{d}\textbf{S} = -\textbf{P}\cdot\mathrm{d}\textbf{S}\),\textcolor{blue}{因为考虑的是点乘,此公式对偶极子的方向和面积元的指向没有特殊要求。}\(\textbf{P}\) 为区域内的极化强度。考虑所有穿过界面的偶极子贡献,我们发现留在 \(\tau\) 区域内总的束缚电荷为:}
	
	\begin{equation}
		Q_P = -\oint_{S}\textbf{P}\cdot\mathrm{d}\textbf{S}
	\end{equation}
	
	\vspace{3mm}
	
	\par{\kaishu 对式(1.4.3)利用高斯定理,进一步得到:}
	
	\vspace{-5mm}
	
	\begin{align}
		\int_{\tau}\rho_P \,\mathrm{d}V &= -\oint_{S}\textbf{P}\cdot\mathrm{d}\textbf{S}\nonumber\\
		\Rightarrow\quad \rho_P &= -\nabla\cdot\textbf{P}
	\end{align}
	
	\vspace{1mm}
	
	\par{\kaishu 其中:\(\mathrm{d}V\) 表示介质的体积微元(包含足够多的偶极子),式(1.4.4)中的 \(\rho_P\) 就是极化电荷密度,我们得到了极化电荷的微分表示形式。}
	
	\begin{center}
		\begin{mdframed}[shadow=false]
			
			\par{\kaishu \textbf{补充:}极化电荷中的问题}
			
			\par{\kaishu 1. 我们在推导极化电荷微分表达式时,认为完全处于区域内部或者完全处于区域外部的偶极子对极化电荷 \(Q_P\) 没有贡献。对后者显然如此。对前者,考虑区域内极化过程(假设区域包含多个完整的电偶极子,且电偶极子指向近似相同),电偶极子的正电荷和相邻电偶极子的负电荷抵消,区域表现为电中性,因此这些偶极子对 \(Q_P\) 没有贡献。}
			
			\par{\kaishu 2. 简单思考式(1.4.4),即可发现问题:假设介质均匀极化 \(\textbf{P} = \textbf{C}\),\(\textbf{C}\) 为常矢量,由式(97)我们得到 \(\rho_P = 0\),介质内无极化电荷分布;但是介质极化后每个分子呈现为一个偶极子,在分子尺度考虑介质,极化电荷分布不均匀,\(\rho_P\ne0\),这个矛盾如何解决?}
			
			\par{\kaishu 我们回顾式(1.4.4)的推导即可发现问题,高斯定理要求:对任意划分作出的微元 \(\mathrm{d}V\),式(1.4.4)的推导成立。但是物理上,划分的微元 \(\Delta V_i\) 如果只包含单个(或者我们可以说——很少的)偶极子,我们就不能简单地使用积分的定义: 
				\[
				\lim_{i\to\infty}\sum_{i}\rho_P(V_i)\Delta V_{i} = \int_\tau\rho_P\mathrm{d}V
				\]
				
				\vspace{3mm}
				
				因为我们无法对单个偶极子定义极化电荷密度。因而此时的讨论是没有物理意义的。\textcolor{blue}{分子尺度计算极化电荷以及其物理量难度很高,而且这里的讨论是经典电动力学,用宏观量计算是一种虽“不正确”但有效的方法。}}
			
			\par{\kaishu 在连续介质的情况下,我们所讨论的微元定义:微观上包含大量粒子(这里是偶极子),粒子足够多能表现物质的宏观性质;宏观上体积或者面积很小,可以将这个微元近似为空间上的几何点。我们取这个微元内微观量的平均值作为在这一几何点物理量的具体数值(因为人类实验上所能测量的是平均值)。}
			
		\end{mdframed}
	\end{center}
	
	\par{\kaishu 我们继续考察极化的介质:当电场随时间改变时,极化过程中正负电荷的相对位移也随时间改变,由此产生的电流称为\textbf{极化电流},密度记为 \(\vec{j}_P\)。极化电流也必须满足连续性方程,有:}
	
	\begin{equation}
		\nabla\cdot\vec{j}_P+\frac{\partial\rho_P}{\partial t} \equiv 0
	\end{equation}
	
	\vspace{3mm}
	
	\par{\kaishu 回想我们得到的极化电荷密度表达式:\(\rho_P = -\nabla\cdot\textbf{P}\),代入式(1.4.5),得到:}
	
	\vspace{-5mm}
	
	\begin{gather}
		\nabla\cdot\vec{j}_P - \frac{\partial}{\partial t}\nabla\cdot\textbf{P} = 0 \nonumber\\
		\Rightarrow \quad \nabla\cdot\left(\vec{j}_P-\frac{\partial}{\partial t}\textbf{P}\right) = 0\nonumber
	\end{gather}	
	
	\vspace{2mm}
	
	\par{\kaishu 我们可以得到结果为:\(\vec{j}_P = \frac{\partial\textbf{P}}{\partial t}+\textbf{C}\),\(\textbf{C}\) 为常矢量,取 \(\textbf{C} = 0\),则有:}
	
	\begin{equation}
		\vec{j}_P = \frac{\partial\textbf{P}}{\partial t}
	\end{equation}
	
	\vspace{3mm}
	
	\par{\kaishu \textbf{B. 磁化:}介质被磁化后,由于内部磁偶极子的作用,产生束缚于磁介质上的磁化电流。下面我们讨论磁化电流密度 \(\vec{j}_M\)。}
	
	\par{\kaishu 假设介质的磁化强度为 \(\textbf{M}(\textbf{r})\),我们在磁介质中选取宏观大小的面积 \(S\),边界光滑,可以通过曲线积分 \(\oint_{L}\mathrm{d}l\) 描述,下面计算穿过 \(S\) 产生的总磁化电流 \(I_M\)。磁介质磁化后产生大量磁偶极子,每个磁偶极子对应一个分子环流。}
	
	\par{\kaishu 如果分子环流完全处于 \(S\) 内,对磁化电流的贡献因为环流两次穿过 \(S\) 而抵消;如果分子环流完全处于 \(S\) 外,则对磁化电流没有贡献。因此,我们只考虑边界处的分子环流(只对 \(S\) 面内贡献一次电流),如图1.4.4所示。}
	
	\begin{center}
		\includegraphics[width=0.5\linewidth]{C:/Users/16965/OneDrive/Pictures/Electrondynamics/Electrodynamics fig 1-4-4.pdf}
		\captionof{figure}{\text{\kaishu 磁介质分子环流示意图,虚线表示分子环流在 \(S\) 的另一侧,分子环流的面积设为\(\Delta s\)}}\label{fig:1-4-4}
	\end{center}
	
	\par{\kaishu 每个分子环流电流为 \(i\),面积为 \(\Delta s\),图18中在 \(\Delta s\cdot\mathrm{d}\vec{l}\) 体积内的所有磁偶极子对 \(S\) 内部的静电流有贡献。若此处偶极子的密度为 \(n\),则 \(\mathrm{d}\vec{l}\) 边界对总磁化电流 \(I_M\) 的贡献为:}
	
	\vspace{-2mm}
	
	\begin{equation}
		\mathrm{d}I_M = \vec{i}\times n\Delta s\cdot\mathrm{d}\vec{l} = \textbf{M}\cdot\mathrm{d}\vec{l}\nonumber
	\end{equation}
	
	\vspace{2mm}
	
	\par{\kaishu 再综合整个环路的贡献,我们得到:}
	
	\vspace{-2mm}
	
	\begin{equation}
		I_M = \int_{L}\mathrm{d}I_M = \oint_{L}\textbf{M}\cdot\mathrm{d}\vec{l}
	\end{equation}
	
	\vspace{2mm}
	
	\par{\kaishu 将磁化电流写成电流密度的积分:\(I_M = \int_{S}\vec{j}_M\cdot\mathrm{d}\textbf{S}\),再对曲线积分利用斯托克斯定理:}
	
	\vspace{-3mm}
	
	\begin{gather}
		\int_{S}\vec{j}_M\cdot\mathrm{d}\textbf{S} = \oint_{L}\textbf{M}\cdot\mathrm{d}\vec{l} = \int_S(\nabla\times\textbf{M})\cdot\mathrm{d}\textbf{S}\nonumber\\
		\Rightarrow\quad \int_S\left(\vec{j}_M-\nabla\times\textbf{M}\right)\cdot\mathrm{d}\textbf{S} = 0\nonumber\\
		\Rightarrow\quad \vec{j}_M = \nabla\times\textbf{M}
	\end{gather}
	
	\par{\kaishu 其中:\(\vec{j}_M\) 为描述束缚于磁介质内部的\textbf{磁化电流密度}。对式(1.4.8)两边取散度,我们有:}
	
	\vspace{-2mm}
	
	\begin{equation}
		\nabla\cdot\vec{j}_M = \nabla\cdot(\nabla\times\textbf{M}) \equiv 0
	\end{equation}
	
	\vspace{2mm}
	
	\par{\kaishu 式(1.4.9)说明:磁化电流没有引起电荷的积累,因此在这里我们无需考虑磁化电荷。}
	
	\subsubsection{\kaishu 介质中的麦克斯韦方程组}
	
	\par{\kaishu 我们先考虑一块电磁介质,此时空间电荷包括自由电荷(产生源电场的电荷)和极化电荷(束缚电荷),将总电荷 \(\rho_t\) 写成:}
	
	\vspace{-5mm}
	
	\begin{equation}
		\rho_t = \rho_f+\rho_P = \rho-\nabla\cdot\textbf{P}
	\end{equation}
	
	\vspace{3mm}
	
	\par{\kaishu 其中:自由电荷 \(\rho_f = \rho\);束缚电荷 \(\rho_P = -\nabla\cdot\textbf{P}\)。此时介质中可能出现的电流包括:传导电流(自由电流)\(\vec{j}_f\)、极化电流和磁化电流,设总电流为 \(\vec{j}_t\),我们有:}
	
	\begin{equation}
		\vec{j}_t = \vec{j}_f+\frac{\partial\textbf{P}}{\partial t}+\nabla\times\textbf{M}
	\end{equation}
	
	\vspace{3mm}
	
	\par{\kaishu 现在,我们将总电荷和总电流代入麦克斯韦方程组,得到:}
	
	\vspace{1mm}
	
	\begin{equation}
		\begin{cases}
			\nabla\cdot\textbf{E} &= \frac{1}{\varepsilon_0}\left(\rho_f-\nabla\cdot\textbf{P}\right)\\
			\nabla\times\textbf{E} &= -\frac{\partial}{\partial t}\textbf{B}\\
			\nabla\cdot\textbf{B} &= 0 \\
			\nabla\times\textbf{B} &= \mu_0\varepsilon_0\frac{\partial}{\partial t}\textbf{E}+\mu_0\left(\vec{j}_f+\frac{\partial\textbf{P}}{\partial t}+\nabla\times\textbf{M}\right)
		\end{cases}
	\end{equation}
	
	\vspace{5mm}
	
	\par{\kaishu 式(1.4.12)中的 \(\textbf{E}\),\(\textbf{B}\) 代表局域的总电磁场(包含了束缚电荷和束缚电流的修正项)。现在我们无法直接求解修正后的方程组,因为增加了两个未知物理量:极化强度 \(\textbf{P}\)、磁化强度 \(\textbf{M}\)。如果我们利用式(1.4.2),将 \(\textbf{P}\)、\(\textbf{M}\) 转换成 \(\textbf{E}\)、\(\textbf{B}\),并代入介质中的麦克斯韦方程组,则问题可以求解。}
	
	\par{\kaishu \textcolor{blue}{注意:如果我们想求解介质中的电、磁问题,必须先知道介质对外场的响应关系(目的是用 \(\textbf{E}\)、\(\textbf{B}\) 表示 \(\textbf{P}\)、\(\textbf{M}\),从而求解)。不同的介质对外场的响应关系不同,我们不能简单地认为代入线性关系(式(1.4.2)所示)是唯一正确的解法。}}
	
	\par{\kaishu 方程组(1.4.12)中,\(\nabla\cdot\textbf{E}\)、\(\nabla\times\textbf{B}\) 较为复杂,由于我们已经知道:极化强度和磁化强度依赖于总电荷、总电流,它们最初仍是电荷源、电流源产生的。基于此,我们将场强度都表示成自由电荷 \(\rho_f\) 和自由电流 \(\vec{j}_f\) 的函数。引入辅助矢量:}
	
	\vspace{-3mm}
	
	\begin{equation}
		\textbf{D} = \varepsilon_0\textbf{E}+\textbf{P} \qquad \textbf{H} = \frac{\textbf{B}}{\mu_0}-\textbf{M}
	\end{equation}
	
	\vspace{3mm}
	
	\par{\kaishu \(\textbf{D}\) 称为电位移矢量,\(\textbf{H}\) 称为磁场强度,代入介质中麦克斯韦方程组可得:}
	
	\begin{equation}
		\begin{cases}
			\nabla\cdot\textbf{D} &= \rho_f \\
			\nabla\times\textbf{E} &= -\frac{\partial\textbf{B}}{\partial t} \\
			\nabla\cdot\textbf{B} &= 0 \\
			\nabla\times\textbf{H} &= \vec{j}_f+\frac{\partial\textbf{D}}{\partial t}
		\end{cases}
	\end{equation}
	
	\vspace{3mm}
	
	\par{\kaishu 导入电位移矢量和磁场强度,使方程组的右边只出现自由电荷和自由电流(源电荷/源电流),我们在形式上就“消除”了束缚电荷/电流。我们的代换是为了简化问题。}
	
	\subsubsection{\kaishu 本构关系}
	
	\par{\kaishu 现在,我们尝试让局域总场的改变与介质的极化/磁化程度自洽;那就需要确定极化/磁化强度与总电磁场的自洽。从数学上看,我们必须确定麦克斯韦方程组(1.4.14)中 \(\textbf{D}\)、\(\textbf{H}\) 与 \(\textbf{E}\)、\(\textbf{B}\) 的关系,才能进一步研究方程。描述 \(\textbf{D}\)、\(\textbf{H}\) 与 \(\textbf{E}\)、\(\textbf{B}\) 的关系式称为\textbf{本构关系},与介质本身的性质有关。}
	
	\par{\kaishu 最简单的本构关系是线性介质,我们在式(1.4.2)给出了极化强度/磁化强度与电/磁场的关系,只需代入 \(\textbf{D}\)、\(\textbf{H}\) 的表达式,有:}
	
	\vspace{-3mm}
	
	\begin{subequations}
		\begin{equation}
			\textbf{D} = (1+\chi_{e})\varepsilon_0\textbf{E} = \varepsilon\textbf{E}
		\end{equation}
		\begin{equation}
			\textbf{H} = \frac{1}{(1+\chi_{m})\mu_0}\textbf{B} = \frac{1}{\mu}\textbf{B}
		\end{equation}
	\end{subequations}
	
	\vspace{3mm}
	
	\par{\kaishu 其中,系数满足:\(\varepsilon = \varepsilon_r\varepsilon_0 = (1+\chi_{e})\varepsilon_0\);\(\mu = \mu_r\mu_0 = (1+\chi_{m})\mu_0\)。\(\varepsilon\)、\(\mu\) 称为介电常数和磁导率;\(\varepsilon_r\)、\(\mu_r\) 称为相对介电常数和相对磁导率(无量纲),\(\varepsilon_r = \mu_r = 1\) 时退化为真空情况。}
	
	\par{\kaishu 观察式(1.4.15b),我们发现可以将磁化强度写成磁场强度的形式 \(\textbf{M} = \frac{1}{\mu_0}\frac{\chi_{m}}{1+\chi_{m}}\textbf{B} = \chi_{m}\textbf{H}\),与极化强度 \(\textbf{P} = \varepsilon_0\chi_{e}\textbf{E}\) 形式类似。历史上人们将磁场强度 \(\textbf{H}\) 作为基本量,与电场强度 \(\textbf{E}\) 的地位相当。\textcolor{blue}{所以,人们最开始定义磁化强度的时候针对的是 \(\textbf{H}\)。我们先介绍的是更便于使用的磁感应强度 \(\textbf{B}\),因此 \(\textbf{M}\) 和 \(\textbf{B}\) 的关系更加复杂。}}
	
	\newpage
	
	\par{\kaishu 我们将本构关系代入无源情况的麦克斯韦方程组,得到:}
	
	\begin{equation}
		\begin{cases}
			\nabla\cdot\textbf{E} &= 0 \\
			\nabla\times\textbf{E} &= -\mu\frac{\partial\textbf{H}}{\partial t}\\
			\nabla\cdot\textbf{H} &= 0 \\
			\nabla\times\textbf{H} &= \varepsilon\frac{\partial\textbf{E}}{\partial t}
		\end{cases}
	\end{equation}
	
	\vspace{3mm}
	
	\par{\kaishu 其中 \(\textbf{E}\)、\(\textbf{H}\) 表达式基本一致(你也可以说它们对称)。另外,导体本身就是一种特殊电磁介质,本构关系为欧姆定律(此处给出微分形式)\(\vec{j} = \sigma\textbf{E}\),我们会在导体章节给出相关推导。}
	
	\par{\kaishu 上述讨论我们将条件限制在线性介质,一般情况下,本构关系非常多样(也更复杂),下面我们做一个简单前瞻\textcolor{blue}{(内容在后续章节均有所介绍和推导,初学者不必担心)}:
		
		\vspace{-3mm}
		
		\begin{enumerate}
			
			\item 在铁磁/反铁磁物质,或者强电磁场情况下,\(\textbf{P}\) 与 \(\textbf{E}\),\(\textbf{M}\) 与 \(\textbf{H}\) 之间不再呈现线性关系。
			
			\vspace{-3mm}
			
			\item  对各向异性的介质,介电常数 \(\varepsilon\) 和磁导率 \(\mu\) 都由张量描述,场强和感应场强之间的关系写成:\(D_i = \varepsilon_{ij}E^{j}\);\(B_{k} = \mu_{km}H^{m}\)(本书凡涉及张量运算均使用爱因斯坦求和规则)。
			
			\vspace{-3mm}
			
			\item 高频情况下,电磁场变化快,我们不能简单地认为:极化电荷和磁化电流的变化同步于场的变化。先给出一般情况下的响应:\(\textbf{D}(t) = \int\varepsilon(t-t')\textbf{E}(t')\mathrm{d}t'\)。当外场随时间以频率 \(\omega\) 简谐变化,我们通过傅里叶分析得到:单频率有 \(\textbf{D}(\omega) = \varepsilon(\omega)\textbf{E}(\omega)\)(对磁化情况有:\(\textbf{B} = \mu(\omega)\textbf{H}(\omega)\))。所以,极化率和磁化率都是频率的函数,\(\varepsilon = \varepsilon(\omega)\),\(\mu = \mu(\omega)\)。
			
			\vspace{-3mm}
			
			\item 有些介质的响应是非局域的:\(\textbf{D}(\textbf{r}) = \int_\tau\varepsilon(\textbf{r}-\textbf{r}')\textbf{E}(\textbf{r}')\mathrm{d}\tau'\),在 \(\textbf{r}'\) 处的扰动会在 \(\textbf{r}\) 处产生响应,这种效应被称为空间色散,类似3中情况。对特定的波矢 \(\textbf{k}\) 在空间中变化的场,我们也可以通过傅里叶分析得到:\(\textbf{D}(\textbf{k}) = \varepsilon(\textbf{k})\textbf{E}(\textbf{k})\)。
			
	\end{enumerate}}
	
	\subsection{\kaishu 麦克斯韦方程组的边界条件}
	
	\par{\kaishu 麦克斯韦方程组在不同介质的边界处“自带”边界条件,无需特意设定。在界面上,微分形式的麦克斯韦方程失去意义,但是我们可以使用积分形式。我们从积分形式出发,推导麦克斯韦方程组的边界条件。}
	
	\par{\kaishu 微分方程 \(\nabla\cdot\textbf{D} = \rho_f\) 对应的积分形式为:}
	
	\vspace{-3mm}
	
	\begin{equation}
		\oint_S\textbf{D}\cdot\mathrm{d}\textbf{S} = \int_{\tau}\nabla\cdot\textbf{D}\,\mathrm{d}\tau = \int_{\tau}\rho_f \,\mathrm{d}\tau
	\end{equation}
	
	\vspace{3mm}
	
	\par{\kaishu 如图1.5.1所示,定义界面的单位方向矢量 \(\textbf{n}\),由介质2指向介质1,跨介质分界面作一个圆柱体,底面平行于界面,分别记为 \(\Delta\textbf{S}_1 = \Delta S \,\textbf{n}\),\(\Delta \textbf{S}_2 = -\Delta S \,\textbf{n}\),圆柱高度为 \(h\),界面平分圆柱。}
	
	\par{\kaishu 现在我们要考虑的是介质分界处的情况,假设介质1、介质2对应的电位移矢量为 \(\textbf{D}_1\)、\(\textbf{D}_2\),圆柱侧面记为 \(\textbf{D}_{\text{side}}\) 代入式(1.5.1)计算得到:}
	
	\vspace{-4mm}
	
	\begin{align}
		\oint_{S}\textbf{D}\cdot\mathrm{d}\textbf{S} &= \textbf{D}_1\cdot \Delta\textbf{S}_1 + \textbf{D}_2\cdot \Delta\textbf{S}_2 +  \textbf{D}_{\text{side}}\,2\sqrt{\pi\Delta S}h \nonumber\\
		&= \frac{h}{2}\rho_1\Delta S_1 + \frac{h}{2}\rho_{2}\Delta S_2 = \int_{\tau}\rho_f\,\mathrm{d}\tau
	\end{align}
	
	\vspace{3mm}
	
	\par{\kaishu 当 \(h\to0\) 时,\(\Delta\textbf{S}_1\)、\(\Delta\textbf{S}_2\) 近似重合,方向相反,我们有:}
	
	\begin{equation}
		\lim_{h\to0}\left[\textbf{D}_1\cdot \Delta\textbf{S}_1 + \textbf{D}_2\cdot \Delta\textbf{S}_2 +  \textbf{D}_{\text{side}}\,2\sqrt{\pi\Delta S}h\right]-\left(\frac{h}{2}\rho_1\Delta S_1 + \frac{h}{2}\rho_{2}\Delta S_2\right) = 0
	\end{equation}
	
	\vspace{3mm}
	
	\begin{center}
		\includegraphics[width=0.7\linewidth]{C:/Users/16965/OneDrive/Pictures/Electrondynamics/Electrodynamics fig 1-5-1.pdf}
		\captionof{figure}{\text{\kaishu 电位移矢量边界条件示意图,介质1、介质2圆柱体内对应的电荷体密度为 \(\rho_{1}\),\(\rho_{2}\)}}\label{fig:1-5-1}
	\end{center}
	
	\par{\kaishu 由于 \(\textbf{D}_{\text{side}}\) 有限,故 \( \lim_{h\to0} \textbf{D}_{\text{side}}\,2\sqrt{\pi\Delta S}h = 0\)。现在我们研究的介质交界面厚度近似为0,定义电荷面密度 \(\sigma = \rho l\),其中:\(l\) 为界面厚度,\(\rho\) 为界面处电荷体密度。在我们这个模型中,\(l = \frac{h}{2}\),\(\rho = \rho_{1}-\rho_{2}\) 我们用电荷面密度将式(1.5.3)改写成:}
	
	\vspace{-5mm}
	
	\begin{gather}
		\textbf{D}_1\cdot\Delta\textbf{S}_1+\textbf{D}_2\cdot\Delta\textbf{S}_2 = \frac{h}{2}(\rho_{1}\Delta S_1+\rho_{2}\Delta S_2)\nonumber\\
		\Rightarrow\quad \textbf{n}\cdot(\textbf{D}_1-\textbf{D}_2)\Delta S = \frac{h}{2}(\rho_{1}-\rho_{2})\Delta S\nonumber\\
		\Rightarrow\quad \textbf{n}\cdot(\textbf{D}_1-\textbf{D}_2) = \sigma
	\end{gather}
	
	\vspace{1mm}
	
	\par{\kaishu \(\sigma\) 是介质交界面上的电荷面密度。这就是电位移矢量在不同介质的边界条件。如果我们讨论的是无源情况,则式(1.5.4)退化为:\(\textbf{n}\cdot(\textbf{D}_1-\textbf{D}_2) = 0\)。}
	
	\par{\kaishu 类似上面的讨论,方程 \(\nabla\cdot\textbf{B} = 0\) 对应的积分形式为:}
	
	\vspace{-2mm}
	
	\begin{equation}
		\oint_{S}\textbf{B}\cdot\mathrm{d}\textbf{S} = \int_{\tau}\nabla\cdot\textbf{B} = 0
	\end{equation}
	
	\vspace{3mm}
	
	\par{\kaishu 根据式(1.5.5)和式(1.5.1)的对比,我们可以用相同的研究方法给出磁感应强度的边界条件:}
	
	\vspace{-3mm}
	
	\begin{equation}
		\textbf{n}\cdot\left(\textbf{B}_1-\textbf{B}_2\right) = 0
	\end{equation}
	
	\vspace{3mm}
	
	\par{\kaishu \(\textbf{n}\) 与图1.5.1中表示的 \(\textbf{n}\) 意义相同,式(1.5.6)说明:不同介质的交界面处,\(\textbf{B}\) 场的法向分量连续。}
	
	\par{\kaishu 我们再关注方程 \(\nabla\times\textbf{H} = \vec{j}_{f}+\frac{\partial\textbf{D}}{\partial t}\),其对应的积分形式为:}
	
	\begin{equation}
		\oint_{L}\textbf{H}\cdot\mathrm{d}\vec{l} = \int_{S}\nabla\times\textbf{H}\cdot\mathrm{d}\textbf{S} =  \int_S\vec{j}_f\cdot\mathrm{d}\textbf{S}+\int_{S}\frac{\partial\textbf{D}}{\partial t}\cdot\mathrm{d}\textbf{S}
	\end{equation}
	
	\vspace{3mm}
	
	\par{\kaishu 假设 \(\textbf{e}_{1}\)、\(\textbf{e}_{2}\) 为介质交界面上相互垂直的两个方向矢量,并与方向矢量 \(\textbf{n}\) 呈右手螺旋,满足:\(\textbf{n}\times\textbf{e}_1 = \textbf{e}_2\);\(\textbf{n}\times\textbf{e}_2 = -\textbf{e}_1\)。在介质界面上,考虑一个长为 \(l\) 宽为 \(h\) 的矩形,如图1.5.2所示。}
	
	\begin{center}
		\includegraphics[width=0.7\linewidth]{C:/Users/16965/OneDrive/Pictures/Electrondynamics/Electrodynamics fig 1-5-2.pdf}
		\captionof{figure}{\text{\kaishu 磁场强度矢量边界条件示意图,介质处于 \(\textbf{e}_1\textbf{e}_2\) 平面内,\(\textbf{H}_1\)、\(\textbf{H}_2\) 与矩形处于同一平面内}}\label{fig:1-5-2}
	\end{center}
	
	\par{\kaishu 我们令矩阵上表面 \(\Delta l\) 方向的单位矢量为 \(\textbf{e}_l\),\(\nabla\times\textbf{H}\) 确保了电流密度 \(\vec{j}_f\) 和矢量 \(\frac{\partial\textbf{D}}{\partial t}\) 的方向垂直于矩形所在平面。我们考虑极限 \(h\to0\),用式(1.5.7)计算得到:}
	
	\vspace{-4mm}
	
	\begin{gather}
		\lim_{h\to0}\oint_{L}\textbf{H}\cdot\mathrm{d}\vec{l} = \textbf{H}_{1}\cdot\Delta l \,\textbf{e}_{l}-\textbf{H}_2\cdot\Delta l \,\textbf{e}_{l}+0\nonumber\\
		=\lim_{h\to0}\left(\vec{j}_f\cdot h\Delta l+\frac{\partial\textbf{D}}{\partial t}\cdot h\Delta l\right) = \lim_{h\to0}\int_{S}\left(\vec{j}_f+\frac{\partial\textbf{D}}{\partial t}\right)\cdot\mathrm{d}\textbf{S}
	\end{gather}
	
	\vspace{3mm}
	
	\par{\kaishu 由于 \(\textbf{H}\) 是有限函数,其得到的\(\frac{\partial\textbf{D}}{\partial t}\) 也是有限的,因此 \(\lim_{h\to0}\frac{\partial\textbf{D}}{\partial t}\cdot h\Delta l = 0\)。介质交界面厚度近似为0,我们定义电流面密度 \(\vec{\alpha} = \vec{j}l\),\(\vec{j}\) 为介质交界面处电流体密度,\(l\) 为介质交界面厚度,电流面密度方向与体密度一致。在此模型中,\(l = h\);\(\vec{j} = \vec{j}_f\),我们有 \(\vec{\alpha}_f = \vec{j}_f h\)。}
	
	\par{\kaishu 为了简化运算,我们将 \(\textbf{e}_l\) 分别替换为 \(\textbf{e}_1\)、\(\textbf{e}_2\),根据两个单位矢量满足的方向关系,有:}
	
	\begin{equation}
		\begin{cases}
			\textbf{e}_1\cdot\left(\textbf{H}_1-\textbf{H}_2\right) = \vec{j}_f\cdot h\textbf{e}_2 =\vec{\alpha}_f\cdot\textbf{e}_2 \\
			\textbf{e}_2\cdot\left(\textbf{H}_1-\textbf{H}_2\right) = -\vec{j}_f\cdot h\textbf{e}_1 = -\vec{\alpha}_f\cdot\textbf{e}_1
		\end{cases}
	\end{equation}
	
	\vspace{5mm}
	
	\par{\kaishu 式(1.5.9)实际表示了磁场强度 \(\textbf{H}_1-\textbf{H}_2\) 和电流面密度 \(\vec{\alpha}_f\) 在 \(\textbf{e}_1\)、\(\textbf{e}_2\) 方向的分量。(\textcolor{blue}{注意:这里我们考虑的是任意方向的 \(\textbf{H}_1-\textbf{H}_2\) 和 \(\vec{\alpha}_f\)})其中并未出现 \(\textbf{H}_1-\textbf{H}_2\) 在 \(\textbf{n}\) 方向的分量,我们注意到:对式(1.5.9)的结果做矢量和,结果如下:}
	
	\vspace{-2mm}
	
	\begin{equation}
		\left(\textbf{H}_1-\textbf{H}_2\right)\cdot\left(\textbf{e}_1+\textbf{e}_2\right) = \vec{\alpha}_f\cdot\left(\textbf{e}_2-\textbf{e}_1\right)\nonumber
	\end{equation}
	
	\vspace{2mm}
	
	\par{\kaishu 我们可以通过向量的外积,将上式改写成更紧凑的形式:}
	
	\vspace{-2mm}
	
	\begin{equation}
		\textbf{n}\times\left(\textbf{H}_1-\textbf{H}_2\right) = \vec{\alpha}_f
	\end{equation}
	
	\vspace{2mm}
	
	\par{\kaishu 在式(1.5.10)我们也用到了 \(\textbf{n}\times\textbf{e}_1 = \textbf{e}_2\)、\(\textbf{n}\times\textbf{e}_2 = -\textbf{e}_1\),这就是磁场强度在不同介质交界面上的边界条件。如果我们考虑的是无电流源的情况,则式(1.5.10)化为:\(\textbf{n}\times\left(\textbf{H}_1-\textbf{H}_2\right) = 0\)。}
	
	\par{\kaishu 类似上面的讨论,方程 \(\nabla\times\textbf{E} = -\frac{\partial\textbf{B}}{\partial t}\) 对应的积分形式为:}
	
	\begin{equation}
		\oint_{L}\textbf{E}\cdot\mathrm{d}\vec{l} = \int_{S}\nabla\times\textbf{E}\cdot\mathrm{d}\textbf{S} = -\int_{S}\frac{\partial\textbf{B}}{\partial t}\cdot\mathrm{d}\textbf{S}
	\end{equation}
	
	\vspace{3mm}
	
	\par{\kaishu 在式(1.5.11)中,\(\frac{\partial\textbf{B}}{\partial t}\) 是有限量,我们使用类似的方法,得到 \(\lim_{h\to0}\int_{S}\frac{\partial\textbf{B}}{\partial t}\cdot\mathrm{d}\textbf{S} = 0\),有边界条件:}
	
	\vspace{-3mm}
	
	\begin{equation}
		\textbf{n}\times\left(\textbf{E}_1-\textbf{E}_2\right) = 0
	\end{equation}
	
	\vspace{2mm}
	
	\par{\kaishu 其中 \(\textbf{n}\) 与图1.5.2中 \(\textbf{n}\) 的意义相同,式(121)说明:不同介质的交界面处,\(\textbf{E}\) 场的切向分量连续。}
	
	\par{\kaishu 如果我们考虑无源麦克斯韦方程组,那么电磁场在介质交界面的边界条件为:\(\textbf{E}\)、\(\textbf{H}\) 场切向连续;\(\textbf{D}\)、\(\textbf{B}\) 场法向连续。只有存在自由面电荷/面电流分布时,\(\textbf{H}\)、\(\textbf{D}\) 场才能出现不连续的情况。}
	
	\par{\kaishu \textcolor{blue}{我们所谓的面电荷/面电流分布实际上是一种近似,是真实体分布的简化。此时,电荷/电流分布在非常薄的一层介质内。在这里的模型中,我们并不关心薄层内的场分布(用面电荷/电流代替),则跨越此薄层的场一定不连续。}}
	
	\newpage
	
	\subsection*{\kaishu 附录1.A 库伦定律的平方反比律}
	
	\addcontentsline{toc}{subsection}{\text{\kaishu 附录 1.A\quad 库伦定律的平方反比律}}
	
	\par{\kaishu 经过第一章的介绍,我们发现,库伦定律的平方反比律在电磁理论的构建中非常重要。库伦定律是实验发现的定律,也许你会提问:如果库伦测定有误差,那为什么真实的库伦定律不能写成 \(\textbf{F} = \frac{1}{4\pi\varepsilon_0}\frac{q_1q_2}{r^{2+\delta}}\) 的形式?我们又该如何通过测量手段确定或否定平方反比律?}
	
	\vspace{3mm}
	
	\begin{center}
		\includegraphics[width=0.7\linewidth]{C:/Users/16965/OneDrive/Pictures/Electrondynamics/Electrodynamics fig 1A-1.pdf}
		\text{\kaishu 图 1A.1 半径为 \(\textbf{R}\) 的导体球壳示意图,球壳电荷量为 \(Q\)}
		\label{fig:A1-1}
	\end{center}
	
	\par{\kaishu 为解答这些问题,我们考虑一个如图1A.1所示的模型:半径为 \(\textbf{R}\) 的带电导体球壳,电荷量为 \(Q\),我们假设库伦定律满足 \(n\) 次方反比律,计算距离球壳中心 \(O\) 为 \(\textbf{r}\) 处(点P)的电场强度。}
	
	\par{\kaishu 我们先取球壳上一条圆环微元,圆环上每一点到P点间距为:\(\sqrt{r^2+R^2-2rR\cos\alpha}\)。由于球壳是均匀的,设表面电荷密度为 \(\sigma\)(常数),P点受到圆环微元上的一点(也近似为微元)的作用,对应的电场强度微元为(\(\textbf{E}_r\) 表示 \(\textbf{r}\) 方向的电场,\(\textbf{r}\) 方向单位矢量记为 \(\textbf{e}_r\)):}
	
	\begin{equation}
		\mathrm{d}\textbf{E}_r = \frac{\textbf{e}_r}{4\pi\varepsilon_0}\frac{\sigma R^2\sin\theta\mathrm{d}\theta\,\mathrm{d}\phi}{\left(r^{2}+R^{2}-2rR\cos\alpha\right)^{\frac{n}{2}}}\cos\alpha\nonumber
	\end{equation}
	
	\vspace{3mm}
	
	\par{\kaishu 我们将 \(\cos\alpha\) 改写成关于 \(r\) 和 \(\theta\) 的函数:}
	
	\begin{equation}
		\cos\alpha = \frac{r-R\cos\theta}{\sqrt{r^2+R^2-2rR\cos\theta}}\nonumber
	\end{equation}
	
	\vspace{3mm}
	
	\par{\kaishu 然后我们先对圆环微元上的微元积分,再对圆环微元积分成球壳:}
	
	\begin{equation}
		\textbf{E}_r = \int\mathrm{d}\textbf{E}_r = \frac{\textbf{e}_r}{4\pi\varepsilon_0}\int_{0}^{\pi}\mathrm{d}\theta\int_{0}^{2\pi}\frac{\sigma R^2\sin\theta\left(r-R\cos\theta\right)}{\left(r^2+R^2-2rR\cos\theta\right)^{\frac{n+1}{2}}}\mathrm{d}\phi\nonumber
	\end{equation}
	
	\vspace{3mm}
	
	\par{\kaishu 我们作积分变量代换:\(u = r^2+R^2-2rR\cos\theta\),\(\mathrm{d}u = 2rR\sin\theta\mathrm{d}\theta\),代入积分得到:}
	
	\begin{equation}
		\textbf{E}_r = \frac{\textbf{e}_r}{4\pi\varepsilon_0}\int_{(r-R)^2}^{(r+R)^2}\frac{\pi\sigma R\left(u+r^{2}-R^{2}\right)}{2r^2u^{\frac{n+1}{2}}}\mathrm{d}u\nonumber
	\end{equation}
	
	\vspace{7mm}
	
	\par{\kaishu 计算上述积分,我们得到(分别讨论 \(r>R\) 和 \(r<R\) 的情况):}
	
	\vspace{2mm}
	
	\begin{equation}
		\nonumber
		\begin{cases}
			\textbf{E}_r = &\frac{2\pi\sigma R\textbf{e}_r}{4\pi\varepsilon_0r^2(3-n)(1-n)}\left\{(r+R)^{2-n}[r(2-n)-R]-\right.\\
			&\left.(r-R)^{2-n}[r(2-n)+R]\right\}\qquad(r>R)\\
			\\
			\textbf{E}_r = &\frac{2\pi\sigma R\textbf{e}_r}{4\pi\varepsilon_0r^2(3-n)(1-n)}\left\{(r+R)^{2-n}[r(2-n)-R]+\right.\\
			&\left.(R-r)^{2-n}[r(2-n)+R]\right\}\qquad(r<R)
		\end{cases}
	\end{equation}
	
	\vspace{5mm}
	
	\par{\kaishu 积分在 \(n=1\)、\(n=3\) 出现奇异性,原因是:\(n=1\)、\(n=3\) 在积分后会出现形如对数函数 \(\ln x\) 的项,所以出现奇异性。下面我们正式讨论这个结果:
		
		\vspace{-3mm}
		
		\begin{enumerate}
			
			\item 如果 \(n=2\),则 \(r<R\) 时,\(\textbf{E}_r = 0\);\(r>R\) 时,\(\textbf{E}_r = \frac{Q}{4\pi\varepsilon_0r^2}\),其中:\(\sigma = \frac{Q}{4\pi R^2}\)。我们得到的结果与高斯定理计算的一致。
			
			\vspace{-3mm}
			
			\item 如果 \(1<n<2\),则 \(r<R\) 时,可以发现:\(\textbf{E}_r>0\)。设想在导体球壳内部还有一个同心导体球,我们将两个导体用细导线连接,因为两球壳之间电场为正,电流将从内部导体球传播到球壳上,球壳会带有负电荷分布。
			
			\vspace{-3mm}
			
			\item 如果 \(2<n<3\),则 \(r<R\) 时,可以发现:\(\textbf{E}_r<0\)。同样使用2. 的设想,两球壳之间电场为负,电流将从球壳传播到内部导体球上,球壳会带有正电荷分布。
			
	\end{enumerate}}
	
	\vspace{-3mm}
	
	\par{\kaishu 我们发现,精确测量内导体球上存在的电荷量值,就可以判断平方反比定律的精确程度。}
	
	\subsection*{\kaishu 附录1.B 麦克斯韦方程组的完备性}
	
	\addcontentsline{toc}{subsection}{\text{\kaishu 附录 1.B\quad 麦克斯韦方程组的完备性}}
	
	\par{\kaishu \textbf{什么是完备性?}如果我们给出确定的初始条件和边界条件,体系的电磁规律完全由一组可解的方程唯一确定,则我们说这组方程是完备的。}
	
	\par{\kaishu 我们只能证明:在经典电磁学体系中,麦克斯韦方程组是完备/不完备的(因为对于能量更高的情况,我们有其他更有效的理论)。在这里的证明只讨论真空中电磁场方程的完备性。}
	
	\par{\kaishu 采用反证法:假设在给定的初始条件和边界条件下,麦克斯韦方程组存在两组不等价的解,分别记为 \(\textbf{E}_1\)、\(\textbf{B}_1\) 和 \(\textbf{E}_2\)、\(\textbf{B}_2\)。两组解满足同一个体系的方程组:}
	
	\vspace{2mm}
	
	\begin{equation}
		\nonumber
		\begin{cases}
			\nabla\cdot\textbf{E}_1 = \frac{\rho}{\varepsilon_0}\\
			\nabla\times\textbf{E}_1 = -\frac{\partial\textbf{B}_1}{\partial t}\\
			\nabla\cdot\textbf{B}_1 = 0\\
			\nabla\times\textbf{B}_1 = \mu_0\vec{j}+\mu_0\varepsilon_0\frac{\partial\textbf{E}_1}{\partial t}
		\end{cases}
		\qquad
		\begin{cases}
			\nabla\cdot\textbf{E}_2 = \frac{\rho}{\varepsilon_0}\\
			\nabla\times\textbf{E}_2 = -\frac{\partial\textbf{B}_2}{\partial t}\\
			\nabla\cdot\textbf{B}_2 = 0\\
			\nabla\times\textbf{B}_2 = \mu_0\vec{j}+\mu_0\varepsilon_0\frac{\partial\textbf{E}_2}{\partial t}
		\end{cases}
	\end{equation}
	
	\vspace{5mm}
	
	\par{\kaishu 因为体系相同,所以两组方程的 \(\rho\)、\(\vec{j}\) 都相同,不仅如此,两组方程的解都满足相同的初始条件和边界条件,\(t=0\) 时,初始条件:}
	
	\begin{equation}
		\nonumber
		\textbf{E}_1(\textbf{r},0) = \textbf{E}_2(\textbf{r},0)\qquad\&\qquad\textbf{B}_1(\textbf{r},0) = \textbf{B}_2(\textbf{r},0)
	\end{equation}
	
	\vspace{2mm}
	
	\par{\kaishu 其在真空界面的边界条件为(假设边界为闭合曲面 \(S\)):}
	
	\vspace{-2mm}
	
	\begin{equation}
		\nonumber
		\left.\textbf{E}_{1}\right|_{S} = \left.\textbf{E}_{2}\right|_{S} \qquad\&\qquad
		\left.\textbf{B}_{1}\right|_{S} = \left.\textbf{B}_{2}\right|_{S}
	\end{equation}
	
	\vspace{2mm}
	
	\par{\kaishu 我们假设 \(\textbf{E} = \textbf{E}_1-\textbf{E}_2\)、\(\textbf{B} = \textbf{B}_1-\textbf{B}_2\),将上述两组方程相减,得到:}
	
	\begin{equation}
		\nonumber
		\begin{cases}
			\nabla\cdot\textbf{E} = 0\\
			\nabla\times\textbf{E} = -\frac{\partial\textbf{B}}{\partial t}\\
			\nabla\cdot\textbf{B} = 0\\
			\nabla\times\textbf{B} = \mu_0\varepsilon_0\frac{\partial\textbf{E}}{\partial t}
		\end{cases}
	\end{equation}
	
	\vspace{3mm}
	
	\par{\kaishu 对应新方程组的初始条件(\(t = 0\))和边界条件可以写成:}
	
	\vspace{-2mm}
	
	\begin{equation}
		\nonumber
		\textbf{E}(\textbf{r},0) = \textbf{B}(\textbf{r},0) = 0 \qquad\&\qquad \left.\textbf{E}\right|_{S} = \left.\textbf{B}\right|_{S} = 0
	\end{equation}
	
	\vspace{3mm}
	
	\par{\kaishu 对于场 \((\textbf{E}_1,\textbf{B}_1)\) 和 \((\textbf{E}_2,\textbf{B}_2)\) 的差 \((\textbf{E},\textbf{B})\),我们可以将其视作一个新的场。\((\textbf{E},\textbf{B})\) 对应的体系无源,无初始扰动,且边界上值恒为0。对这样的体系,考虑如下体积分:}
	
	\begin{equation}
		\nonumber
		\mathcal{I} = \frac{\mathrm{d}}{\mathrm{d}t}\int_{V}\left(\varepsilon_0\textbf{E}\cdot\textbf{E}+\frac{1}{\mu_0}\textbf{B}\cdot\textbf{B}\right)\mathrm{d}V
	\end{equation}
	
	\vspace{3mm}
	
	\par{\kaishu 体系的边界不随时间改变,我们将积分和时间导数交换顺序,作变换:\(\frac{\mathrm{d}}{\mathrm{d}t}\to\frac{\partial}{\partial t}\),得到:}
	
	\begin{equation}
		\nonumber
		\mathcal{I} = 2\int_{V}\left(\varepsilon_0\textbf{E}\cdot\frac{\partial\textbf{E}}{\partial t}+\frac{1}{\mu_0}\textbf{B}\cdot\frac{\partial\textbf{B}}{\partial t}\right)\mathrm{d}V
	\end{equation}
	
	\vspace{3mm}
	
	\par{\kaishu 将得到的关于 \(\textbf{E}\)、\(\textbf{B}\) 的新方程组代入上式,得到:}
	
	\vspace{-4mm}
	
	\begin{align*}
		\mathcal{I} &= 2\int_{V}\left[\frac{1}{\mu_0}\textbf{E}\cdot\left(\nabla\times\textbf{B}\right)-\frac{1}{\mu_0}\textbf{B}\cdot\left(\nabla\times\textbf{E}\right)\right]\mathrm{d}V \\
		&\overset{a}{=} -\frac{2}{\mu_0}\int_{V}\nabla\cdot\left[\textbf{E}\times\textbf{B}\right]\mathrm{d}V\\
		&\overset{b}{=}\frac{2}{\mu_0}\oint_{S}(\textbf{B}\times\textbf{E})\cdot\mathrm{d}\textbf{S}
	\end{align*}
	
	\vspace{3mm}
	
	\par{\kaishu a处使用了矢量分析的公式:\(\nabla\cdot(\textbf{A}\times\textbf{B}) = (\nabla\times\textbf{A})\cdot\textbf{B}-(\nabla\times\textbf{B})\cdot\textbf{A}\),b处使用了高斯定律,\(S\) 是体积 \(V\) 对应的面积边界。根据边界条件,我们知道:\(\mathcal{I} = 0\),因此有:}
	
	\begin{equation}
		\nonumber
		\int_{V}\left(\varepsilon_{0}\textbf{E}\cdot\textbf{E}+\frac{1}{\mu_0}\textbf{B}\cdot\textbf{B}\right)\mathrm{d}V = \text{constant}
	\end{equation}
	
	\vspace{8mm}
	
	\par{\kaishu 再考虑体系的初始条件:\(t = 0\) 时,\(\textbf{E} = 0\)、\(\textbf{B} = 0\),上式的常数为0。我们有:}
	
	\begin{equation}
		\nonumber	\int_{V}\left(\varepsilon_{0}\textbf{E}\cdot\textbf{E}+\frac{1}{\mu_0}\textbf{B}\cdot\textbf{B}\right)\mathrm{d}V = \int_{V}\left(\varepsilon_{0}E^2+\frac{1}{\mu_0}B^2\right)\mathrm{d}V = 0
	\end{equation}
	
	\vspace{3mm}
	
	\par{\kaishu 由于被积函数是连续且恒大于等于0的,我们得到:\(E^2 = B^2 = 0\),因此 \(\textbf{E} = \textbf{B} = 0\),即:}
	
	\vspace{-1mm}
	
	\begin{equation}
		\nonumber
		\textbf{E}_1 = \textbf{E}_2 \qquad\&\qquad \textbf{B}_1 = \textbf{B}_2 
	\end{equation}
	
	\vspace{3mm}
	
	\par{\kaishu 可见我们所设的两组解是等价的,因此麦克斯韦方程组的完备性得证。}
	
	\vspace{2mm}
	
	\par{\kaishu\textcolor{blue}{在反证法的过程中,考虑体积分 \(\mathcal{I} = \frac{\mathrm{d}}{\mathrm{d}t}\int_{V}\left(\varepsilon_0\textbf{E}\cdot\textbf{E}+\frac{1}{\mu_0}\textbf{B}\cdot\textbf{B}\right)\mathrm{d}V\) 这一步需要“惊人的注意力”,实际上这种构造并不是从天而降的,我们会在第二章内容讨论相关的构造。}}
	
	\subsection*{\kaishu 附录1.C 磁单极子简介}
	
	\addcontentsline{toc}{subsection}{\text{\kaishu 附录 1.C\quad 磁单极子}}
	
	\par{\kaishu 磁单极子的物理内容非常丰富,受限于篇幅和知识内容,这里只给出简单的介绍。}
	
	\par{\kaishu 磁单极子是一类假想粒子的名称,我们也将单独的带电粒子(如经典的质子,电子)称为“电单极子”。类比电荷,我们说磁单极子中带有磁荷(magnetic charge)。磁荷的国际单位是 \(\mathrm{A}\cdot\mathrm{m}\)(安培\(\cdot\)米),将磁荷记为 \(q_m\),类比电荷,我们也可以定义磁荷密度 \(\rho_m\),满足:\(q_m = \int_{V}\rho_m\mathrm{d}V\)。}
	
	\par{\kaishu 对于存在磁单极子的真空体系,麦克斯韦方程组变为:}
	
	\begin{equation}
		\nonumber
		\begin{cases}
			\nabla\cdot\textbf{E} = \frac{\rho}{\varepsilon_0}\\
			\nabla\times\textbf{E} = -\mu_0\vec{j}_{m}-\frac{\partial\textbf{B}}{\partial t}\\
			\nabla\cdot\textbf{B} = \mu_0\rho_m\\
			\nabla\times\textbf{B} = \mu_0\vec{j}+\mu_0\varepsilon_{0}\frac{\partial\textbf{E}}{\partial t}
		\end{cases}
		\quad\xrightarrow{\textbf{\kaishu 高斯单位制}}\qquad
		\begin{cases}
			\nabla\cdot\textbf{E} = 4\pi\rho\\
			\nabla\times\textbf{E} = -\frac{4\pi}{c}\vec{j}_m-\frac{1}{c}\frac{\partial\textbf{B}}{\partial t}\\
			\nabla\cdot\textbf{B} = 4\pi\rho_m\\
			\nabla\times\textbf{B} = \frac{4\pi}{c}\vec{j}+\frac{1}{c}\frac{\partial\textbf{E}}{\partial t}
		\end{cases}
	\end{equation}
	
	\vspace{3mm}
	
	\par{\kaishu 其中:\(\vec{j}_m\) 是对应于电流密度的“磁流密度”,同样也满足磁流守恒定律。我们发现:采用高斯单位制后,电场 \(\textbf{E}\) 和磁场 \(\textbf{B}\) 的系数也相同了,方程的结构更相似(对称性更容易发现)。}
	
	\par{\kaishu 引入磁单极子后,电荷和磁荷的总洛伦兹力,满足:}
	
	\vspace{-2mm}
	
	\begin{equation}
		\nonumber
		\textbf{F} = q\left(\textbf{E}+\textbf{v}\times\textbf{B}\right)+q_m\left(\textbf{B}-\textbf{v}\times\textbf{E}\right)
	\end{equation}
	
	\vspace{3mm}
	
	\par{\kaishu 如果采用高斯单位制,将上式的 \(\textbf{v}\) 替换为 \(\frac{\textbf{v}}{c}\)。经典的磁单极子内容讨论将在后续章节展开。}
	
	\newpage
	
	
	\section{\text{\kaishu 电磁场守恒定律和对称性}}
	
	\pagestyle{fancy}
	\lhead{Lecture Notes}
	\chead{Electrodynamics}
	\rhead{\kaishu Chapter 2 电磁场守恒定律和对称性}
	
	\par{\kaishu 在第一章中,我们已经得到了电磁场的一个守恒定律——电流守恒定律。经典电磁理论体系中,这个定律无论在真空条件或者介质中均满足。下面我们关注电磁场其他守恒定律,以及电磁场的部分对称性(\textcolor{blue}{如果你觉得需要先知道如何求解第一章的麦克斯韦方程组,请先阅览第三章、第四章、第七章内容;另外,在涉及狭义相对论以后,我们会给出更多对称性})。}
	
	\subsection{\kaishu 真空中电磁场的能量守恒定律}
	
	\subsubsection{\kaishu 电磁场能量图像}
	
	\par{\kaishu 电磁场无法直接被“看到”,我们只能观测物质中的电荷/电流。电场/磁场对物质的作用力,那我们如何描述电磁场能量的图像呢?\textcolor{blue}{如果你有一定的物理理解并想直接进行计算,跳过此节即可。}}
	
	\par{\kaishu 假设空间中存在电磁场 \(\textbf{E}(\textbf{r},t)\)、\(\textbf{B}(\textbf{r},t)\),由置于远处的源电荷/电流激发。我们先从更贴近动力学的角度考虑:源电荷/电流在形成电磁场时的能量、动量变化。}
	
	\par{\kaishu 最终电场的电荷可以视作散落在无穷远处的一系列点电荷组成的。为了建立最终电场的分布 \(\textbf{E}(\textbf{r},t)\),我们假设这些电荷经过准静态过程(经过无限长时间的移动)移动到最终电场分布的对应位置。在这个准静态过程中,电荷必须通过假想的外力与电场对新搬入电荷的作用力,因此这个假想的外力必须对体系持续做功。当终态电场建立完成后,外力(假想力)做的总功,根据能量守恒定律,它们全部转化为电磁场的能量(对动量也是同理)。}
	
	\par{\kaishu 现在我们在场的范畴考虑:在电磁场中放置一个带电体,带电体会受到电磁场的作用力而产生动量以及能量的变化。考察物体,这些机械动量和机械能量的改变只能来自于电磁场本身的能量和动量。这些动量和能量的改变,为什么不可能是源电荷/源电流的机械动量和机械能转化的呢?}
	
	\par{\kaishu 因为实验发现,即使我们关掉电荷源/电流源,空间的电磁场不会立即消失。在一段时间 \(\tilde{\tau}\) 内,空间电磁场仍然可以通过电磁相互作用对带电体做功,并改变其动量。这证明:带电体能量/动量的改变量不是由源电荷/电流直接提供的,而是由电磁场提供的。我们将利用第二种场论的语言,研究电磁场的局域能量动量密度。}
	
	\vspace{3mm}
	
	\begin{center}
		\includegraphics[width=0.5\linewidth]{C:/Users/16965/OneDrive/Pictures/Electrondynamics/Electrodynamics fig 2-1-1.pdf}
		\captionof{figure}{\text{\kaishu 物体在电场中示意图,其能量和动量受到电磁场作用改变}}\label{fig:2-1-1}
	\end{center}
	
	\vspace{3mm}
	
	\subsubsection{\kaishu 电磁场能量密度}
	
	\par{\kaishu 考察电磁场对带电体所做的功,根据洛伦兹力,我们发现:磁场作用于带电体的力总是与带电体的速度方向垂直,所以磁场对带电体不做功。我们只需要求电场对带电体所作的功即可。}
	
	\par{\kaishu 假设带电体电荷分布为 \(\rho\),则体积元 \(\mathrm{d}\tau\) 中的电荷为 \(\rho\,\mathrm{d}\tau\),其运动的速度为 \(\textbf{v}\),我们考虑 \(\mathrm{d}t\) 时间内电场 \(\textbf{E}\) 对 \(\rho\,\mathrm{d}\tau\) 做功:}
	
	\vspace{-3mm}
	
	\begin{equation}
		\nonumber
		\rho\,\mathrm{d}\tau\textbf{E}\cdot\textbf{v}\mathrm{d}t = \textbf{E}\cdot(\rho\textbf{v})\,\mathrm{d}t\mathrm{d}\tau = \textbf{E}\cdot\vec{j}\,\mathrm{d}t\mathrm{d}\tau
	\end{equation}
	
	\vspace{3mm}
	
	\par{\kaishu 单位时间内,电场对空间选定区域内的电流做功为(单位时间令 \(\mathrm{d}t = 1\)):}
	
	\vspace{-1mm}
	
	\begin{equation}
		\int_{\tau}\textbf{E}\cdot\vec{j}\,\mathrm{d}\tau
	\end{equation}
	
	\vspace{3mm}
	
	\par{\kaishu 电场对带电体做功增加了带电体的机械能,我们有:}
	
	\begin{equation}
		\frac{\mathrm{d}W_{\text{m}}}{\mathrm{d}t} = \int_{\tau}\textbf{E}\cdot\vec{j}\,\mathrm{d}\tau
	\end{equation}
	
	\vspace{3mm}
	
	\par{\kaishu 考虑真空、有源的麦克斯韦方程组,即式(1.3.21)的第四式,将电流密度改写为:}
	
	\begin{equation}
		\nonumber
		\vec{j} = \frac{1}{\mu_0}\nabla\times\textbf{B}-\varepsilon_{0}\frac{\partial\textbf{E}}{\partial t}
	\end{equation}
	
	\vspace{3mm}
	
	\par{\kaishu 将式(2.1.3)代入 \(\vec{j}\cdot\textbf{E}\),有:}
	
	\vspace{-3mm}
	
	\begin{equation}
		\vec{j}\cdot\textbf{E} = \frac{1}{\mu_0}\left(\nabla\times\textbf{B}\right)\cdot\textbf{E}-\varepsilon_{0}\textbf{E}\frac{\partial\textbf{E}}{\partial t}
	\end{equation}
	
	\vspace{3mm}
	
	\par{\kaishu 再考虑式(1.3.21)的第二式:\(\nabla\times\textbf{E}+\frac{\partial\textbf{B}}{\partial t} = 0\),注意到式(2.1.3)中有 \((\nabla\times\textbf{B})\cdot\textbf{E}\) 的项,我们可以改写式(1.3.21)的第二式如下:}
	
	\begin{equation}
		-\frac{1}{\mu_0}\left(\nabla\times\textbf{E}+\frac{\partial\textbf{B}}{\partial t}\right)\cdot\textbf{B} = 0
	\end{equation}
	
	\vspace{3mm}
	
	\par{\kaishu 将式(2.1.4)加到式(2.1.3)两边,我们得到:}
	
	\vspace{-5mm}
	
	\begin{align}
		\vec{j}\cdot\textbf{E} &= \frac{1}{\mu_0}\left[(\nabla\times\textbf{B})\cdot\textbf{E}-(\nabla\times\textbf{E})\cdot\textbf{B}\right]-\varepsilon_{0}\textbf{E}\frac{\partial\textbf{E}}{\partial t}-\frac{1}{\mu_0}\textbf{B}\frac{\partial\textbf{B}}{\partial t}\nonumber\\
		&= -\frac{1}{\mu_0}\nabla\cdot(\textbf{E}\times\textbf{B})-\frac{1}{2}\frac{\partial}{\partial t}\left(\varepsilon_{0}E^2+\frac{1}{\mu_0}B^2\right)
	\end{align}
	
	\vspace{3mm}
	
	\par{\kaishu 式(2.1.5)利用了矢量分析的公式:\(\nabla\cdot(\textbf{A}\times\textbf{B}) = (\nabla\times\textbf{A})\cdot\textbf{B}-(\nabla\times\textbf{B})\cdot\textbf{A}\)。}
	
	\par{\kaishu 在得到了式(2.1.5)以后,我们定义以下两个量:}
	
	\begin{subequations}
		\begin{equation}
			\textbf{S}_{\text{P}}(\textbf{r},t) = \frac{1}{\mu_0}\textbf{E}\times\textbf{B} = \textbf{E}\times\textbf{H}
		\end{equation}
		\begin{equation}
			u(\textbf{r},t) = \frac{1}{2}\left(\varepsilon_{0}E^2+\frac{1}{\mu_0}B^2\right)
		\end{equation}
	\end{subequations}
	
	\vspace{3mm}
	
	\par{\kaishu 利用上面的等式,我们将式(2.1.2)改写成如下形式:}
	
	\vspace{-5mm}
	
	\begin{gather}
		\frac{\mathrm{d}W_{\text{m}}}{\mathrm{d}t} = -\frac{1}{\mu_0}\int_{\tau}\nabla\cdot(\textbf{E}\times\textbf{B})\,\mathrm{d}\tau-\frac{\mathrm{d}}{\mathrm{d}t}\int_{\tau}\left(\frac{1}{2}\varepsilon_{0}E^2+\frac{1}{2\mu_0}B^2\right)\mathrm{d}\tau\nonumber\\
		\Rightarrow\quad \int_{\tau}\textbf{E}\cdot\vec{j}\mathrm{d}\tau = -\underbrace{\oint_{S}\textbf{S}_{\text{P}}\cdot\mathrm{d}\textbf{S}}_{\text{\kaishu 使用高斯定理}}-\frac{\mathrm{d}}{\mathrm{d}t}\int u(\textbf{r},t)\,\mathrm{d}\tau
	\end{gather}
	
	\vspace{3mm}
	
	\par{\kaishu \(S\) 是有限闭合区域 \(\tau\) 的边界,将上式改写成下面的形式(对 \(t\) 的导数结合):}
	
	\begin{equation}
		\frac{\mathrm{d}}{\mathrm{d}t}\left[W_{\text{m}}+\int_{\tau}u(\textbf{r},t)\,\mathrm{d}\tau\right] = -\oint_{S}\textbf{S}_{\text{P}}\cdot\mathrm{d}\textbf{S}
	\end{equation}
	
	\vspace{3mm}
	
	\par{\kaishu 式(2.1.8)的物理意义非常明确:在一个有限的闭合空间 \(\tau\) 内,物理量 \(W_{\text{m}}+\int_{\tau}u(\textbf{r},t)\,\mathrm{d}\tau\) 随时间的变化,等于矢量 \(\textbf{S}_{P}\) 通过表面 \(S\) 流入/流出闭合空间 \(\tau\) 的大小。}
	
	\par{\kaishu 等式左边 \(W_{\text{m}}\) 是物质的机械能量,\(\int_{\tau}u(\textbf{r},t)\,\mathrm{d}\tau\) 描述了其他的能量形式,在这个模型中只可能是电磁场的能量。我们发现:这个等式可以对应我们在1.2.1节中提到的守恒定律,等式左边是“\textbf{能量荷}”随时间的变化,等式右边是“\textbf{能量流}”的变化。}
	
	\par{\kaishu 因此我们知道:类比电荷密度,\(u(\textbf{r},t)\) 表示电磁场的能量密度;类比电流密度,\(\textbf{S}_{\text{P}}\) 表示电磁场的能流密度,也被称为玻印廷 (Poynting) 矢量。如果我们考察全空间,对无穷远处的边界,电磁场强度为0,\(\textbf{S}_{\text{P}} = 0\),此时有:}
	
	\begin{equation}
		\frac{\mathrm{d}W_{\text{m}}}{\mathrm{d}t} = -\frac{\mathrm{d}}{\mathrm{d}t}\int_{\infty}u(\textbf{r},t)\,\mathrm{d}\tau = -\frac{\mathrm{d}W_{\text{em}}}{\mathrm{d}t}
	\end{equation}
	
	\vspace{3mm}
	
	\par{\kaishu 其中 \(W_{\text{em}}\) 表示空间电磁场总能量。式(2.1.9)表明物体机械能和电磁场能量可以相互转化,但是总量维持不变。如果我们考虑无源空间,此时没有其他的能量形式,\(W_{\text{m}} = 0\),式(2.1.8)变为:}
	
	\vspace{-4mm}
	
	\begin{gather}
		\oint_{S}\textbf{S}_{\text{P}}\cdot\mathrm{d}\textbf{S} + \frac{\mathrm{d}}{\mathrm{d}t}\int_{\tau}u(\textbf{r},t)\,\mathrm{d}\tau = 0\nonumber\\
		\Rightarrow\quad\int_{\tau}\nabla\cdot\textbf{S}_{\text{P}}\,\mathrm{d}\tau+\int_{\tau}\frac{\partial}{\partial t}u(\textbf{r},t)\,\mathrm{d}\tau = 0\nonumber\\
		\Rightarrow\quad\int_{\tau}\left(\nabla\cdot\textbf{S}_{\text{P}}+\frac{\partial}{\partial t}u(\textbf{r},t)\right)\mathrm{d}\tau = 0\nonumber\\
		\Rightarrow\quad \nabla\cdot\textbf{S}_{\text{P}}+\frac{\partial}{\partial t}u(\textbf{r},t) = 0
	\end{gather}
	
	\vspace{3mm}
	
	\par{\kaishu 我们得到的式(2.1.10)与电荷守恒定理的形式类似,能看出能量密度和能流密度的物理意义。}
	
	\begin{center}
		\begin{mdframed}[shadow=false]
			
			\par{\kaishu \textbf{补充:}附录1.B中的函数构造}
			
			\par{\kaishu 在附录1.B中,我们证明了麦克斯韦方程组的完备性,在证明过程中构造了积分函数:}
			
		\end{mdframed}
	\end{center}
	
	\begin{center}
		\begin{mdframed}[shadow = false]
			
			\vspace{3mm}
			
			\begin{equation}
				\nonumber
				\mathcal{I} = \frac{\mathrm{d}}{\mathrm{d}t}\int_V\left(\varepsilon_0 E^2+\frac{1}{\mu_0}B^2\right)\mathrm{d}V
			\end{equation}
			
			\vspace{3mm}
			
			\par{\kaishu 不难发现:这个积分内函数的形式与电磁场能量密度 \(u(\textbf{r},t)\) 相似。而且我们在代入 \(\frac{\partial\textbf{E}}{\partial t}\)、\(\frac{\partial\textbf{B}}{\partial t}\) 对应的麦克斯韦方程后,出现了 \(\frac{1}{\mu_0}\textbf{B}\times\textbf{E}\) 的项,对应能流密度 \(-\textbf{S}_{\text{P}}\)。}
			
			\vspace{2mm}
			
			\par{\kaishu 实际上,对积分 \(\mathcal{I}\) 的化简过程已经包含了式(2.1.10)所描述的电磁场能量守恒性。化简后再利用边界条件 \(\textbf{E}|_{S} = \textbf{B}|_{S} = 0\),得到流入/流出 \(V\) 体积区域的能流为0,\(V\) 区域内电磁场的能量没有变化,\(\mathcal{I} = 0\),再根据边界条件得到 \(E^2 = B^2 = 0\)。}
			
			\vspace{2mm}
			
			\par{\kaishu 如果我们构造 \(\varepsilon_{0}\textbf{E}^{2n}+\frac{1}{\mu_0}\textbf{B}^{2n}\) 形式的函数也没有问题(\(n\in\mathbb{N}^*\)),但是电磁场能量为我们提供了最简便的形式:\(n=1\),这也说明麦克斯韦方程的完备性蕴含了能量的性质。}
			
		\end{mdframed}
	\end{center}
	
	\subsubsection{\kaishu 典例:缓慢充电的电容器}
	
	\par{\kaishu 我们考察缓慢充电的电容器的能流。假设此电容器由两块等大的圆形平板构成,圆形平板的半径为 \(r\),间距为 \(h\),内部的均匀电场为 \(\textbf{E}\),不考虑边界效应。}
	
	\vspace{3mm}
	
	\begin{center}
		\includegraphics[width=0.5\linewidth]{C:/Users/16965/OneDrive/Pictures/Electrondynamics/Electrodynamics fig 2-1-2.pdf}
		\captionof{figure}{\text{\kaishu 圆形平行板电容器示意图,电场 \(\textbf{E}\) 指向为 \(z\) 轴正方向}}\label{fig:2-1-2}
	\end{center}
	
	\par{\kaishu 如图2.1.2所示,电容器内部的能量为:}
	
	\begin{equation}
		W = \int_{V}\frac{1}{2}\varepsilon_0E^2\,\mathrm{d}V = \frac{1}{2}\varepsilon_0E^2\pi r^2h
	\end{equation}
	
	\vspace{3mm}
	
	\par{\kaishu 因为我们考虑的是缓慢充电过程,均匀电场 \(\textbf{E}\) 是时间的函数,电容器中的能量随充电时间不断增加,能量的增加速率为:}
	
	\begin{equation}
		\frac{\mathrm{d}W}{\mathrm{d}t} = \varepsilon_{0}\pi r^2h\textbf{E}\cdot\frac{\mathrm{d}\textbf{E}}{\mathrm{d}t}
	\end{equation}
	
	\vspace{3mm}
	
	\par{\kaishu 接下来我们看能量流动情况,\(\textbf{E} = E(t)\,\hat{e}_{z}\),考虑麦克斯韦方程组第四式(此系统无电流源):}
	
	\begin{equation}
		\oint_{L}\textbf{B}\cdot\mathrm{d}\vec{l} = \mu_0\varepsilon_{0}\int_{S}\frac{\partial\textbf{E}}{\partial t}\cdot\mathrm{d}\textbf{S}
	\end{equation}
	
	\vspace{3mm}
	
	\par{\kaishu 其中,\(L\) 为圆心处于圆柱形轴上,半径为 \(\rho\) 的圆环(如图2.1.3所示),\(\rho\) 可以小于、等于或大于柱形电容器底面半径 \(r\)。但是我们不考虑边界效应,超出电容器的部分不再考虑电场。}
	
	\begin{center}
		\includegraphics[width=0.55\linewidth]{C:/Users/16965/OneDrive/Pictures/Electrondynamics/Electrodynamics fig 2-1-3.pdf}
		\captionof{figure}{\text{\kaishu \(L_1\)圆环的半径小于\(r\),\(L_2\)圆环的半径等于\(r\),能流密度\(\textbf{S}_{\text{P}}\)的方向从圆柱外指向中心轴线}}\label{fig:2-1-3}
	\end{center}
	
	\par{\kaishu 根据式(2.1.13),我们先计算磁场的磁感应强度:}
	
	\vspace{-3mm}
	
	\begin{equation}
		\begin{cases}
			B\cdot2\pi\rho = \mu_0\varepsilon_{0}\pi\rho^2\dfrac{\partial\textbf{E}}{\partial t}\quad(\rho<r)\\[10pt]
			B\cdot2\pi\rho = \mu_0\varepsilon_{0}\pi r^2\dfrac{\partial\textbf{E}}{\partial t}\quad (\rho\ge r)
		\end{cases}
		\quad\Rightarrow\qquad
		\begin{cases}
			\textbf{B} = \frac{1}{2}\mu_0\varepsilon_{0}\rho\dfrac{\partial\textbf{E}}{\partial t}\hat{e}_{\phi}\quad(\rho<r)\\[10pt]
			\textbf{B} = \frac{r^2}{2\rho}\mu_0\varepsilon_{0}\dfrac{\partial\textbf{E}}{\partial t}\hat{e}_{\phi}\quad (\rho\ge r)
		\end{cases}
	\end{equation}
	
	\vspace{3mm}
	
	\par{\kaishu 我们进一步得到:体系内部能流密度 \(\textbf{S}_{\text{P}}\) 为:}
	
	\begin{equation}
		\textbf{S}_{\text{P}} = \frac{1}{\mu_0}\textbf{E}\times\textbf{B} = -\frac{\varepsilon_0}{2}\rho\,\textbf{E}\frac{\partial\textbf{E}}{\partial t}\hat{e}_{\phi}\qquad(\rho<r)
	\end{equation}
	
	\vspace{3mm}
	
	\par{\kaishu \(\textbf{S}_{\text{P}}\) 方向指向电容器中心轴线。如果我们想知道能量流入电容器的过程,需要取 \(\rho = r\),在电容器侧面上对能量密度进行积分,得到流入电容器的能量速率:}
	
	\begin{equation}
		-\int_{\text{side}}\textbf{S}_{\text{P}}\cdot\mathrm{d}\textbf{S} = \frac{\varepsilon_0}{2}r\,\textbf{E}\frac{\partial\textbf{E}}{\partial t}\cdot2\pi rh = \pi r^2h\varepsilon_0\textbf{E}\frac{\partial\textbf{E}}{\partial t} = \frac{\partial W_{\text{em}}}{\partial t}
	\end{equation}
	
	\vspace{3mm}
	
	\par{\kaishu 式(2.1.16)说明:能流密度流入电容器的能量速率,恰好等于电容器中能量的增加速率。这说明:能量不是从导线流向电容器的,而是从电容器外面的空间中通过电容器侧面流进电容器的。}
	
	\par{\kaishu \textcolor{blue}{如何理解这一点呢?我们设想电容器两端的很远处各有一些正负电荷,电荷距离电容器很远时,其包围着电容器的电场很微弱;当这些电荷流向电容器时,电容器附近的场逐渐增强。因此,远处的场能量会不断移向电容器,最后,当正负电荷分别到达电容器的两个极板时,这部分电磁场的能量也流进了电容器。}}
	
	\subsection{\kaishu 电磁场的动量守恒定律}
	
	\subsubsection{\kaishu 动量守恒定律}
	
	\par{\kaishu 我们用相似的方法考虑电磁场的动量。带电体的机械动量变化率等于场对带电体的作用力。在洛伦兹力作用下,考虑带电体的机械动量(记为 \(\textbf{G}_{\text{m}}\))变化:}
	
	\begin{equation}
		\frac{\mathrm{d}\textbf{G}_{\text{m}}}{\mathrm{d}t} = \int_{\tau}\left(\rho\,\textbf{E}\,\mathrm{d}\tau+\rho\,\textbf{v}\times\textbf{B}\,\mathrm{d}\tau\right) = \int_{\tau}\left(\rho\,\textbf{E}+\vec{j}\times\textbf{B}\right)\,\mathrm{d}\tau = \int_{\tau}\vec{f}\mathrm{d}\tau
	\end{equation}
	
	\vspace{3mm}
	
	\par{\kaishu 其中:\(\tau\) 为带电体体积;\(\rho\) 表示带电体的电荷密度;\(\vec{f}\) 表示洛伦兹力密度。为简化方程,我们可以利用麦克斯韦方程组将 \(\rho\)、\(\vec{j}\) 消去,力密度改写成:}
	
	\begin{equation}
		\vec{f} = \underbrace{\varepsilon_0(\nabla\cdot\textbf{E})}_{\text{\kaishu 使用}\,\nabla\cdot\textbf{E} = \frac{\rho}{\varepsilon_0}}\textbf{E}+\underbrace{\frac{1}{\mu_0}\left(\nabla\times\textbf{B}-\mu_0\varepsilon_{0}\frac{\partial\textbf{E}}{\partial t}\right)}_{\text{\kaishu 使用}\,\nabla\times\textbf{B} = \mu_0\vec{j}+\mu_0\varepsilon_{0}\frac{\partial\textbf{E}}{\partial t}}\times\textbf{B}
	\end{equation}
	
	\vspace{3mm}
	
	\par{\kaishu 我们仿照电磁场能量的方法,将上式改写成对时间和空间的全微分,先研究时间部分:}
	
	\begin{equation}
		\frac{\partial\textbf{E}}{\partial t}\times\textbf{B} = \frac{\partial}{\partial t}(\textbf{E}\times\textbf{B})-\textbf{E}\times\frac{\partial\textbf{B}}{\partial t} = \frac{\partial}{\partial t}(\textbf{E}\times\textbf{B})+\textbf{E}\times(\nabla\times\textbf{E})
	\end{equation}
	
	\vspace{3mm}
	
	\par{\kaishu 其中的代换使用了麦克斯韦方程组的第二式,我们得到了只与电场有关的项:\(\textbf{E}\times(\nabla\times\textbf{E})\)。将上面的结果代入式(2.2.2),并提取出只与电场有关的项进行计算,有:}
	
	\vspace{-5mm}
	
	\begin{align}
		(\nabla\cdot\textbf{E})\,\textbf{E}-\textbf{E}\times(\nabla\times\textbf{E}) &\overset{a}{=}  (\nabla\cdot\textbf{E})\,\textbf{E}-\frac{1}{2}\nabla(\textbf{E}\cdot\textbf{E})+(\textbf{E}\cdot\nabla)\textbf{E}\nonumber\\
		&\overset{b}{=} \nabla\cdot(\textbf{E}\,\textbf{E})-\nabla\left(\frac{1}{2}E^2\right)\nonumber\\
		&\overset{c}{=} \nabla\cdot(\textbf{E}\,\textbf{E})-\nabla\cdot\left(\frac{1}{2}E^2\overleftrightarrow{I} \right)
	\end{align}
	
	\vspace{1mm}
	
	\par{\kaishu 其中:a处等号使用了矢量叉乘公式:}
	
	\vspace{-1mm}
	
	\[\textbf{A}\times(\nabla\times\textbf{B}) = \nabla(\textbf{A}\cdot\textbf{B})-\textbf{B}\times(\nabla\times\textbf{A})-(\textbf{B}\cdot\nabla)\textbf{A}-(\textbf{A}\cdot\nabla)\textbf{B}\]
	
	\vspace{5mm}
	
	\par{\kaishu 我们只需要令 \(\textbf{A} = \textbf{B} = \textbf{E}\),就可以完成这一步化简。}
	
	\par{\kaishu b处等号利用了并矢的运算性质(\textcolor{blue}{注意:如果为了完成考试,写成并矢化简更快速,但是更好的方法是用分量化简式(2.2.4),我们将在补充内容中见到}):}
	
	\vspace{-6mm}
	
	\begin{gather*}
		\nabla\cdot(\textbf{A}\,\textbf{B}) = (\nabla\cdot\textbf{A})\textbf{B}+(\textbf{A}\cdot\nabla)\textbf{B}\\
		\text{proof:}\quad\partial_{i}(A_{i}B_{j})\hat{e}_{j} = (\partial_{i}A_{i})B_{j}\hat{e}_{j}+A_{i}(\partial_{i}B_{j})\hat{e}_{j}
	\end{gather*}
	
	\vspace{1mm}
	
	\par{\kaishu 仍然是令 \(\textbf{A} = \textbf{B} = \textbf{E}\),我们就能完成这一步化简。}
	
	\vspace{2mm}
	
	\par{\kaishu c处等号使用了恒等式:\(\nabla\varphi = \nabla\cdot\left(\varphi\overleftrightarrow{I}\right)\),\(\overleftrightarrow{I}\) 表示的是单位张量,此节之后,我们将频繁使用张量分量表示张量。恒等式证明如下:}
	
	\vspace{-3mm}
	
	\[\nabla\cdot\left(\varphi\overleftrightarrow{I}\right) = \partial_{i}\left(\varphi\overleftrightarrow{I}\right)_{ij}\hat{e}_{j} = \partial_{i}\left(\varphi\delta_{ij}\right)\hat{e}_{j} = \partial_{i}\varphi\,\hat{e}_{i} = \nabla\varphi\]
	
	\vspace{3mm}
	
	\par{\kaishu 下面我们先考虑磁场相关的项:\((\nabla\times\textbf{B})\times\textbf{B}\),与电场项 \(\textbf{E}\times(\nabla\times\textbf{E})\) 类似。由于电场另一项为 \((\nabla\cdot\textbf{E})\textbf{E}\),考虑到 \(\nabla\cdot\textbf{B} = 0\),我们也可以构造新的项:\((\nabla\cdot\textbf{B})\textbf{B}\)。利用式(2.2.4)的结果,将磁场相关的项化简为:}
	
	\vspace{-3mm}
	
	\begin{equation}
		(\nabla\cdot\textbf{B})\textbf{B}+(\nabla\times\textbf{B})\times\textbf{B} = \nabla\cdot(\textbf{B}\,\textbf{B})-\nabla\cdot\left(\frac{1}{2}B^2\overleftrightarrow{I}\right)
	\end{equation}
	
	\vspace{3mm}
	
	\par{\kaishu 将式(2.2.3)、式(2.2.4)、式(2.2.5)的结果代入式(2.2.2),我们将其写成:}
	
	\vspace{-5mm}
	
	\begin{gather}
		\vec{f} = -\nabla\cdot\overleftrightarrow{T}-\frac{\partial\vec{\textbf{g}}}{\partial t}\\[8pt]
		\text{\kaishu 其中:}\quad\overleftrightarrow{T} = \frac{1}{2}\left(\varepsilon_0E^2+\frac{1}{\mu_0}B^2\right)\overleftrightarrow{I}-\varepsilon_{0}\textbf{E}\,\textbf{E}-\frac{1}{\mu_0}\textbf{B}\,\textbf{B}\nonumber\\[10pt]
		\vec{\textbf{g}} = \varepsilon_{0}(\textbf{E}\times\textbf{B}) = \frac{1}{c^2}\,\textbf{S}_{\text{P}}\qquad(c = \frac{1}{\sqrt{\varepsilon_{0}\mu_0}})\nonumber
	\end{gather}
	
	\vspace{3mm}
	
	\par{\kaishu 将上述结果代入式(2.2.1)中,我们得到:}
	
	\begin{equation}
		\frac{\mathrm{d}\textbf{G}_{\text{m}}}{\mathrm{d}t} = -\int_{\tau}\nabla\cdot\overleftrightarrow{T}\mathrm{d}\tau-\int_{\tau}\frac{\partial\vec{\textbf{g}}}{\partial t}\mathrm{d}\tau = -\oint_{S}\overleftrightarrow{T}\cdot\mathrm{d}\textbf{S}-\frac{\mathrm{d}}{\mathrm{d}t}\int_{\tau}\vec{\textbf{g}}\mathrm{d}\tau
	\end{equation}
	
	\vspace{3mm}
	
	\par{\kaishu 类似对电磁场能量的讨论,式(2.2.8)说明:\(\vec{\textbf{g}}\) 是电磁场的\textbf{动量密度},\(\overleftrightarrow{T}\) 是\textbf{动量流密度},是一个张量。因为动量本身就是一个矢量,我们对动量定义“流”,数学上相当于并矢,故 \(\overleftrightarrow{T}\) 必须是一个二阶张量。\textcolor{blue}{虽然并矢一定是张量,但是不是所有形式的张量都可以表示成并矢形式。}}
	
	\par{\kaishu 利用式(2.2.7),我们可以更方便地讨论电磁场对物质的作用。考虑一个置于电磁场中的物体,取光滑闭合有向曲面 \(\textbf{S}\),其恰好包围此物体,电磁场对此闭合区域 \(\tau\) 内的作用力可以计算如下:}
	
	\begin{equation}
		\textbf{F}_{\text{em}} = \frac{\partial\textbf{G}_{\text{em}}}{\partial t} = -\oint_{S}\overleftrightarrow{T}\cdot\mathrm{d}\textbf{S}-\frac{\mathrm{d}}{\mathrm{d}t}\int_{\tau}\vec{\textbf{g}}\,\mathrm{d}\tau
	\end{equation}
	
	\vspace{5mm}
	
	\par{\kaishu 
		
		\begin{enumerate}
			
			\item 式(2.2.8)右边第一项:\(\oint_{S}\overleftrightarrow{T}\cdot\mathrm{d}\textbf{S}\) 表示电磁场单位时间内穿过边界曲面 \(S\) 内部的总动量。根据“力”的定义,这一项就是外界电磁场通过边界对物体施加的总力。类似弹性力学中的张力,表面外的场作用再表面内\textbf{单位面积上的应力}可以写成:\(-\textbf{e}_{n}\cdot\overleftrightarrow{T}\)(\(\textbf{e}_{n}\) 表示此处界面对应切面的单位法向量)。所以 \(\overleftrightarrow{T}\) 又被称为麦克斯韦应力张量,作用示意图如下。
			
			\begin{center}
				\includegraphics[width=0.5\linewidth]{C:/Users/16965/OneDrive/Pictures/Electrondynamics/Electrodynamics fig 2-2-1.pdf}
				\captionof{figure}{\text{\kaishu \(\overleftrightarrow{T}\)对物体作用示意图,将电磁场视为网格,物体视为其中的球体}}\label{fig:2-2-1}
			\end{center}
			
			\vspace{-3mm}
			
			如果对图2.2.1中的网格施加扰动,则动量通过界面 \(S\)“流入”闭合区域,一部分变成了区域内的动量,另一部分传递给球体,给球体施加了作用力。
			
			\vspace{-3mm}
			
			\item 在式(2.2.8)中考虑的总力作用在区域内的物体和电磁场上,不仅改变物体动量,也会引发区域内电磁场动量的改变,这正是式(2.2.8)右边第二项所对应的物理意义。
			
			\vspace{-3mm}
			
			\item 在稳态或者外场随时间简谐变化下,式(2.2.8)右边第二项对时间平均为0,此时物体在电磁场中的受力完全由麦克斯韦应力张量的曲面积分表示。
			
	\end{enumerate}}
	
	\vspace{-2mm}
	
	\par{\kaishu 我们已经结束了对电磁场动量守恒的讨论,在下一节开始之前,有必要补充讨论并矢的问题:}
	
	\begin{center}
		\begin{mdframed}[shadow = false]
			
			\par{\kaishu\textbf{补充:}并矢和张量表示的问题讨论\textcolor{blue}{(内容较多,出现理解困难时可以看)}}
			
			\par{\kaishu 并矢最早由吉布斯(Gibbs)提出,并矢张量(dyadic tensor)是一个以特别标记法写出的二阶张量,是由成对的矢量并置形成的。并矢张量的每一对矢量的并置称为并矢(dyad)。下面我们先给出并矢张量的定义:}
			
			\par{\kaishu 设 \(V\) 是域 \(\mathbb{F}\) 上的一个线性空间,\(V\) 上的并矢张量可以按照下述规则建立:
				
				(1) 任意矢量 \(\boldsymbol{v}\)、\(\boldsymbol{w}\) 并置排列形成一个并矢积 \(\boldsymbol{v}\boldsymbol{w}\).
				
				(2) 对于任意的 \(\alpha\in\mathbb{F}\) 和任意的 \(\boldsymbol{v},\boldsymbol{w}\in V\),规定 \((\alpha\boldsymbol{v})\boldsymbol{w} = \boldsymbol{v}(\alpha\boldsymbol{w}) = \alpha(\boldsymbol{v}\boldsymbol{w})\),统一记为 \(\alpha\,\boldsymbol{v}\boldsymbol{w}\).
				
				(3) 称有限个并矢积的形式和(不追究具体含义的和式)为一个并矢张量. }
			
		\end{mdframed}
	\end{center}
	
	\begin{center}
		\begin{mdframed}[shadow = false]
			
			\par{\kaishu (4) 对任意正整数 \(k\),如果 \(\boldsymbol{v}_1,\boldsymbol{v}_2,\cdots,\boldsymbol{v}_{k}\in V\) 线性无关,则 \(\left\{\boldsymbol{v}_{i}\boldsymbol{v}_{j}\,|\, i,j = 1,2,\cdots,k\right\}\) 是线性无关矢量组,特别地,\(\boldsymbol{v}\boldsymbol{w} = 0\) 的充要条件是 \(\boldsymbol{v} = 0\) 或 \(\boldsymbol{w} = 0\).
				
				(5) 对于任意的 \(\boldsymbol{u},\boldsymbol{v},\boldsymbol{w}\in V\),其并矢成立分配律:
				
				\vspace{-4mm}
				
				\[\boldsymbol{u}(\boldsymbol{v}+\boldsymbol{w}) = \boldsymbol{u}\boldsymbol{v}+\boldsymbol{u}\boldsymbol{w}\qquad (\boldsymbol{u}+\boldsymbol{v})\boldsymbol{w} = \boldsymbol{u}\boldsymbol{w}+\boldsymbol{v}\boldsymbol{w}\]
				
				\vspace{1mm}
				
				在物理学中,我们更加关心这种张量和矢量之间的乘积(或者说缩并)。记 \(V\) 上所有的并矢张量构成线性空间 \(V\otimes V\),如果 \(V\) 是一个内积空间,\(\boldsymbol{v},\boldsymbol{w}\in V\) 的内积记为 \(\boldsymbol{v}\cdot\boldsymbol{w}\)(当 \(\mathbb{F} = \mathbb{C}\) 时,约定 \(\boldsymbol{v}\cdot\boldsymbol{w}\) 对 \(\boldsymbol{v}\) 是共轭线性的)。定义并矢张量 \(\boldsymbol{T}\) 和矢量 \(\boldsymbol{v}\) 的缩并:\(\boldsymbol{T}\cdot\boldsymbol{v}\)、\(\boldsymbol{v}\cdot\boldsymbol{T}\),两者都是 \(V\) 中的矢量,满足如下运算规律:
				
				(6) 对任意的 \(\alpha\in\mathbb{F}\)、\(\boldsymbol{T}\in V\otimes V\)、\(\boldsymbol{v}\in V\),满足:
				
				\vspace{-4mm}
				
				\[(\alpha\boldsymbol{T})\cdot\boldsymbol{v} = \boldsymbol{T}\cdot(\alpha^*\boldsymbol{v}) = \alpha(\boldsymbol{T}\cdot\boldsymbol{v})\qquad \boldsymbol{v}\cdot(\alpha\boldsymbol{T}) = (\alpha^*\boldsymbol{v})\cdot\boldsymbol{T} = \alpha(\boldsymbol{v}\cdot\boldsymbol{T})\]
				
				\vspace{1mm}
				
				上述公式中,如果 \(\mathbb{F} = \mathbb{C}\),\(\alpha^*\) 表示 \(\alpha\) 的复共轭.
				
				(7) 分配律1:对于任意的 \(\boldsymbol{S},\boldsymbol{T}\in V\otimes V\),\(\boldsymbol{v}\in V\),有:
				
				\vspace{-4mm}
				
				\[(\boldsymbol{S}+\boldsymbol{T})\cdot\boldsymbol{v} = \boldsymbol{S}\cdot\boldsymbol{v}+\boldsymbol{T}\cdot\boldsymbol{v}\qquad\boldsymbol{v}\cdot(\boldsymbol{S}+\boldsymbol{T}) = \boldsymbol{v}\cdot\boldsymbol{S}+\boldsymbol{v}\cdot\boldsymbol{T}\]
				
				\vspace{1mm}
				
				(8) 分配律2:对于任意的 \(\boldsymbol{T}\in V\otimes V\),\(\boldsymbol{v},\boldsymbol{w}\in V\),有:
				
				\vspace{-4mm}
				
				\[\boldsymbol{T}\cdot(\boldsymbol{v}+\boldsymbol{w}) = \boldsymbol{T}\cdot\boldsymbol{v}+\boldsymbol{T}\cdot\boldsymbol{w}\qquad(\boldsymbol{v}+\boldsymbol{w})\cdot\boldsymbol{T} = \boldsymbol{v}\cdot\boldsymbol{T}+\boldsymbol{w}\cdot\boldsymbol{T}\]
				
				\vspace{1mm}
				
				(9) 结合律:对于任意的 \(\boldsymbol{u},\boldsymbol{v},\boldsymbol{w}\in V\),有:
				
				\vspace{-4mm}
				
				\[(\boldsymbol{u}\boldsymbol{v})\cdot\boldsymbol{w} = \boldsymbol{u}\cdot(\boldsymbol{v}\boldsymbol{w})\qquad\boldsymbol{u}\cdot(\boldsymbol{v}\boldsymbol{w}) = (\boldsymbol{u}\cdot\boldsymbol{v})\boldsymbol{w}\]
				
				\vspace{1mm}
				
				性质(1)\(\sim\)(9)足以讲清楚并矢张量的运算性质,但是不够直观,我们将其展开成分量形式:
				
				考虑 \(V\) 是实数域 \(\mathbb{R}\) 上的有限维线性空间(\(\mathrm{dim}V = n\)),内积正定,设 \((\boldsymbol{e}_1,\cdots,\boldsymbol{e}_{n})\) 是 \(V\) 中的一组基,则对任意 \(\boldsymbol{v},\boldsymbol{w}\in V\),展开得到:\(\boldsymbol{v} = \sum_{i}v^{i}\boldsymbol{e}_{i}\),\(\sum_{i}w^{i}\boldsymbol{e}_{i}\),对于并矢 \(\boldsymbol{v}\boldsymbol{w}\),有:}
			
			\vspace{-5mm}
			
			\begin{align*}
				\boldsymbol{v}\boldsymbol{w}&= \left(\sum_{i}v^i\boldsymbol{e}_{i}\right)\left(\sum_{j}w^{i}\boldsymbol{e}_{j}\right) \overset{(5)}{=} \sum_{i}\sum_{j}(v^{i}\boldsymbol{e}_{i})(w^{j}\boldsymbol{e}_{j})\\
				&\overset{(2)}{=} \sum_{i}\sum_{j}\boldsymbol{e}_{i}\left(v^{i}(w^{j}\boldsymbol{e}_{j})\right)\overset{(2)}{=}\sum_{i}\sum_{j}\left(v^{i}w^{j}\right)\boldsymbol{e}_{i}\boldsymbol{e}_{j}
			\end{align*}
			
			\vspace{1mm}
			
			\par{\kaishu 我们说明了,所有形如 \(\boldsymbol{v}\boldsymbol{w}\) 的并矢张量都可以写成 \(\boldsymbol{e}_{i}\boldsymbol{e}_{j}\) 的线性组合。接着使用规则(3)和上述结论,所有的二阶张量也能表达为 \(\boldsymbol{e}_{i}\boldsymbol{e}_{j}\) 的线性组合。
				
				由规则(1)和(3),每一个 \(\boldsymbol{e}_{i}\boldsymbol{e}_{j}\) 都是一个二阶张量,再由规则(3),它们的任意线性组合也是二阶张量,因此有:二阶张量等价于 \(\boldsymbol{e}_{i}\boldsymbol{e}_{j}\) 的线性组合。
				
				根据规则(4)可知:全部的 \(\boldsymbol{e}_{i}\boldsymbol{e}_{j}\) 线性无关,构成了线性空间 \(V\otimes V\) 的一组基。
				
				最后利用规则(6)\(\sim\)(9)不难把所有的缩并化简为 \(\left(\boldsymbol{e}_{i}\boldsymbol{e}_{j}\right)\cdot\boldsymbol{e}_{k}\)。特别的,如果我们将 \(V\) 选取为欧氏空间(或者闵氏空间),\(\left\{\boldsymbol{e}_{1},\boldsymbol{e}_{2},\cdots,\boldsymbol{e}_{n}\right\}\) 选取为\textbf{标准正交基},那么结果就非常简单了。
				
				下面我们讨论张量的表示:}
			
		\end{mdframed}
	\end{center}
	
	\begin{center}
		\begin{mdframed}[shadow = false]
			
			\par{\kaishu 物理学中提到的张量一般是张量场,其定义独立于任何选取的基(在物理学中一般选取的是坐标)但是我们会用特定坐标系中的基的分量表示;这些分量形成一个数组,可以将其视为高维矩阵。对于 \(n\) 维空间上的 \(r\) 阶张量,其有 \(n^r\) 个分量,\(r\) 也被称为该张量的秩(不同于矩阵的秩(rank))。第零阶张量(\(r=0\))为标量,第一阶张量(\(r=1\))为矢量,第二阶张量(\(r = 2\))则成为矩阵,三阶以上张量则无法直观表示,想要计算我们只能研究其分量。
				
				考虑线性空间 \(V\)(\(\mathrm{dim}V = n\)),为了便于理解,给定一个标准正交基 \(\left(\boldsymbol{e}_1,\boldsymbol{e}_2,\cdots,\boldsymbol{e}_{n}\right)\),对于矢量 \(\boldsymbol{v}\in V\),可以写成:\(\boldsymbol{v} = v^1\boldsymbol{e}_1+\cdots+v^n\boldsymbol{e}_{n} = v^{i}\boldsymbol{e}_{i}\)(使用爱因斯坦求和规则)。
				
				爱因斯坦约定的优点是:可以被应用到 \(V\) 的对偶空间 \(V^*\) 和张量积空间 \(V\otimes V\) 中,\(V\otimes V\) 的基可以写成 \(\boldsymbol{e}_i\boldsymbol{e}_{j} = \boldsymbol{e}_i\otimes\boldsymbol{e}_j\),任意张量 \(\boldsymbol{T}\in V\otimes V\) 可以写成:}
			
			\vspace{-3mm}
			
			\begin{equation*}
				\boldsymbol{T} = T^{ij}\boldsymbol{e}_{i}\boldsymbol{e}_{j} \qquad (T^{ij}\,\text{\kaishu 表示张量的ij分量})
			\end{equation*}
			
			\vspace{1mm}
			
			\par{\kaishu \(V\) 的对偶空间 \(V^*\) 有对应的标准正交基 \(\left(\boldsymbol{e}^{1},\boldsymbol{e}^{2},\cdots,\boldsymbol{e}^{n}\right)\),满足:\(\boldsymbol{e}^{i}\cdot\boldsymbol{e}_{j} = \delta^{i}_{j}\)。}
			
			\par{\kaishu 我们通过一个例子来说明分量表示张量的优势:}
			
			\par{\kaishu 考虑4维时空(分量 \(\mu,\,\nu\) 记号为0,1,2,3)二阶张量\(\boldsymbol{A}\) 分量为\(A^{\mu\nu}\),我们作如下运算:}
			
			\vspace{-7mm}
			
			\begin{gather*}
				C^{\mu} = A^{\mu\nu}B_{\nu}+f^{\mu}\\
				\Leftrightarrow\qquad C^{\mu} = A^{\mu 0}B_{0}+A^{\mu 1}B_{1}+A^{\mu 2}B_{2}+A^{\mu 3}B_{3}+f^{\mu}
			\end{gather*}
			
			\par{\kaishu 用分量表示的张量使我们更加直观地看出张量分量的代数关系,而且可以写出任意阶的任意张量,其有效范围覆盖了并矢表示(因此我们更多使用分量表示张量,前文介绍的并矢表示作为张量内容的补充出现,理解即可)。}
			
		\end{mdframed}
	\end{center}
	
	\subsubsection{\kaishu 典例:麦克斯韦应力张量计算 \& 带电粒子动量}
	
	\par{\kaishu \textbf{例1}:考虑放置在均匀电场 \(\textbf{E}_{0}\) 中的均匀带电板,其形状如图2.2.2所示。带电板矩形表面面积为 \(A\),电荷密度为 \(\sigma\),不考虑边缘效应。}
	
	\begin{center}
		\includegraphics[width=0.4\linewidth]{C:/Users/16965/OneDrive/Pictures/Electrondynamics/Electrodynamics fig 2-2-2.pdf}
		\captionof{figure}{\text{\kaishu 长方形带电极板示意图,如图示只考虑与外电场平行/反平行的电场}}\label{fig:2-2-2}
	\end{center}
	
	\par{\kaishu 带电板电荷写为 \(Q = A\sigma\);根据高斯定理,考虑如图2.2.2示的曲面将带电板包围,我们写出:}
	
	\[
	E_{1}\cdot2A = \frac{1}{\varepsilon_{0}}A\sigma \qquad\Rightarrow\qquad \textbf{E}_{1} = \frac{\sigma}{2\varepsilon_0}\hat{e}_{z}
	\]
	
	\vspace{3mm}
	
	\par{\kaishu 此时空间的总电场可以写成:}
	
	\vspace{-2mm}
	
	\begin{equation}
		\textbf{E}_{\text{up}} = \textbf{E}_0+\frac{\sigma}{2\varepsilon_0}\hat{e}_{z}\qquad\textbf{E}_{\text{down}} = \textbf{E}_0-\frac{\sigma}{2\varepsilon_{0}}\hat{e}_{z}
	\end{equation}
	
	\vspace{3mm}
	
	\par{\kaishu 现在我们计算麦克斯韦应力张量(此处采用并矢写法):}
	
	\begin{equation}
		\overleftrightarrow{T} = \frac{1}{2}\varepsilon_0E^2\overleftrightarrow{I}-\varepsilon_{0}\textbf{E}\,\textbf{E} = \frac{1}{2}\varepsilon_{0}E^2\left(\hat{e}_{x}\hat{e}_{x}+\hat{e}_{y}\hat{e}_{y}-\hat{e}_{z}\hat{e}_{z}\right)
	\end{equation}
	
	\vspace{3mm}
	
	\par{\kaishu 继续利用高斯定理的曲面(记为 \(S\)),计算电磁场对带电板的作用:}
	
	\vspace{-5mm}
	
	\begin{align}
		\textbf{F}_{\text{em}} &= -\oint_{S}\left(\textbf{n}\cdot\overleftrightarrow{T}\right)\cdot\mathrm{d}\textbf{S} = -A\hat{e}_{z}\cdot\left(\overleftrightarrow{T}_{\text{up}}-\overleftrightarrow{T}_{\text{down}}\right)\nonumber\\[6pt]
		&= A\hat{e}_{z}\,\frac{1}{2}\varepsilon_{0}\left(E^{2}_{\text{up}}-E^{2}_{\text{down}}\right) = A\hat{e}_{z}\,\frac{1}{2}\varepsilon_{0}\cdot4E_{0}\,\frac{\sigma}{2\varepsilon_0} = A\sigma E_0\hat{e}_z
	\end{align}
	
	\vspace{3mm}
	
	\par{\kaishu 我们代入电荷 \(Q\),发现式(2.2.11)可以被写成 \(\textbf{F}_{\text{em}} = Q \,\textbf{E}_0\) 的形式,符合第一章的结论。}
	
	\par{\kaishu \textcolor{blue}{例1对应的情景较为简单,如果不用麦克斯韦应力张量,也可以得到正确的结果;但是当电磁场以及介质较为混乱时,我们不得不使用应力张量来解决。}}
	
	\vspace{3mm}
	
	\par{\kaishu \textbf{例2}:在均匀磁场 \(\textbf{B}_0\) 中存在线电流 \(I\),电流方向与 \(\textbf{B}_0\) 垂直。若以电流为中心轴线作一圆柱面,计算单位长度圆柱体外的场对柱内场的作用力(只考虑侧面)。}
	
	\par{\kaishu 我们取 \(\textbf{B}_0\) 为 \(x\) 方向,取 \(I\) 方向为 \(z\) 方向,电流产生的磁场和均匀磁场为(假设考察点距离中心轴为 \(r\)):}
	
	\vspace{-3mm}
	
	\begin{equation}
		\textbf{B} = \frac{\mu_0 I}{2\pi r^2}\,\hat{e}_z\times\textbf{r}+\textbf{B}_0 = \left(B_0-\frac{\mu_0 I}{2\pi r}\sin\phi\right)\hat{e}_x+\frac{\mu_0 I}{2\pi r}\cos\phi\,\hat{e}_y
	\end{equation}
	
	\vspace{3mm}
	
	\par{\kaishu 在半径 \(r\) 处的麦克斯韦应力张量为:}
	
	\vspace{-5mm}
	
	\begin{align}
		\overleftrightarrow{T} &= \frac{1}{2\mu_0}B^{2}\overleftrightarrow{I}-\frac{1}{\mu_0}\textbf{B}\,\textbf{B}\nonumber\\[6pt]
		&= \frac{1}{2\mu_0}\left[B_0^2-\frac{\mu_0 I B_0}{\pi r}\sin\phi+\left(\frac{\mu_0 I}{2\pi r}\right)^2\right]\left(\hat{e}_x\hat{e}_x+\hat{e}_y\hat{e}_y+\hat{e}_z\hat{e}_z\right)\nonumber\\[6pt]
		&-\frac{1}{\mu_0}\left[\left(B_0-\frac{\mu_0 I}{2\pi r}\sin\phi\right)^{2}\hat{e}_x\hat{e}_x+\left(\frac{\mu_0 I}{2\pi r}\cos\phi\right)^2\hat{e}_y\hat{e}_y\right]\nonumber\\[6pt]
		&-\frac{1}{\mu_0}\left(B_0-\frac{\mu_0 I}{2\pi r}\sin\phi\right)\frac{\mu_0 I}{2\pi r}\cos\phi\left(\hat{e}_x\hat{e}_y+\hat{e}_{y}\hat{e}_x\right) 
	\end{align}
	
	\vspace{3mm}
	
	\par{\kaishu 相对应的,写出力密度(此处应力张量写成并矢形式),由于是静磁场,电场 \(\textbf{E} = 0\):}
	
	\vspace{-3mm}
	
	\begin{align}
		\vec{f} &= -\hat{e}_{\rho}\cdot\overleftrightarrow{T} = -\left(\cos\phi\,\hat{e}_x+\sin\phi\,\hat{e}_y\right)\cdot\left(\frac{1}{2\mu_0}B^2\overleftrightarrow{I}-\frac{1}{\mu_0}\textbf{B}\,\textbf{B}\right)\nonumber\\[8pt]
		&= \frac{1}{2\mu_0}\left[B_0^2-\left(\frac{\mu_0 I}{2\pi r}\right)^2\right]\cos\phi\,\hat{e}_x+\left\{\frac{I B_0}{2\pi r}-\frac{1}{2\mu_0}\left[B_0^2+\left(\frac{\mu_0 I}{2\pi r}\right)^2\right]\sin\phi\right\}\hat{e}_y
	\end{align}
	
	\vspace{3mm}
	
	\par{\kaishu 将式(2.2.13)代入力密度的过程较为冗长,需要仔细计算。此时我们求圆柱面外部磁场对圆柱内场的作用力(单位长度 \(r = 1\)),总力为:}
	
	\vspace{-2mm}
	
	\begin{equation}
		\textbf{F}_{\text{em}} = \int_{0}^{2\pi}\vec{f}_{\rho}\,\mathrm{d}\phi = \frac{IB_0}{2\pi}\hat{e}_y\int_{0}^{2\pi}\mathrm{d}\phi = IB_0\,\hat{e}_y
	\end{equation}
	
	\vspace{3mm}
	
	\par{\kaishu 不难注意到:\(\int_{0}^{2\pi}\cos\phi\,\mathrm{d}\phi =\int_{0}^{2\pi}\sin\phi\,\mathrm{d}\phi = 0\),我们积分时无需考虑带有因子 \(\cos\phi\)、\(\sin\phi\) 的项。式(2.2.15)说明:用麦克斯韦应力张量得到的作用等同于磁场对电流的作用力。}
	
	\vspace{3mm}
	
	\par{\kaishu \textbf{例3}:质量为 \(m\),电荷量为 \(q\) 的粒子在稳恒磁场中运动,求带电粒子的总动量。}
	
	\par{\kaishu 回顾上一节内容,我们得到的电磁场动量守恒表达式为(即式(2.2.7)):}
	
	\vspace{-2mm}
	
	\begin{equation*}
		\frac{\mathrm{d}\textbf{G}_{\text{m}}}{\mathrm{d}t} = -\oint_{S}\overleftrightarrow{T}\cdot\mathrm{d}\textbf{S}-\frac{\mathrm{d}}{\mathrm{d}t}\int_{V}\frac{1}{c^2}\textbf{S}_{\text{P}}\,\mathrm{d}\tau
	\end{equation*}
	
	\vspace{3mm}
	
	\par{\kaishu 现在我们将积分区域拓展到无穷远处,则项 \(\oint_{S}\overleftrightarrow{T}\cdot\mathrm{d}\textbf{S} = 0\),上式可以写成:}
	
	\begin{equation}
		\frac{\mathrm{d}\textbf{G}_{\text{m}}}{\mathrm{d}t} = -\frac{\mathrm{d}}{\mathrm{d}t}\int_{V}\frac{1}{c^2}\textbf{S}_{\text{P}}\,\mathrm{d}\tau \qquad\Rightarrow\qquad \textbf{G}_{\text{em}} = \int_{\infty}\varepsilon_{0}\left(\textbf{E}\times\textbf{B}\right)\mathrm{d}\tau
	\end{equation}
	
	\vspace{3mm}
	
	\par{\kaishu 式(2.2.16)中 \(\textbf{G}_{\text{em}}\) 被称为电磁场的电磁动量。对应地 \(\vec{\textbf{g}}\) 被称为电磁动量密度。}
	
	\par{\kaishu 电荷在磁场中运动时,不仅自己携带机械动量,而且会在空间中改变电磁动量。后者是由电荷产生的电磁场与空间原有的电磁场相互作用产生的。此处将电磁场的动量写成:}
	
	\begin{equation}
		\textbf{G}_{\text{em}} = \int_{\infty}\varepsilon_0(\textbf{E}\times\textbf{B})\,\mathrm{d}\tau = \int_{\infty}\varepsilon_{0}\textbf{E}\times(\nabla\times\textbf{A})\,\mathrm{d}\tau
	\end{equation}
	
	\vspace{3mm}
	
	\par{\kaishu 问题中电荷运动的速度远小于光速,其产生的磁场可以忽略,因此我们近似认为:电荷只产生电场,此处我们近似为静电场(同样是因为速度远小于光速):}
	
	\begin{equation*}
		\nabla\times\textbf{E} \simeq 0 \qquad\&\qquad \nabla\cdot\textbf{E} = \frac{q}{\varepsilon_{0}}\,\delta\left[\textbf{r}'-\textbf{r}(t)\right]
	\end{equation*}
	
	\vspace{3mm}
	
	\par{\kaishu 其中:\(\textbf{r}(t)\) 表示粒子在 \(t\) 时刻所处的位矢。基于我们上面的假设,空间的磁场完全由外界磁场贡献,因此:\(\nabla\cdot\textbf{A} = 0\),这是恒定磁场的要求。}
	
	\par{\kaishu 接下来我们使用矢量叉乘公式:\(\textbf{A}\times(\nabla\times\textbf{B}) = \nabla(\textbf{A}\cdot\textbf{B})-\textbf{B}\times(\nabla\times\textbf{A})-(\textbf{B}\cdot\nabla)\textbf{A}-(\textbf{A}\cdot\nabla)\textbf{B}\),根据条件 \(\nabla\times\textbf{E} = 0\),代入后有:}
	
	\vspace{-2mm}
	
	\begin{equation*}
		\textbf{E}\times(\nabla\times\textbf{A}) = \nabla(\textbf{E}\cdot\textbf{A})-(\textbf{A}\cdot\nabla)\textbf{E}-(\textbf{E}\cdot\nabla)\textbf{A}
	\end{equation*}
	
	\vspace{5mm}
	
	\par{\kaishu 我们再使用并矢的公式 \(\nabla\cdot(\textbf{A}\,\textbf{B}) = (\nabla\cdot\textbf{A})\,\textbf{B}+(\textbf{A}\cdot\nabla)\,\textbf{B}\),将这两项 \(-(\textbf{A}\cdot\nabla)\textbf{E}-(\textbf{E}\cdot\nabla)\textbf{A}\) 化为:\(-\nabla\cdot\left(\textbf{A}\,\textbf{E}+\textbf{E}\,\textbf{A}\right)+(\nabla\cdot\textbf{E})\textbf{A}+(\nabla\cdot\textbf{A})\textbf{E}\),将此处的结果代入上式得到(\(\nabla\cdot\textbf{A} = 0\)):}
	
	\vspace{-4mm}
	
	\begin{align}
		\textbf{E}\times(\nabla\times\textbf{A}) &= \nabla(\textbf{A}\cdot\textbf{E})-\nabla\cdot\left(\textbf{A}\,\textbf{E}+\textbf{E}\,\textbf{A}\right)+(\nabla\cdot\textbf{E})\,\textbf{A}\nonumber\\[6pt]
		&= \nabla\cdot\left[(\textbf{A}\cdot\textbf{E})\overleftrightarrow{I}-\textbf{A}\,\textbf{E}-\textbf{E}\,\textbf{A}\right]+\frac{\rho}{\varepsilon_0}\delta\left[\textbf{r}'-\textbf{r}(t)\right]\textbf{A}
	\end{align}
	
	\vspace{3mm}
	
	\par{\kaishu 我们先考虑等式(2.2.18)右边第一项,根据高斯定理,积分区域拓展到无穷远处,有:}
	
	\vspace{-3mm}
	
	\begin{equation}
		\int_{\infty}\nabla\cdot\left[(\textbf{A}\cdot\textbf{E})\overleftrightarrow{I}-\textbf{A}\,\textbf{E}-\textbf{E}\,\textbf{A}\right]\,\mathrm{d}\tau = \oint_{S}\left[(\textbf{A}\cdot\textbf{E})\overleftrightarrow{I}-\textbf{A}\,\textbf{E}-\textbf{E}\,\textbf{A}\right]\cdot\mathrm{d}\textbf{S} = 0
	\end{equation}
	
	\vspace{3mm}
	
	\par{\kaishu 下面我们考虑式(2.2.18)右边第二项,我们写出:}
	
	\begin{equation}
		\int_{\infty}\frac{\rho}{\varepsilon_0}\textbf{A}\delta\left[\textbf{r}'-\textbf{r}(t)\right]\,\mathrm{d}\tau = \int_{\infty}\frac{\rho}{\varepsilon_0}\textbf{A}\,\delta\left[\textbf{r}'-\textbf{r}(t)\right]\,\mathrm{d}^3\textbf{r}' = \frac{q}{\varepsilon_0}\textbf{A}
	\end{equation}
	
	\vspace{3mm}
	
	\par{\kaishu \textcolor{blue}{上式的 \(\mathrm{d}^{3}\textbf{r}'\) 与 \(\mathrm{d}\tau\) 表示相同的意义,这种写法只是为了方便对 \(\delta\)-函数进行积分。}因此我们得到:\(\textbf{G}_{\text{em}} = \varepsilon_0\cdot\frac{q}{\varepsilon_0}\textbf{A} = q\textbf{A}\),加上带电粒子的机械动量,我们可以得到带电粒子在磁场中运动的总动量:}
	
	\vspace{-1mm}
	
	\begin{equation}
		\textbf{p} = m\textbf{v}+q\textbf{A}
	\end{equation}
	
	\vspace{3mm}
	
	\par{\kaishu \(\textbf{p}\) 不仅是总动量,还是正则动量。式(2.2.21)对应的附加动量 \(q\textbf{A}\) 是由电荷和电磁场相互作用得到的,单独电荷存在或者电磁场存在都无法得到这个结果;而且,这个结果是在低速条件 \(v\ll c\) 下得到的。\textcolor{blue}{此式在量子力学中相当重要。}}
	
	\subsubsection{\kaishu 角动量守恒定律}
	
	\par{\kaishu 1936年,贝斯(bethe)在实验上证实了电磁场角动量的存在,下面我们导出电磁场角动量表达式和其对应的角动量守恒定律。}
	
	\par{\kaishu 考虑电磁场对带电体的作用力,施加在带电体微元 \(\mathrm{d}\tau\) 上的力为:}
	
	\vspace{-2mm}
	
	\begin{equation}
		\mathrm{d}\textbf{F} = \left(\rho\,\textbf{E}+\vec{\textbf{j}}\times\textbf{B}\right)\,\mathrm{d}\tau
	\end{equation}
	
	\vspace{3mm}
	
	\par{\kaishu 设微元 \(\mathrm{d}\tau\) 距离带电体中心为 \(\textbf{r}\),那么考虑 \(\mathrm{d}\textbf{F}\) 对带电体施加的力矩(记力矩为\(\boldsymbol{M}\)):}
	
	\vspace{-2mm}
	
	\begin{equation}
		\mathrm{d}\boldsymbol{M} = \textbf{r}\times\mathrm{d}\textbf{F} = \textbf{r}\times\left(\rho\,\textbf{E}+\vec{j}\times\textbf{B}\right)\,\mathrm{d}\tau
	\end{equation}
	
	\vspace{3mm}
	
	\par{\kaishu 设带电体的机械角动量为 \(\textbf{L}_{\text{m}}\),根据经典力学,我们知道:角动量的时间变化率等于作用在物体上的合力矩,那么有:}
	
	\vspace{-3mm}
	
	\begin{equation}
		\frac{\mathrm{d}\textbf{L}_{\text{m}}}{\mathrm{d}t} = \int_{\tau}\textbf{r}\times\left(\rho\,\textbf{E}+\vec{\textbf{j}}\times\textbf{B}\right)\,\mathrm{d}\tau
	\end{equation}
	
	\vspace{3mm}
	
	\par{\kaishu 利用动量守恒章节得到的结果:\(\vec{f} = \rho\,\textbf{E}+\vec{j}\times\textbf{B} = -\nabla\cdot\overleftrightarrow{T}-\frac{\partial\vec{\textbf{g}}}{\partial t}\),将此式代入式(2.2.24)得到:}
	
	\vspace{2mm}
	
	\begin{equation}
		\frac{\mathrm{d}\textbf{L}_{\text{m}}}{\mathrm{d}t} = -\frac{\mathrm{d}}{\mathrm{d}t}\int_{\tau}\textbf{r}\times\vec{\textbf{g}}\,\mathrm{d}\tau-\int_{\tau}\textbf{r}\times\left(\nabla\cdot\overleftrightarrow{T}\right)\,\mathrm{d}\tau
	\end{equation}
	
	\vspace{3mm}
	
	\par{\kaishu 对于等式(2.2.25)右边第二项,我们想要把 \(\nabla\) 算符提出括号,可以利用关系式 \(\textbf{r}\times\left(\nabla\cdot\overleftrightarrow{T}\right) = -\nabla\cdot\left(\overleftrightarrow{T}\times\textbf{r}\right)\),证明如下(作为补充内容,读者理解可跳过)。}
	
	\begin{center}
		\begin{mdframed}[shadow = false]
			
			\par{\kaishu \textbf{补充:}证明 \(\textbf{r}\times\left(\nabla\cdot\overleftrightarrow{T}\right) = -\nabla\cdot\left(\overleftrightarrow{T}\times\textbf{r}\right)\)
				
				\vspace{2mm}
				
				定义三阶张量 \(M_{ijk} = T_{ik}x_{j}-T_{ij}x_{k}\),其中;\(T_{ij} = \frac{1}{2}\left(\varepsilon_0E^2+\frac{1}{\mu_0}B^2\right)\delta_{ij}-\varepsilon_0E_{i}E_{j}-\frac{1}{\mu_0}B_{i}B_{j}\),此处我们使用分量形式。不难注意到:\(T_{ji} = T_{ij}\),并且:}
			
			\vspace{-2mm}
			
			\begin{equation*}
				M_{ikj} = T_{ij}x_{k}-T_{ik}x_{j} = -M_{ijk}
			\end{equation*}
			
			\vspace{3mm}
			
			\par{\kaishu \(M_{ijk}\) 是关于 \(j\)、\(k\) 的反对称张量,接下来我们对 \(x^{i}\) 求导(其中:\(\partial_{i} = \frac{\partial}{\partial x^{i}}\),且不区分角标):}
			
			\vspace{-5mm}
			
			\begin{align}
				\partial_{i}M_{ijk} &= \partial_{i}\left(T_{ik}x_{j}-T_{ij}x_{k}\right) = x_{j}\,\partial_{i}T_{ik}-x_{k}\,\partial_{i}T_{ij}+T_{ik}\,\partial_{i}x_{j}-T_{ij}\,\partial_{i}x_{k}\nonumber\\
				&= x_{j}\,\partial_{i}T_{ik}-x_{k}\,\partial_{i}T_{ij}+T_{ik}\,\delta_{ij}-T_{ij}\,\delta_{ik}\nonumber\\
				&= x_{j}\,\partial_{i}T_{ik}-x_{k}\,\partial_{i}T_{ij}+T_{jk}-T_{kj}\nonumber\\
				&= x_{j}\,\partial_{i}T_{ik}-x_{k}\,\partial_{i}T_{ij}
			\end{align}
			
			\vspace{1mm}
			
			\par{\kaishu \(\partial_{i}T_{ik}\) 写成并矢的形式为 \(\nabla\cdot\overleftrightarrow{T}\)(\(\partial_{i}T_{ij}\) 同理),\(\partial_{i}M_{ijk}\) 写成张量形式为:\(\nabla\cdot\overleftrightarrow{M}\),将式(2.2.26)写成矢量形式,我们就得到了:}
			
			\vspace{-3mm}
			
			\begin{equation*}
				\textbf{r}\times\left(\nabla\cdot\overleftrightarrow{T}\right) = -\nabla\cdot\left(\overleftrightarrow{T}\times\textbf{r}\right)
			\end{equation*}
			
			\vspace{1mm}
			
		\end{mdframed}
	\end{center}
	
	\par{\kaishu 现在我们用此关系式,将式(2.2.25)写成更好处理的形式:}
	
	\begin{equation}
		\frac{\mathrm{d}\textbf{L}_{\text{m}}}{\mathrm{d}t} = -\frac{\mathrm{d}}{\mathrm{d}t}\int_{\tau}\textbf{r}\times\vec{\textbf{g}}\,\mathrm{d}\tau+\int_{\tau}\nabla\cdot\left(\overleftrightarrow{T}\times\textbf{r}\right)\,\mathrm{d}\tau
	\end{equation}
	
	\vspace{3mm}
	
	\par{\kaishu 对式(2.2.27)使用高斯定理,假设带电体闭合区域边界为 \(S\):}
	
	\begin{equation}
		\frac{\mathrm{d}\textbf{L}_{\text{m}}}{\mathrm{d}t} = -\frac{\mathrm{d}}{\mathrm{d}t}\int_{\tau}\textbf{r}\times\vec{\textbf{g}}\,\mathrm{d}\tau+\oint_{S}\mathrm{d}\textbf{S}\cdot\left(-\overleftrightarrow{T}\times\textbf{r}\right)
	\end{equation}
	
	\vspace{3mm}
	
	\par{\kaishu 式(2.2.28)就是电磁体系角动量守恒的表达式,等式右边第二项可以定义:\(\overleftrightarrow{M} = -\overleftrightarrow{T}\times\textbf{r}\),这就是\textbf{角动量流密度张量}。类似前面能量守恒和动量守恒定律的讨论,考虑无穷大区域 \(\tau\to\infty\),我们消去等式右边第二项,即:}
	
	\vspace{-3mm}
	
	\[\lim_{\tau\to\infty}\oint_{S = \partial\tau}\mathrm{d}\textbf{S}\cdot\left(\overleftrightarrow{T}\times\textbf{r}\right)\,\mathrm{d}\tau = 0\]
	
	\vspace{3mm}
	
	\par{\kaishu 考虑上式,我们将会得到:}
	
	\begin{equation}
		\frac{\mathrm{d}\textbf{L}_{\text{m}}}{\mathrm{d}t} = -\frac{\mathrm{d}}{\mathrm{d}t}\int_{\tau}\textbf{r}\times\vec{\textbf{g}}\,\mathrm{d}\tau \qquad\Rightarrow\qquad \textbf{L}_{\text{em}} = \int_{\tau} \textbf{r}\times\varepsilon_{0}(\textbf{E}\times\textbf{B})\,\mathrm{d}\tau
	\end{equation}
	
	\vspace{3mm}
	
	\par{\kaishu 式(2.2.29)中 \(\textbf{L}_{\text{em}}\) 表示电磁场总角动量,我们也可以定义电磁场角动量密度 \(\vec{\boldsymbol{l}}_{\text{em}} = \textbf{r}\times\varepsilon_0(\textbf{E}\times\textbf{B})\)。}
	
	\vspace{1mm}
	
	\par{\kaishu 电磁场角动量的概念,在静场已经富有成效(我们将在下面的例子中看到这一点);但是在处理电磁场辐射理论和带电粒子碰撞问题时,角动量守恒定律才能发挥真正的威力。}
	
	\vspace{1mm}
	
	\par{\kaishu \textbf{典例:}考虑一个长度为 \(h\) 的圆柱形电容器,外筒面半径为 \(R_{1}\),质量为 \(m_{1}\),且接地;内筒面半径为 \(R_{2}\),质量为 \(m_{2}\),且电势为 \(\varphi_{0}\),均匀外磁场 \(\textbf{B}_0\) 与轴线平行,如图2.2.3所示。求 \(\textbf{B}_0\) 移除瞬间电容器旋转的角速度(不考虑电容器边界效应)。}
	
	\begin{center}
		\includegraphics[width=0.48\linewidth]{C:/Users/16965/OneDrive/Pictures/Electrondynamics/Electrodynamics fig 2-2-3.pdf}
		\captionof{figure}{\text{\kaishu 同轴圆柱形电容器,我们在图中构建了柱坐标系方便研究}}\label{fig:2-2-3}
	\end{center}
	
	\par{\kaishu 我们考虑整个电磁体系,有:\(\oint_{S}\mathrm{d}\textbf{S}\cdot\left(-\overleftrightarrow{T}\times\textbf{r}\right)\,\mathrm{d}\tau = 0\),将此式代入式(2.2.28),我们知道:}
	
	\begin{equation}
		\frac{\mathrm{d}}{\mathrm{d}t}\left(\textbf{L}_{m}+\textbf{L}_{\text{em}}\right) = 0 \qquad\Rightarrow\qquad\nonumber \textbf{L}_{m}+\textbf{L}_{\text{em}} = \textbf{C}\qquad(\textbf{C}\,\text{\kaishu 为常矢量})
	\end{equation}
	
	\vspace{3mm}
	
	\par{\kaishu 在初始时刻(\(t=0\)),圆筒静止,机械角动量 \(\textbf{L}_{\text{m}} = 0\);电磁场角动量 \(\textbf{L}_{\text{em}} = \textbf{C}\);\(t\) 时刻去除 \(\textbf{B}_0\),此时体系内所有电磁角动量必须全部转化为机械角动量,即 \(\textbf{L}_{\text{m}} = \textbf{C}\),下面先计算 \(\textbf{C}\):}
	
	\vspace{-2mm}
	
	\begin{equation}
		\textbf{C} = \textbf{L}_{\text{em}} = \int_{V}\textbf{r}\times\vec{\textbf{g}}\,\mathrm{d}\tau = \varepsilon_{0}\int_{0}^{h}\mathrm{d}z\int_{0}^{2\pi}\int_{R_{2}}^{R_{1}}\boldsymbol{\rho}\times(\textbf{E}\times\textbf{B})\, \rho\mathrm{d}\rho\,\mathrm{d}\phi
	\end{equation}
	
	\vspace{3mm}
	
	\par{\kaishu 下面计算电场,假设内层圆柱带电荷 \(Q\),我们根据高斯定理,取闭合面为圆柱面,半径为 \(\rho\):}
	
	\begin{equation}
		\oint_{S}\textbf{E}\cdot\mathrm{d}\textbf{S} = \frac{Q}{\varepsilon_0} \qquad\Rightarrow\qquad \textbf{E} = \frac{Q}{2\pi\varepsilon_0\rho h}\hat{e}_{\rho}\nonumber
	\end{equation}
	
	\vspace{3mm}
	
	\par{\kaishu 我们根据电势的定义:\(U = -\int\textbf{E}\cdot\mathrm{d}\textbf{r}\),可以写出下面的等式:}
	
	\begin{equation}
		\varphi_{0} = \int_{R_{2}}^{R_{1}}\frac{Q}{2\pi\varepsilon_0\rho h}\,\mathrm{d}\rho = \frac{Q}{2\pi\varepsilon_0 h}\ln\frac{R_{1}}{R_{2}}
	\end{equation}
	
	\vspace{3mm}
	
	\par{\kaishu 根据式(2.2.31),我们可以得到 \(Q = 2\pi\varepsilon_0 h\varphi_{0}\ln\frac{R_{1}}{R_{2}}\)(电容为 \(C = 2\pi\varepsilon_0 h\ln\frac{R_{1}}{R_{2}}\)),电场为:}
	
	\begin{equation}
		\textbf{E} = \frac{\varphi_{0}}{\ln R_{1}-\ln R_{2}}\,\frac{1}{\rho}\,\hat{e}_{\rho}
	\end{equation}
	
	\vspace{3mm}
	
	\par{\kaishu 将得到的电场和磁场代入电磁角动量计算,矢量方向为:\(-\hat{e}_{\rho}\times\left(\hat{e}_{\rho}\times\hat{e}_{z}\right) = \hat{e}_{z}\),计算得:}
	
	\vspace{-5mm}
	
	\begin{align}
		\textbf{C} = \textbf{L}_{\text{em}} &= \varepsilon_{0}\int_{0}^{h}\mathrm{d}z\int_{0}^{2\pi}\int_{R_{2}}^{R_{1}}\frac{\varphi_{0}\,B_{0}}{\ln R_{1}-\ln R_{2}}\,\rho\mathrm{d}\rho\,\mathrm{d}\phi\,\hat{e}_{z}\nonumber\\[8pt]
		&= 2\pi\varepsilon_0h\cdot\frac{\varphi_{0}B_0}{\ln R_{1}-\ln R_{2}}\left[\frac{\rho^2}{2}\right]_{R_{2}}^{R_{1}}\,\hat{e}_{z}= \frac{\pi\varepsilon_0h\varphi_{0}B_{0}\left(R_{1}^{2}-R_{2}^{2}\right)}{\ln R_{1}-\ln R_{2}}\hat{e}_{z}
	\end{align}
	
	\vspace{3mm}
	
	\par{\kaishu 假设角速度为 \(\boldsymbol{\omega}\),根据经典力学,我们可以写出此圆柱体系的角动量表达式:}
	
	\vspace{-2mm}
	
	\begin{equation}
		\textbf{L}_{\text{m}} = \left(m_{1}R_{1}^{2}+m_{2}R_{2}^{2}\right)\boldsymbol{\omega}
	\end{equation}
	
	\vspace{3mm}
	
	\par{\kaishu 将式(2.2.33)的结果代入机械角动量,并且与式(2.2.34)联立,我们得到角速度为:}
	
	\begin{equation}
		\boldsymbol{\omega} = \frac{1}{m_{1}R_{1}^{2}+m_{2}R_{2}^{2}}\cdot\frac{\pi\varepsilon_0h\varphi_{0}B_{0}\left(R_{1}^{2}-R_{2}^{2}\right)}{\ln R_{1}-\ln R_{2}}\hat{e}_{z}
	\end{equation}
	
	\vspace{2mm}
	
	\subsection{\kaishu 介质中电磁场的守恒定律}
	
	\subsubsection{\kaishu 介质中的能量守恒}
	
	\par{\kaishu 电磁场在真空中只对传导电流做功;但在电磁介质中,存在极化电流和磁化电流,电磁场\textbf{同时}对传导电流、极化电流和磁化电流做功。}
	
	\par{\kaishu 电磁场对 \(\mathrm{d}\tau\) 体积内的电荷/电流在单位时间内做的总功为:}
	
	\vspace{-1mm}
	
	\begin{equation}
		\frac{\mathrm{d}W_{\text{em}}}{\mathrm{d}t} = \int_{\tau}\left(\vec{j}_{\mathrm{f}}+\vec{j}_{\mathrm{p}}+\vec{j}_{\mathrm{m}}\right)\cdot\textbf{E}\,\mathrm{d}\tau
	\end{equation}
	
	\vspace{3mm}
	
	\par{\kaishu 我们先观察式(2.3.1)的各项物理意义:
		
		\vspace{-3mm}
		
		\begin{enumerate}
			
			\item \(\vec{j}_{\text{f}}\cdot\textbf{E}\) 对应的是电磁场对自由电荷做的功,这部分功转化为电荷运动的机械动能;在除超导体外的电磁介质中,杂质发生散射,机械动能在此过程中转化为环境的热能。\textcolor{blue}{在我们熟悉的大多数情况下,机械动能向热能的转化过程不可逆。}
			
			\vspace{-3mm}
			
			\item \(\left(\vec{j}_{\text{p}}+\vec{j}_{\text{m}}\right)\cdot\textbf{E}\) 对应的是电磁场对电磁介质中的束缚电荷/束缚电流所做的功。在介质中,这部分功转化成电荷拉开后的弹性势能或化学势能,以及这些电荷跟随电场运动而具有的机械动能。\textcolor{blue}{这部分能量束缚在电磁介质中,不转化为环境热能。当电磁场离开介质时,这些能量会通过电磁辐射的形式返还给电磁场。这部分能量依附于电磁场存在。}
			
	\end{enumerate}}
	
	\vspace{-2mm}
	
	\par{\kaishu 现在我们已经基本理解了两类功的不同。我们可以将式(2.3.1)分为两部分:传导电流;极化、磁化电流。由于后者不改变电磁场的能量,我们将其与电磁场能量合并,统称为电磁场在介质中的能量。此时我们只需考虑电磁场对自由电流做功即可。}
	
	\par{\kaishu 考虑电磁场对 \(\mathrm{d}\tau\) 体内的自由电荷/电流在单位时间内做的总功:}
	
	\begin{equation}
		\frac{\mathrm{d}W_{\text{f}}}{\mathrm{d}t} = \int_{\tau}\vec{j}_{\text{f}}\cdot\textbf{E}\,\mathrm{d}\tau
	\end{equation}
	
	\vspace{3mm}
	
	\par{\kaishu 利用 \(\nabla\times\textbf{H} = \vec{j}_{\text{f}}+\frac{\partial\textbf{D}}{\partial t}\),将自由电流消去,我们得到:}
	
	\begin{equation}
		\vec{j}_{\text{f}}\cdot\textbf{E} = (\nabla\times\textbf{H})\cdot\textbf{E}-\textbf{E}\cdot\frac{\partial\textbf{D}}{\partial t}
	\end{equation}
	
	\vspace{3mm}
	
	\par{\kaishu 将式(2.3.3)右边第一项凑配成为全微分的形式:}
	
	\vspace{-5mm}
	
	\begin{gather}
		(\nabla\times\textbf{H})\cdot\textbf{E} = \nabla\cdot(\textbf{H}\times\textbf{E}) + \textbf{H}\cdot(\nabla\times\textbf{E})\quad\longleftarrow\quad\text{\kaishu 代入方程}\,\nabla\times\textbf{E} = -\frac{\partial\textbf{B}}{\partial t}\nonumber\\[8pt]
		\Rightarrow\qquad (\nabla\times\textbf{H})\cdot\textbf{E} = \nabla\cdot(\textbf{H}\times\textbf{E})-\textbf{H}\cdot\frac{\partial\textbf{B}}{\partial t}
	\end{gather}
	
	\vspace{1mm}
	
	\par{\kaishu 对于线性无色散介质,有本构关系:\(\textbf{D} = \varepsilon\,\textbf{E}\),\(\textbf{B} = \mu\,\textbf{H}\)。我们将式(2.3.4)和本构关系代入式(2.3.3)中,得到:}
	
	\vspace{-7mm}
	
	\begin{align}
		\vec{j}_{\text{f}}\cdot\textbf{E} &= -\nabla\cdot(\textbf{E}\times\textbf{H})-\textbf{E}\cdot\frac{\partial\textbf{D}}{\partial t}-\textbf{H}\cdot\frac{\partial\textbf{B}}{\partial t}\nonumber\\[6pt]
		&= -\nabla\cdot\frac{1}{\mu}\left(\textbf{E}\times\textbf{B}\right)-\frac{\partial}{\partial t}\left(\frac{1}{2}\varepsilon E^{2}+\frac{1}{2\mu}B^{2}\right)\nonumber\\[6pt]
		&= -\nabla\cdot\textbf{S}_{\text{P}}-\frac{\partial}{\partial t}u(\textbf{r},t)
	\end{align}
	
	\vspace{2mm}
	
	\par{\kaishu 这里的能量密度定义为:\(u(\textbf{r},t) = \frac{1}{2}(\textbf{E}\cdot\textbf{D}+\textbf{H}\cdot\textbf{B})\) 能流密度定义为:\(\textbf{S}_{\text{P}}(\textbf{r},t) = \textbf{E}\times\textbf{H}\)。根据式(2.3.5),我们可以写出电磁介质中的能量转化方程:}
	
	\begin{equation}
		\frac{\mathrm{d} W'}{\mathrm{d}t} = \int_{\tau}\vec{j}_{\text{f}}\cdot\textbf{E}\,\mathrm{d}\tau = -\oint_{S}\textbf{S}_{\text{P}}\cdot\mathrm{d}\textbf{S}-\frac{\partial}{\partial t}\int_{\tau}u\,\mathrm{d}\tau
	\end{equation}
	
	\vspace{3mm}
	
	\par{\kaishu 其中:\(\int_{\tau}u\,\mathrm{d}\tau\) 为体积内的电磁场能量以及电磁介质中储存的能量,\(W'\) 为体积内其他的能量形式(如传导电荷的机械能,或者传导电流与环境交换的热能)。\(\textbf{S}_{\text{P}}\) 为能流密度,其物理意义是:电磁场和介质拥有的能量在单位时间内流过单位面积的量。\textcolor{blue}{显然,真空可以认为是 \(\varepsilon = \varepsilon_0\),\(\mu = \mu_0\) 的特殊介质,是式(2.3.6)的特殊情形。}}
	
	\subsubsection{\kaishu 介质中的动量守恒}
	
	\par{\kaishu 利用同样的方法讨论动量守恒,电磁介质中总的电荷分布为:\(\rho_f+\rho_P\),总电流分布为:\(\vec{j}_{f}+\vec{j}_p+\vec{j}_{m}\),因此电磁场对带电体以及介质的作用力也就是对上述电荷、电流的作用力,其力密度为:}
	
	\vspace{-1mm}
	
	\begin{equation}
		\vec{f} = \left(\rho_f+\rho_P\right)\,\textbf{E}+\left(\vec{j}_{f}+\vec{j}_p+\vec{j}_{m}\right)\times\textbf{B}
	\end{equation}
	
	\vspace{3mm}
	
	\par{\kaishu 将电荷、电流取为总电荷、总电流,此时麦克斯韦方程中的场均为空间总场,形式如式(1.4.12),我们此处只是将矢量 \(\textbf{P}\)、\(\textbf{M}\) 写成 \(\rho_P\)、\(\vec{j}_{p}\)、\(\vec{j}_{m}\) 的形式。参照2.2.2节的推导过程,我们得到下式:}
	
	\vspace{-3mm}
	
	\begin{gather}
		\vec{f} = -\nabla\cdot\overleftrightarrow{T}^{*}-\frac{\partial\vec{\textbf{g}}^{*}}{\partial t}\\[12pt]
		\text{\kaishu 其中:}\qquad
		\overleftrightarrow{T}^{*} = \frac{1}{2}\left(\varepsilon_0 E^{2}+\frac{1}{\mu_0}B^{2}\right)\overleftrightarrow{I}-\varepsilon_0\textbf{E}\,\textbf{E}-\frac{1}{\mu_0}\textbf{B}\,\textbf{B}\\[10pt]
		\vec{\textbf{g}}^{*} = \varepsilon_0\,\textbf{E}\times\textbf{B}
	\end{gather}
	
	\vspace{1mm}
	
	\par{\kaishu 虽然式(2.3.8)、(2.3.9)、(2.3.10)的形式与真空中的动量守恒一致,但是其中的 \(\textbf{E}\)、\(\textbf{B}\) 分别表示总电场、总磁场。其中不仅含有自由电荷、自由电流激发的“源场”,还包含了 \(\rho_P\)、\(\vec{j}_{p}\)、\(\vec{j}_{m}\) 对应的贡献。现在我们可以将 \(\vec{f}\) 分为两部分:}
	
	\vspace{-1mm}
	
	\begin{equation}
		\vec{f} = \underbrace{\rho_f\,\textbf{E}+\vec{j}_{f}\times\textbf{B}}_{\vec{f}_{1}}\,+\,\underbrace{\rho_P\,\textbf{E}+\left(\vec{j}_{p}+\vec{j}_{m}\right)\times\textbf{B}}_{\vec{f}_{2}}
	\end{equation}
	
	\vspace{2mm}
	
	\par{\kaishu 式(2.3.11)中 \(\vec{f}_{1}\) 代表场对带自由电荷的带电体(导体)的作用力,\(\vec{f}_{2}\) 代表场对介质的作用力。\textcolor{blue}{当我们把电磁场与电磁介质看作一个系统时,$\vec{f}_{2}$ 属于内力,它只是将一些电磁动量存储在介质中。\(\vec{f}_{1}\) 则是纯粹的电磁场对传导电荷(带电体)的作用力,其作用是改变物体的动量。}}
	
	\par{\kaishu 考虑线性无色散介质:\(\textbf{D} = \varepsilon\,\textbf{E}\)、\(\textbf{B} = \mu\,\textbf{H}\),我们可以通过类似2.3.1节的计算得到:}
	
	\vspace{-3mm}
	
	\begin{gather}
		\vec{f}_{1} = -\nabla\cdot\overleftrightarrow{T}'-\frac{\partial\vec{\textbf{g}}'}{\partial t}\\[12pt]
		\text{\kaishu 其中:}\qquad
		\overleftrightarrow{T}' = \frac{1}{2}(\textbf{E}\cdot\textbf{D}+\textbf{B}\cdot\textbf{H})\overleftrightarrow{I}-\textbf{D}\,\textbf{E}-\textbf{B}\,\textbf{H}\\[10pt]
		\vec{\textbf{g}}' = \textbf{D}\times\textbf{B}
	\end{gather}
	
	\vspace{1mm}
	
	\par{\kaishu 在这种定义下,我们将电磁场和电磁介质看作一个整体。电磁动量密度 \(\vec{\textbf{g}}' = \textbf{D}\times\textbf{B}\) 不仅包含了电磁场本身携带的动量,而且包含了束缚于电磁介质体上的机械动量。\textcolor{blue}{后者不依赖于电磁场动量,但它们只能依赖于电磁场存在,因此我们将它们归到电磁动量内。}}
	
	\par{\kaishu 我们也可以写出电磁介质中 \(\vec{\textbf{g}}'\) 和 \(\textbf{S}_{\text{P}}\) 的关系(介质内光速为 \(c^{2} = \frac{1}{\varepsilon\mu}\)):}
	
	\vspace{-1mm}
	
	\begin{equation}
		\vec{\textbf{g}}' = \frac{1}{c^2}\,\textbf{S}_{\text{P}}
	\end{equation}
	
	\vspace{5mm}
	
	\par{\kaishu 下面我们简单分析表达式(2.3.8)和表达式(2.3.12)的适用范围:
		
		\vspace{-3mm}
		
		\begin{enumerate}
			
			\item 式(2.3.8)描述了场对一部分体积内的所有电荷/电流的作用力。现在考虑一个电磁介质物体(导体或介质)处于真空中的电磁场中,此时就应该用式(2.3.8)计算物体受到电磁场的力,有:
			
			\vspace{-2mm}
			
			\begin{equation}
				\textbf{F} = -\oint_{S}\mathrm{d}\textbf{S}\cdot\overleftrightarrow{T}^{*}+\frac{\mathrm{d}}{\mathrm{d}t}\int\vec{\textbf{g}}^{*}\,\mathrm{d}\tau
			\end{equation}
			
			\vspace{2mm}
			
			因为高斯面(图2.3.1标出)内包含所有的电荷/电流,当电磁场处于静态或者随时间简谐变化时:式(2.3.16)等式右边第二项对时间的平均为0。有效力平均为:
			
			\vspace{-2mm}
			
			\begin{equation}
				\left\langle\,\textbf{F}\,\right\rangle = -\oint_{S}\mathrm{d}\textbf{S}\cdot\left\langle\overleftrightarrow{T}\right\rangle
			\end{equation}
			
			\begin{center}
				\includegraphics[width=0.7\linewidth]{C:/Users/16965/OneDrive/Pictures/Electrondynamics/Electrodynamics fig 2-3-1.pdf}
				\captionof{figure}{\kaishu 真空电磁场对电磁介质作用示意图}
			\end{center}
			
			\item 式(2.3.12)适用于计算介质环境中金属导体受到电磁场的作用力,此时我们用下式计算受力(此处我们直接写出平均力,前提条件与式(2.3.17)相同):
			
			\vspace{-2mm}
			
			\begin{equation}
				\left\langle\textbf{F}'\right\rangle = -\oint_{S}\mathrm{d}\textbf{S}\cdot\left\langle\overleftrightarrow{T}'\right\rangle
			\end{equation}
			
			\vspace{2mm}
			
			因为我们只需要对高斯面(如图2.3.2所示)内的“自由电荷”计算受力,而束缚在导体和介质环境界面上的极化电荷并不隶属于金属物体,其受力不应被计算。如果我们利用式(2.3.17)计算之,那我们同时计算了处于介质表面处的极化/磁化电荷的受力,这些受力并不能传递到导体物体上。
			
	\end{enumerate}}
	
	\begin{center}
		\includegraphics[width=0.55\linewidth]{C:/Users/16965/OneDrive/Pictures/Electrondynamics/Electrodynamics fig 2-3-2.pdf}
		\vspace{-3mm}
		\captionof{figure}{\kaishu 电磁介质内部电磁场对导体作用示意图}
	\end{center}
	
	\par{\kaishu \textcolor{blue}{注意:如果我们使用式(2.3.8)计算介质中的电磁场对束缚电荷的作用力,此时的“场”和电荷密度是一个宏观平均值,分别记作 \(\textbf{E}_{\text{mac}} = \left\langle \textbf{E}_{\text{mic}}\right\rangle\),\(\rho_{\text{mac}} = \left\langle \rho_{\text{mic}}\right\rangle\),实际力的宏观平均值为:\(\vec{f}_{\text{mac}} = \left\langle\textbf{E}_{\text{mic}}\rho_{\text{mic}}\right\rangle\),有些情况下,\(\left\langle\textbf{E}_{\text{mic}}\rho_{\text{mic}}\right\rangle\ne\left\langle \textbf{E}_{\text{mic}}\right\rangle\left\langle \rho_{\text{mic}}\right\rangle\),但是这两者的差是内部应力 \(\vec{f}_{\text{mac}}-\left\langle\textbf{E}_{\text{mic}}\right\rangle\left\langle \rho_{\text{mic}}\right\rangle\),属于物体“内力”,对一个宏观物体的总和为0。因此用式(2.3.17)计算总受力没有问题,但是计算局域力密度会出现一些状况,我们将在附录2. C、附录2. D 中讨论。}}
	
	\subsection{\kaishu 电磁场的对称与规范}
	
	\subsubsection{\kaishu 电磁场的对称性(部分)}
	
	\par{\kaishu 本节我们将在\textbf{经典电动力学}框架下讨论电磁场的对称性,并不会在此用群论的语言讨论这些对称性(但在量子场论等讲义内我们不得不这样做),具体可以列出4个有代表性的对称:
		
		\vspace{-3mm}
		
		\begin{enumerate}
			
			\item \textbf{线性:}麦克斯韦方程对于电磁场是线性的,这一点我们从线性叠加原理就可以看出(\textcolor{blue}{若我们考虑量子电动力学效应,这种线性会被破坏,但这并不影响经典电动力学线性带来的简洁})。考虑真空,存在电荷分布 \(\rho_{1}\)、电流分布 \(\vec{j}_{1}\),其产生的电场为 \(\textbf{E}_1\)、磁场为 \(\textbf{B}_{1}\);电荷分布 \(\rho_{2}\)、电流分布 \(\vec{j}_{f}\),其产生的电场为 \(\textbf{E}_{2}\)、磁场为 \(\textbf{B}_{2}\)。那么当 \(\rho_{1}\)、\(\vec{j}_{1}\) 和 \(\rho_{2}\)、\(\vec{j}_{2}\) 同时存在时,它们所产生的电磁场一定是 \(\textbf{E}_{1}+\textbf{E}_{2}\) 和 \(\textbf{B}_{1}+\textbf{B}_{2}\)。也就是说,我们只需要知道一个单位电荷以及单位电流源产生的电磁场,就可以通过线性叠加原理得到任意一个电荷电流分布产生的电磁场。\textcolor{blue}{这一特性在数学上的体现是:可以运用格林函数来求解线性方程(具体见第四章)。}
			
			\vspace{-2mm}
			
			\item \textbf{洛伦兹协变性:}麦克斯韦方程所具有的连续对称性有两个:\textbf{洛伦兹不变性},体现了电磁场在时空变换下的性质;另一个是\textbf{规范不变性}。洛伦兹不变性源于爱因斯坦的狭义相对论,狭义相对论认为:不同的惯性参考系中的物理规律的形式应当是相同的,这被称为\textbf{相对性原理};不同参考系之间的时空变换由\textbf{洛伦兹变换(Lorentz transformation)}描述。麦克斯韦方程恰好满足这个特性,它在不同惯性参考系之间的洛伦兹变换下保持形式不变。\textcolor{blue}{这个对称性非常重要,我们将利用一整章的篇幅讨论之。}
			
			\vspace{-2mm}
			
			\item \textbf{规范不变性:}麦克斯韦方程具有的另一个连续对称性就是\textbf{规范不变性}(又称规范对称性),我们将在2.4.2节中详细讨论之。
			
			\vspace{-2mm}
			
			\item \textbf{分立对称性:}除了上面我们提到的两个连续对称性,麦克斯韦方程的分立对称性也相当重要。这里我们重点讨论两种分立对称性:\textbf{空间反射对称性}和\textbf{时间反演对称性}。
			
	\end{enumerate}}
	
	\vspace{-2mm}
	
	\par{\kaishu \textbf{空间反射:}将所有的空间坐标都改变一个符号 \(\mathbf{x}\to-\mathbf{x}\) 的变换,此变换又被称为\textbf{宇称变换}。在这个变换下,电荷密度不变;电流密度矢量会改变符号(因为造成电流的微观带电粒子的速度改变了符号);梯度算符 \(\nabla\) 也会改变符号。此时,电磁场在空间反射变换下应当按照下列形式变化:}
	
	\vspace{-2mm}
	
	\begin{equation}
		\textbf{E}\to\,-\textbf{E}\qquad\&\qquad\textbf{B}\to\,\textbf{B}
	\end{equation}
	
	\vspace{3mm}
	
	\par{\kaishu 如果电磁场按照式(2.4.1)变换,麦克斯韦方程组的形式就在空间反射变换下不变。我们发现:三维空间中的矢量电场 \(\textbf{E}\) 和磁场 \(\textbf{B}\),在空间反射变换下的性质是不同的。电场 \(\textbf{E}\) 在空间反射变换下的变换性质与坐标 \(\textbf{x}\) 的变换性质是相同的,即改变一个符号;磁场 \(\textbf{B}\) 则不变。一般地,我们将具有电场或坐标这样变换性质(即在空间反射下变号) 的三维矢量称为极矢量(或简称矢量);将磁场这类在空间反射下不变的矢量称为轴矢量(\textcolor{blue}{这不是我们的讲义系列第一次提到轴矢量。经典力学中的角速度、角动量都是轴矢量})。}
	
	\par{\kaishu \textbf{时间反演:}从形式上讲,时间反演变换相当于将时间反号 \(t\to\,-t\) 的变换。在这个变换下,电流密度矢量会变号,另外对于时间的偏微分算符 \(\frac{\partial}{\partial t}\) 也会变号:\(\frac{\partial}{\partial t}\to-\frac{\partial}{\partial t}\)。如果我们考察四个麦克斯韦方程,我们发现要保证麦克斯韦方程的形式在时间反演下不变,相应的电磁场应当在时间反演时满足如下变换:}
	
	\vspace{-3mm}
	
	\begin{equation}
		\textbf{E}\to\,\textbf{E}\qquad\&\qquad\textbf{B}\to\,-\textbf{B}
	\end{equation}
	
	\vspace{2mm}
	
	\par{\kaishu 在时间反演操作下,电场和磁场也跟空间反射时类似,具有不同的变换性质。}
	
	\subsubsection{\kaishu 电磁场的势与规范对称性}
	
	\par{\kaishu 上一节我们提到:规范对称性是电磁场所具有的一个重要的对称性。它的重要性在经典电动力学中还并没有体现得非常充分,但是在现代量子场论的体系中,规范对称性可以说是处于核心的地位。在经典电动力学中,规范对称性是通过所谓的电磁势来体现的。}
	
	\par{\kaishu 此处我们先将真空中的麦克斯韦方程列出:}
	
	\begin{equation*}
		\begin{cases}
			\nabla\cdot\textbf{E} &= \frac{\rho}{\varepsilon_0}\\[10pt]
			\nabla\times\textbf{E} &= -\frac{\partial\textbf{B}}{\partial t}
		\end{cases}\qquad\&\qquad
		\begin{cases}
			\nabla\cdot\textbf{B} &= 0\\[8pt]
			\nabla\times\textbf{B} &= \mu_0\vec{j}+\mu_0\varepsilon_{0}\frac{\partial\textbf{E}}{\partial t}
		\end{cases}
	\end{equation*}
	
	\vspace{5mm}
	
	\par{\kaishu 真空中的麦克斯韦方程是关于电磁场的一阶偏微分方程组。事实上,麦克斯韦原初的方程就经常使用\textbf{电磁势} \((\Phi,\,\textbf{A})\),关于电磁势的方程是相互独立的二阶偏微分方程(\textcolor{blue}{我们将在第四章求解这些方程})。电磁势 \((\Phi,\,\textbf{A})\) 中 \(\Phi\) 被称为\textbf{标量势}(简称标势);\(\textbf{A}\) 被称为\textbf{矢量势}(简称矢势)。}
	
	\vspace{2mm}
	
	\par{\kaishu 首先考察两个齐次方程(\(\nabla\cdot\textbf{B} = 0\) 和 \(\nabla\times\textbf{E}+\frac{\partial\textbf{B}}{\partial t} = 0\))。通过磁的高斯定律 \(\nabla\cdot\textbf{B} = 0\),我们可以将磁感应强度写成某个矢量场的旋度:}
	
	\vspace{-2mm}
	
	\begin{equation}
		\textbf{B} = \nabla\times\textbf{A}
	\end{equation}
	
	\vspace{3mm}
	
	\par{\kaishu 其中 \(\textbf{A}\) 一般被称为磁矢势,此时 \(\nabla\times\textbf{E}+\frac{\partial\textbf{B}}{\partial t} = 0\) 可以被写成:}
	
	\begin{equation}
		\nabla\times\left(\textbf{E}+\frac{\partial\textbf{A}}{\partial t}\right) = 0
	\end{equation}
	
	\vspace{3mm}
	
	\par{\kaishu 由于式(2.4.4)中括号内的是一个无旋度矢量场,所有其一定能表达为一个标量场的负梯度:\(-\nabla\Phi\),于是我们得到下式:}
	
	\vspace{-3mm}
	
	\begin{equation}
		\textbf{E} = -\nabla\Phi-\frac{\partial\textbf{A}}{\partial t}
	\end{equation}
	
	\vspace{3mm}
	
	\par{\kaishu \textcolor{blue}{注意:只有在静电学中,\(\Phi\) 这个普遍意义上的标量势才会回到电势(电压)这个概念。}}
	
	\par{\kaishu 需要指出:电磁势的定义是不唯一的。即,可以存在两套不同的电磁势:\((\Phi,\,\textbf{A})\)、\((\Phi',\,\textbf{A}')\),它们对应于相同的电磁场 \(\textbf{E}\)、\(\textbf{B}\)。要实现这一点,这两套电磁势是相互有关联的。具体地说,如果我们任意选取一个性质足够好的标量场 \(\Lambda\),并且我们令:}
	
	\vspace{-1mm}
	
	\begin{equation}
		\textbf{A}' = \textbf{A}+\nabla\Lambda\qquad\&\qquad \Phi' = \Phi-\frac{\partial\Lambda}{\partial t}
	\end{equation}
	
	\vspace{3mm}
	
	\par{\kaishu 我们将式(2.4.6)代入式(2.4.3)和式(2.4.5)进行计算:}
	
	\vspace{-3mm}
	
	\begin{align*}
		\textbf{B} &= \nabla\times\textbf{A}' = \nabla\times\left(\textbf{A}+\nabla\Lambda\right) = \nabla\times\textbf{A}+\nabla\times(\nabla\Lambda) = \nabla\times\textbf{A}
	\end{align*}
	
	\vspace{3mm}
	
	\par{\kaishu 其中我们已经使用了结论 \(\nabla\times\nabla\Lambda = 0\),对于电场的情况,我们仍然有:}
	
	\vspace{-3mm}
	
	\begin{align*}
		\textbf{E} &= -\nabla\Phi'-\frac{\partial\textbf{A}'}{\partial t} = -\nabla\Phi+\nabla\left(\frac{\partial\Lambda}{\partial t}\right)-\frac{\partial}{\partial t}\left(\textbf{A}+\nabla\Lambda\right)\\[8pt]
		&= -\nabla\Phi+\nabla\left(\frac{\partial\Lambda}{\partial t}\right)-\frac{\partial\textbf{A}}{\partial t}-\frac{\partial}{\partial t}(\nabla\Lambda)\\[8pt]
		&= -\nabla\Phi-\frac{\partial\textbf{A}}{\partial t}
	\end{align*}
	
	\vspace{3mm}
	
	\par{\kaishu 其中我们也使用了结论 \(\frac{\partial}{\partial t}(\nabla\Lambda) = \nabla\left(\frac{\partial\Lambda}{\partial t}\right)\)。
		
		\vspace{2mm}
		
		我们验证了,这两套电磁势所对应的电磁场 \(\textbf{E}\) 和 \(\textbf{B}\) 是相同的。电磁场所具有的这种不变性(你也可以称其为对称性)就被称为\textbf{规范对称性}。式(2.4.6)中所描述的两套等价的电磁势之间的变换就被称为\textbf{规范变换}。}
	
	\par{\kaishu 由于电磁场存在规范对称性,而在经典电动力学过程中,所有可以直接测量的物理量都仅仅与电磁场(而不是电磁势)有关,这就意味着在经典电动力学中,电磁势本身具有一定的不确定性,它们并不是可以直接测量的物理量。\textcolor{blue}{这一点仅在纯粹经典的范畴中是正确的。如果考虑量子效应,那么电磁势可以具有可观测的物理效果,典型的例子就是所谓的 Aharonov-Bohm 效应(将在量子力学讲义登场)。}}
	
	\par{\kaishu 在经典或量子电动力学中,取决于所处理问题的不同,往往必须为电磁势 \((\Phi,\,\textbf{A})\),加上一些附加的条件从而将其形式完全确定。这类条件被通称为\textbf{规范条件},而选取一定规范条件的步骤被称为确定规范。满足一定规范条件的电磁势就被称为某种规范中的电磁势。例如,在经典电动力学中十分常用的一个规范是所谓的\textbf{洛伦兹规范(Lorenz Gauge)}(此洛伦兹不是提出洛伦兹力的那位,而是丹麦物理学家,因为翻译同名),又被称为协变规范。在洛仑兹规范中电磁势满足:}
	
	\begin{equation}
		\nabla\cdot\textbf{A}+\frac{1}{c^{2}}\frac{\partial\Phi}{\partial t} = 0
	\end{equation}
	
	\vspace{3mm}
	
	\par{\kaishu 另外一个常用的规范是所谓的\textbf{库仑规范},它又被称为辐射规范或横规范。在库仑规范中的电磁势满足的数学条件是:}
	
	\vspace{-3mm}
	
	\begin{equation}
		\nabla\cdot\textbf{A} = 0
	\end{equation}
	
	\vspace{2mm}
	
	\par{\kaishu 电磁势的引入使得真空中麦克斯韦方程组中的两个齐次方程自动得到满足。如果将电场的表达式(2.4.5)代入到麦克斯韦方程中的第一个方程(电的高斯定律),我们得到:}
	
	\begin{equation}
		\nabla^{2}\Phi+\frac{\partial}{\partial t}(\nabla\cdot\textbf{A}) = -\frac{\rho}{\varepsilon_{0}}
	\end{equation}
	
	\vspace{1mm}
	
	\par{\kaishu 类似的,将 \(\textbf{B} = \nabla\times\textbf{A}\) 代入到麦克斯韦方程组的第二个方程中,我们有(\(c^{2} = \frac{1}{\varepsilon_{0}\mu_0}\)):}
	
	\begin{equation}
		\nabla^{2}\textbf{A}-\frac{1}{c^{2}}\frac{\partial^{2}\textbf{A}}{\partial t^{2}}-\nabla\left(\nabla\cdot\textbf{A}+\frac{1}{c^{2}}\frac{\partial\Phi}{\partial t}\right) = -\mu_0\vec{j}
	\end{equation}
	
	\vspace{3mm}
	
	\par{\kaishu 现在,我们取洛伦兹规范,也就是要求电磁势满足式(2.4.7),将此式代入式(2.4.9)和式(2.4.10)之中,得到两个有源的波动方程:}
	
	\vspace{-3mm}
	
	\begin{equation}
		\begin{cases}
			\nabla^{2}\Phi-\frac{1}{c^{2}}\frac{\partial^{2}\Phi}{\partial t^{2}} = -\frac{\rho}{\varepsilon_{0}}\\[10pt]
			\nabla^{2}\textbf{A}-\frac{1}{c^{2}}\frac{\partial^{2}\textbf{A}}{\partial t^{2}} = -\mu_0\vec{j}
		\end{cases}
	\end{equation}
	
	\vspace{3mm}
	
	\par{\kaishu 因此,在洛伦兹规范下真空中的麦克斯韦方程组化为相互独立的标量势和磁矢势的波动方程,它预示着电磁势(从而电磁场) 具有波动形式的解。理所当然的,这种波被称为电磁波。根据波动方程的形式,电磁波在真空中的波速为 \(c\),我们就得到了光速,也将在经典范畴统一电、磁、光。}
	
	\subsection*{\kaishu 附录2.A 光子的自旋}
	
	\addcontentsline{toc}{subsection}{\text{\kaishu 附录 2.A\quad 光子的自旋}}
	
	\par{\kaishu 本节会涉及一点点量子力学的知识,我们会在其中指出,读者不必担心。本节推导的是比较经典的光子自旋内容,在量子场论中,我们将见到现代观点的粒子自旋。}
	
	\par{\kaishu 我们从角动量开始,真空中的电磁角动量为:}
	
	\vspace{-2mm}
	
	\begin{equation*}
		\textbf{L}_{\text{em}} = \varepsilon_{0}\int_{\tau}\textbf{r}\times\left(\textbf{E}\times\textbf{B}\right)\,\mathrm{d}\tau
	\end{equation*}
	
	\vspace{3mm}
	
	\par{\kaishu 利用 \(\textbf{B} = \nabla\times\textbf{A}\),将上式改写成:}
	
	\vspace{-2mm}
	
	\begin{equation*}
		\textbf{L}_{\text{em}} = \varepsilon_{0}\int_{\tau}\textbf{r}\times\left[\textbf{E}\times\left(\nabla\times\textbf{A}\right)\right]\,\mathrm{d}\tau
	\end{equation*}
	
	\vspace{3mm}
	
	\par{\kaishu 利用矢量公式:\(\textbf{E}\times\left(\nabla\times\textbf{A}\right) = \nabla\textbf{A}\cdot\textbf{E}-\left(\textbf{E}\cdot\nabla\right)\textbf{A}\),上式写成:}
	
	\vspace{-4mm}
	
	\begin{align*}
		\textbf{L}_{\text{em}} &= \varepsilon_{0}\int_{\tau}\textbf{r}\times\nabla\textbf{A}\cdot\textbf{E}\,\mathrm{d}\tau-\varepsilon_{0}\int_{\tau}\textbf{r}\times\left(\textbf{E}\cdot\nabla\right)\textbf{A}\,\mathrm{d}\tau\\[5pt]
		&= \varepsilon_{0}\int_{\tau}E_{j}\left(\textbf{r}\times\nabla\right)A_{j}\,\mathrm{d}\tau - \varepsilon_{0}\int_{\tau}E_{a}\,\textbf{r}\times\frac{\partial}{\partial x_{a}}\textbf{A}\,\mathrm{d}\tau
	\end{align*}
	
	\vspace{1mm}
	
	\par{\kaishu 利用关系:\(\left(\textbf{r}\times\frac{\partial}{\partial t}\textbf{A}\right)_{i} = \epsilon_{ijk}\, x_{j}\frac{\partial}{\partial x_{a}}A_{k}\),其中 \(\epsilon_{ijk}\) 是三阶反对称张量,上式第二项化简为:}
	
	\vspace{-2mm}
	
	\begin{gather*}
		E_{a}\left(\textbf{r}\times\frac{\partial}{\partial t}\textbf{A}\right)_{i} = E_{a}\epsilon_{ijk}\, x_{j}\frac{\partial}{\partial x_{a}}A_{k}\\[8pt]
		\Rightarrow\quad \epsilon_{ijk}\frac{\partial}{\partial x_{a}}\left(E_{a}x_{j}A_{k}\right) - \epsilon_{ijk}\, x_{j}A_{k}\frac{\partial}{\partial x_{a}}E_{a}-E_{a}\epsilon_{ijk}A_{k}\delta_{ja}
	\end{gather*}
	
	\vspace{2mm}
	
	\par{\kaishu 如果我们讨论无源区域,其中只有电磁波,即 \(\vec{j} = 0\)、\(\rho = 0\),有:\(\frac{\partial E_{a}}{\partial x_{a}} = \nabla\cdot\textbf{E} = 0\),那么:}
	
	\begin{equation*}
		E_{a}\left(\textbf{r}\times\frac{\partial}{\partial t}\textbf{A}\right)_{i} = \frac{\partial}{\partial x_{a}}\left(E_{a}\epsilon_{ijk}\, x_{j}A_{k}\right)-\epsilon_{ijk}E_{j}A_{k}
	\end{equation*}
	
	\vspace{3mm}
	
	\par{\kaishu 此时体系的总角动量可以写成:}
	
	\begin{equation*}
		\textbf{L}_{\text{em}} = \varepsilon_{0}\int_{\tau}E_{j}\left(\textbf{r}\times\nabla\right)A_{j}\,\mathrm{d}\tau-\varepsilon_{0}\int_{\tau}\nabla\cdot\left[\textbf{E}\left(\textbf{r}\times\textbf{A}\right)\right]\,\mathrm{d}\tau+\varepsilon_{0}\int_{\tau}\textbf{E}\times\textbf{A}\,\mathrm{d}\tau
	\end{equation*}
	
	\vspace{3mm}
	
	\par{\kaishu 我们所讨论的体系内的电磁波,是在这之前(指我们研究的时刻之前)有限的时间间隔内产生的一列波,这一列在有限时间内产生的电磁波只能局限在有限的空间区域内,这就是我们考察的区域,记为 \(\tau\)。我们\textbf{假设}此区域外电磁场为0。将上式的右边第二项化为面积分,我们有:}
	
	\begin{equation*}
		\int_{\tau}\nabla\cdot\left[\textbf{E}\left(\textbf{r}\times\textbf{A}\right)\right]\,\mathrm{d}\tau = \oint_{S}\mathrm{d}\textbf{S}\cdot\textbf{E}\left(\textbf{r}\times\textbf{A}\right) = 0
	\end{equation*}
	
	\vspace{4mm}
	
	\par{\kaishu \textcolor{blue}{我们在上面的过程中只是用了电磁波的概念,对电磁波深入的探讨在第七章。}总角动量为:}
	
	\vspace{3mm}
	
	\begin{equation*}
		\textbf{L}_{\text{em}} = \varepsilon_{0}\int_{\tau}E_{j}\left(\textbf{r}\times\nabla\right)A_{j}\,\mathrm{d}\tau+\varepsilon_{0}\int_{\tau}\textbf{E}\times\textbf{A}\,\mathrm{d}\tau
	\end{equation*}	
	
	\vspace{3mm}
	
	\par{\kaishu 此时电磁角动量被我们分成两项,根据经典物理和量子力学的对应关系:将 \(\textbf{r}\times\nabla\) 算符化可以得到量子力学中的角动量算符:\(\hat{L} = -i \, \textbf{r}\times\nabla\),基于此,我们将上式右边第一项看作\textbf{轨道角动量}。再根据量子力学知识,粒子的总角动量包括轨道角动量和自旋角动量,因此第二项 \(\varepsilon_{0}\int_{\tau}\textbf{E}\times\textbf{A}\,\mathrm{d}\tau\) 可以被看作\textbf{自旋角动量}。}
	
	\par{\kaishu \textcolor{blue}{此处提到的两种角动量的来源需要等到量子力学部分才能清晰地描述。}我们再运用能量量子概念(即能量是分立的),令:}
	
	\vspace{-2mm}
	
	\begin{equation*}
		\hat{S} = \varepsilon_{0}\int_{\tau}\textbf{E}\times\textbf{A}\,\mathrm{d}\tau
	\end{equation*}
	
	\vspace{3mm}
	
	\par{\kaishu 因为 \(\textbf{B} = \nabla\times\textbf{A}\),\(\nabla\times\textbf{E} = -\frac{\partial\textbf{B}}{\partial t}\),对于真空中的电磁波(详细过程在第七章给出),有:}
	
	\vspace{2mm}
	
	\begin{equation*}
		\left|\textbf{E}\right| = \left|\frac{\partial\textbf{A}}{\partial t}\right|\qquad\left|\textbf{E}\right| = c\left|\textbf{B}\right|
	\end{equation*}
	
	\vspace{3mm}
	
	\par{\kaishu 为方便推导,我们假定:电磁波是以正弦波传播的,因此我们有:}
	
	\vspace{-1mm}
	
	\begin{equation*}
		\left|\textbf{E}\right| = \omega\left|\textbf{A}\right|\qquad\text{\kaishu 波动圆频率为}\,\omega
	\end{equation*}
	
	\vspace{3mm}
	
	\par{\kaishu 如果是圆偏振(见第七章)情况下,\(\textbf{E}\) 和 \(\textbf{A}\) 垂直,所以:}
	
	\begin{equation*}
		\left|\hat{S}\right| = \varepsilon_{0}\int_{\tau}\,\frac{1}{\omega}\, E^{2}\,\mathrm{d}\tau = \frac{1}{\omega}\int_{\tau}\varepsilon_{0}E^{2}\,\mathrm{d}\tau
	\end{equation*}
	
	\vspace{3mm}
	
	\par{\kaishu 再考虑电磁场能量(\(B^{2} = \mu_0\varepsilon_{0}\, E^{2}\)):}
	
	\vspace{-1mm}
	
	\begin{equation*}
		W_{\text{em}} = \frac{1}{2}\int_{\tau}\left(\varepsilon_{0}E^{2}+\frac{1}{\mu_0}B^{2}\right)\,\mathrm{d}\tau = \int_{\tau}\varepsilon_{0}E^{2}\,\mathrm{d}\tau
	\end{equation*}
	
	\vspace{3mm}
	
	\par{\kaishu 现在我们得到了两者之间的关系:}
	
	\vspace{-3mm}
	
	\begin{equation*}
		\left|\hat{S}\right| = \frac{1}{\omega}\left|W_{\text{em}}\right| = \frac{1}{2\pi\nu}\left|W_{\text{em}}\right|
	\end{equation*}
	
	\vspace{3mm}
	
	\par{\kaishu 根据量子力学,电磁波能量是量子化的,若我们将前面的一列波看作一个光子,则有:}
	
	\vspace{-1mm}
	
	\begin{equation*}
		W_{\text{em}} = h\nu = 2\pi\hbar\nu \qquad \text{\kaishu 其中:h 为普朗克常量}
	\end{equation*}
	
	\vspace{3mm}
	
	\par{\kaishu 将上式代入 \(W_{\text{em}}\) 和 \(\hat{S}\) 的关系中,我们将得到:}
	
	\begin{equation*}
		\hat{S} = \frac{1}{2\pi\nu}\cdot2\pi\hbar\nu = \hbar
	\end{equation*}
	
	\vspace{5mm}
	
	\par{\kaishu 此处自旋常用自然单位制:\(c = \hbar = 1\),因此我们推出电磁场光子自旋为1。我们使用了量子力学的部分结论,按照经典电动力学的框架推出了这个结果。\textcolor{blue}{也许你觉得此过程漏洞不少,如果你有一定的耐心,你将在后续的量子场论中基本解决它们。}}
	
	\subsection*{\kaishu 附录 2.B 介质中极化/磁化的能量}
	
	\addcontentsline{toc}{subsection}{\text{\kaishu 附录 2.B\quad 介质中极化/磁化的能量}}
	
	\par{\kaishu 我们在介质的电磁能量守恒一节中并未计算介质中的能量,实际上这并不是一个平凡的问题。}
	
	\par{\kaishu 考虑如下情形:一个具有线性介质的电磁介质物体被置于一个电场中,并且场的源是固定的。假定最初在电介质中存在一个电荷分布:\(\rho(\textbf{x})\) 产生电场 \(\textbf{E}_{0}\),电介质的介电常数为 \(\varepsilon_{0}\)(此处的 \(\varepsilon_{0}\) 不表示在真空中)初始静电能量:}
	
	\vspace{-2mm}
	
	\begin{equation*}
		W_{0} = \frac{1}{2}\int\textbf{E}_{0}\cdot\textbf{D}_{0}\,\mathrm{d}^{3}\textbf{x}\qquad\text{\kaishu 其中:}\textbf{D}_{0} = \varepsilon_{0}\,\textbf{E}_{0}
	\end{equation*}
	
	\vspace{3mm}
	
	\par{\kaishu 然后我们将一个体积为 \(V_{1}\) 的电介质物体置于场中,此时电场从 \(\textbf{E}_{0}\) 变成 \(\textbf{E}\),由于该物体加入了这个静电体系,介电常数分布记为:\(\varepsilon(\textbf{x})\)。\textcolor{blue}{为了避免数学上过于困难,我们设想 \(\varepsilon(\textbf{x})\) 是一个随位置变换的平滑函数(并非光滑)。}\(\varepsilon(\textbf{x})\) 在体积 \(V_{1}\) 内的值为 \(\varepsilon_{1}\),且在 \(V_{1}\) 外取值为 \(\varepsilon_{0}\),并在边界上连续变化。静电能量变为:}
	
	\vspace{-2mm}
	
	\begin{equation*}
		W_{1} = \frac{1}{2}\int\textbf{E}\cdot\textbf{D}\,\mathrm{d}^{3}\textbf{x}\qquad\text{\kaishu 其中:}\textbf{D} = \varepsilon(\textbf{x})\,\textbf{E}
	\end{equation*} 
	
	\vspace{3mm}
	
	\par{\kaishu 能量变化量为:}
	
	\vspace{-5mm}
	
	\begin{align*}
		\Delta W &= \frac{1}{2}\int\left(\textbf{E}\cdot\textbf{D}-\textbf{E}_{0}\cdot\textbf{D}_{0}\right)\,\mathrm{d}^{3}\textbf{x}\quad\longleftarrow\,\text{\kaishu 作恒等变换}\\[6pt]
		&=\underbrace{\frac{1}{2}\int\left(\textbf{E}\cdot\textbf{D}_{0}-\textbf{E}_{0}\cdot\textbf{D}\right)\,\mathrm{d}^{3}\textbf{x}}_{\mathcal{I}_{1}}\,+\,\underbrace{\frac{1}{2}\int\left(\textbf{E}+\textbf{E}_{0}\right)\cdot\left(\textbf{D}-\textbf{D}_{0}\right)\,\mathrm{d}^{3}\textbf{x}}_{\mathcal{I}_{2}}
	\end{align*}
	
	\vspace{3mm}
	
	\par{\kaishu 注意到:\(\nabla\times\left(\textbf{E}+\textbf{E}_{0}\right) = 0\)(对静电场取旋度),可以写成势场 \(\Phi\) 的负梯度:}
	
	\vspace{-1mm}
	
	\begin{equation*}
		\textbf{E}+\textbf{E}_{0} = -\nabla\Phi
	\end{equation*}
	
	\vspace{3mm}
	
	\par{\kaishu \(\Delta W\) 等式第二项可以写成如下形式:}
	
	\vspace{-1mm}
	
	\begin{equation*}
		\mathcal{I}_{2} = -\frac{1}{2}\int\nabla\Phi\cdot\left(\textbf{D}-\textbf{D}_{0}\right)\,\mathrm{d}^{3}\textbf{x}
	\end{equation*}
	
	\vspace{3mm}
	
	\par{\kaishu 接下来我们使用分部积分处理之:}
	
	\vspace{-4mm}
	
	\begin{align*}
		\mathcal{I}_{2} &= -\underbrace{\left.\frac{1}{2}\left[\Phi\left(\textbf{D}-\textbf{D}_{0}\right)\right]\right|_{V}}_{ = 0}+\frac{1}{2}\int\Phi\,\nabla\cdot\left(\textbf{D}-\textbf{D}_{0}\right)\,\mathrm{d}^{3}\textbf{x}\\[6pt]
		&= \frac{1}{2}\int\Phi\,\nabla\cdot\left(\textbf{D}-\textbf{D}_{0}\right)\,\mathrm{d}^{3}\textbf{x} = 0
	\end{align*}
	
	\vspace{7mm}
	
	\par{\kaishu 最后一步是因为:\(\nabla\cdot\left(\textbf{D}-\textbf{D}_{0}\right) = 0\),选取的全局区域 \(V\),我们所设定的电场“源”\(\rho(\textbf{x})\) 不会因为介质改变。现在 \(\Delta W\) 只剩下了 \(\mathcal{I}_{1}\) 部分。}
	
	\begin{equation*}
		\Delta W = \mathcal{I}_{1} = \frac{1}{2}\int\left(\textbf{E}\cdot\textbf{D}_{0}-\textbf{E}_{0}\cdot\textbf{D}\right)\,\mathrm{d}^{3}\textbf{x}
	\end{equation*}
	
	\vspace{3mm}
	
	\par{\kaishu 上面我们说积分区域是全局区域 \(V\),实际上将区域局限在 \(V_{1}\) 也是可以的,因为在 \(V_{1}\) 区域外,\(\textbf{D} = \varepsilon_{0}\,\textbf{E}_{0}\),此时积分就可以化为:}
	
	\begin{equation*}
		\Delta W = -\frac{1}{2}\int_{V_{1}}\left(\varepsilon_{1}-\varepsilon_{0}\right)\,\textbf{E}\cdot\textbf{E}_{0}\,\mathrm{d}^{3}\textbf{x}
	\end{equation*}
	
	\vspace{3mm}
	
	\par{\kaishu 如果 \(V_{1}\) 外界的介质是真空,我们再使用极化强度的定义:\(\textbf{P} = \textbf{D}-\varepsilon_{0}\textbf{E} = (\varepsilon_{1}-\varepsilon_{0})\,\textbf{E}\),有:}
	
	\begin{equation*}
		\Delta W = -\frac{1}{2}\int_{V_{1}}\textbf{P}\cdot\textbf{E}_{0}\,\mathrm{d}^{3}\textbf{x}
	\end{equation*}
	
	\vspace{3mm}
	
	\par{\kaishu 上式说明:在固定“源”的电磁场内,电介质的能量密度为:}
	
	\begin{equation*}
		u = -\frac{1}{2}\,\textbf{P}\cdot\textbf{E}_{0}
	\end{equation*}
	
	\vspace{3mm}
	
	\par{\kaishu 上式中系数 \(\frac{1}{2}\) 的出现是因为:\(u\) 实际上表示的是在外部场中可极化电介质的能量密度;并不是永久性偶极子的能量密度。}
	
	\vspace{1mm}
	
	\par{\kaishu 如果我们考虑的是静磁场中的线性磁介质,固定电流分布 \(\vec{j}(\textbf{x})\) 产生磁场 \(\textbf{B}_{0}\),磁介质的磁导率为 \(\mu_0\)。我们考虑用\textbf{磁感应强度} \(\textbf{B}\) 替换前面的电场强度,因为:\(\textbf{B} = \mu\,\textbf{H}\)。按照上面的步骤,如果我们将区域局限在放入的磁介质 \(V_{2}\) 中(假设其中的磁导率为 \(\mu_2\)),则有:}
	
	\begin{equation*}
		\Delta W' = -\frac{1}{2}\int_{V_{2}}\left(\frac{1}{\mu_2}-\frac{1}{\mu_0}\right)\,\textbf{B}\cdot\textbf{B}_{0}\,\mathrm{d}^{3}\textbf{x}
	\end{equation*}
	
	\vspace{3mm}
	
	\par{\kaishu 如果 \(V_{2}\) 外界的介质是真空,我们再使用磁化强度的定义:\(\textbf{M} = \frac{\textbf{B}}{\mu_0}-\frac{\textbf{B}}{\mu_2}\),有:}
	
	\begin{equation*}
		\Delta W' = \frac{1}{2}\int_{V_{2}}\textbf{M}\cdot\textbf{B}_{0}\,\mathrm{d}^{3}\textbf{x}
	\end{equation*}
	
	\vspace{3mm}
	
	\par{\kaishu 上式说明:在固定“源”的电磁场内,磁介质的能量密度为:}
	
	\begin{equation*}
		u' = \frac{1}{2}\,\textbf{M}\cdot\textbf{B}_{0}
	\end{equation*}
	
	\subsection*{\kaishu 附录 2.C 液态介电体内受到的电力}
	
	\addcontentsline{toc}{subsection}{\text{\kaishu 附录 2.C\quad 液态介电体内受到的电力}}
	
	\par{\kaishu 在2.3.2节中我们提到:计算局域力密度会遇到问题。实际上,计算处于任意不均匀电场内的介电体上所受的力(称为有质动力)的间题相当复杂,并且要求对液体(气体)和固体分别研究。首先,我们研究比较简单的液态介电体的情况。}
	
	\par{\kaishu \textcolor{blue}{作用在物体某有限体积上的力,可以化为作用于该体积表面上的力(见经典力学中的弹性力学)。这实际上是动量守恒的结果。}体积 \(\mathrm{d}\tau\) 内物质所受到的力是单位时间内它的动量的变化,这个动量变化必须等于相同时间内经过表面进入该体积内的动量。使用 \(-\sigma_{ij}\) 表示\textbf{动量流张量}(此处包括后面我们将一直使用张量分量形式),有:}
	
	\vspace{-1mm}
	
	\begin{equation*}
		\int_{\tau}f_{i}\,\mathrm{d}\tau = \oint_{S}\sigma_{ij}\,\mathrm{d}S_{j}\qquad\text{\kaishu 表面积分量为:}\mathrm{d}S_{j}
	\end{equation*}
	
	\vspace{3mm}
	
	\par{\kaishu \(S\) 为体积 \(\tau\) 表面,\(\sigma_{ij}\) 为\textbf{应力张量},显然有:}
	
	\vspace{-3mm}
	
	\begin{equation*}
		\sigma_{ij}\,\mathrm{d}S_{j} = \sigma_{ij}n_{j}\,\mathrm{d}S
	\end{equation*}
	
	\vspace{3mm}
	
	\par{\kaishu 上式表示作用于面积元 \(\mathrm{d}S\) 上的第 \(i\) 个分量(\(\textbf{n}\) 是表面单位法向量,指向体积外侧)。}
	
	\vspace{1mm}
	
	\par{\kaishu 类似地,作用于给定体积上的总力矩也可以化成面积分(来源于应力张量的对称性 \(\sigma_{ij} = \sigma_{ji}\))。把上面张量的面积分转换为体积分,我们得到:}
	
	\vspace{-2mm}
	
	\begin{equation*}
		\int_{\tau}f_{i}\,\mathrm{d}\tau = \int_{\tau}\frac{\partial\sigma_{ij}}{\partial x_{j}}\,\mathrm{d}\tau
	\end{equation*}
	
	\vspace{3mm}
	
	\par{\kaishu 由于积分体积的任意性,由此得到:\(f_{i} = \frac{\partial\sigma_{ij}}{\partial x_{j}}\),我们可以用应力张量的导数表示体积力。}
	
	\vspace{1mm}
	
	\par{\kaishu 下面可以计算应力张量。假设表面上任一小区域均可以视为平面,而这一小区域附近的物体性质及电场也可以看成是均匀的。为了简化问题,我们研究处于均匀电场中的均匀(指的是成分、密度、温度)物质平行板(厚度 \(h\));内部电场可以设想为由加在物质层表面上的导电平面(电容器极板)所产生的。}
	
	\par{\kaishu 使用经典力学中的方法:我们使其中一个“电容器极板”(不妨取为上极版)经过无穷小的平行\textbf{虚位移} \(\boldsymbol{\xi}\),\(\boldsymbol{\xi}\) 的方向任意。假定导体的电势(在导体每一点处)在位移过程中保持不变,而由位移所引起的介电体层的均匀形变是等温的。}
	
	\par{\kaishu 表面的单位面积上受到的平行平面层的作用力为 \(-\sigma_{ij}n_{j}\),在虚位移过程中这个力所做的功为 \(-\sigma_{ij}n_{j}\xi_{i}\)。另一方面,导体产生等温形变并保持电势不变的前提下,所做的功等于 \(\int\widetilde{F}\,\mathrm{d}\tau\);或者等于物质层单位表面积上 \(h\widetilde{F}\) 的减小,有:}
	
	\vspace{-2mm}
	
	\begin{equation*}
		\sigma_{ij}\xi_{i}n_{j} = \delta\left(h\widetilde{F}\right) = h\,\delta\widetilde{F}+\widetilde{F}\,\delta h
	\end{equation*}
	
	\vspace{3mm}
	
	\par{\kaishu 液体的热力学量只与它的密度有关(在给定温度和电场强度下);不改变密度的形变(纯剪切)不影响热力学状态,因此我们将液体内的等温变分 \(\delta\widetilde{F}\) 写为:}
	
	\vspace{2mm}
	
	\begin{equation*}
		\delta\widetilde{F} = \left(\frac{\partial\widetilde{F}}{\partial\textbf{E}}\right)_{T,\,\rho}\cdot\delta\,\textbf{E}+\left(\frac{\partial\widetilde{F}}{\partial\rho}\right)_{T,\,\textbf{E}}\delta\rho = -\textbf{D}\cdot\delta\,\textbf{E}+\left(\frac{\partial\widetilde{F}}{\partial\rho}\right)_{T,\,\textbf{E}}\delta\rho
	\end{equation*}
	
	\vspace{3mm}
	
	\par{\kaishu 物质层的密度变化与其厚度变化关系式为:\(\delta\rho = -\rho\frac{\delta h}{h}\),下面我们计算场的变分 \(\delta\,\textbf{E}\)。}
	
	\vspace{1mm}
	
	\par{\kaishu 假设在位移时,质点从 \(\textbf{r}-\textbf{u}\) 位置移动到给定点处 \(\textbf{r}\),此处 \(\textbf{u}\) 是物质层体积内质点的位移矢量(和 \(\boldsymbol{\xi}\) 量级相同),因为在我们所研究的条件下(均匀形变和电容极板上的电势为常量),每一物质质点都带有自己的场势值一起位移,因此,在空间一给定点处的场势改变为:}
	
	\vspace{-2mm}
	
	\begin{equation*}
		\delta \varphi = \varphi(\textbf{r}-\textbf{u})-\varphi(\textbf{r}) = -\textbf{u}\cdot\nabla\varphi = \textbf{u}\cdot\textbf{E}
	\end{equation*}
	
	\vspace{3mm}
	
	\par{\kaishu 其中:\(\textbf{E}\) 为未形变层内的均匀电场,由于我们假设形变是均匀的,有:}
	
	\vspace{-2mm}
	
	\begin{equation*}
		\textbf{u} = \frac{z}{h}\,\boldsymbol{\xi}
	\end{equation*}
	
	\vspace{3mm}
	
	\par{\kaishu 式中 \(z\) 是距下表面的距离,我们用单位方向矢量 \(\textbf{n}\) 替换之,就得到了场的变分:}
	
	\begin{equation*}
		\delta\,\textbf{E} = -\frac{1}{h}\left(\textbf{E}\cdot\boldsymbol{\xi}\right)\,\textbf{n}
	\end{equation*}
	
	\vspace{3mm}
	
	\par{\kaishu 将上面的结果代入 \(\delta\widetilde{F}\) 中,且我们有:\(\delta h = \xi_{z} = \boldsymbol{\xi}\cdot\textbf{n}\),将 \(\sigma_{ij}\xi_{i}n_{j}\) 化为:}
	
	\vspace{-3mm}
	
	\begin{align*}
		\sigma_{ij}\xi_{i}n_{j} &= \left(\textbf{n}\cdot\textbf{D}\right)\left(\boldsymbol{\xi}\cdot\textbf{E}\right)-\left(\boldsymbol{\xi}\cdot\textbf{n}\right)\rho\frac{\partial\widetilde{F}}{\partial\rho}+\left(\boldsymbol{\xi}\cdot\textbf{n}\right)\widetilde{F}\\[6pt]
		&= \left(E_{i}D_{j}-\rho\frac{\partial\widetilde{F}}{\partial\rho}\delta_{ij}+\widetilde{F}\delta_{ij}\right)\xi_{i}n_{j}
	\end{align*}
	
	\vspace{2mm}
	
	\par{\kaishu 不难注意到应力张量的表达式:}
	
	\begin{equation*}
		\sigma_{ij} = \left[\widetilde{F}-\rho\left(\frac{\partial\widetilde{F}}{\partial\rho}\right)_{\textbf{E},\, T}\right]\delta_{ij}+E_{i}D_{j}
	\end{equation*}
	
	\vspace{3mm}
	
	\par{\kaishu 在各向同性介质中,即本节研究的情况,\(\textbf{E}\) 和 \(\textbf{D}\) 的方向重合。因此,\(E_{i}D_{j} = D_{i}E_{j}\),则 \(\sigma_{ij}\) 显然是对称张量。如果我们再假设液体介质是线性的(\(\textbf{D} = \varepsilon\,\textbf{E}\)),有:}
	
	\begin{equation*}
		\widetilde{F} = F_{0}(\rho,T)-\frac{1}{2}\varepsilon E^{2}
	\end{equation*}
	
	\vspace{3mm}
	
	\par{\kaishu 其中:\(F_{0}\) 是电场不存在时物质单位体积的亥姆霍兹自由能。上式是通过在热力学关系式中引入电场做到的(\textcolor{blue}{详细见热力学与统计物理的讲义})。根据热力学关系式,单位质量物质的自由能对比体积的导数等于压强:}
	
	\vspace{-2mm}
	
	\begin{equation*}
		\left[\frac{\partial}{\partial\left(\frac{1}{\rho}\right)}\frac{F_{0}}{\rho}\right]_{T} = F_0-\rho\left(\frac{\partial F_0}{\partial\rho}\right)_{T} = -P_{0}
	\end{equation*}
	
	\vspace{5mm}
	
	\par{\kaishu \(P_{0} = P_{0}(\rho,T)\) 是电场不存在时 \(\rho\) 和 \(T\) 保持定值的介质内的压强。将此式代入应力张量得到:}
	
	\begin{equation*}
		\sigma_{ij} = -P_{0}(\rho,T)\,\delta_{ij}-\frac{1}{2}E^{2}\left[\varepsilon-\rho\left(\frac{\partial\varepsilon}{\partial\rho}\right)_{T}\right]\delta_{ij}+\varepsilon E_{i}E_{j}
	\end{equation*}
	
	\vspace{7mm}
	
	\par{\kaishu 真空内,上式退化为电场的麦克斯韦应力张量(因为定义差一个负号,所以此处恰好相反)。}
	
	\vspace{1mm}
	
	\par{\kaishu 作用于两种相互接触的不同介质分界面上的力,必须大小相等、方向相反,即:\(\sigma_{ij}n_{j} = -\sigma'_{ij}n'_{j}\)(\(\sigma\) 和 \(\sigma'\) 代表了两种介质,法向量 \(\textbf{n}\) 和 \(\textbf{n}'\) 方向相反),也可以写成 \(\sigma_{ij}n_{j} = \sigma'_{ij}n_{j}\)。}
	
	\vspace{1mm}
	
	\par{\kaishu 在两种各向同性介质的交界面上,\textbf{力的切向分量相等}的条件恒满足。将应力张量代入上面的边界条件,取出切向分量,我们得到:}
	
	\vspace{-2mm}
	
	\begin{equation*}
		E_{t}\, D_{n} = E'_{t}\, D'_{n}\qquad t \,\text{\kaishu 表示切向,}n \,\text{\kaishu 表示法向}
	\end{equation*}
	
	\vspace{3mm}
	
	\par{\kaishu 但是这是一个已经被满足的边界条件(麦克斯韦方程组对应的边界条件)由力的法向分量相等的条件,可以得到施加在两种介质内的非平庸边界条件。}
	
	\par{\kaishu 我们选取液体与大气的交界面研究,大气中 \(\varepsilon = \varepsilon_{0}\)。用带撇号的量表示大气中的物理量,并用应力张量表达式,我们有(已经利用了上面的边界条件):}
	
	\begin{equation*}
		-P_0(\rho,T)+\frac{E^{2}}{2}\rho\left(\frac{\partial\varepsilon}{\partial\rho}\right)_{T}+\frac{1}{2}\varepsilon\left(E_{t}^{2}-E_{n}^{2}\right) = -\underbrace{P_{\text{atm}}}_{\text{\kaishu 大气压强}}+\frac{1}{2}\varepsilon_{0}\left(\left.E'_{t}\right.^{2}-\left.E'_{n}\right.^{2}\right)
	\end{equation*}
	
	\vspace{3mm}
	
	\par{\kaishu 上式我们研究的是介质交界面法向的力作用。考虑到边界条件:\(E_{t} = E'_{t}\),\(D_{n} = \varepsilon E_{n} = D'_{n} = \varepsilon_{0}E'_{n}\),表面压强差化为:}
	
	\begin{equation*}
		P_0(\rho,T)-P_{\text{atm}} = \frac{1}{2}\,\rho E^{2}\left(\frac{\partial\varepsilon}{\partial\rho}\right)_{T}-\frac{1}{2}(\varepsilon-\varepsilon_{0})\cdot\left(\frac{\varepsilon}{\varepsilon_{0}}E_{n}^{2}+E_{t}^{2}\right)
	\end{equation*}
	
	\vspace{3mm}
	
	\par{\kaishu 这个关系式可以被理解为:由液体内的电场强度确定液面附近液体密度 \(\rho\) 的方程式。}
	
	\vspace{1mm}
	
	\par{\kaishu 现在求作用在介质内的体积力,利用 \(f_{i} = \frac{\partial\sigma_{ij}}{\partial x_{j}}\),计算得到:}
	
	\vspace{2mm}
	
	\begin{equation*}
		f_{i} = \frac{\partial}{\partial x_{i}}\left[-P_{0}+\frac{1}{2}\,\rho E^{2}\left(\frac{\partial\varepsilon}{\partial\rho}\right)_{T}\right]-\frac{1}{2}\, E^{2}\frac{\partial\varepsilon}{\partial x_{i}}-\frac{1}{2}\,\varepsilon\frac{\partial}{\partial x_{i}}E^{2}+\frac{\partial}{\partial x_{j}}\left(E_{i}D_{j}\right)
	\end{equation*}
	
	\vspace{4mm}
	
	\par{\kaishu 再考虑无源麦克斯韦方程组:\(\nabla\cdot\textbf{D} = \frac{\partial D_{k}}{\partial x_{k}} = 0\),上式最后两项化简为:}
	
	\vspace{2mm}
	
	\begin{equation*}
		-\frac{1}{2}\,\varepsilon\frac{\partial}{\partial x_{i}}E^{2}+\frac{\partial}{\partial x_{j}}\left(E_{i}D_{j}\right) = -\varepsilon E_{j}\frac{\partial E_{j}}{\partial x_{i}}+D_{j}\frac{\partial E_{i}}{\partial x_{j}} = -D_{j}\left(\frac{\partial E_{j}}{\partial x_{i}}-\frac{\partial E_{i}}{\partial x_{j}}\right)
	\end{equation*}
	
	\vspace{4mm}
	
	\par{\kaishu 最后一项其实是:\(\nabla\times\textbf{E}\),不考虑磁场其值为0,将 \(f_{i}\) 用微分算符写成更紧凑的形式:}
	
	\begin{equation*}
		\vec{f} = -\nabla P_{0}(\rho,T)+\frac{1}{2}\,\nabla\left[E^{2}\rho\left(\frac{\partial\varepsilon}{\partial\rho}\right)_{T}\right]-\frac{1}{2}\, E^{2}\nabla\varepsilon
	\end{equation*}
	
	\vspace{3mm}
	
	\par{\kaishu 如果介电体内有体密度为 \(\rho_{\text{e}}\) 的外电荷,\(\nabla\cdot\textbf{D} = \rho_{\text{e}}\),上式添加一项:\(\rho_{\text{e}}\textbf{E}\)}
	
	\vspace{1mm}
	
	\begin{equation*}
		\vec{f} = -\nabla P_{0}(\rho,T)+\frac{1}{2}\,\nabla\left[E^{2}\rho\left(\frac{\partial\varepsilon}{\partial\rho}\right)_{T}\right]-\frac{1}{2}\, E^{2}\nabla\varepsilon+\rho_{\text{e}}\textbf{E}
	\end{equation*}
	
	\vspace{3mm}
	
	\par{\kaishu 在气体中,可以认为 \(\varepsilon-\varepsilon_0\) 与密度 \(\rho\) 成正比,\(\rho\frac{\partial\varepsilon}{\partial\rho} = \varepsilon-\varepsilon_{0}\),此时 \(\nabla\varepsilon = 0\),力密度为:}
	
	\begin{equation*}
		\vec{f} = -\nabla P_{0}+\frac{1}{2}\left(\varepsilon-\varepsilon_{0}\right)\nabla E^{2}
	\end{equation*}
	
	\vspace{3mm}
	
	\par{\kaishu 如果介质组分是非均匀的,\(\varepsilon\) 不仅是 \(\rho\) 和 \(T\) 的函数,而且也是沿介质变化的混合物浓度的函数,假定组分均匀,则 \(\varepsilon\) 可以被展开成:}
	
	\begin{equation*}
		\nabla\varepsilon = \left(\frac{\partial\varepsilon}{\partial T}\right)_{\rho}\nabla T+\left(\frac{\partial\varepsilon}{\partial\rho}\right)_{T}\nabla\rho
	\end{equation*}
	
	\vspace{3mm}
	
	\par{\kaishu 代入力密度中,我们将得到:}
	
	\begin{equation*}
		\vec{f} = -\nabla P_{0}+\frac{1}{2}\,\rho\,\nabla\left[E^{2}\left(\frac{\partial\varepsilon}{\partial\rho}\right)_{T}\right]-\frac{1}{2}E^{2}\left(\frac{\partial\varepsilon}{\partial T}\right)_{\rho}\nabla T
	\end{equation*}
	
	\vspace{3mm}
	
	\par{\kaishu 如果物体内的温度为常量,那么第三项变为零,而在第一项内,可以用 \(\rho\nabla\zeta_{0}\) 代替 \(P_{0}\)(\textcolor{blue}{因为无电场时的化学势的热力学关系:\(\rho\,\mathrm{d}\zeta_{0} = \mathrm{d}P_{0}-S_{0}\mathrm{d}T\),\(S_{0}\) 为熵}),上式写成:}
	
	\begin{equation*}
		\vec{f} = -\rho\,\nabla\left[\zeta_{0}-\frac{1}{2}E^{2}\left(\frac{\partial\varepsilon}{\partial\rho}\right)_{T}\right] = -\rho\nabla\zeta
	\end{equation*}
	
	\vspace{3mm}
	
	\par{\kaishu 其中:\(\zeta\) 为电场内物质的化学势。特别地,温度为常量时,力学平衡条件 \(\vec{f} = 0\) 为:}
	
	\begin{equation*}
		\zeta = \zeta_{0}-\frac{1}{2}E^{2}\left(\frac{\partial\varepsilon}{\partial\rho}\right)_{T} = \text{const}
	\end{equation*}
	
	\vspace{3mm}
	
	\par{\kaishu 这与普遍的热力学平衡条件一致,这个条件通常可写为更简单的形式。在电场作用下,介质密度的变化与 \(E^{2}\) 成正比。\textcolor{blue}{如果电场为0的条件下,介质密度是均匀的;那么即使存在电场,\(\rho = \text{const}\);因为我们本来假设的线性关系 \(\textbf{D} = \varepsilon\textbf{E}\) 不足以求出 \(\rho\) 的精确分布。}平衡条件还可以写成:}
	
	\begin{equation*}
		P_{0}(\rho,T)-\frac{1}{2}\rho E^{2}\left(\frac{\partial\varepsilon}{\partial\rho}\right)_{T} = \text{const}
	\end{equation*}
	
	\vspace{3mm}
	
	\par{\kaishu 这两个表达式的区别就是用 \(\frac{P_{0}}{\rho}\) 代替了化学式 \(\zeta_{0}\)。我们已经较为详细的讨论了各向同性液态介电体受到的电力。}
	
	\subsection*{\kaishu 附录 2.D 固体受到的电力}
	
	\addcontentsline{toc}{subsection}{\text{\kaishu 附录 2.D\quad 固体受到的电力}}
	
	\par{\kaishu 固体的介电性质相较于液体更加复杂。介电性质不但随其密度改变而改变,还受到不改变密度的形变(即剪切)的影响。首先我们研究在无电场时各向同性的固体,物体的形变会破坏其各向同性,导致其介电性质也变成各向异性了,我们用\textbf{介电张量}描述其介电性质。}
	
	\vspace{2mm}
	
	\par{\kaishu 弱形变物体的状态用应变张量描述:\(u_{ij} = \frac{1}{2}\left(\frac{\partial u_i}{\partial x_{j}}+\frac{\partial u_j}{\partial x_{i}}\right)\),式中 \(\textbf{u}(x,y,z)\) 为物体上各个点的位移矢量。由千这些量很小,因此在分量 \(\varepsilon_{ij}\) 的变化中只要取 \(u_{ij}\) 的一次项就够了。}
	
	\par{\kaishu 我们可以把形变物体的介电张量表示成以下形式:}
	
	\vspace{-2mm}
	
	\begin{equation*}
		\varepsilon_{ij} = \varepsilon\delta_{ij}+a_{1}u_{ij}+a_{2}u_{kk}\delta_{ij}
	\end{equation*}
	
	\vspace{2mm}
	
	\par{\kaishu 其中:\(\varepsilon_{0}\) 为未形变物体的介电张量,后面两项含有两个标量函数 \(a_{1}\) 和 \(a_{2}\),代表最普遍形式的二秩张量。\textcolor{blue}{这种张量可以由张量 \(u_{ij}\) 的分量以线性方式构成。}}
	
	\par{\kaishu 由于已经推导了液态介电体的受力,我们只需作出修正即可,无需再次推出全部过程。固体内的亥姆霍兹自由能 \(\widetilde{F}\) 与应变张量的分量有关,变分 \(\delta\widetilde{F}\) 改写成:}
	
	\begin{equation*}
		\delta\widetilde{F} = -\textbf{D}\cdot\delta\,\textbf{E}+\frac{\partial\widetilde{F}}{\partial u_{ij}}\delta u_{ij}
	\end{equation*}
	
	\vspace{3mm}
	
	\par{\kaishu 在上一节我们提到:虚位移 \(\boldsymbol{\xi}\) 对应的矢量 \(\textbf{u} = \frac{z}{h}\,\boldsymbol{\xi}\),应变张量为:}
	
	\begin{equation*}
		u_{ij} = \frac{1}{2h}\left(\xi_{i}n_{j}+\xi_{j}n_{i}\right)
	\end{equation*}
	
	\vspace{3mm}
	
	\par{\kaishu 将上式代入 \(\delta\widetilde{F}\) 中,考虑张量的对称性,即 \(\frac{\partial\widetilde{F}}{\partial u_{ij}}\) 的对称性,有:}
	
	\begin{equation*}
		\delta\widetilde{F} = -\textbf{D}\cdot\delta\,\textbf{E}+\frac{1}{h}\frac{\partial\widetilde{F}}{\partial u_{ij}}\xi_{i}n_{j}
	\end{equation*}
	
	\vspace{3mm}
	
	\par{\kaishu 此时,固体中的应力张量表达为:}
	
	\vspace{-2mm}
	
	\begin{equation*}
		\sigma_{ij} = \widetilde{F}\delta_{ij}+\left(\frac{\partial\widetilde{F}}{\partial u_{ij}}\right)_{T,\textbf{E}}+E_{i}D_{j}
	\end{equation*}
	
	\vspace{3mm}
	
	\par{\kaishu 无论 \(\textbf{D}\) 对 \(\textbf{E}\) 的关系如何,上式均是适用的。对于非热释电物体和非压电物体,其中 \(D_{i} = \varepsilon_{ij}E_{j}\)(此时 \(\widetilde{F} = F_{0}-\frac{1}{2}\varepsilon_{ij}E_{i}E_{j}\)),于是得到我们所求的导数:}
	
	\begin{equation*}
		\frac{\partial\widetilde{F}}{\partial u_{ij}} = \frac{\partial F_{0}}{\partial u_{ij}}-\frac{1}{2}\left(a_{1}E_{i}E_{j}+a_{2}E^{2}\delta_{ij}\right)
	\end{equation*}
	
	\vspace{3mm}
	
	\par{\kaishu 然后我们在应力张量中令:\(\varepsilon_{ij} = \varepsilon_{0}\,\delta_{ij}\),求得以下应力张量:}
	
	\begin{equation*}
		\sigma_{ij} = \sigma_{ij}^{(0)}+\frac{1}{2}\left(2\varepsilon_{0}-a_{1}\right)E_{i}E_{j}-\frac{1}{2}\left(\varepsilon_{0}+a_{2}\right)E^{2}\delta_{ij}
	\end{equation*}
	
	\vspace{3mm}
	
	\par{\kaishu \(\sigma_{ij}^{(0)}\) 是没有电场时的应力张量,按照弹性理论的通常公式,它由应变张量以及切变模量和压缩模量决定。下面我们计算各向异性固体(此处不讨论压电晶体)。}
	
	\par{\kaishu 当物质层发生虚形变时,它的晶轴也发生转动,因而晶轴相对于外电场的取向也发生改变。由于晶体的介电性质是各向异性的,我们需要考虑 \(\widetilde{F}\) 的附加变化。在计算这一变化时,假定晶轴相对于电场 \(\textbf{E}\) 旋转了角度 \(\boldsymbol{\delta\varphi}\)(如果我们反向旋转晶轴 \(-\boldsymbol{\delta\varphi}\),结论是一样的),或者假设电场相对晶轴旋转 \(-\boldsymbol{\delta\varphi}\)。这说明:在上一节提到的电场的变分 \(\delta\textbf{E}\),必须添加转动角度为 \(-\boldsymbol{\delta\varphi}\) 时电场的变化。因此我们有:}
	
	\begin{equation*}
		\delta\textbf{E} = -\frac{1}{h}\,\textbf{n}\left(\textbf{E}\cdot\boldsymbol{\xi}\right)-\boldsymbol{\delta\varphi}\times\textbf{E}
	\end{equation*}
	
	\vspace{3mm}
	
	\par{\kaishu 角 \(\boldsymbol{\delta\varphi}\) 和形变时的位移矢量 \(\textbf{u}\) 的关系为:\(\boldsymbol{\delta\varphi} = \frac{1}{2}\,\nabla\times\textbf{u}\)(因为物体转动角度 \(\boldsymbol{\delta\varphi}\) 时,物体上各点的位移可以写成:\(\textbf{u} = \boldsymbol{\delta\varphi}\times\textbf{r}\),两边取旋度并移项即可得到)。代入上一节中的 \(\textbf{u}\),得到:}
	
	\begin{equation*}
		\boldsymbol{\delta\varphi} = \frac{1}{2h}(\nabla z)\times\boldsymbol{\xi} = \frac{1}{2h}\textbf{n}\times\boldsymbol{\xi}
	\end{equation*}
	
	\vspace{3mm}
	
	\par{\kaishu 代入 \(\delta\textbf{E}\):}
	
	\vspace{-3mm}
	
	\begin{equation*}
		\delta\textbf{E} = -\frac{1}{h}\,\textbf{n}\left(\textbf{E}\cdot\boldsymbol{\xi}\right)+\frac{1}{2h}\,\textbf{E}\times\left(\textbf{n}\times\boldsymbol{\xi}\right) = -\frac{1}{2h}\left[\textbf{n}\left(\textbf{E}\cdot\boldsymbol{\xi}\right)+\boldsymbol{\xi}\left(\textbf{n}\cdot\textbf{E}\right)\right]
	\end{equation*}
	
	\vspace{3mm}
	
	\par{\kaishu 现在,我们可以将 \(\textbf{D}\cdot\delta\textbf{E}\) 化为:}
	
	\begin{equation*}
		-\textbf{D}\cdot\delta\textbf{E} = \frac{1}{2h}\left[\left(\textbf{n}\cdot\textbf{D}\right)\left(\boldsymbol{\xi}\cdot\textbf{E}\right)+\left(\boldsymbol{\xi}\cdot\textbf{D}\right)\left(\textbf{n}\cdot\textbf{E}\right)\right] = \frac{\xi_{i}n_{j}}{2h}\left(E_{i}D_{j}+E_{j}D_{i}\right)
	\end{equation*}
	
	\vspace{3mm}
	
	\par{\kaishu 现在,应力张量的表达式可以改写为:}
	
	\begin{equation*}
		\sigma_{ij} = \widetilde{F}\delta_{ij}+\frac{\partial\widetilde{F}}{\partial u_{ij}}+\frac{1}{2}\left(E_{i}D_{j}+E_{j}D_{i}\right)
	\end{equation*}
	
	\vspace{3mm}
	
	\par{\kaishu 我们所求得的表达式对于下标 \(i\) 和 \(j\) 自动对称,这一点毋庸置疑。}
	
	\vspace{1mm}
	
	\par{\kaishu \textcolor{blue}{注意:普遍情况下,形变晶体的介电张量可以写成:\(\varepsilon_{ij} = \varepsilon_{ij}^{(0)}+a_{ijkl}u_{kl}\)。其中:\(a_{ijkl}\) 为四阶张量,对于两组指标 \(i\)、\(j\) 和 \(k\)、\(l\)对称,但是对于两组指标的换位(如 \(a_{klij}\))不对称。这个张量的不为零的独立分量数目取决于晶体的对称性,即取决于晶体的晶类。}}
	
	\newpage
	
	\section{\kaishu 静电学 \Rmnum{1}}
	
	\pagestyle{fancy}
	\lhead{Lecture Notes}
	\chead{Electrodynamics}
	\rhead{\kaishu Chapter 3 静电学 \Rmnum{1}}
	
	\par{\kaishu 我们在第一章和第二章给出电磁场理论的框架,下面我们需要在框架中解决具体的问题。如同经典力学,静力学的难度要低于动力学;在电动力学中,静电、静磁学的难度要低于“动力学”部分。我们先详细地解决静电和静磁问题,本章将介绍静电学和其重要定理,下一章将用基于这些原理的不同方法解决静电问题;此后静磁问题可以类比推理之。}
	
	\subsection{\kaishu 静电学基本问题}
	
	\subsubsection{\kaishu 静电方程}
	
	\par{\kaishu 静电现象(electrostatics)研究的是电磁学中这样的问题:\(\frac{\partial}{\partial t}(\text{\kaishu 物理量}) = 0\) 和 \(\vec{j} = 0\),所有的物理量都不随时间改变,而且电荷静止不动。}
	
	\par{\kaishu 如果我们把静电条件代入麦克斯韦方程中,空间中不会激发电流,也没有变化电场,所以空间不会激发磁场。满足关系:}
	
	\vspace{-3mm}
	
	\begin{equation}
		\nabla\cdot\textbf{D} = \rho \qquad \nabla\times\textbf{E} = 0
	\end{equation}
	
	\vspace{3mm}
	
	\par{\kaishu 由于 \(\nabla\times\textbf{E} = 0\),我们引入标量势 \(\varphi\) 满足 \(\textbf{E} = -\nabla\varphi\)。然后根据本构关系 \(\textbf{D}(\textbf{r}) = \varepsilon(\textbf{r})\,\textbf{E}\),代入 \(\nabla\cdot\textbf{D} = \rho\) 中,我们得到:}
	
	\vspace{-3mm}
	
	\begin{equation}
		\nabla\cdot\left[\varepsilon(\textbf{r})\,\nabla\varphi(\textbf{r})\right] = -\rho(\textbf{r})
	\end{equation}
	
	\vspace{3mm}
	
	\par{\kaishu 根据式(3.1.2),我们可以描述几乎所有经典的静电现象。如果我们假设研究的介质是均匀的,有:\(\varepsilon(\textbf{r}) = \varepsilon\),那么式(3.1.2)转化成一种特殊的偏微分方程——\textbf{泊松方程}(Poisson equation)。}
	
	\begin{equation}
		\nabla^{2}\varphi(\textbf{r}) = -\frac{\rho(\textbf{r})}{\varepsilon}
	\end{equation}
	
	\vspace{3mm}
	
	\par{\kaishu 如果我们考虑的系统中没有电荷源(\(\rho = 0\)),标量势 \(\varphi\) 的方程退化成\textbf{拉普拉斯方程}(Laplace equation),我们得到:}
	
	\vspace{-3mm}
	
	\begin{equation}
		\nabla^{2}\varphi = 0
	\end{equation}
	
	\vspace{3mm}
	
	\par{\kaishu 其实我们在第一章的电势可以作为泊松方程的一个特解。此处将电势改写成:}
	
	\begin{equation}
		\varphi(\textbf{x}) = \frac{1}{4\pi\varepsilon}\int\frac{\rho(\textbf{x}')}{\left|\textbf{x}-\textbf{x}'\right|}\,\mathrm{d}^{3}\textbf{x}'
	\end{equation}
	
	\vspace{3mm}
	
	\par{\kaishu 下面我们将直接验证式(3.1.5)满足泊松方程,我们对等式两边作用微分算符 \(\nabla^{2}\)。为了避免数学上的奇异性,我们可以定义一个新的标量势 \(\varphi_a(\textbf{x})\),其满足:}
	
	\begin{equation}
		\varphi_a(\textbf{x}) = \frac{1}{4\pi\varepsilon}\int\frac{\rho(\textbf{x})}{\sqrt{\left(\textbf{x}-\textbf{x}'\right)^{2}+a^{2}}}\mathrm{d}^{3}\textbf{x}'
	\end{equation}
	
	\vspace{3mm}
	
	\par{\kaishu 现在,式(3.1.5)的 \(\varphi(\textbf{x})\) 可以看作 \(\varphi_a(\textbf{x})\) 取 \(a\to0\) 情况的极限。我们对 \(\varphi_a\) 作用微分算符 \(\nabla^{2}\):}
	
	\vspace{-5mm}
	
	\begin{align*}
		\nabla^{2}\varphi_a(\textbf{x}) &= \frac{1}{4\pi\varepsilon}\int\rho(\textbf{x}')\nabla^{2}\left(\frac{1}{\sqrt{r^{2}+a^{2}}}\right)\mathrm{d}^{3}\textbf{x}'\\[6pt]
		&= -\frac{1}{4\pi\varepsilon}\int\rho(\textbf{x}')\left[\frac{3a^{2}}{\left(r^{2}+a^{2}\right)^{\frac{5}{2}}}\right]\,\mathrm{d}^{3}\textbf{x}'
	\end{align*}
	
	\vspace{3mm}
	
	\par{\kaishu 其中 \(r = \left|\textbf{x}-\textbf{x}'\right|\)。中括号内的项,在 \(a\ne 0\) 的情况下不存在奇异性;但是 \(a\to0\) 情况下,如果令 \(r= 0\) 则会得到无穷大,\(r>0\) 完全消失。为了对上式进行积分,我们将空间划分为两个区域:以 \(\textbf{x}\) 为球心固定半径 \(R\) 的球体和剩余区域。\(R\) 的选择依据是:使得 \(\rho(\textbf{x}')\) 在球体内部的变化很小;\(a\ll R\) 并且 \(a\to0\)。如果 \(\rho(\textbf{x}')\) 满足式(3.1.5),在球体外区域 \(\nabla^{2}\varphi_a(\textbf{x})\) 和 \(a\to0\) 时的 \(a^{2}\) 类似。即:}
	
	\begin{equation}
		\nabla^2\varphi_a(\textbf{x}) = -\frac{1}{4\pi\varepsilon}\int_{V}\rho(\textbf{x}')\left[\frac{3a^{2}}{\left(r^{2}+a^{2}\right)^{\frac{5}{2}}}\right]\,\mathrm{d}^{3}\textbf{x}'+\mathcal{O}(a^{2})
	\end{equation}
	
	\vspace{3mm}
	
	\par{\kaishu 其中 \(V\) 表示球体区域。下面我们要考虑 \(\rho(\textbf{x}')\) 在 \(\textbf{x}' = \textbf{x}\) 处的泰勒展开式,并略去其中的小量。}
	
	\vspace{1mm}
	
	\begin{center}
		\begin{mdframed}[shadow = false]
			
			\par{\kaishu \textbf{补充:}式(3.1.7)泰勒展开过程,以及积分后产生的小量化简。
				
				球心为 \(\textbf{x}\),记 \(\textbf{x}' = \textbf{x}+\textbf{r}\),其中 \(\textbf{r} = r\hat{n}\)(\(\hat{n}\) 为单位方向向量,\(r = |\textbf{r}|\))。假定 \(\rho(\textbf{x})\) 为光滑标量场,在 \(r = 0\) 做泰勒展开到二阶(分量使用爱因斯坦求和约定):}
			
			\vspace{-5mm}
			
			\begin{align*}
				\rho(\textbf{x}+\textbf{r}) &= \rho(\textbf{x})+r_{i}\frac{\partial\rho}{\partial x_{i}}+\frac{1}{2}r_{i}r_{j}\frac{\partial^{2}\rho}{\partial x_{i}\partial x_{j}}+\mathcal{O}(r^{3})\\[6pt]
				&= \rho(\textbf{x})+r\hat{n}_{i}\partial_{i}\rho+\frac{1}{2}r^{2}\hat{n}_{i}\hat{n}_{j}\left(\partial_{i}\partial_{j}\rho\right)+\mathcal{O}(r^{3})
			\end{align*}
			
			\vspace{2mm}
			
			\par{\kaishu 接下来对展开项分别进行体积分,我们考虑单位球面 \(\mathbb{S}^{2}\),其方向向量 \(\hat{n}\) 满足(角积分为 \(\mathrm{d}\Omega\)):}
			
			\begin{equation*}
				\int_{\mathbb{S}^{2}}\mathrm{d}\Omega = 4\pi \qquad
				\int_{\mathbb{S}^{2}}\hat{n}_{i}\,\mathrm{d}\Omega = 0 \qquad	\int_{\mathbb{S}^{2}}\hat{n}_{i}\hat{n_{j}}\,\mathrm{d}\Omega = \frac{4\pi}{3}\delta_{ij}
			\end{equation*}
			
			\vspace{3mm}
			
			\par{\kaishu 因此在我们所考察的积分中(\(\mathrm{d}^{3}\textbf{x}' = r^{2}\mathrm{d}r\mathrm{d}\Omega\)),线性项角积分为:}
			
			\begin{equation*}
				\int_{\mathbb{S}^{2}}\hat{n}_{i}\,\mathrm{d}\Omega = 0 \quad\Rightarrow\quad \int r\hat{n}_{i}\partial_{i}\rho\,\mathrm{d}\Omega = 0
			\end{equation*}
			
			\vspace{3mm}
			
			\par{\kaishu 二阶项角积分:}
			
			\vspace{-3mm}
			
			\begin{equation*}
				\int_{\mathbb{S}^{2}}\frac{1}{2}r^{2}\hat{n}_{i}\hat{n}_{j}\left(\partial_{i}\partial_{j}\rho\right)\,\mathrm{d}\Omega = \frac{1}{2}r^{2}\partial_{i}\partial_{j}\rho\int_{\mathbb{S}^{2}}\hat{n}_{i}\hat{n_{j}}\,\mathrm{d}\Omega = \frac{4\pi}{6}r^{2}\partial_{i}\partial_{i}\rho(\textbf{x}) = \frac{4\pi}{6}r^{2}\nabla^{2}\rho
			\end{equation*}
			
			\vspace{3mm}
			
			\par{\kaishu 将结果代入式(3.1.7)中,积分转化为:}
			
			\vspace{-2mm}
			
			\begin{equation*}
				\nabla^{2}\varphi_a(\textbf{x}) = -\frac{1}{\varepsilon}\int_{0}^{R}\frac{3a^{2}}{\left(r^{2}+a^{2}\right)^{\frac{5}{2}}}\left[\rho(\textbf{x})+\frac{r^{2}}{6}\nabla^{2}\rho+\mathcal{O}(r^{3})\right]r^{2}\mathrm{d}r+\mathcal{O}(a^{2})
			\end{equation*}
			
		\end{mdframed}
	\end{center}
	
	\begin{center}
		\begin{mdframed}[shadow = false]
			
			\par{\kaishu 我们不考虑 \(\mathcal{O}(r^{3})\) 表示的项(会归类到 \(\mathcal{O}(a^{2})\) 中),积分只剩下两项,对常数项 \(\rho(\textbf{x})\),令:}
			
			\begin{equation*}
				\mathcal{I}_{1} = \int_{0}^{R}\frac{3a^2r^2}{\left(r^{2}+a^{2}\right)^{\frac{5}{2}}}\mathrm{d}r = \int_{0}^{\frac{R}{a}}\frac{3t^{2}}{\left(1+t^{2}\right)^{\frac{5}{2}}}\mathrm{d}t
			\end{equation*}
			
			\vspace{3mm}
			
			\par{\kaishu 其中:\(r = at\),根据我们的假设 \(a\to0\),因此积分化为:}
			
			\begin{equation*}
				\mathcal{I}_{1} = \int_{0}^{+\infty}\frac{3t^{2}}{\left(1+t^{2}\right)^{\frac{5}{2}}}\,\mathrm{d}t
			\end{equation*}
			
			\vspace{3mm}
			
			\par{\kaishu 此积分可以用mathematica计算,但是,注意到:\(\frac{\mathrm{d}}{\mathrm{d}t}\left[\frac{t^{3}}{\left(1+t^{2}\right)^{\frac{3}{2}}}\right] = \frac{3t^{2}}{\left(1+t^{2}\right)^{\frac{5}{2}}}\),因此:}
			
			\vspace{3mm}
			
			\begin{equation*}
				\mathcal{I}_{1} = \int_{0}^{+\infty}\frac{3t^{2}}{\left(1+t^{2}\right)^{\frac{5}{2}}}\,\mathrm{d}t = \left.\frac{t^{3}}{\left(1+t^{2}\right)^{\frac{3}{2}}}\right|_{0}^{\infty} = 1
			\end{equation*}
			
			\vspace{3mm}
			
			\par{\kaishu \textcolor{blue}{严格地,对于固定的 \(R\) 且 \(a\to0\),从 \(0\) 到 \(R\) 的积分趋向于 1。}对二次项也做相同代换 \(r = at\):}
			
			\vspace{-5mm}
			
			\begin{align*}
				\mathcal{I}_{2} &= \int_{0}^{R}\frac{a^{2}r^{4}}{2\left(r^{2}+a^{2}\right)^{\frac{5}{2}}}\,\mathrm{d}r = a^{2}\int_{0}^{\frac{R}{a}}\frac{t^{4}}{2\left(1+t^{2}\right)^{\frac{5}{2}}}\mathrm{d}t \\[8pt]
				&= \frac{a^{2}}{2}\left[-\frac{t\left(3+4t^{2}\right)}{3\left(1+t^{2}\right)^{\frac{3}{2}}}-\ln\left(\sqrt{1+t^{2}}-t\right)\right]^{\frac{R}{a}}_{0}\\[8pt]
				&= -\frac{a^{2}R\left(3a^{2}+4R^{2}\right)}{6\left(a^{2}+R^{2}\right)^{\frac{3}{2}}}-\frac{a^{2}}{2}\ln\left(\sqrt{a^{2}+R^{2}}-R\right)+\frac{a^{2}}{2}\ln a
			\end{align*}
			
			\vspace{1mm}
			
			\par{\kaishu 考虑 \(a\to0\) 极限,我们发现上式为 0(使用mathematica验证即可),更严格地说,上式对积分值的贡献是 \(\mathcal{O}(a^{2})\) 和 \(\mathcal{O}(a^{2}\ln a)\) 级别,因此将上面的所有结果代入 \(\nabla^{2}\varphi_a\) 中,得到:}
			
			\begin{equation}
				\nabla^{2}\varphi_a(\textbf{x}) = -\frac{1}{\varepsilon}\left[\rho(\textbf{x})+\mathcal{O}\left(a^{2},\,a^{2}\ln a\right)\nabla^{2}\rho\right]+\mathcal{O}(a^{2})
			\end{equation}
			
			\vspace{2mm}
			
		\end{mdframed}
	\end{center}
	
	\vspace{1mm}
	
	\par{\kaishu 我们将式(3.1.8)内的诸多小量忽略,再使用 \(\varphi_a\to\varphi\),最终得到泊松方程:}
	
	\begin{equation*}
		\nabla^{2}\varphi(\textbf{x}) = -\frac{\rho(\textbf{x})}{\varepsilon}
	\end{equation*}
	
	\vspace{3mm}
	
	\par{\kaishu 如果要验证拉普拉斯方程,我们可以使用 \(\nabla^{2}\left(\frac{1}{\left|\textbf{x}-\textbf{x}'\right|}\right) = -4\pi\delta\left(\textbf{x}-\textbf{x}'\right)\) 直接验证。}
	
	\subsubsection{\kaishu 静电边界条件}
	
	\par{\kaishu 上文我们验证了电势 \(\varphi(\textbf{x})\) 可以作为泊松方程(或拉普拉斯方程的解),但是我们没有限制任何的边界,即:我们是在无穷大空间中来解泊松方程的。如果存在边界,那么在边界面上会产生额外(不光滑)的电荷分布,而且这些面电荷的分别在我们求解出静电势之前是未知的(\textcolor{blue}{如果我们解出了区域静电势,则很快就能知道面电荷分布})。因此,仅知道体电荷分布函数 \(\rho(\textbf{x})\),便不足以确定一个体系的静电势。}
	
	\vspace{1mm}
	
	\par{\kaishu 如果我们考虑的空间区域是有限的,那就不得不考虑边界的影响。设想一个空间区域 \(V\),其闭合边界曲面为 \(S\equiv\partial V\),我们现在要求解区域 \(V\) 内满足泊松方程(式(3.1.3)),同时在边界 \(S\) 上满足给定边界条件的静电势 \(\varphi\)。数学上称之为边界值问题。}
	
	\par{\kaishu 最简单的情形下,静电势在界面 \(S\) 上满足的边界条件有两类:
		
		\vspace{-3mm}
		
		\begin{enumerate}
			
			\item \textbf{\textcolor{blue}{狄利克雷(Dirichlet)边界条件}}:已知静电势在界面 \(S\) 上的取值 \(\left.\varphi\right|_{S}\)。
			
			\vspace{-3mm}
			
			\item \textbf{\textcolor{blue}{诺伊曼(Neumann) 边界条件}}:已知静电势在边界面上法向偏微分的取值 \(\left.\frac{\partial\varphi}{\partial\hat{n}}\right|_{S}\)。
			
	\end{enumerate}}
	
	\vspace{-3mm}
	
	\par{\kaishu 在这两类边界条件下,静电场边值问题的解是唯一的。}
	
	\subsubsection{\kaishu 格林定理}
	
	\par{\kaishu 我们需要在边界条件下证明静电场边值问题的解唯一——\textbf{唯一性定理},但是在证明唯一性定理之前,我们需要处理静电势的边界条件,那就需要新的数学工具,即:格林定理(或恒等式)。这些定理势通过对散度定理的应用得到的,先列出散度定理:}
	
	\vspace{-2mm}
	
	\begin{equation*}
		\int_{V}\nabla\cdot\textbf{A}\,\mathrm{d}^{3}x = \oint_{S}\textbf{A}\cdot\textbf{n}\,\mathrm{d}S\qquad\left(\mathrm{d}^{3}x = \mathrm{d}V\right)
	\end{equation*}
	
	\vspace{3mm}
	
	\par{\kaishu 这一定理适用于封闭曲面 \(S\) 所围成的体积 \(V\) 内定义的任意守恒向量场 \(\textbf{A}\)。我们将 \(\textbf{A}\) 改写成:\(\textbf{A} = \phi\,\nabla\psi\),其中:\(\phi\) 和 \(\psi\) 都是任意标量场,现在取 \(\textbf{A}\) 的散度:}
	
	\vspace{-2mm}
	
	\begin{equation}
		\nabla\cdot\left(\phi\,\nabla\psi\right) = \phi\,\nabla^{2}\psi+\nabla\phi\cdot\nabla\psi
	\end{equation}
	
	\vspace{3mm}
	
	\par{\kaishu \(\textbf{A}\cdot\textbf{n}\) 也可以简化为:}
	
	\vspace{-5mm}
	
	\begin{equation}
		\phi\,\nabla\psi\cdot\textbf{n} = \phi\,\frac{\partial\psi}{\partial n}
	\end{equation}
	
	\vspace{3mm}
	
	\par{\kaishu 其中:\(\frac{\partial}{\partial n}\) 是表面 \(S\) 上的方向导数(从体积 \(V\) 内部指向外部)。我们现在将式(3.1.9)代入式(3.1.10)以后,就得到了格林第一恒等式。即:}
	
	\begin{equation}
		\int_{V}\left(\phi\,\nabla^{2}\psi+\nabla\phi\cdot\nabla\psi\right)\,\mathrm{d}^{3}x = \oint_{S}\phi\,\frac{\partial\psi}{\partial n}\,\mathrm{d}S
	\end{equation}
	
	\vspace{3mm}
	
	\par{\kaishu 如果我们将式(3.1.11)重新写,两个标量场交换仍然成立等式。有:}
	
	\begin{equation}
		\int_{V}\left(\psi\,\nabla^{2}\phi+\nabla\phi\cdot\nabla\psi\right)\,\mathrm{d}^{3}x = \oint_{S}\psi\,\frac{\partial\phi}{\partial n}\,\mathrm{d}S
	\end{equation}
	
	\vspace{3mm}
	
	\par{\kaishu 现在我们将式(3.1.11)减去式(3.1.12),就得到了格林第二恒等式(也是较为常见的格林定理):}
	
	\begin{equation}
		\int_{V}\left(\phi\,\nabla^{2}\psi-\psi\,\nabla^{2}\phi\right)\,\mathrm{d}^{3}x = \oint_{S}\left(\phi\,\frac{\partial\psi}{\partial n}-\psi\,\frac{\partial\phi}{\partial n}\right)\,\mathrm{d}S
	\end{equation}
	
	\begin{center}
		\begin{mdframed}[shadow = false]
			
			\par{\kaishu \textbf{补充:}格林第三恒等式}
			
			\vspace{1mm}
			
			\par{\kaishu 实际上,我们还可以得到格林第三恒等式,下面推导一般情况下的格林第三恒等式。}
			
			\par{\kaishu 假设函数 \(G\) 是拉普拉斯方程的基本解:}
			
			\vspace{-2mm}
			
			\begin{equation}
				\nabla^{2}G(\textbf{x},\,\textbf{x}') = \delta(\textbf{x}-\textbf{x}')
			\end{equation}
			
			\vspace{3mm}
			
			\par{\kaishu 其中:\(\delta\left(\textbf{x}-\textbf{x}'\right)\) 是狄拉克函数,而函数 \(G(\textbf{x},\,\textbf{x}')\) 被称为格林函数(Green's function)。对于变量 \(\textbf{x}\) 和 \(\textbf{x}'\) 的交换,格林函数具有对称性,即 \(G(\textbf{x},\,\textbf{x}') = G(\textbf{x}',\,\textbf{x})\)。}
			
			\par{\kaishu 现在令 \(\phi = G\),在我们考虑的区域 \(\mathbb{U}\) 内,\(\psi\) 是二阶连续可微函数。假设 \(\textbf{x}\) 在积分区域 \(\mathbb{U}\) 内,则我们使用格林第二恒等式,即式(3.1.14):}
			
			\vspace{-4mm}
			
			\begin{equation}
				\int_{\mathbb{U}}\left[G(\textbf{x},\,\textbf{x}')\,\nabla'^{2}\psi(\textbf{x}')-\psi(\textbf{x}')\,\delta(\textbf{x}-\textbf{x}')\right]\,\mathrm{d}V' = \oint_{\partial\mathbb{U}}\left[G(\textbf{x},\,\textbf{x}')\,\frac{\partial\psi(\textbf{x}')}{\partial n'}-\psi(\textbf{x}')\frac{\partial G(\textbf{x},\,\textbf{x}')}{\partial n'}\right]\,\mathrm{d}S'\nonumber
			\end{equation}
			
			\vspace{2mm}
			
			\par{\kaishu 我们再利用狄拉克函数的性质:\(\int_{\mathbb{U}}\psi(\textbf{x}')\,\delta(\textbf{x}-\textbf{x}')\,\mathrm{d}V' = \psi(\textbf{x})\),将上式化简为:}
			
			\vspace{-2mm}
			
			\begin{equation}
				\psi(\textbf{x})-\int_{\mathbb{U}}\left[G(\textbf{x},\,\textbf{x}')\,\nabla'^{2}\psi(\textbf{x}')\right]\,\mathrm{d}V' = \oint_{\partial\mathbb{U}}\left[\psi(\textbf{x}')\frac{\partial G(\textbf{x},\,\textbf{x}')}{\partial n'}-G(\textbf{x},\,\textbf{x}')\,\frac{\partial\psi(\textbf{x}')}{\partial n'}\right]\,\mathrm{d}S'
			\end{equation}
			
			\vspace{3mm}
			
			\par{\kaishu 其中:\(\mathrm{d}V'\)、\(\mathrm{d}S'\) 表示在区域 \(\mathbb{U}\) 内对变量 \(\textbf{x}'\) 积分。式(3.1.15)就是格林第三恒等式。如果 \(\psi\) 是拉普拉斯方程式的解:\(\nabla'^{2}\psi(\textbf{x}') = 0\),则恒等式简化为:}
			
			\begin{equation}
				\psi(\textbf{x}) = \oint_{\partial\mathbb{U}}\left[\psi(\textbf{x}')\,\frac{\partial G(\textbf{x},\,\textbf{x}')}{\partial n'}-G(\textbf{x},\,\textbf{x}')\,\frac{\partial\psi(\textbf{x}')}{\partial n'}\right]
			\end{equation}
			
			\vspace{2mm}
			
		\end{mdframed}
	\end{center}
	
	\subsubsection{\kaishu 边界条件下的唯一性定理}
	
	\par{\kaishu 在狄利克雷边界条件和诺伊曼边界条件下,我们在此小节将使用格林定理来证明:静电场边值问题的解是唯一的。}
	
	\par{\kaishu 我们考虑这样一个体系:体系区域为 \(V\),体系受到封闭边界表面 \(S\) 上的狄利克雷或者诺伊曼边界条件的约束。假设体系内存在两个解 \(\varphi_{1}\)、\(\varphi_{2}\),都满足泊松方程(\(\nabla^{2}\varphi = -\frac{\rho}{\varepsilon}\))以及相同的边界条件。现在我们令:}
	
	\vspace{-5mm}
	
	\begin{equation}
		U = \varphi_{2}-\varphi_{1}
	\end{equation}
	
	\vspace{2mm}
	
	\par{\kaishu 由于两个解均满足泊松方程,因此有:}
	
	\vspace{-3mm}
	
	\begin{equation}
		\nabla^{2}U = \nabla^{2}\left(\varphi_{2}-\varphi_{1}\right) = \frac{\rho}{\varepsilon}-\frac{\rho}{\varepsilon} = 0
	\end{equation}
	
	\vspace{3mm}
	
	\par{\kaishu 而且在区域边界上,满足 \(U = 0\) 或者 \(\frac{\partial U}{\partial n} = 0\)。此处我们利用格林第一恒等式(3.1.11),令 \(\phi = \psi = U\),我们得到:}
	
	\begin{equation}
		\int_{V}\left(U\,\nabla^{2}U+\nabla U\cdot\nabla U\right)\,\mathrm{d}^{3}x = \oint_{S}U \,\frac{\partial U}{\partial n}\,\mathrm{d}S
	\end{equation}
	
	\vspace{3mm}
	
	\par{\kaishu 代入 \(\nabla^{2} U = 0\),上式将化简为:}
	
	\vspace{-3mm}
	
	\begin{equation*}
		\int_{V}\left|\nabla U\right|^{2}\,\mathrm{d}^{3}x = \oint_{S}U \,\frac{\partial U}{\partial n}\,\mathrm{d}S
	\end{equation*}
	
	\vspace{3mm}
	
	\par{\kaishu 我们发现:等式右边无论代入狄利克雷边界条件还是诺伊曼边界条件,积分内的结果都为0,也就是说我们一定能得到:}
	
	\vspace{-3mm}
	
	\begin{equation}
		\int_{V}\left|\nabla U\right|^{2}\,\mathrm{d}^{3}x = 0
	\end{equation}
	
	\vspace{3mm}
	
	\par{\kaishu 我们很容易证明:\(\nabla U = 0\),因此在区域 \(V\) 内,\(U\) 是常数。对于狄利克雷边界条件,\(U = 0\) 在边界 \(S\) 处满足,因此在区域内,有:\(\varphi_{1} = \varphi_{2}\);对于诺伊曼边界条件,\(\varphi_{1}\) 与 \(\varphi_{2}\) 之间差一个常数,其函数结构相同。只要选取合适的势能0点,电势解是唯一的。}
	
	\vspace{1mm}
	
	\par{\kaishu \textbf{第三边界条件:}又被称为混合边界条件(mixed boundary condition),给定研究对象(电势)在区域边界 \(\partial V = \Gamma_{1}\cup\Gamma_{2}\) 各自满足的条件。此处假设在 \(\Gamma_{1}\) 处电势满足狄利克雷边界条件;在 \(\Gamma_{2}\) 处电势满足诺伊曼边界条件。即,我们已知:}
	
	\vspace{-3mm}
	
	\begin{equation}
		\left.\varphi(\textbf{x})\right|_{\Gamma_{1}}\quad\left(\textbf{x}\in\Gamma_{1}\right) \qquad\&\qquad \left.\frac{\partial}{\partial n}\varphi(\textbf{x})\right|_{\Gamma_{2}}\quad\left(\textbf{x}\in\Gamma_{2}\right)
	\end{equation}
	
	\vspace{1mm}
	
	\par{\kaishu 上面的例子作出示意图如下:}
	
	\vspace{4mm}
	
	\begin{center}
		\includegraphics[width=0.53\linewidth]{C:/Users/16965/OneDrive/Pictures/Electrondynamics/Electrodynamics fig 3-1-1.pdf}
		\captionof{figure}{\text{\kaishu 混合边界条件示意图(区域边界可以作更复杂的有限次划分)}}\label{fig:3-1-1}
	\end{center}
	
	\par{\kaishu 下面我们根据混合边界条件证明唯一性定理:仍然假设 \(\varphi_{1}\)、\(\varphi_{2}\) 满足泊松方程和相同的边界条件,定义 \(U = \varphi_{2}-\varphi_{1}\),\(U\) 满足的边界条件为:}
	
	\begin{equation}
		\left.U(\textbf{x})\right|_{\Gamma_{1}} = 0 \qquad\&\qquad \left.\frac{\partial U}{\partial n}\right|_{\Gamma_{2}} = 0\nonumber
	\end{equation}
	
	\vspace{2mm}
	
	\par{\kaishu 此时,式(3.1.19)将被改写成:}
	
	\vspace{-2mm}
	
	\begin{equation}
		\int_{V}\left|\nabla U\right|^{2}\,\mathrm{d}^{3}x = \oint_{\Gamma_{1}}U \,\frac{\partial U}{\partial n}\,\mathrm{d}S+\oint_{\Gamma_{2}}U \,\frac{\partial U}{\partial n}\,\mathrm{d}S
	\end{equation}
	
	\vspace{3mm}
	
	\par{\kaishu 根据 \(U\) 满足的边界条件,式(3.1.22)的右侧第一项和第二项均为0。我们仍能得到式(3.1.20),也就是说,我们能得到:\(\nabla U = 0\),\(U\) 为常数恒成立。因为在边界上存在部分 \(U = 0\) 的区域,在 \(V\) 内部 \(U \equiv0\),即 \(\varphi_{1} = \varphi_{2}\)。说明\textbf{混合边界条件也能唯一的确定一个电势解}。}
	
	\begin{center}
		\begin{mdframed}[shadow = false]
			
			\par{\kaishu \textbf{补充:}第四边界条件电势解不唯一
				
				第四边界条件:又被称为 Robin 边界条件,给定研究对象(电势)以及其外侧法向导数的线性组合在边界 \(\partial V\) 上的取值:}
			
			\vspace{-2mm}
			
			\begin{equation*}
				\alpha \varphi + \beta\,\frac{\partial\varphi}{\partial n} = f(\textbf{x})\,\qquad\textbf{x}\in\partial V
			\end{equation*}
			
			\vspace{2mm}
			
			\par{\kaishu 其中:\(\alpha,\,\beta\ne0\)。\textcolor{blue}{更一般地,允许 \(\alpha\) 和 \(\beta\) 是给定的函数 \(\alpha(\textbf{x})\) 和 \(\beta(\textbf{x})\)。}}
			
			\vspace{1mm}
			
			\par{\kaishu 现在我们沿用相同的思路,假设 \(\varphi_{1}\)、\(\varphi_{2}\) 满足泊松方程和相同的边界条件(\(\alpha,\,\beta\ne0\)):}
			
			\begin{equation*}
				\begin{cases}
					\alpha\varphi_{1}(\textbf{x})+\beta\,\frac{\partial}{\partial n}\varphi_{1}(\textbf{x}) = f(\textbf{x}) \\[12pt]
					\alpha\varphi_{2}(\textbf{x})+\beta\,\frac{\partial}{\partial n}\varphi_{2}(\textbf{x}) = f(\textbf{x})
				\end{cases}\qquad \textbf{x}\in\partial V
			\end{equation*}
			
			\vspace{5mm}
			
			\par{\kaishu 令 \(U = \varphi_{1}-\varphi_{2}\),其满足的边界条件是:}
			
			\begin{equation*}
				\alpha U(\textbf{x})+\beta\,\frac{\partial U(\textbf{x})}{\partial n} = 0 \qquad\textbf{x}\in\partial V
			\end{equation*}
			
			\vspace{2mm}
			
			\par{\kaishu 现在我们将得到:\(\frac{\partial U}{\partial n} = -\frac{\alpha}{\beta}\, U\),此时代入式(3.1.19)便可以得到:}
			
			\begin{equation*}
				\int_{V}\left|\nabla U\right|^{2}\,\mathrm{d}^{3}x = -\oint_{\partial V}\frac{\alpha}{\beta}\, U^{2}\,\mathrm{d}S
			\end{equation*}
			
			\vspace{2mm}
			
			\par{\kaishu 下面我们需要对 \(\alpha\) 和 \(\beta\) 进行讨论。
				
				\vspace{1mm}
				
				\textbf{若 \(\boldsymbol{\alpha}\)、\(\boldsymbol{\beta}\) 同号},简单分析上式左右两侧可知:}
			
			\begin{equation*}
				\int_{V}\left|\nabla U\right|^{2}\,\mathrm{d}^{3}x\ge0\qquad\&\qquad -\oint_{\partial V}\frac{\alpha}{\beta}\, U^{2}\,\mathrm{d}S\le0
			\end{equation*}
			
			\vspace{3mm}
			
			\par{\kaishu 为保证等式成立,只能左右两侧同时为0,即区域内 \(\nabla U = 0\),又因为边界处取 \(U = 0\),所以在区域内有:\(U\equiv0\),\(\varphi_{1} = \varphi_{2}\)。\(\alpha\)、\(\beta\) 同号时,Robin 边界条件能够唯一确定一个电势解。}
			
		\end{mdframed}
	\end{center}
	
	\begin{center}
		\begin{mdframed}[shadow = false]
			
			\par{\kaishu \textbf{若 \(\boldsymbol{\alpha}\)、\(\boldsymbol{\beta}\) 不同号},则有:}
			
			\begin{equation*}
				\int_{V}\left|\nabla U\right|^{2}\,\mathrm{d}^{3}x\ge0\qquad\&\qquad -\oint_{\partial V}\frac{\alpha}{\beta}\, U^{2}\,\mathrm{d}S\ge0
			\end{equation*}
			
			\vspace{3mm}
			
			\par{\kaishu 此时我们无法仅凭符号推断解的唯一性。如果我们想说明唯一性定理此处不成立,那么只需要找一个反例即可。
				
				\textbf{反例:}一维泊松方程,区域为 \(x\in\left[0,\, l\right]\),此时 \(U\) 满足的方程为:}
			
			\begin{equation*}
				\frac{\mathrm{d}^{2}U}{\mathrm{d}x^{2}} = 0\quad\Rightarrow\quad U = cx+d\qquad\left(c,\, d\in\mathbb{R}\right)
			\end{equation*}
			
			\vspace{3mm}
			
			\par{\kaishu 此时的边界条件写成:\(\frac{\mathrm{d}U}{\mathrm{d}n}-a \, U = 0\),(\(a>0\))。代入两个端点,得到 \(c\) 和 \(d\) 的线性方程组:}
			
			\begin{equation*}
				\begin{cases}
					\left(1-al\right)c-ad = 0\qquad\left(x = l\right)\\[4pt]
					c+ad = 0\qquad\left(x = 0\right)
				\end{cases}
			\end{equation*}
			
			\vspace{3mm}
			
			\par{\kaishu 如果齐次线性方程组有非0解,则它的系数矩阵行列式值为0:}
			
			\begin{equation*}
				\det\begin{pmatrix}
					1-al & -a\\
					1 & a
				\end{pmatrix} = \left(1-al\right)a+a = 0
			\end{equation*}
			
			\vspace{2mm}
			
			\par{\kaishu 可以解出 \(a = \frac{2}{l}\) 时,存在 \(c\)、\(d\) 的非0解,此时 \(U = cx+d\) 不是常数函数,与 \(x\) 不同的值有关,即:\(\varphi_{1}\ne\varphi_{2}\),这种情况下解并不唯一。如果我们取 \(a\ne\frac{2}{l}\),\(c\)、\(d\) 只有0解,于是有 \(\varphi_{1} = \varphi_{2}\)。}
			
			\vspace{1mm}
			
			\par{\kaishu \(\alpha\)、\(\beta\) 异号时,Robin 边界条件并不能总是唯一的确定电势解,静电场边值问题的唯一性定理并不总是成立的。\textcolor{blue}{Robin 边界条件的反例和更多的边界条件将在附录中给出讨论。}}
			
			\vspace{2mm}
			
		\end{mdframed}
	\end{center}
	
	\subsubsection{\kaishu 静电场边值问题的形式解}
	
	\par{\kaishu 对于静电方程的形式解,我们在前四小节已经做好了充分的准备。电势在有限区域 \(V\) 内满足泊松方程或拉普拉斯方程,在边界 \(S\) 上满足合适的边界条件。我们需要用格林函数(Green's function)给出正式解。}
	
	\par{\kaishu 回顾第一章的静电内容,我们已经讨论过点电荷电势的泊松方程的解,即:}
	
	\begin{equation}
		\nabla'^{2}\left(\frac{1}{\left|\textbf{x}-\textbf{x}'\right|}\right) = -4\pi\delta\left(\textbf{x}-\textbf{x}'\right)
	\end{equation}
	
	\vspace{3mm}
	
	\par{\kaishu 根据我们补充的格林第三恒等式的知识,函数 \(-\frac{1}{4\pi\left|\textbf{x}-\textbf{x}'\right|}\) 可以作为格林函数的一类。}
	
	\vspace{2mm}
	
	\par{\kaishu 事实上,我们甚至可以利用格林第三恒等式本身,只需要令:\(\psi = \varphi\)(\(\varphi\) 为电势),\(G(\textbf{x},\,\textbf{x}')\) 为格林函数,区域 \(V\) 内的静电边值问题的形式解为:}
	
	\begin{equation}
		\varphi(\textbf{x}) = \frac{1}{4\pi\varepsilon}\int_{V}G(\textbf{x},\,\textbf{x}')\rho(\textbf{x}')\,\mathrm{d}^{3}x'+\frac{1}{4\pi}\oint_{S}\left[G(\textbf{x},\,\textbf{x}')\,\frac{\partial\varphi}{\partial n'}-\varphi(\textbf{x}')\,\frac{\partial G(\textbf{x},\,\textbf{x}')}{\partial n'}\right]\,\mathrm{d}S'
	\end{equation}
	
	\vspace{3mm}
	
	\par{\kaishu \textcolor{blue}{注意:函数 \(\frac{1}{\left|\textbf{x}-\textbf{x}'\right|}\) 虽然满足格林函数对应的方程,但它不一定满足区域 \(V\) 的边界条件。}
		
		\vspace{2mm}
		
		因此一般来说,式(3.1.23)不是我们最终想得到的格林函数,但我们可以写出一般的格林函数(为了简化计算,将 \(-4\pi\) 因子移动到等式右边):}
	
	\begin{equation}
		G\left(\textbf{x},\,\textbf{x}'\right) = \frac{1}{\left|\textbf{x}-\textbf{x}'\right|}+F(\textbf{x},\,\textbf{x}')
	\end{equation}
	
	\vspace{3mm}
	
	\par{\kaishu 其中:\(F(\textbf{x},\,\textbf{x}')\) 在区域 \(V\) 内满足拉普拉斯方程:\(\nabla'^{2}F(\textbf{x},\,\textbf{x}') = 0\),格林函数增加了一个\textbf{调和函数}(拉普拉斯方程的解)。我们需要选取合适的 \(F(\textbf{x},\,\textbf{x}')\) 使得式(3.1.24)确定的静电势在边界 \(S\) 上满足相应的边界条件。}
	
	\par{\kaishu 对于狄利克雷边界条件的边值问题,我们选取函数 \(F(\textbf{x},\,\textbf{x}')\) 使得格林函数满足:}
	
	\vspace{-3mm}
	
	\begin{equation*}
		G_{D}\left(\textbf{x},\,\textbf{x}'\right) = 0\qquad\textbf{x}'\in S
	\end{equation*}
	
	\vspace{2mm}
	
	\par{\kaishu \(G_{D}\) 表示狄利克雷边界条件下的格林函数,为简化计算,我们将边界定值取为0。于是,在边界 \(S\) 上满足狄利克雷边界条件的区域 \(V\) 内的静电势为:}
	
	\begin{equation}
		\varphi(\textbf{x}) = \frac{1}{4\pi\varepsilon}\int_{V}G_{D}\left(\textbf{x},\,\textbf{x}'\right)\rho(\textbf{x}')\,\mathrm{d}^{3}x'-\frac{1}{4\pi}\oint_{S}\varphi(\textbf{x}')\,\frac{\partial G_{D}\left(\textbf{x},\,\textbf{x}'\right)}{\partial n'}\,\mathrm{d}S'
	\end{equation}
	
	\vspace{5mm}
	
	\par{\kaishu 在诺伊曼边界条件下需要注意,我们\textbf{不能}简单地认为 \(G_{N}(\textbf{x},\,\textbf{x}')\)(\(G_{N}\) 表示诺伊曼边界条件下的格林函数)满足 \(\left.\frac{\partial G_{N}}{\partial n}\right|_{S} = 0\),原因是格林函数物理上相当于区域 \(\textbf{x}\) 内的一个点电荷,因此我们在此区域利用高斯定理:}
	
	\begin{equation*}
		\oint_{S}\frac{\partial G_{N}\left(\textbf{x},\,\textbf{x}'\right)}{\partial n'}\,\mathrm{d}S' = -4\pi
	\end{equation*}
	
	\vspace{3mm}
	
	\par{\kaishu 选取最简单的选择,令:}
	
	\vspace{-3mm}
	
	\begin{equation*}
		\frac{\partial G_{N}\left(\textbf{x},\,\textbf{x}'\right)}{\partial n'} = -\frac{4\pi}{A_{S}}\qquad\textbf{x}'\in S
	\end{equation*}
	
	\vspace{2mm}
	
	\par{\kaishu 其中:\(A_{S}\) 表示边界 \(S\) 的面积。在边界 \(S\) 上满足诺伊曼边界条件的区域 \(V\) 内的静电势为:}
	
	\vspace{-5mm}
	
	\begin{align}
		\varphi(\textbf{x})&= \frac{1}{4\pi\varepsilon}\int_{V}G(\textbf{x},\,\textbf{x}')\rho(\textbf{x}')\,\mathrm{d}^{3}x'+\frac{1}{4\pi}\oint_{S}\left[G(\textbf{x},\,\textbf{x}')\,\frac{\partial\varphi}{\partial n'}+\frac{4\pi}{A_{S}}\varphi(\textbf{x}')\right]\,\mathrm{d}S'\nonumber\\[6pt]
		&= \langle\varphi\rangle_{S}+\frac{1}{4\pi\varepsilon}\int_{V}G(\textbf{x},\,\textbf{x}')\rho(\textbf{x}')\,\mathrm{d}^{3}x'+\frac{1}{4\pi}\oint_{S}G(\textbf{x},\,\textbf{x}')\,\frac{\partial\varphi}{\partial n'}\,\mathrm{d}S'
	\end{align}
	
	\vspace{3mm}
	
	\par{\kaishu 其中:\(\langle\varphi\rangle_{S}\) 表示静电势在边界面 \(S\) 上的平均值。下面我们讨论一下格林函数的性质:
		
		\begin{enumerate}
			
			\item 格林函数满足的边界条件非常简单,这些条件与狄利克雷/诺伊曼边界条件的详细形式无关。但是,由于格林函数的形式取决于曲面 \(S\) 的形状,因此确定它是一件非常复杂的事情,后面我们将讨论这类问题。
			
			\vspace{-3mm}
			
			\item 我们已经知道格林函数具有对称性,\(G(\textbf{x},\,\textbf{x}') = G(\textbf{x}',\,\textbf{x})\)。格林函数的物理意义是一个单位点源产生的势,因此格林函数的对称性仅表示“源”和观测点之间的可交换性。\textcolor{blue}{对于诺伊曼边界条件,这种对称性并非默认存在,可以作为单独的要求设定。}
			
			\vspace{-1mm}
			
			\item 最后我们讨论 \(\frac{F(\textbf{x},\,\textbf{x}')}{4\pi\varepsilon}\) 的物理意义,它是区域 \(V\) 内满足拉普拉斯方程的解,代表区域 \(V\) 外部电荷系统的电势。我们可以认为:这一项是由外部电荷分布产生的电势,这些电荷分布与点 \(\textbf{x}'\) 处的“源”\textbf{共同产生}的电势,满足区域边界 \(S\) 处电势为0(或者方向导数为0)的边界条件。由于表面 \(\textbf{x}\) 处的电势取决于源点的位置,故 \(F(\textbf{x},\,\textbf{x}')\) 也必须依赖于“参数”\(\textbf{x}'\)。对于系统中包含导体的狄利克雷问题,\(\frac{F(\textbf{x},\,\textbf{x}')}{4\pi\varepsilon}\) 可以被解释为:由源点 \(\textbf{x}'\) 处的点电荷在导体诱导表面电荷分布所产生的电势。
			
	\end{enumerate}}
	
	\subsection{\kaishu 导体系统}
	
	\subsubsection{\kaishu 导体的边界条件}
	
	\par{\kaishu 此处我们只讨论静电学范畴内的导体,所以我们将\textbf{假设}导体满足欧姆定律 \(\vec{j} = \sigma\,\textbf{E}\),且它的直流电导率足够大,以至于可以被视为理想导体。}
	
	\par{\kaishu 我们设想某一时刻导体内部的电场不为0,根据欧姆定律,导体内部将产生强大的电流密度。导体内部的自由电子存在宏观流动,改变空间的电荷分布 \(\rho(\textbf{x})\),此时的电场并不是静态的。反过来讲,电场达到静态的\textbf{静电体系},导体内部一定没有宏观的电流密度。}
	
	\par{\kaishu 再考虑导体与非导体的交界面处,如果导体表面的电势 \(\varphi(\textbf{y})\) 不是常数(处处相等),那么就会在表面产生局域电场,进而造成导体表面电荷的宏观移动,从而电场不处于静态。综上所述,我们得到的结论是:}
	
	\par{\kaishu \textbf{\textcolor{blue}{静电学中,理想导体内部的电场恒为0,导体的表面是等势面(电势为常数),导体上所有的电荷只可能分布在导体的表面。}}}
	
	\vspace{1mm}
	
	\par{\kaishu 此处如果我们对导体表面作高斯面并使用高斯定理,将会发现:静电势在导体表面的法向梯度与该处\textbf{自由面电荷密度}联系在一起。}
	
	\vspace{-3mm}
	
	\begin{equation}
		\varepsilon\,\frac{\partial\varphi}{\partial n} = -\sigma
	\end{equation}
	
	\vspace{2mm}
	
	\par{\kaishu 我们假定:在导体外部是介电常数为 \(\varepsilon\) 的各向同性的、均匀线性电介质。这里的法向定义为从导体的内部指向外部。因此,导体的边界条件为:}
	
	\vspace{-2mm}
	
	\begin{equation}
		\left.\varphi\right|_{S} = \varphi_{0} \qquad \left.\frac{\partial\varphi}{\partial n}\right|_{S} = -\frac{\sigma}{\varepsilon}
	\end{equation}
	
	\vspace{3mm}
	
	\par{\kaishu 静电学中的导体问题可以总结为:在式(3.2.2)的边界条件下,求解泊松方程。如果导体的表面电势确定,则导体的表面电荷不能预先设定,需要用 \(Q = -\oint\varepsilon\,\frac{\partial\varphi}{\partial n}\,\mathrm{d}S\) 求得;如果考虑孤立导体问题,\(Q\) 已知,则需要求解 \(\varphi\)。}
	
	\subsubsection{\kaishu 格林互易定理}
	
	\par{\kaishu 接下来我们需要研究多个理想导体构成的静电系统,讨论具体问题之前需要先介绍一个一般的定理——\textbf{格林互易定理}。这个定理在导体静电学中非常有用,其表述如下:}
	
	\begin{center}
		\begin{mdframed}[frametitle={\text{\kaishu 格林互易定理}}, frametitlerule=true, frametitlebackgroundcolor=blue!05]
			
			\par{\kaishu 给定一个由 \(m\) 个导体组成的系统,导体上的电荷为 \(q_{1}\)、\(q_{2}\)、\(\cdots\)、\(q_{m}\),对应的电势为 \(\varphi_{1}\)、\(\varphi_{2}\)、\(\cdots\)、\(\varphi_{m}\);而对应另外一种电荷分布,\(q_{1}'\)、\(q_{2}'\)、\(\cdots\)、\(q_{m}'\),导体的电势分布为 \(\varphi_{1}'\)、\(\varphi_{2}'\)、\(\cdots\)、\(\varphi_{m}'\),两种分布之间满足关系式:}
			
			\vspace{-2mm}
			
			\begin{equation}
				\sum_{i = 1}^{m}q_{i}\varphi_{i}' = \sum_{i = 1}^{m}q_{i}'\varphi_{i}
			\end{equation}
			
			\vspace{1mm}	
			
		\end{mdframed}
	\end{center}
	
	\vspace{3mm}
	
	\par{\kaishu 下面我们简要地证明格林互易定理,这里需要用到上一节我们给出的格林第二恒等式,即:}
	
	\vspace{-2mm}
	
	\begin{equation*}
		\int_{V}\left(\phi\,\nabla^{2}\psi-\psi\,\nabla^{2}\phi\right)\,\mathrm{d}^{3}x = \oint_{S}\left(\phi\,\frac{\partial\psi}{\partial n}-\psi\,\frac{\partial\phi}{\partial n}\right)\,\mathrm{d}S
	\end{equation*}
	
	\vspace{2mm}
	
	\par{\kaishu 现在我们考虑含有 \(m\) 个导体的空间 \(\mathbb{U}\),取无限远处为封闭连续曲面 \(\boldsymbol{S}_{0}\),并在其中“挖掉”所有导体所处的区域 \(V_{i}\),挖掉的区域在 \(\boldsymbol{S}_{0}\) 内部形成了 \(m\) 个封闭的连续曲面 \(S_{i}\)(\(i = 1,\,2,\cdots,\,m\))。我们将积分区域 \(V\) 选取如下:}
	
	\vspace{-3mm}
	
	\begin{equation*}
		V = \complement_{\mathbb{U}}V_{1}\,\cap\,\complement_{\mathbb{U}}V_{2}\,\cap\,\cdots\,\cap\,\complement_{\mathbb{U}}V_{m} = \bigcap_{i = 1}^{m}\,\complement_{\mathbb{U}}V_{i}
	\end{equation*}
	
	\vspace{1mm}
	
	\par{\kaishu 积分区域 \(V\) 就是全空间减去导体空间的剩余部分,对应的封闭表面 \(S\) 取为:}
	
	\vspace{-1mm}
	
	\begin{equation*}
		S = \boldsymbol{S}_{0}\,\cup\, S_{1}\,\cup\, S_{2}\,\cup\,\cdots\,\cup S_{m} = \left(\bigcup_{i = 1}^{m}S_{i}\right)\,\cup\,\boldsymbol{S}_{0}
	\end{equation*}
	
	\begin{center}
		\includegraphics[width=0.53\linewidth]{C:/Users/16965/OneDrive/Pictures/Electrondynamics/Electrodynamics fig 3-2-1.pdf}
		\captionof{figure}{\text{\kaishu \(m\) 个导体的空间 \(\mathbb{U}\) 示意图,取的封闭曲面 \(S_{i}\) 尽可能接近导体表面}}\label{fig:3-2-1}
	\end{center}
	
	\vspace{1mm}
	
	\par{\kaishu 现在我们分别考虑系统的两个静电态:\(\boldsymbol{\Gamma}_{1}\)、\(\boldsymbol{\Gamma}_{2}\),导体表面电荷分别记为 \(\left\{q_{i}\right\}\)、\(\left\{q_{i}'\right\}\),对应的空间电势分布记为 \(\phi\) 和 \(\psi\),都满足泊松方程和相应的边界条件。}
	
	\par{\kaishu 由于静电平衡条件下,电荷分布在导体表面,所以在区域 \(V\) 内,电势满足的是拉普拉斯方程,即 \(\nabla^{2}\phi = 0\)、\(\nabla^{2}\psi = 0\),代入格林第二恒等式,我们得到:}
	
	\begin{equation}
		0 = \int_{V}\left(\phi\,\nabla^{2}\psi-\psi\,\nabla^{2}\phi\right)\,\mathrm{d}^{3}x = \oint_{S}\left(\phi\,\frac{\partial\psi}{\partial n}-\psi\,\frac{\partial\phi}{\partial n}\right)\,\mathrm{d}S
	\end{equation}
	
	\vspace{3mm}
	
	\par{\kaishu 因为我们将 \(\boldsymbol{S}_{0}\) 取在无穷远处,此处电势均为0,有:}
	
	\begin{equation*}
		\oint_{S}\phi\,\frac{\partial\psi}{\partial n}\,\mathrm{d}\boldsymbol{S}_{0} = \oint_{S}\psi\,\frac{\partial\phi}{\partial n}\,\mathrm{d}\boldsymbol{S}_{0} \equiv 0
	\end{equation*}
	
	\vspace{3mm}
	
	\par{\kaishu 因此我们只需考虑 \(\bigcup_{i = 1}^{m}S_{i}\) 部分即可,式(3.2.4)改写为:}
	
	\begin{equation}
		\sum_{i = 1}^{m}\,\oint_{S_{i}}\left(\phi\,\frac{\partial\psi}{\partial n}-\psi\frac{\partial\phi}{\partial n}\right)\,\mathrm{d}S_{i} = 0
	\end{equation}
	
	\vspace{3mm}
	
	\par{\kaishu 根据静电平衡时导体的边界条件,导体表面的电荷分布可以写出(根据式(3.2.1)):}
	
	\begin{equation}
		\sigma_{i} = -\left.\varepsilon\,\frac{\partial\phi}{\partial n}\right|_{S_{i}} \qquad \sigma_{i}' = -\left.\varepsilon\,\frac{\partial\psi}{\partial n}\right|_{S_{i}}
	\end{equation}
	
	\vspace{3mm}
	
	\par{\kaishu 此处 \(S_{i}\) 方向与式(3.2.1)相同,再根据我们的假设:导体表面是等势体,即 \(\varphi_{i} = \left.\phi\right|_{S_{i}}\),\(\varphi'_{i} = \left.\psi\right|_{S_{i}}\)(\(\varphi_{i}\) 表示第 \(i\) 个导体表面的电势)。现在我们将式(3.2.6)和边界条件代入式(3.2.5)中,有:}
	
	\vspace{-3mm}
	
	\begin{gather}
		\sum_{i = 1}^{m}\oint_{S_{i}}\left(-\varphi_{i}\sigma_{i}'+\varphi_{i}'\sigma_{i}\right)\,\mathrm{d}S_{i} = \sum_{i = 1}^{m}\left(\oint_{S_{i}}\varphi_{i}'\sigma_{i}\,\mathrm{d}S_{i}-\oint_{S_{i}}\varphi_{i}\sigma_{i}'\,\mathrm{d}S_{i}\right) = 0\nonumber\\[8pt]
		\Rightarrow\quad\sum_{i = 1}^{m}\left(\varphi_{i}'\oint_{S_{i}}\sigma_{i}\,\mathrm{d}S_{i}-\varphi_{i}\oint_{S_{i}}\sigma_{i}'\,\mathrm{d}S_{i}\right) = 0 \nonumber\\[8pt]
		\Rightarrow\quad\sum_{i = 1}^{m}q_{i}\varphi_{i}' = \sum_{i = 1}^{m}q_{i}'\varphi_{i}
	\end{gather}
	
	\vspace{3mm}
	
	\par{\kaishu 最后一步我们利用了电荷的积分式:\(q = -\varepsilon\oint_{S}\frac{\partial\varphi}{\partial n}\,\mathrm{d}S\),移项得到结论,这就是格林互易定理。}
	
	\vspace{1mm}
	
	\par{\kaishu 根据格林互易定理,我们可以得到一个简单的结果:将两个静电态分别设置为:
		
		\vspace{-2mm}
		
		\begin{equation*}
			\begin{cases}
				\boldsymbol{\Gamma}_{1}:\, q_{1} = q_{2} = \cdots = q_{i-1} = q_{i+1} = \cdots = q_{m} = 0\\
				\boldsymbol{\Gamma}_{2}:\, q_{1}' = q_{2}' = \cdots  = q_{j-1}' = q_{j+1}' = \cdots = q_{m}' = 0
			\end{cases}\qquad\left(i\ne j\right)
		\end{equation*}
		
		\vspace{3mm}
		
		此时我们将得到:\(q_{i}\,\varphi_{j}' = q_{j}'\,\varphi_{i}\),若令 \(q_{i} = q_{j}' = Q\),则有:\(\varphi_{i} = \varphi_{j}'\)。这个结果的物理意义是:带电量 \(Q\) 的导体 \(\mathcal{I}\) 在不带电的导体 \(\mathcal{J}\) 表面感应产生的电势;等于带电量 \(Q\) 的导体 \(\mathcal{J}\) 在不带电的导体 \(\mathcal{I}\) 表面感应产生的电势。}
	
	\vspace{1mm}
	
	\par{\kaishu 同理,假定 \(\varphi_{i} = \varphi_{j}' = \varphi_{0}\),则有:\(q_{i} = q_{j}'\)。这说明,固定电势为 \(\varphi_{0}\) 的导体 \(\mathcal{I}\) 在接地导体(电势为0)\(\mathcal{J}\) 表面感应出的电荷;等于固定电势为 \(\varphi_{0}\) 的导体 \(\mathcal{J}\) 在接地导体(电势为0)\(\mathcal{I}\) 表面感应出的电荷(\(\varphi_{0}\)、\(Q\) 均为固定常数)。下面我们简单讨论一下:
		
		\vspace{-3mm}
		
		\begin{enumerate}
			
			\item \(i = j\),格林互易定理将退化为 \(q_{i}\varphi_{i}' = q_{i}'\varphi_{i}\),我们可以得到与导体本身性质相关的一个定值:\(C_{i} = \frac{q_{i}}{\varphi_{i}} = \frac{q_{i}'}{\varphi_{i}'}\)(\(\varphi_{i},\varphi_{i}'\ne0\))。这就是单个导体的\textbf{电容}。
			
			\item 格林互易定理并未规定导体必须是点电荷,即:对任意简单连通形状的导体,格林互易定理均可以使用(即使电场会引发导体电荷的重新分布)。互易定理本质代表了源点和观察点的对称关系,我们将在电镜像法一节中看到格林互易定理的威力。\textcolor{blue}{但是,格林互易定理解决问题很依赖观察和技巧,而且对于一般性的导体问题,其限制条件又较多。它更像是精巧的手工艺品,是不适合一般方法的求解,切勿花费大量精力追求这种“技巧”!}
			
	\end{enumerate}}
	
	\subsubsection{\kaishu 导体组的能量}
	
	\par{\kaishu 我们在第二章就已经给出了电磁场能量的表示,现在除去磁场,单一的电场能量为:}
	
	\vspace{-2mm}
	
	\begin{equation*}
		W = \frac{1}{2}\int_{\tau}\varepsilon\, E^{2}\,\mathrm{d}\tau = \frac{1}{2}\int_{\tau}\textbf{E}\cdot\textbf{D}\,\mathrm{d}\tau
	\end{equation*}
	
	\vspace{3mm}
	
	\par{\kaishu 现在为了方便导体能量的研究,我们将其用电势 \(\varphi\) 表示,利用 \(\textbf{E} = -\nabla\varphi\),我们得到:}
	
	\vspace{-5mm}
	
	\begin{align}
		W &= -\frac{1}{2}\int_{\tau}\nabla\varphi\,\cdot\textbf{D}\,\mathrm{d}\tau\nonumber \\[5pt]
		&\overset{a}{=}-\frac{1}{2}\int_{\tau}\nabla\cdot\left(\varphi\,\textbf{D}\right)\,\mathrm{d}\tau+\frac{1}{2}\int_{\tau}\varphi\,\nabla\cdot\textbf{D}\,\mathrm{d}\tau\nonumber\\[5pt]
		&\overset{b}{=}-\frac{1}{2}\oint_{S}\varphi\,\textbf{D}\cdot\mathrm{d}\textbf{S}+\frac{1}{2}\int_{\tau}\varphi\,\rho\,\mathrm{d}\tau
	\end{align}
	
	\vspace{3mm}
	
	\par{\kaishu 其中:\(a\) 处等号运用了矢量分析公式:\(\nabla\cdot\left(\varphi\,\textbf{D}\right) = \nabla\varphi\cdot\textbf{D}+\varphi\,\nabla\cdot\textbf{D}\),\(b\) 处等号对右侧第一项使用高斯定理,对第二项使用 \(\nabla\cdot\textbf{D} = \rho\) 代换。}
	
	\par{\kaishu 在式(3.2.8) 中,我们考虑的空间 \(\tau\) 包含导体。如果我们将 \(\tau\) 扩大到全空间,则可以考察导体体系的总能量,此时 \(S\) 为无穷远边界,假设电荷体系分布在有限区域内,式(3.2.8)右侧第一项为0。另外,导体上的电荷分布在导体表面,且表面电势为常数,对于 \(n\) 个导体体系积分,得到:}
	
	\vspace{-1mm}
	
	\begin{equation}
		W = \frac{1}{2}\int_{\tau}\varphi\,\rho\,\mathrm{d}\tau = \frac{1}{2}\sum_{i=1}^{n}q_{i}\varphi_{i}
	\end{equation}
	
	\vspace{3mm}
	
	\par{\kaishu 其中:\(\varphi_{i}\)、\(q_{i}\) 分别表示第 \(i\) 个导体表面的电势和电荷,此式对点电荷模型也成立。}
	
	\par{\kaishu 式(3.2.9)中,我们将能量改写成与电荷密度相关的积分,但是这\textbf{\textcolor{blue}{不意味着}}能量只存在于有电荷分布的区域。对于两种静电场的能量表达式:\(W = \frac{1}{2}\int\textbf{E}\cdot\textbf{D}\,\mathrm{d}^{3}x\) 和 \(W = \frac{1}{2}\int\varphi\,\rho(\textbf{x})\,\mathrm{d}^{3}x\),只有求静电场的\textbf{\textcolor{blue}{总能量}}时才能等价。
		
		如果我们讨论空间有限范围内的电磁能量,两者不能等效使用,因为此时式(3.2.8)中的曲面积分不再为0,所以这种情况我们使用 \(W = \frac{1}{2}\int\textbf{E}\cdot\textbf{D}\,\mathrm{d}^{3}x\) 计算局域的电场能量。}
	
	\vspace{1mm}
	
	\par{\kaishu 下面我们将更进一步,讨论多个带电体的能量关系(使用 \(W = \frac{1}{2}\int\textbf{E}\cdot\textbf{D}\,\mathrm{d}^{3}x\) 进行推导)。考虑两个带电体 \(\mathcal{A}\)、\(\mathcal{B}\),它们各自独立存在时,空间激发的电场分别记为 \(\textbf{E}_{1}\) 和 \(\textbf{E}_{2}\)。现在我们将 \(\mathcal{A}\) 和 \(\mathcal{B}\) 置于同一个系统中,当如下条件之一存在:
		
		\vspace{-3mm}
		
		\begin{enumerate}
			
			\item 两个带电体自身尺寸远小于它们之间的距离。
			
			\vspace{-3mm}
			
			\item 一个带电体的电量以及尺寸远小于另一个带电体的电量以及尺寸(如图3.2.2所示)。
			
	\end{enumerate}}
	
	\vspace{3mm}
	
	\begin{center}
		\includegraphics[width=0.6\linewidth]{C:/Users/16965/OneDrive/Pictures/Electrondynamics/Electrodynamics fig 3-2-2.pdf}
		\captionof{figure}{\text{\kaishu 带电体 \(\mathcal{A}\)、\(\mathcal{B}\) 间距满足:\(a,\, b\ll|\textbf{r}|\),或者满足:\(a\ll b\);\(q\ll Q\)}}\label{fig:3-2-2}
	\end{center}
	
	\vspace{2mm}
	
	\par{\kaishu 那么,\textcolor{blue}{两个带电体上的电荷分布不会因为相对构型的改变产生显著的变化(我们忽略微小的变化)。}空间电场可以近似写成:\(\textbf{E} \simeq \textbf{E}_{1}+\textbf{E}_{2}\),体系总能量为:}
	
	\begin{equation}
		W = \frac{1}{2}\int\varepsilon\,E^{2}\,\mathrm{d}^{3}x = \underbrace{\frac{1}{2}\int\varepsilon\left(E_{1}^{2}+E_{2}^{2}\right)\,\mathrm{d}^{3}x}_{W_{1}+W_{2}}+\underbrace{\varepsilon\int\textbf{E}_{1}\cdot\textbf{E}_{2}\,\mathrm{d}^{3}x}_{W_{\text{int}}}
	\end{equation}
	
	\vspace{3mm}
	
	\par{\kaishu 从式(3.2.10)看出,系统总能量由两部分构成:\(W_{1}\) 和 \(W_{2}\) 表示的是带电体 \(\mathcal{A}\) 或者 \(\mathcal{B}\) 单独存在时的静电能量,我们称之为\textbf{固有能}。\(W_{\text{int}}\) 表示的是 \(\mathcal{A}\)、\(\mathcal{B}\) 激发的电场相互作用的项,我们称之为\textbf{相互作用能}。用电势进一步表示成:}
	
	\vspace{-2mm}
	
	\begin{equation}
		W_{\text{int}} = \varepsilon\int\textbf{E}_{1}\cdot\textbf{E}_{2}\,\mathrm{d}^{3}x = \varepsilon\int\nabla\varphi_{1}\cdot\nabla\varphi_{2}\,\mathrm{d}^{3}x
	\end{equation}
	
	\vspace{3mm}
	
	\par{\kaishu 其中:\(\varphi_{1}\)、\(\varphi_{2}\) 分别表示带电体 \(\mathcal{A}\) 和 \(\mathcal{B}\) 独立存在时空间的电势分布。}
	
	\vspace{1mm}
	
	\par{\kaishu 实际上,我们可以利用矢量运算公式进一步化简,使用 \(\nabla\cdot\left(\varphi\,\textbf{A}\right) = \nabla\varphi\cdot\textbf{A}+\varphi\,\nabla\cdot\textbf{A}\),现在在式(3.2.11)的例子中,令 \(\textbf{A} = \nabla\varphi_{2}\),我们将得到:}
	
	\vspace{-6mm}
	
	\begin{align*}
		W_{\text{int}} &= \varepsilon\int\nabla\varphi_{1}\cdot\nabla\varphi_{2}\,\mathrm{d}^{3}x\\[5pt]
		&= \varepsilon\int\nabla\cdot\left(\varphi_{1}\,\nabla\varphi_{2}\right)\,\mathrm{d}^{3}x-\varepsilon\int\varphi_{1}\nabla\cdot\left(\nabla\varphi_{2}\right)\,\mathrm{d}^{3}x\\[5pt]
		&= \varepsilon\int\nabla\cdot\left(\varphi_{1}\,\nabla\varphi_{2}\right)\,\mathrm{d}^{3}x-\varepsilon\int\varphi_{1}\,\nabla^{2}\varphi_{2}\,\mathrm{d}^{3}x
	\end{align*}
	
	\vspace{3mm}
	
	\par{\kaishu \(\varphi_{1}\)、\(\varphi_{2}\) 都满足泊松方程 \(\nabla^{2}\varphi_{1,\,2} = \frac{\rho_{1,\,2}}{\varepsilon}\),\(\rho_{1}\)、\(\rho_{2}\) 分别表示带电体 \(\mathcal{A}\)、\(\mathcal{B}\) 的电荷分布。我们将泊松方程代入上式,并对右侧第一项使用高斯定理(因为是全空间积分,曲面积分结果为0)得:}
	
	\vspace{-3mm}
	
	\begin{equation}
		W_{\text{int}} = \int\varphi_{1}\rho_{2}\,\mathrm{d}^{3}x
	\end{equation}
	
	\vspace{3mm}
	
	\par{\kaishu 在我们研究的系统中,多数时候带电体自身的尺寸远小于不同带电体之间的间距,即:\(a,\, b\ll|\textbf{r}|\)。此时,我们将电势为 \(\varphi_{1}\) 的带电体 \(\mathcal{A}\) 置于坐标原点附近,电势 \(\varphi_{1}\) 在带电体 \(\mathcal{B}\) 处近似为常数,且电荷分布在导体表面,全空间积分转化为导体所在空间积分:\(\mathbb{U}\to V_{2}\),我们得到:}
	
	\vspace{-2mm}
	
	\begin{equation}
		W_{\text{int}} \simeq \varphi_{1}\int_{V_{2}}\rho_{2}\,\mathrm{d}^{3}x = q_{2}\,\varphi_{1}
	\end{equation}
	
	\vspace{3mm}
	
	\par{\kaishu 式(3.2.13)就是简化以后的导体(带电体)相互作用能的表达式,适用于计算大体系中局部的小电荷体系,以及点电荷之间的相互作用。}
	
	\vspace{1mm}
	
	\par{\kaishu 下面我们关注一个特例:\textbf{点电荷的能量以及相互作用能}。根据式(3.2.13),点电荷的能量为:}
	
	\vspace{-2mm}
	
	\begin{equation}
		W = q_{i}\varphi(\textbf{x}_{i})\qquad \left(i = 1,\,2,\,\cdots,\,n\right)
	\end{equation}
	
	\vspace{3mm}
	
	\par{\kaishu 电势 \(\varphi(\textbf{x}_{i})\) 由另外的 \(n-1\) 个电荷 \(q_{j}\)(\(j = 1,\,2,\,\cdots,\, n-1\))产生,我们假设电荷 \(q_{j}\) 的位置是 \(\textbf{x}_{j}\),得到电势函数为(我们讨论的电荷均为正电荷):}
	
	\vspace{-2mm}
	
	\begin{equation}
		\varphi\left(\textbf{x}_{i}\right) = \frac{1}{4\pi\varepsilon}\sum_{j = 1}^{n-1}\frac{q_{i}}{\left|\textbf{x}_{i}-\textbf{x}_{j}\right|}
	\end{equation}
	
	\vspace{2mm}
	
	\par{\kaishu 对于第 \(i\) 个点电荷 \(q_{i}\),我们得到:}
	
	\vspace{-3mm}
	
	\begin{equation}
		W_{i} = \frac{q_{i}}{4\pi\varepsilon}\sum_{j = 1}^{n-1}\frac{q_{i}}{\left|\textbf{x}_{i}-\textbf{x}_{j}\right|}
	\end{equation}
	
	\vspace{2mm}
	
	\par{\kaishu 系统中所有电荷由于彼此间相互作用而产生的总势能为:}
	
	\begin{equation}
		W = \frac{1}{2}\sum_{i = 1}^{n}W_{i} = \frac{1}{8\pi\varepsilon}\,\sum_{i = 1}^{n}\sum_{j = 1}^{n-1}\,\frac{q_{i}q_{j}}{\left|\textbf{x}_{i}-\textbf{x}_{j}\right|}
	\end{equation}
	
	\vspace{3mm}
	
	\par{\kaishu 因为我们在计算电荷之间两两相互作用时重复了一次(比如我们同时计算了 \(q_{i}\) 对 \(q_{j}\) 的作用;和 \(q_{j}\) 对 \(q_{i}\) 的作用),所以出现了因子 \(\frac{1}{2}\)。\textcolor{blue}{注意:计算势能时不需要考虑电子自能项(式(3.2.17)已经将其消去),我们需要量子场论才能解决电子自能问题,经典电动力学中不作讨论。}}
	
	\vspace{1mm}
	
	\par{\kaishu 对于连续带电体,式(3.2.17)形式改变为(点电荷可以用狄拉克函数表示电荷密度):}
	
	\begin{equation}
		W = \frac{1}{8\pi\varepsilon}\,\iint\,\frac{\rho(\textbf{x})\,\rho(\textbf{x}')}{\left|\textbf{x}-\textbf{x}'\right|}\,\mathrm{d}^{3}x \,\mathrm{d}^{3}x'
	\end{equation}
	
	\begin{center}
		\begin{mdframed}[shadow = false]
			
			\par{\kaishu \textbf{补充:}我们并未将能量表示成电场形式 \(W = \frac{1}{2}\int\varepsilon\, E^{2}\,\mathrm{d}^{3}x\) 计算点电荷系统的能量,那么这种方法能否正确计算点电荷系统的能量呢?}
			
		\end{mdframed}
	\end{center}
	
	\begin{center}
		\begin{mdframed}[shadow = false]
			
			\par{\kaishu 不难发现:电场形式的能量总是非负的;但是我们假设式(3.2.17)中存在相反电荷,那计算出的电势能值可能是负的。这种矛盾的来源是:电场形式的能量考虑了电子“自能”项。为了说明这一点,我们只考虑两个点电荷的体系(如图3.2.3所示):}
			
			\begin{center}
				\includegraphics[width=0.5\linewidth]{C:/Users/16965/OneDrive/Pictures/Electrondynamics/Electrodynamics fig 3-2-3.pdf}
				\captionof{figure}{\text{\kaishu 两个点电荷的系统,\(P\) 为我们的观察点}}\label{fig:3-2-3}
			\end{center}
			
			\par{\kaishu 我们写出点 \(P\) 处的电场:}
			
			\vspace{-2mm}
			
			\begin{equation*}
				\textbf{E} = \frac{1}{4\pi\varepsilon}\,\frac{q_{1}\left(\textbf{x}-\textbf{x}_{1}\right)}{\left|\textbf{x}-\textbf{x}_{1}\right|^{3}}+\frac{1}{4\pi\varepsilon}\,\frac{q_{2}\left(\textbf{x}-\textbf{x}_{2}\right)}{\left|\textbf{x}-\textbf{x}_{2}\right|^{3}}
			\end{equation*}
			
			\vspace{2mm}
			
			\par{\kaishu 对应的能量密度 \(u\) 为:}
			
			\vspace{-2mm}
			
			\begin{equation*}
				u = \frac{1}{32\pi^{2}\varepsilon}\left[\frac{q_{1}^2}{\left|\textbf{x}-\textbf{x}_{1}\right|^{4}}+\frac{q_{2}^{2}}{\left|\textbf{x}-\textbf{x}_{2}\right|}+\frac{2q_{1}q_{2}\left(\textbf{x}-\textbf{x}_{1}\right)\cdot\left(\textbf{x}-\textbf{x}_{2}\right)}{\left|\textbf{x}-\textbf{x}_{1}\right|^{3}\,\left|\textbf{x}-\textbf{x}_{2}\right|^{3}}\right]
			\end{equation*}
			
			\vspace{3mm}
			
			\par{\kaishu 上式的等式右边前两项相当于点电荷“自能”贡献(本书不做讨论),我们要证明第三项结果积分后符合式(3.2.17) 对应的结果。我们写出:}
			
			\vspace{-2mm}
			
			\begin{equation*}
				W_{\text{int}} = \frac{q_{1}q_{2}}{16\pi^{2}\varepsilon}\,\int\frac{\left(\textbf{x}-\textbf{x}_{1}\right)\cdot\left(\textbf{x}-\textbf{x}_{2}\right)}{\left|\textbf{x}-\textbf{x}_{1}\right|^{3}\,\left|\textbf{x}-\textbf{x}_{2}\right|^{3}}\,\mathrm{d}^{3}x
			\end{equation*}
			
			\vspace{2mm}
			
			\par{\kaishu 我们现在令 \(\boldsymbol{\xi} = \frac{\left(\textbf{x}-\textbf{x}_{1}\right)}{\left|\textbf{x}_{1}-\textbf{x}_{2}\right|}\),\(\textbf{n} = \frac{\left(\textbf{x}_{1}-\textbf{x}_{2}\right)}{\left|\textbf{x}_{1}-\textbf{x}_{2}\right|}\) 为 \(\textbf{x}_{1}-\textbf{x}_{2}\) 方向的单位向量,上式化简为:}
			
			\vspace{2mm}
			
			\begin{equation*}
				W_{\text{int}} = \frac{1}{16\pi^{2}\varepsilon}\,\frac{q_{1}q_{2}}{\left|\textbf{x}_{1}-\textbf{x}_{2}\right|}\,\int\frac{\boldsymbol{\xi}\cdot\left(\boldsymbol{\xi}+\textbf{n}\right)}{\xi^{3}\left|\boldsymbol{\xi}+\textbf{n}\right|^{3}}\,\mathrm{d}^{3}\xi
			\end{equation*}
			
			\vspace{3mm}
			
			\par{\kaishu 下面我们需要计算的就只剩上式的积分,记:\(\phi(\boldsymbol{\xi}) = \frac{1}{\xi}\),\(\psi(\boldsymbol{\xi}) = \frac{1}{\left|\boldsymbol{\xi}+\textbf{n}\right|}\),事实上,我们取这两个函数的原因是:}
			
			\vspace{-3mm}
			
			\begin{equation*}
				\nabla\phi(\boldsymbol{\xi}) = \nabla\left(\frac{1}{|\boldsymbol{\xi}|}\right) = -\frac{\boldsymbol{\xi}}{\xi^{3}} \qquad \nabla\psi(\boldsymbol{\xi}) = -\frac{\boldsymbol{\xi}+\textbf{n}}{\left|\boldsymbol{\xi}+\textbf{n}\right|^{3}}
			\end{equation*}
			
		\end{mdframed}
	\end{center}
	
	\begin{center}
		\begin{mdframed}[shadow = false]
			
			\par{\kaishu 因此积分式满足:}
			
			\vspace{-2mm}
			
			\begin{equation*}
				\int\frac{\boldsymbol{\xi}\cdot\left(\boldsymbol{\xi}+\textbf{n}\right)}{\xi^{3}\left|\boldsymbol{\xi}+\textbf{n}\right|^{3}}\,\mathrm{d}^{3}\xi = \int\,\nabla\phi\cdot\nabla\psi\,\mathrm{d}^{3}\xi
			\end{equation*}
			
			\vspace{3mm}
			
			\par{\kaishu 由于 \(\phi\)、\(\psi\) 都是标量场,我们再次利用矢量运算公式:\(\nabla\cdot\left(\varphi\,\textbf{A}\right) = \nabla\varphi\cdot\textbf{A}+\varphi\,\nabla\cdot\textbf{A}\),有:}
			
			\vspace{-2mm}
			
			\begin{equation*}
				\nabla\phi\cdot\nabla\psi = \nabla\cdot\left(\phi\,\nabla\psi\right)-\phi\,\nabla\cdot\left(\nabla\psi\right) = \nabla\cdot\left(\phi\,\nabla\psi\right)-\phi\,\nabla^{2}\psi
			\end{equation*}
			
			\vspace{2mm}
			
			\par{\kaishu 由于 \(W_{\text{int}}\) 针对全空间 \(\mathbb{U}\) 积分有效,结合上式将积分化简为:}
			
			\vspace{-2mm}
			
			\begin{equation*}
				\int_{\mathbb{U}}\nabla\phi\cdot\nabla\psi\,\mathrm{d}^{3}\xi = \int_{\mathbb{U}}\nabla\cdot\left(\phi\,\nabla\psi\right)\,\mathrm{d}^{3}\xi-\int_{\mathbb{U}}\phi\,\nabla^{2}\psi\,\mathrm{d}^{3}\xi
			\end{equation*}
			
			\vspace{2mm}
			
			\par{\kaishu 对上式等式右边第一项使用高斯定理(对全空间曲面积分趋近于0):}
			
			\vspace{-2mm}
			
			\begin{equation*}
				\int_{\mathbb{U}}\nabla\cdot\left(\phi\,\nabla\psi\right)\,\mathrm{d}^{3}\xi = \int_{\partial\mathbb{U}}\phi\,\nabla\psi\cdot\mathrm{d}\textbf{S} = 0
			\end{equation*}
			
			\vspace{2mm}
			
			\par{\kaishu 对于第二项,我们先写出:}
			
			\vspace{-2mm}
			
			\begin{equation*}
				\nabla^{2}\psi(\boldsymbol{\xi}) = \nabla^{2}\frac{1}{|\boldsymbol{\xi}+\textbf{n}|} = -4\pi\,\delta(\boldsymbol{\xi}+\textbf{n})
			\end{equation*}
			
			\vspace{1mm}
			
			\par{\kaishu 代入积分式,我们得到:}
			
			\vspace{-2mm}
			
			\begin{equation*}
				-\int_{\mathbb{U}}\phi(\boldsymbol{\xi})\,\nabla^{2}\psi(\boldsymbol{\xi})\,\mathrm{d}^{3}\xi = \int_{\mathbb{U}}\phi(\boldsymbol{\xi})\,4\pi\,\delta(\boldsymbol{\xi}+\textbf{n})\,\mathrm{d}^{3}\xi = 4\pi\phi(-\textbf{n}) = \frac{4\pi}{|\textbf{n}|}
			\end{equation*}
			
			\vspace{2mm}
			
			\par{\kaishu 由于 \(\textbf{n}\) 是单位向量,得到:}
			
			\vspace{-2mm}
			
			\begin{equation*}
				\int\frac{\boldsymbol{\xi}\cdot\left(\boldsymbol{\xi}+\textbf{n}\right)}{\xi^{3}\left|\boldsymbol{\xi}+\textbf{n}\right|^{3}}\,\mathrm{d}^{3}\xi = 4\pi
			\end{equation*}
			
			\vspace{2mm}
			
			\par{\kaishu 此时计算得到的点电荷相互作用能是:}
			
			\vspace{-2mm}
			
			\begin{equation*}
				W_{\text{int}} = \frac{1}{16\pi^{2}\varepsilon}\,\frac{q_{1}q_{2}}{\left|\textbf{x}_{1}-\textbf{x}_{2}\right|}\cdot4\pi = \frac{1}{4\pi^{2}\varepsilon}\frac{q_{1}q_{2}}{\left|\textbf{x}_{1}-\textbf{x}_{2}\right|}
			\end{equation*}
			
			\vspace{2mm}
			
			\par{\kaishu 我们得到的结果与正文一致。实际上,这部分内容就是从点电荷模型证明:相互作用能不仅可以用电场表示,还可以用电势和电荷表示。\textcolor{blue}{也说明电场表示相互作用能更加普适。}}
			
			\vspace{2mm}
			
		\end{mdframed}
	\end{center}
	
	\par{\kaishu 最后我们讨论一道经典的点电荷例题:相同数量的正离子和负离子,电荷量为 \(\pm q\),相间排列在一条直线上(如图3.2.4所示),计算第 \(i\) 个离子受到的相互作用能(正负离子数均为 \(N\))。}
	
	\vspace{1mm}
	
	\par{\kaishu 设离子间距为 \(a\),第 \(i\) 个粒子(不妨考虑正电子)受到的静电相互作用能表示为:}
	
	\vspace{-5mm}
	
	\begin{align*}
		W_{i} &= \frac{1}{2}\, q_{i}\varphi_{i} = \frac{1}{2}\, q \,\varphi_{i}\\[5pt]
		&= \frac{q}{8\pi\varepsilon}\left[-\frac{2q}{a}+\frac{2q}{2a}-\frac{2q}{3a}+\frac{2q}{4a}-\cdots+\left(-1\right)^{n}\frac{2q}{na}\right]\\[5pt]
		&= \frac{1}{8\pi}\,\frac{2q^{2}}{\varepsilon\, a}\left[\sum_{n=1}^{N}\frac{\left(-1\right)^{n}}{n}\right]
	\end{align*}
	
	\begin{center}
		\includegraphics[width=0.7\linewidth]{C:/Users/16965/OneDrive/Pictures/Electrondynamics/Electrodynamics fig 3-2-4.pdf}
		\captionof{figure}{\text{\kaishu \(N\) 个正离子和负离子交替等距排列示意图}}\label{fig:3-2-4}
	\end{center}
	
	\par{\kaishu 取 \(N\to \infty\) 极限,上式括号内部分改写成:}
	
	\vspace{-2mm}
	
	\begin{equation*}
		\sum_{n=1}^{N}\frac{\left(-1\right)^{n}}{n} = \left[x-\frac{x^2}{2}+\frac{x^3}{3}-\cdots+\frac{\left(-1\right)^n}{n}\, x^n\right]_{x = 1}
	\end{equation*}
	
	\vspace{3mm}
	
	\par{\kaishu 根据级数知识,我们将得到:}
	
	\vspace{-2mm}
	
	\begin{equation}
		W_{i} = -\left.\frac{q^{2}}{4\pi\varepsilon a}\ln\left(1+x\right)\right|_{x = 1} = -\frac{q^{2}}{4\pi\varepsilon a}\ln 2
	\end{equation}
	
	\vspace{2mm}
	
	\par{\kaishu 第 \(i\) 个离子受到的相互作用能为负值,如果考虑负电荷,结果相同。这个模型实际上是一维链状离子晶体,静电能是马德隆(Madelung)结合能,在固体物理中相当重要。}
	
	\subsubsection{\kaishu 电容}
	
	\par{\kaishu 在3.2.2节中,我们已经“得到”了孤立带电体的电容计算公式。那对于多个导体组成的体系,又该如何描述电容呢?}
	
	\par{\kaishu 考虑 \(n\) 个导体组成的体系,每个导体都是等势体,电势为 \(\left\{\phi_{i}\right\}\) 同时导体带有的电荷量为:\(\left\{q_{i}\right\}\),我们希望建立电势和电荷量之间的关系以便统一地描述这个导体体系的性质。}
	
	\vspace{1mm}
	
	\par{\kaishu 我们在第二章第四节提到,电磁场的对称性之一就是线性叠加,静电场自然也满足\textbf{线性叠加}。我们可以说:一个导体上的电势 \(\phi_{i}\) 是导体上各电量的线性函数。运用线性代数的语言,我们写出:}
	
	\begin{equation}
		\varphi_{i} = \sum_{j = 1}^{n}P_{ij}\, q_{j}
	\end{equation}
	
	\vspace{3mm}
	
	\par{\kaishu \(P_{ij}\) 表示 \(j\) 对 \(i\) 的作用(称为电势系数),整个系统的线性关系用矩阵描述:}
	
	\begin{equation}
		\begin{pmatrix}
			\varphi_{1}\\
			\varphi_{2}\\
			\varphi_{3}\\
			\vdots\\
			\varphi_{n}
		\end{pmatrix} = 
		\begin{pmatrix}
			P_{11} & P_{12} & P_{13} & \cdots & P_{1n} \\
			P_{21} & P_{22} & P_{23} & \cdots & P_{2n} \\
			P_{31} & P_{32} & P_{33} & \cdots & P_{3n} \\
			\vdots & \vdots & \vdots &  & \vdots \\
			P_{n1} & P_{n2} & P_{n3} & \cdots & P_{nn}
		\end{pmatrix}
		\begin{pmatrix}
			q_{1}\\
			q_{2}\\
			q_{3}\\
			\vdots\\
			q_{n}
		\end{pmatrix} 
	\end{equation}
	
	\vspace{3mm}
	
	\par{\kaishu 反过来讲,我们也可以求出矩阵 \(\mathrm{P}\) 的逆矩阵 \(\mathrm{P}^{-1} = \mathrm{C}\),用电势表示电荷:}
	
	\begin{equation}
		\begin{pmatrix}
			q_{1}\\
			q_{2}\\
			q_{3}\\
			\vdots\\
			q_{n}
		\end{pmatrix} =
		\begin{pmatrix}
			C_{11} & C_{12} & C_{13} & \cdots & C_{1n} \\
			C_{21} & C_{22} & C_{23} & \cdots & C_{2n} \\
			C_{31} & C_{32} & C_{33} & \cdots & C_{3n} \\
			\vdots & \vdots & \vdots &  & \vdots \\
			C_{n1} & C_{n2} & C_{n3} & \cdots & C_{nn}
		\end{pmatrix}
		\begin{pmatrix}
			\varphi_{1}\\
			\varphi_{2}\\
			\varphi_{3}\\
			\vdots\\
			\varphi_{n}
		\end{pmatrix}
	\end{equation}
	
	\vspace{3mm}
	
	\par{\kaishu 其中:\(\mathrm{C}\) 就是这个导体系统的电容矩阵,其对角元 \(C_{ii}\) 表示第 \(i\) 个导体的电容(capacitance);非对角元 \(C_{jk}\)(\(j\ne k\))被称为感应系数(coefficients of induction)。}
	
	\par{\kaishu 下面我们证明:非对角元满足 \(C_{jk} = C_{kj}\)。利用格林互易定理,两种不同的静电荷为 \(\left\{q_{i}\right\}\) 和 \(\left\{q_{i}'\right\}\),对应的电势为 \(\left\{\varphi_{i}\right\}\) 和 \(\left\{\varphi_{i}'\right\}\):}
	
	\vspace{-3mm}
	
	\begin{equation*}
		\sum_{i = 1}^{n}q_{i}\varphi_{i}' = \sum_{i= 1}^{n}q_{i}'\varphi_{i}
	\end{equation*}
	
	\par{\kaishu 我们将电荷改写为:\(q_{i} = \sum_{j = 1}^{n}C_{ij}\,\varphi_{j}\),代入上式,有:}
	
	\begin{equation}
		\sum_{i = 1}^{n}\sum_{j = 1}^{n}\varphi_{i}'\, C_{ij}\,\varphi_{j} = \sum_{i = 1}^{n}\sum_{j = 1}^{n}\varphi_{i}\, C_{ij}\varphi_{j}'
	\end{equation}
	
	\vspace{2mm}
	
	\par{\kaishu 因为我们没有改变导体的构型,所以电容矩阵的元素不会改变。下面我们对式(3.2.22)等式左边的项,\(i\) 和 \(j\) 指标相互交换,有:}
	
	\vspace{-5mm}
	
	\begin{gather}
		\sum_{i = 1}^{n}\sum_{j = 1}^{n}\varphi_{j}'\, C_{ji}\,\varphi_{i} = \sum_{i = 1}^{n}\sum_{j = 1}^{n}\varphi_{i}\, C_{ij}\varphi_{j}'\nonumber\\[5pt]
		C_{ij} = C_{ji}
	\end{gather}
	
	\vspace{1mm}
	
	\par{\kaishu 下面我们看一道电容的例题:考虑 \(N+1\) 个理想导体等距排列在一条直线上,分别记为 \(m = 0,\,1,\,2,\,\cdots,\, N\),我们写出电容矩阵:}
	
	\begin{equation}
		C_{ij} = \begin{cases}
			C_{0} \qquad &\qquad i = j\\
			-C \qquad &\qquad \left|i-j\right| = 1\\
			0 \qquad & \qquad \left|i-j\right| > 1
		\end{cases}\nonumber
	\end{equation}
	
	\vspace{2mm}
	
	\par{\kaishu 假设 \(m = 0\) 导体在这个一维系统的边界(以在前端为例),电荷量为 \(Q\),剩余导体均不带电。现在我们需要求出在第 \(m\) 个导体上的电势 \(\varphi_{m}\),取 \(C\ll C_{0}\) 近似。}
	
	\vspace{1mm}
	
	\par{\kaishu 我们先列出导体电势满足的条件:}
	
	\vspace{-2mm}
	
	\begin{equation}
	\begin{cases}
		Q = C_{0}\,\varphi_{0}-C\varphi_{1}\\
		0 = C_{0}\,\varphi_{m}-C\left(\varphi_{m+1}+\varphi_{m-1}\right)\\
		0 = C_{0}\,\varphi_{N}-C\,\varphi_{N-1}	
	\end{cases}\qquad m\ne 0,\, N
	\end{equation}
	
	\vspace{2mm}
	
	\par{\kaishu 我们根据上面第三式得到:\(\varphi_{N} = \frac{C}{C_{0}}\,\varphi_{N-1}\),可以代入第二式反向递推得到(\(m = N-1\)):}
	
	\begin{equation}
		\left(\frac{C_{0}}{C}-\frac{C}{C_{0}}\right)\varphi_{N-2} = \varphi_{N-1}\qquad\Rightarrow\qquad\varphi_{N-1}\simeq\frac{C}{C_{0}}\,\varphi_{N-2}
	\end{equation}
	
	\vspace{3mm}
	
	\par{\kaishu 我们使用了 \(C\ll C_{0}\) 近似,对于第 \(m\) 个导体,根据上面的结果有:}
	
	\begin{equation*}
		\varphi_{m} = \frac{C}{C_{0}}\,\varphi_{m-1} \qquad\Rightarrow\qquad \varphi_{m} = \left(\frac{C}{C_{0}}\right)^{m}\varphi_{0}
	\end{equation*}
	
	\vspace{3mm}
	
	\par{\kaishu 对于 \(\varphi_{0}\),有 \(Q = C_{0}\,\varphi_{0}\),所以我们最终得到:\(\varphi_{m} = \frac{Q}{C_{0}}\left(\frac{C}{C_{0}}\right)^{m}\)。}
	
	\vspace{1mm}
	
	\par{\kaishu 最后我们来讨论如何用电容表示能量,将 \(q_{i} = \sum_{j = 1}^{n}C_{ij}\,\varphi_{j}\) 代入式(3.2.9)中(\(W = \frac{1}{2}\sum_{i=1}^{n}q_{i}\varphi_{i}\)),得到下面的二次型多项式(体系导体数量为 \(n\)):}
	
	\begin{equation}
		W = \frac{1}{2}\sum_{i=1}^{n}\sum_{j=1}^{n}C_{ij}\,\varphi_{i}\varphi_{j}
	\end{equation}
	
	\vspace{2mm}
	
	\par{\kaishu 用式(3.2.9)计算得到的能量是恒正的,对于第 \(i\) 个导体有 \(W = \frac{1}{2}C_{ii}\varphi_{i}^{2}>0\),进而有 \(C_{ii}>0\),故矩阵 \(\mathrm{C}\) 是正定矩阵。}
	
	\par{\kaishu \textcolor{blue}{我们无法通过纯粹的数学手段证明 \(C_{jk}<0\)(\(j\ne k\) 下同),但是可以考虑:导体 \(j\) 携带的电荷发出的电场终止于另外的导体或者无穷远处,后者能容纳更多电能。如果系统满足:\(j\) 导体与剩余的导体以外没有电场发散到无穷远处(例如理想无穷大平板电容器),则有:\(\sum_{k = 1}^{n}C_{jk} = 0\),否则一定有:\(\sum_{k = 1}^{n}C_{jk}>0\)。根据这一结论,两体系统中 \(C_{12}<0\),用归纳法分析可知:\(C_{jk}<0\)。}}
	
	\vspace{1mm}
	
	\par{\kaishu 对于单独的导体,我们可以用导体本身的电容表示其静电能。假定导体的电荷为 \(Q\)(或者电势为 \(\varphi\)),电容为 \(C\),导体能量可以表示为:}
	
	\begin{equation}
		W = \frac{Q^{2}}{2C} = \frac{1}{2}C\varphi^{2}
	\end{equation}
	
	\subsubsection{\kaishu 作用在导体上的力}
	
	\par{\kaishu 最后,我们简单讨论一下作用在导体上的力。电场中的动量流密度由麦克斯韦应力张量给出:}
	
	\vspace{-2mm}
	
	\begin{equation*}
		T_{ij} = \frac{1}{2}\varepsilon\, E^{2}\,\delta_{ij}-\varepsilon\, E_{i}E_{j}
	\end{equation*}
	
	\vspace{2mm}
	
	\par{\kaishu 我们已经知道:在导体内部的电场满足:\(\textbf{E}_{\text{in}} = 0\);然后导体外表面的场只有法向分量 \(\textbf{E}_{n}\),导体表面单位面积的受力为:}
	
	\vspace{-3mm}
	
	\begin{equation}
		\textbf{F}_{n} = \frac{1}{2}\varepsilon\, E_{n}^{2}\,\boldsymbol{e}_{n}
	\end{equation}
	
	\vspace{3mm}
	
	\par{\kaishu 引入面电荷密度 \(\sigma\),导体表面的场写成面电荷的形式:\(\textbf{E}_{n} = \frac{\sigma}{\varepsilon}\boldsymbol{e}_{n}\),代入式(3.2.28)得:}
	
	\begin{equation}
		\textbf{F}_{n} = \frac{1}{2\varepsilon}\,\sigma^{2}\boldsymbol{e}_{n}
	\end{equation}
	
	\vspace{2mm}
	
	\par{\kaishu 因此,导体表面受到“负压力”作用,这个力指向导体表面的外法线方向,数值等于电场能量密度。作用在导体上的总力 \(\textbf{F}\),将式(3.2.28)的力对导体表面积分求出:}
	
	\begin{equation}
		\textbf{F} = \oint_{S}\frac{1}{2}\,\varepsilon E^{2}\,\mathrm{d}S
	\end{equation}
	
	\vspace{2mm}
	
	\par{\kaishu 如果作用在导体上的总力和总力矩等于零,则导体在电场内保持静止,于是与导体形变有关的效应(称为\textbf{电致伸缩})提到首要地位。作用在导体表面上的力引起导体的形状和体积改变。这时由于力的拉伸特性,导体的体积增大。但我们要完全确定形变,必须在已知导体表面力的分布条件下求解弹性理论方程。如果我们只对体积变化感兴趣,问题的解法就要简单得多。}
	
	\par{\kaishu 为此必须考虑到如果形变很小(实际上电致伸缩就是这种情况),则形状变化对体积变化所产生的影响是二级小量效应。因此,在初级近似下可以把体积变化看作形状不发生变化的形变,亦即可以把它看成由某种有效剩余压强 \(\Delta P\) 作用下的体膨胀,这种有效剩余压强均匀分布于物体的表面,代替了式(3.2.28)式那样的精确分布。
	
	将 \(\Delta P\) 乘上物体的各向伸长系数(均匀膨胀系数),就得到体积的相对变化。此处压强 \(\Delta P\) 由导体的电能 \(W\) 对其体积的导数 \(\Delta P = -\frac{\partial W}{\partial V}\) 确定。}
	
	\vspace{1mm}
	
	\par{\kaishu 假设引起形变的电场由带电的导体本身所产生的,能量用电容表示:\(W = \frac{q^{2}}{2C}\),压强为:}
	
	\begin{equation}
		\Delta P = -\frac{q^{2}}{2}\,\frac{\partial}{\partial t}\left(\frac{1}{C}\right)
	\end{equation}
	
	\vspace{3mm}
	
	\par{\kaishu 当导体形状不变时,导体的电容与其线度成正比(例如半径为 \(R\) 的导体球,其电容为 \(4\pi\varepsilon R\)),可以写成 \(C\propto V^{\frac{1}{3}}\),因此我们有:}
	
	\vspace{-2mm}
	
	\begin{equation}
		\Delta P \,\propto\,\frac{q^{2}}{6CV} = \frac{q\varphi}{6V}
	\end{equation}
	
	\vspace{3mm}
	
	\par{\kaishu 如果未带电的导体处于均匀的外电场 \(\textbf{E}\) 中,我们先写出导体的电介质张量 \(\varepsilon_{ij}\),此时整个导体总能量为 \(W = \frac{1}{2}V\varepsilon_{ij}E_{i}E_{j}\),这种情况下拉伸压强为:}
	
	\begin{equation}
		\Delta P = \frac{1}{2}\varepsilon_{ij}E_{i}E_{j}
	\end{equation}
	
	\vspace{1mm}
	
	\begin{center}
		\begin{mdframed}[shadow = false]
			
		\par{\kaishu \textbf{补充:}阶跃函数的简单应用
		
		首先我们简单介绍一种特殊函数——\textbf{阶跃函数},我们先定义单位阶跃函数:}
		
		\vspace{-2mm}
		
		\begin{equation*}
			\boldsymbol{\theta}(x) = \begin{cases}
				0 \qquad& t<0 \\
				1 \qquad& t\ge0
			\end{cases}
		\end{equation*}
		
		\vspace{2mm}
		
		\end{mdframed}
	\end{center}
	
	\begin{center}
		\begin{mdframed}[shadow = false]
			
		\par{\kaishu 显然在黎曼(Riemann)条件下,单位阶跃函数并不是初等函数。阶跃函数和我们介绍过的狄拉克函数满足特殊的关系,即:}
		
		\begin{equation*}
			\frac{\mathrm{d}\boldsymbol{\theta}(x)}{\mathrm{d}x} = \delta(x)\qquad\&\qquad\int\delta(x)\,\mathrm{d}x = \boldsymbol{\theta}(x)+\text{const}
		\end{equation*}
		
		\vspace{2mm}
		
		\par{\kaishu \textcolor{blue}{在傅里叶变换和更复杂的物理中,阶跃函数有更大的威力,此处只是简单的引入。}
			
		那么导体表面受力是如何联系阶跃函数的呢?是因为导体边界内部电场为0,边界外部“突然”出现电场,可以用阶跃函数描述。电场在导体边界有更微观复杂的行为,此处作用太小无需考虑,就用阶跃函数粗糙地描绘。
		
		考虑局域导体,其法向定义为 \(z\),电荷体密度可以表示为:\(\rho_{z} = \sigma_{0}\delta(z)\),在 \(z\) 方向对 \(\rho_{z}\) 积分,就得到了理想导体的电场分布:}
		
		\vspace{-3mm}
		
		\begin{equation*}
			\textbf{E}(z) = \frac{\sigma_{0}}{\varepsilon}\,\boldsymbol{\theta}(z)\boldsymbol{e}_{z}
		\end{equation*}
		
		\vspace{2mm}
		
		\par{\kaishu 接下来我们用静电力就可以计算结果:}
		
		\vspace{-5mm}
		
		\begin{align*}
			|\textbf{F}| &= \int\rho\,\textbf{E}\,\mathrm{d}\tau = \frac{1}{\varepsilon}\int\mathrm{d}S\,\int\sigma_{0}^{2}\,\delta(z)\,\boldsymbol{\theta}(z)\,\mathrm{d}z\\[6pt]
			&= \frac{1}{\varepsilon}\int\sigma_{0}^{2}\,\mathrm{d}S\int_{-\infty}^{+\infty}\delta(z)\,\boldsymbol{\theta}(z)\,\mathrm{d}z		
		\end{align*}
		
		\vspace{2mm}
		
		\par{\kaishu 下面我们只需要用数学工具计算积分 \(\int_{-\infty}^{+\infty}\delta(z)\,\boldsymbol{\theta}(z)\,\mathrm{d}z\),如果使用最初的定义计算则此积分无法求出,我们需要用傅里叶变换(Fourier transformation)处理之(如果读者对傅里叶变换有了解则可以跳过)。}
		
		\vspace{1mm}
		
		\par{\kaishu 为便于傅里叶变换操作,我们构造一个辅助函数逼近阶跃函数,如下所示:}
		
		\vspace{-2mm}
		
		\begin{equation*}
			f(t) = \begin{cases}
				0\qquad& t<0\\
				e^{-at}\qquad& t\ge0
			\end{cases}\qquad\Rightarrow\qquad
			\lim_{a\to0^{+}}f(t) = u(t)
		\end{equation*}
		
		\vspace{2mm}
		
		\par{\kaishu 考虑辅助函数的傅里叶变换:}
		
		\vspace{-4mm}
		
		\begin{equation*}
			\mathcal{F}[f(t)] = \int_{0}^{+\infty}e^{-at}e^{-i\omega t}\,\mathrm{d}t = \frac{1}{a+i\omega}
		\end{equation*}
		
		\vspace{2mm}
		
		\par{\kaishu 辅助函数取 \(a\to0^+\) 的极限就得到阶跃函数,那么我们对辅助函数的傅里叶变换取极限也能得到阶跃函数的傅里叶变换,即:}
		
		\vspace{-4mm}
		
		\begin{equation*}
			\mathcal{F}[\boldsymbol{\theta}(t)] = \lim_{a\to0^+}\mathcal{F}[f(t)] = \lim_{a\to0^+}\frac{1}{a+i\omega} = \lim_{a\to0^+}\frac{a}{a^2+\omega^2}-i\lim_{a\to0^+}\frac{\omega}{a^2+\omega^2}
		\end{equation*}
		
		\vspace{2mm}
		
		\par{\kaishu 考虑实部的极限,有:}
		
		\vspace{-3mm}
		
		\begin{equation*}
			\lim_{a\to0^+}\frac{a}{a^2+\omega^2} = \begin{cases}
				0 \qquad& \omega\ne0 \\
				\infty\qquad& \omega = 0
			\end{cases}\qquad\Rightarrow\quad C\delta(\omega)
		\end{equation*}
		
		\end{mdframed}
	\end{center}
		
	\begin{center}
		\begin{mdframed}[shadow = false]
			
		\par{\kaishu 下面我们利用 \(\delta(\omega)\) 的性质求出常数 \(C\):}	
		
		\vspace{-3mm}
		
		\begin{equation*}
			\int_{-\infty}^{+\infty}\frac{a}{a^2+\omega^2}\,\mathrm{d}\omega = \pi = C = C\int_{-\infty}^{+\infty}\delta(\omega)\,\mathrm{d}\omega
		\end{equation*}
		
		\vspace{2mm}
		
		\par{\kaishu 我们得到:\(\lim_{a\to0^+}\frac{a}{a^2+\omega^2} = \pi\delta(\omega)\),虚部的极限为 \(\lim_{a\to0^+}\frac{\omega}{a^2+\omega^2} = \frac{1}{\omega}\),整合以上的内容得到:}
		
		\begin{equation*}
			\mathcal{F}[\boldsymbol{\theta}] = \pi\delta(\omega)-\frac{i}{\omega}
		\end{equation*}
		
		\vspace{2mm}
		
		\par{\kaishu 下面我们再利用得到的结果,计算其傅里叶反变换:}
		
		\vspace{-6mm}
		
		\begin{align*}
			\mathcal{F}^{-1}\left[\pi\delta(\omega)-\frac{i}{\omega}\right] &= \frac{1}{2\pi}\int_{-\infty}^{+\infty}\left[\pi\delta(\omega)-\frac{i}{\omega}\right]\, e^{i\omega t}\,\mathrm{d}\omega\\[6pt]
			&= \frac{1}{2}+\frac{1}{\pi}\int_{0}^{+\infty}\frac{\sin\omega t}{\omega}\,\mathrm{d}\omega = \boldsymbol{\theta}(t)
		\end{align*}
		
		\vspace{2mm}
		
		\par{\kaishu 于是我们得到:\(\boldsymbol{\theta}(t) = \frac{1}{2}+\frac{1}{\pi}\int_{0}^{+\infty}\frac{\sin\omega t}{\omega}\,\mathrm{d}\omega\),根据积分的值,我们得到:}
		
		\begin{equation*}
			\boldsymbol{\theta}(t) = \begin{cases}
				\,0\qquad& t<0\\[4pt]
				\,\frac{1}{2}\qquad& t = 0\\[4pt]
				\,1\qquad& t>0
			\end{cases}\qquad\Leftarrow\qquad \boldsymbol{\theta}(t) = \frac{1}{2}+\frac{1}{\pi}\int_{0}^{+\infty}\frac{\sin\omega t}{\omega}\,\mathrm{d}\omega
		\end{equation*}
		
		\vspace{2mm}
		
		\par{\kaishu 现在我们得到了 \(\boldsymbol{\theta}(t)\) 的新定义,将上式代入 \(|\textbf{F}|\) 中:}
		
		\vspace{-1mm}
		
		\begin{equation*}
			|\textbf{F}| = \frac{1}{\varepsilon}\int\sigma_{0}^{2}\,\mathrm{d}S\cdot\frac{1}{2} = \int\frac{\sigma_{0}^{2}}{2\varepsilon}\,\mathrm{d}S
		\end{equation*}
		
		\vspace{2mm}
		
		\par{\kaishu 于是我们再次得到了单位面积的受力:\(\textbf{F}_{S} = \frac{\sigma_{0}^2}{2\varepsilon}\,\boldsymbol{e}_{n}\)。}
			
		\end{mdframed}
	\end{center}
	
	\subsection{\kaishu 静电体系的稳定性问题}
	
	\par{\kaishu 在前面研究静电体系的过程中,我们并没有考虑导体构型稳定性的问题,下面将较为系统地考察静电体系的稳定性。静电体系的稳定包括两个问题:}
	
	\vspace{-3mm}
	
	\par{\kaishu \begin{enumerate}
			
	\item 静电体系处于某一个假定的构型下是否稳定?
	
	\vspace{-3mm}
	
	\item 稳定时导体体系中电荷分布以及导体的构型满足的条件是什么?
			
	\end{enumerate}}

	\vspace{-3mm}
	
	\par{\kaishu 我们通过研究静电体系的能量来确定这两个问题,静电体系(真空中)的总静电能为:}
	
	\begin{equation}
		W = \frac{1}{2}\int\varepsilon_{0}E^{2}\,\mathrm{d}^3x = \frac{1}{2}\int\rho(\textbf{x})\varphi(\textbf{x})\,\mathrm{d}^{3}x
	\end{equation}
	
	\vspace{3mm}
	
	\par{\kaishu 体系的状态可以由电荷分布 \(\rho(\textbf{x})\) 或电势 \(\varphi(\textbf{x})\) 描述,两个函数实际上是对应的。给定不同的电荷分布 \(\rho(\textbf{x})\),体系就具有不同的能量,我们可以将能量 \(W\) 视为 \(\rho(\textbf{x})\) 的泛函 \(\mathcal{W}[\rho(\textbf{x})]\)。}
		
	\par{\kaishu 现在,体系的稳定性问题转化成:\textcolor{blue}{对于什么样的电荷分布,体系能量取极小值?}我们使用变分法,对电荷分布 \(\rho(\textbf{x})\) 作变分,有:}
	
	\vspace{-3mm}
	
	\begin{equation}
		\rho(\textbf{x}) \to \rho(\textbf{x})+\delta\rho\quad\Rightarrow\quad \frac{\delta\mathcal{W}}{\delta\rho} = 0
	\end{equation}
	
	\vspace{3mm}
	
	\par{\kaishu 接下来我们的讨论基于式(3.3.2)。电荷的变分 \(\delta\rho\) 存在两种类型:}
	
	\vspace{-3mm}
	
	\par{\kaishu \begin{enumerate}
			
		\item 导体位置不动,电荷在导体上的重新分布;
		
		\vspace{-3mm}
		
		\item 导体位置变化,同时也会引起单个导体上电荷的重新分布。
		 
	\end{enumerate}
	
	\vspace{-3mm}
	
	\par{\kaishu 下面我们将讨论这两种情况,并给出两个研究经典体系稳定性的定理。}
	
	\subsubsection{\kaishu 汤姆孙定理(Thomson's theorem)}
	
	\par{\kaishu 考虑简单的情况:假设每个导体都是静止不动的(被固定在特定位置),但是电荷在每个导体上可以重新分布,扰动满足:单个导体总电荷守恒,即:}
	
	\vspace{-2mm}
	
	\begin{equation}
		\int_{V_{i}}\delta\rho_{i}\,\mathrm{d}V_{i} = \delta Q_{i}\equiv0
	\end{equation}
	
	\vspace{2mm}
	
	\par{\kaishu \(Q_{i}\) 为第 \(i\) 个导体上的总电荷,式(3.3.3)说明:\(Q_{i}\) 是守恒量。假设电荷分布 \(\rho\) 对应的电场分布为 \(\textbf{E}\);\(\rho+\delta\rho\) 对应的电场分布为 \(\textbf{E}+\delta\textbf{E}\),我们考察电荷分布扰动引起的能量变化为:}
	
	\vspace{-6mm}
	
	\begin{align}
		\delta\mathcal{W} &= \mathcal{W}[\rho+\delta\rho]-\mathcal{W}[\rho]\nonumber\\[6pt]
		&= \frac{\varepsilon_{0}}{2}\left[\int\left(E^{2}+2\textbf{E}\cdot\boldsymbol{\delta}\textbf{E}+\delta E^{2}\right)\mathrm{d}^3x-\int E^{2}\,\mathrm{d}^3x\right]\nonumber\\[6pt]
		&= \varepsilon_{0}\int\textbf{E}\cdot\boldsymbol{\delta}\textbf{E}\,\mathrm{d}^3x+\mathcal{O}(\delta E^{2})
	\end{align}
	
	\vspace{2mm}
	
	\par{\kaishu \(\textbf{E}\) 可以用电势代替:\(\textbf{E} = -\nabla\varphi\),而且电场变分 \(\boldsymbol{\delta}\textbf{E}\) 也满足泊松方程:}
	
	\vspace{-2mm}
	
	\begin{equation*}
		\nabla\cdot\boldsymbol{\delta}\textbf{E} = \frac{\delta\rho}{\varepsilon_{0}}
	\end{equation*}
	
	\vspace{3mm}
	
	\par{\kaishu 将上面的结果代入式(3.3.4),忽略二阶变分项,得到:}
	
	\vspace{-5mm}
	
	\begin{align}
		\delta\mathcal{W} &= -\varepsilon_{0}\int\nabla\varphi\cdot\boldsymbol{\delta}\textbf{E}\,\mathrm{d}^3x\nonumber\\[6pt]
		&\overset{a}{=} -\varepsilon_0\int\nabla\cdot\left(\varphi\,\boldsymbol{\delta}\textbf{E}\right)\,\mathrm{d}^3x+\varepsilon_0\int\varphi\,\nabla\cdot\delta\textbf{E}\,\mathrm{d}^3x\nonumber\\[6pt]
		&\overset{b}{=}\varepsilon_0\oint_{S}\varphi\,\boldsymbol{\delta}\textbf{E}\cdot\mathrm{d}\textbf{S}+\int\varphi\,\delta\rho\,\mathrm{d}^3x\nonumber\\[6pt]
		&\overset{c}{=}\sum_{i = 1}^{N}\int\varphi(\textbf{x}_i)\delta\rho(\textbf{x}_i)\,\mathrm{d}^3x
	\end{align}
	
	\vspace{2mm}
	
	\par{\kaishu 其中:a处等号使用了分部积分法;b处等号第一项使用了高斯定理,第二项代入了泊松方程的结果;c处我们将 \(\delta\rho\) 拆成对 \(N\) 个导体的求和(曲面积分项因为考虑全空间所以为0)。}
	
	\vspace{5mm}
	
	\par{\kaishu 针对这个给定了限制条件(单个导体总电荷守恒),同时求出了能量变分 \(\delta\mathcal{W}\) 的体系,我们可以使用拉格朗日数乘法(Lagrange multiplier method)计算这个问题。\textcolor{blue}{实际上在经典力学求解约束力的问题中,我们已经使用过这种思想。}}
	
	\begin{center}
		\begin{mdframed}[shadow = false]
		
		\par{\kaishu \textbf{补充:}泛函拉格朗日数乘法}
		
		\par{\kaishu 实际上我们在上面使用的拉格朗日数乘法不在数学分析的范畴,我们针对变分 \(\delta\mathcal{W}\) 和 \(\delta Q\),需要用到泛函拉格朗日数乘法。此处我们给出简要的证明(不感兴趣/已经学会的读者可以跳过):}
		
		\vspace{1mm}
		
		\par{\kaishu 考虑 \(X\)、\(Y\) 为实巴纳赫(Banach)空间,设 \(F: X\to\mathbb{R}\);\(G: X\to Y\)。两者都是在开集 \(\Omega\subset X\) 上的 \(C^2\) 映射,同时我们定义约束流形(Constrained manifold):}
		
		\vspace{-3mm}
		
		\begin{equation*}
			\mathcal{M} = \left\{x\in\Omega: G(x) = 0\right\}
		\end{equation*}
		
		\vspace{2mm}
		
		\par{\kaishu 我们取 \(x_{0}\in\mathcal{M}\) 且满足正则性:即 \(\boldsymbol{D}G(x_{0}): X\to Y\) 为满射(像为整个 \(Y\) 空间),并且它的核 \(\ker\boldsymbol{D}G(x_{0})\) 是补空间良好的(意思是我们可以使用隐函数定理)。假设 \(x_{0}\) 是 \(F\) 在 \(\mathcal{M}\) 上的局部极值点(极小或者极大)。
			
		我们要证明:存在唯一的拉格朗日乘数(Lagrange multiplier)\(\lambda\in Y^{*}\)(\(Y\) 的对偶空间)使得:}
		
		\vspace{-3mm}
		
		\begin{equation*}
			\boldsymbol{D}F(x_0) = \boldsymbol{D}G(x_0)^{*}[\lambda]
		\end{equation*}
		
		\vspace{2mm}
		
		\par{\kaishu 且 \(G(x_0) = 0\);或者可以说:对于任意 \(v\in X\) 有:}
		
		\vspace{-3mm}
		
		\begin{equation*}
			\boldsymbol{D}F(x_0)[v] = \lambda\left(\boldsymbol{D}G(x_0)[v]\right)
		\end{equation*}
		
		\vspace{2mm}
		
		\par{\kaishu 上面的 \(\boldsymbol{D}G(x_0)^*:Y^*\to X^*\) 是 \(\boldsymbol{D}G(x_0)\) 的对偶(伴随)算子。\textcolor{blue}{在希尔伯特空间(Hilbert space)中我们可以将 \(\lambda\in Y\) 写成更直观的内积形式。}
			
		\vspace{1mm}
		
		\textbf{证明必要性:}\(\boldsymbol{D}G(x_0)\) 满射并且 \(\boldsymbol{D}G(x_0)\) 为连续线性算子,巴纳赫空间中的隐函数定理保证:
		
		对于 \(X\),存在分解:}
		
		\vspace{-4mm}
		
		\begin{equation*}
			X = \ker\boldsymbol{D}G(x_0)\oplus Z
		\end{equation*}
		
		\vspace{1mm}
		
		\par{\kaishu 其中:\(Z\) 与 \(\ker\boldsymbol{D}G(x_0)\) 为互补闭子空间。以及存在邻域 \(U\)(\(x_0\) 的邻域)和映射 \(\phi: W\to Z\),其中:\(W\) 是 \(\ker\boldsymbol{D}G(x_0)\) 中的一个小邻域;使得任意的 \(x\in U\cap\mathcal{M}\) 可以被唯一地写成:}
		
		\vspace{-3mm}
		
		\begin{equation*}
			x = u+\phi(u)\qquad u\in W\subset\ker\boldsymbol{D}G(x_0)
		\end{equation*}
		
		\vspace{1mm}
		
		\par{\kaishu 因为 \(x_0\) 是 \(F\) 在 \(\mathcal{M}\) 上的局部极值,则 \(u_0\) 为 \(\widetilde{F}\) 的局部极值,因此对 \(\widetilde{F}\) 的导数有:}
		
		\vspace{-3mm}
		
		\begin{equation*}
			\boldsymbol{D}\widetilde{F}(u_0) = 0
		\end{equation*}
		
		\vspace{1mm}
		
		\par{\kaishu 我们使用导数的链式法则:}
		
		\vspace{-4mm}
		
		\begin{equation*}
			\boldsymbol{D}\widetilde{F}(u_0)[h] = \boldsymbol{D}F(x_0)\left[h+D\phi(u_0)[h]\,\right]\qquad\forall\, h\in\ker\boldsymbol{D}G(x_0)
		\end{equation*}
		
		\vspace{1mm}
		
		\par{\kaishu 所以 \(\boldsymbol{D}F(x_0)\) 对于所有的 \(v\in\ker\boldsymbol{D}G(x_0)\) 为0,因为 \(h+\boldsymbol{D}\phi(u_0)[h]\) 覆盖了 \(\mathcal{M}\) 在 \(x_0\) 的切空间。}
		
		\end{mdframed}
	\end{center}
	
	\begin{center}
		\begin{mdframed}[shadow = false]
			
		\par{\kaishu 或者我们可以说,切空间 \(\boldsymbol{T}_{x_0}\mathcal{M} = \ker\boldsymbol{D}G(x_0)\)(由隐函数定理保证),因此:}
		
		\vspace{-3mm}
		
		\begin{equation*}
			\boldsymbol{D}F(x_0)[v] = 0\qquad\forall\, v\in\ker\boldsymbol{D}G(x_0)
		\end{equation*}
		
		\vspace{1mm}
		
		\par{\kaishu 下面我们延拓到整个 \(X\) 空间,由线性代数的一般引理:如果连续线性泛函 \(\mathscr{L}\in X^*\) 在 \(\ker\boldsymbol{D}G(x_0)\) 上为0,则 \(\mathscr{L}\) 可以被唯一地表示为某个 \(\lambda\in Y^*\) 经过伴随算子以后作用的形式,即:存在唯一的 \(\lambda\in Y^*\),使得:}
		
		\vspace{-5mm}
		
		\begin{equation*}
			\mathscr{L} = \boldsymbol{D}G(x_0)^*[\lambda]
		\end{equation*}
		
		\vspace{1mm}
			
		\par{\kaishu 因为 \(\boldsymbol{D}G(x_0)\) 满射,伴随 \(\boldsymbol{D}G(x_0)^*:Y^*\to X^*\) 为单射,且像空间 \(\Im \boldsymbol{D}G(x_0)^*\) 恰好是:\(\ker\boldsymbol{D}G(x_0)\) 上为0的那些线性泛函的子空间 \(\left\{\mathscr{L}\in X^*:\left.\mathscr{L}\right|_{\ker\boldsymbol{D}G(x_0)} = 0\right\}\)。因此令 \(\mathscr{L}:=\boldsymbol{D}F(x_0)\),由于 \(\mathscr{L}\) 在 \(\ker\boldsymbol{D}G(x_0)\) 上为0,所以存在唯一的 \(\lambda\in Y^*\) 满足 \(\boldsymbol{D}F(x_0) = \boldsymbol{D}G(x_0)^*[\lambda]\)。至此我们得到了所需要的拉格朗日乘子关系。}
		
		\vspace{1mm}
		
		\par{\kaishu \textbf{充分性:}类似数学分析中对拉格朗日乘数的证明,定义二次型:}
		
		\vspace{-3mm}
		
		\begin{equation*}
			H[v,v]:= \boldsymbol{D}^2F(x_0)[v,v]-\lambda\left(\boldsymbol{D}^2G(x_0)[v,v]\right)\qquad v\in X
		\end{equation*}
		
		\vspace{1mm}
		
		\par{\kaishu 在必要性的证明中,我们已经使用了隐函数定理得到了 \(\widetilde{F}\)。还根据链式法则计算结果,定义有界线性算子 \(J:\ker DG(x_0)\to X\):}
		
		\vspace{-3mm}
		
		\begin{equation*}
			J(h):= h+\boldsymbol{D}\phi(x_0)[h] \qquad \Im J = \boldsymbol{T}_{x_0}\mathcal{M}
		\end{equation*}
		
		\vspace{1mm}
		
		\par{\kaishu 我们知道:对于 \(J =\mathbbm{1} +\boldsymbol{D}\phi(0)\) 的逆也是有界的,因此存在常数 \(c>0\) 使得对于所有 \(h\in K\)(\(K:= \ker\boldsymbol{D}G(x_0)\))有:}
		
		\vspace{-5mm}
		
		\begin{equation*}
			\lVert Jh\rVert_{X}\ge c \,\lVert h\rVert_{K}
		\end{equation*}
		
		\vspace{1mm}
		
		\par{\kaishu 现在我们需要确定 \(x=x_0\) 是 \(F\) 在 \(\mathcal{M}\) 上的局部极小(或者极大);假设 \(H\) 在核空间严格正定,则存在 \(\alpha>0\) 满足:}
		
		\vspace{-5mm}
		
		\begin{equation*}
			H[v,v]\ge\alpha\,\lVert v\rVert^{2}_{X} \qquad\forall\, v\in\boldsymbol{T}_{x_0}\mathcal{M}
		\end{equation*}
		
		\vspace{1mm}
		
		\par{\kaishu 下面我们对 \(\widetilde{F}\) 求二阶导数:\(\boldsymbol{D}^{2}\widetilde{F}(x_0)\),运用 \(G(x(u)) = 0\) 求二阶导化简,代入最终得到:}
		
		\vspace{-3mm}
		
		\begin{equation*}
			\boldsymbol{D}^{2}\widetilde{F}(x_0)[h,h] = H[Jh,Jh]
		\end{equation*}
		
		\vspace{1mm}
		
		\par{\kaishu \textcolor{blue}{你可以认为这就是二次型的构造过程。}现在由于 \(H\) 正定,有:}
		
		\vspace{-3mm}
		
		\begin{equation*}
			\boldsymbol{D}^{2}\widetilde{F}(x_0)[h,h] = H[Jh,Jh]\ge\alpha\,\lVert Jh\rVert_{X}^{2}\ge\alpha c^{2}\lVert h\rVert_{K}^{2}
		\end{equation*}
		
		\vspace{1mm}
		
		\par{\kaishu 对 \(\widetilde{F}\) 渐近展开有:}
		
		\vspace{-4mm}
		
		\begin{equation*}
			\widetilde{F}(h) = \widetilde{F}(x_0)+\frac{1}{2}\boldsymbol{D}^2\widetilde{F}(x_0)[h,h]+\mathcal{O}\left(\lVert h\rVert_{K}^2\right)
		\end{equation*}
		
		\vspace{2mm}
		
		\par{\kaishu 我们可以找到正实数 \(\delta>0\),使得 \(\lVert h\rVert_{K}<\delta\) 时 \(\mathcal{O}\left(\lVert h\rVert_{K}^2\right)\le\frac{1}{4}\alpha c^2\,\lVert h\rVert_{K}^2\),那么有:}
			
		\end{mdframed}
	\end{center}
	
	\begin{center}
		\begin{mdframed}[shadow = false]
		
		\begin{equation*}
			\widetilde{F}(h)-\widetilde{F}(x_0)\ge\frac{1}{2}\alpha c^2\,\lVert h\rVert_{K}^2-\frac{1}{4}\alpha c^2\,\lVert h\rVert_{K}^2 = \frac{1}{4}\alpha c^2\,\lVert h\rVert_{K}^2>0
		\end{equation*}
		
		\vspace{1mm}
		
		\par{\kaishu 因此在邻域内总有:\(\widetilde{F}(h)>\widetilde{F}(x_0)\),\(u = 0\) 是严格的局部极小点,由于 \(x = u+\phi(u)\),回到变量 \(x\)。这说明:\(x = x_0\) 是 \(F\) 在 \(\mathcal{M}\) 上的严格局部极小点。
		
		如果 \(H\) 在核空间严格负定,那么我们用相同的方法(\(\alpha<0\))可以证明严格局部极大的情况。}
		
		\par{\kaishu 至此我们完成了泛函条件下拉格朗日乘数法的证明。}
		
		\vspace{1mm}
		
		\end{mdframed}
	\end{center}
	
	\par{\kaishu 此处引入拉格朗日不定乘数 \(\lambda_{i}\),我们有:}
	
	\vspace{-6mm}
	
	\begin{align}
		0 &= \delta\mathcal{W}-\sum_{i = 1}^{N}\lambda_{i}\,\delta Q_{i}\nonumber\\[6pt]
		&= \sum_{i = 1}^{N}\int\varphi(\textbf{x}_i)\,\delta\rho(\textbf{x}_{i})\,\mathrm{d}^3x-\sum_{i = 1}^{N}\int\delta\rho(\textbf{x}_i)\,\mathrm{d}^3x\nonumber\\[6pt]
		&= \sum_{i = 1}^{N}\int\left[\varphi(\textbf{x}_{i})-\lambda_{i}\right]\,\delta\rho(\textbf{x}_{i})\,\mathrm{d}^3x
	\end{align}
	
	\vspace{2mm}
	
	\par{\kaishu 引入拉格朗日不定乘数以后,我们认为 \(\delta\rho(\textbf{x}_{i})\) 相互独立,根据式(3.3.6),我们有:}
	
	\vspace{-3mm}
	
	\begin{equation}
		\varphi(\textbf{x}_{i}) = \lambda_{i}
	\end{equation}
	
	\vspace{2mm}
	
	\par{\kaishu 因此,如果导体系统中每个导体的位置不变,当电荷的分布(假定是正电荷)使得所有导体均为等势体的时候,体系的能量达到极值,处于平衡状态。\textcolor{blue}{严格来说我们还需证明:\(\frac{\delta^{2}\mathcal{W}}{\delta\rho^2}>0\)},我们先求 \(\delta^{2}\mathcal{W}\),根据 \(\delta\mathcal{W}\) 的结果:}
	
	\vspace{-6mm}
	
	\begin{align}
		\delta^{2}\mathcal{W} &= \varepsilon_{0}\int\delta\left(\textbf{E}\cdot\boldsymbol{\delta}\textbf{E}\right)\,\mathrm{d}^3x+\mathcal{O}(\delta^{3})\nonumber\\[5pt]
		&= \varepsilon_0\int\boldsymbol{\delta}\textbf{E}\cdot\boldsymbol{\delta}\textbf{E}\,\mathrm{d}^3x+\varepsilon_{0}\int\textbf{E}\cdot\boldsymbol{\delta}^2\textbf{E}\,\mathrm{d}^3x
	\end{align}
	
	\vspace{2mm}
	
	\par{\kaishu 对式(3.3.8)而言,等式右边第一项恒为正,我们只需讨论右边第二项的符号:代入 \(\textbf{E} = -\nabla\varphi\),\(\nabla\cdot\boldsymbol{\delta}^2\textbf{E} = \frac{\delta^2\rho}{\varepsilon_0}\),类似对 \(\delta\mathcal{W}\) 的讨论,我们得到:}
	
	\vspace{-6mm}
	
	\begin{align}
		\varepsilon_{0}\int\textbf{E}\cdot\boldsymbol{\delta}^2\textbf{E}\,\mathrm{d}^3x &= -\varepsilon_0\int\nabla\cdot\left(\varphi\,\boldsymbol{\delta}^2\textbf{E}\right)\,\mathrm{d}^3x+\int\varphi\,\delta^2\rho\,\mathrm{d}^3x\nonumber\\[5pt]
		&= \int\varphi\,\delta^2\rho\,\mathrm{d}^3x
	\end{align}
	
	\vspace{3mm}
	
	\par{\kaishu 现在我们结合式(3.3.8)、(3.3.9)的结果,有:}
	
	\begin{equation}
		\frac{\delta^2\mathcal{W}}{\delta\rho^2} = \varepsilon_{0}\int\left(\frac{\partial\textbf{E}}{\partial\rho}\right)^2\,\mathrm{d}^3x+\int\varphi\,\mathrm{d}^3x>0
	\end{equation}
	
	\vspace{2mm}
	
	\par{\kaishu 式(3.3.10)等式右侧第一项为正,由于我们假设的是正电荷,故一定满足 \(\frac{\delta^2\mathcal{W}}{\delta\rho^2}>0\)。即,所有导体达到等势体,能量达到极小值,体系才能处于平衡状态。}
	
	\vspace{8mm}
	
	\par{\kaishu 如果存在负电荷,我们考虑符号的转换,体系也能达到平衡。这就是\textbf{汤姆孙定理}。}
	
	\par{\kaishu \textcolor{blue}{得到这样的静电平衡有两个前提:1. 导体上的电荷不离开导体表面;2. 每个导体的位置不会改变。条件1可以由导体表面的束缚能量保证;但是条件2不得不引入非静电力固定导体位置。如果我们去除条件2,则导体构型也会随电荷密度的改变而变化,过程如下:}}
	
	\begin{center}
		\tikzstyle{arrow} = [thick,->,>=stealth]
		\begin{tikzpicture}[
		squarednode/.style={rectangle, rounded corners, draw=blue, fill=blue!2, thick, minimum size=10mm},
		]
			
			\node[squarednode](maintopic) at (0,0) {\textbf{\kaishu 电荷分布改变}\,\(\Delta\rho\)};
			\node[squarednode, node distance=2cm](leftsquare)[left=of maintopic]{\textbf{\kaishu 静电体系}\,\(\rho,\,\varphi\)};
			\node[squarednode, node distance=2cm](rightsquare) at (2,0)[right=of maintopic]{\textbf{\kaishu 导体位置 \(\textbf{x}_{i}\) 改变}};
			\node at (2.7,-1) {\textbf{\kaishu 反作用}};
			
			\draw [arrow] (leftsquare.east) -- node[anchor=south]{\textbf{\kaishu 扰动}}(maintopic.west);
			\draw [arrow] (maintopic.east) -- node[anchor=south]{\textbf{\kaishu 作用}}(rightsquare.west);
			\draw [arrow] (rightsquare.230) to[out = 230, in = 310] (maintopic.310);
			
		\end{tikzpicture}
	\end{center}
	
	\par{\kaishu \textcolor{blue}{我们猜测:是否导体位置和电荷分布能自发形成新的平衡态呢?如果可以,这个平衡态就是无外力状态下体系的稳定态。这就是我们下一小节讨论的内容。}}
	
	\subsubsection{\kaishu 恩绍定理(Earnshaw's theorem)}
	
	\par{\kaishu 现在我们开始讨论:对于一个导体系统,导体位置发生改变和电荷分布改变的作用能否达到平衡,进而成为静电系统。这个问题较为复杂,我们先进行如下简化,并展开讨论:}
	
	\vspace{-3mm}
	
	\par{\kaishu \begin{enumerate}
			
		\item “绝热近似”:导体运动速度足够慢,使得每个时刻电荷分布有足够的时间达到平衡(即:每一时刻导体都是等势体);
		
		\vspace{-3mm}
		
		\item 导体之间的距离足够远,导体运动导致其他导体上的电荷再分布过程可以忽略。
		
	\end{enumerate}}

	\vspace{-3mm}
	
	\par{\kaishu 因为导体的构型改变不会影响固有的电能,所以在上述近似中,我们不考虑体系的固有能,只考虑相互作用能:}
	
	\vspace{-5mm}
	
	\begin{equation}
		\mathcal{W}_{\text{int}} = \frac{1}{2}\sum_{a = 1}^{N}q_{a}\varphi_{a}
	\end{equation}
	
	\vspace{2mm}
	
	\par{\kaishu 其中:\(\varphi_{a}\) 表示除去第 \(a\) 个导体,其他导体在 \(\textbf{r}_{a}\)(表示第 \(a\) 个导体位置)处产生的电势。此时 \(\mathcal{W}_{\text{int}}\) 与导体位置 \(\left\{\textbf{r}_{a}\right\}\) 有关。为了使系统稳定,我们需要 \(\mathcal{W}_{\text{int}}\) 具有局部极小值,即:\(\mathcal{W}_{\text{int}}\) 对坐标的一阶变分为0;二阶变分大于0。}
	
	\par{\kaishu 考虑坐标变分 \(\textbf{r}_{a}+\boldsymbol{\delta}\textbf{r}_{a}\),要求能量变分为0,根据一阶变分的知识,我们将得到:}
	
	\vspace{-3mm}
	
	\begin{equation*}
		\delta\mathcal{W}_{\text{int}} = 0\qquad\Rightarrow\qquad \frac{\partial\mathcal{W}}{\partial r_{a}} = 0
	\end{equation*}
	
	\vspace{2mm}
	
	\par{\kaishu 考虑式(3.3.11)中能量的表达式,由于我们在假设中忽略了电荷分布的作用,关注电势,有:}
	
	\vspace{-3mm}
	
	\begin{equation*}
		\nabla\varphi_{a} = 0\qquad\Rightarrow\qquad \textbf{E}_{a} = 0
	\end{equation*}
	
	\vspace{2mm}
	
	\par{\kaishu 上式说明:在第 \(a\) 个导体处,其他导体产生的电场必须相互抵消为0。也就是说,平衡条件就是——体系中任意一个导体所处位置的电场均为0(见图3.3.1)。}
	
	\vspace{1mm}
	
	\par{\kaishu 下面我们关注具体的导体构型。仍考虑 \(N\) 个导体的体系,对处于平衡位置的第 \(a\) 个导体作扰动 \(\textbf{r}_a\to\textbf{r}_a+\boldsymbol{\delta}\textbf{r}_a\),有 \(\delta\mathcal{W} = 0\),考虑二阶变分:}
	
	\begin{center}
		\includegraphics[width=0.6\linewidth]{C:/Users/16965/OneDrive/Pictures/Electrondynamics/Electrodynamics fig 3-3-1.pdf}
		\captionof{figure}{\text{\kaishu 多个自由导体静电平衡示意图,网格表示假想电场,导体所处位置无电场}}\label{fig:3-3-1}
	\end{center}
	
	\vspace{-5mm}
	
	\begin{align}
		\delta^2\mathcal{W}_a &= \frac{1}{2}\,\frac{\partial^2\mathcal{W}}{\partial r_{a}^i \,\partial r_{a}^j}\,\delta r_{a}^i \,\delta r_{a}^j\nonumber\\[6pt]
		&= \frac{q_a}{2}\,\frac{\partial^{2}\varphi_a}{\partial r_{a}^i \,\partial r_{a}^j}\,\delta r_{a}^i \,\delta r_{a}^j = B_{ij}\,\delta r_{a}^i \,\delta r_{a}^j
	\end{align} 
	
	\par{\kaishu 其中:\(i\),\(j\) 表示坐标分量,这两个角标满足爱因斯坦求和规则。\(B_{ij} = \frac{q_a}{2}\,\frac{\partial^{2}\varphi_a}{\partial r_{a}^i \,\partial r_{a}^j}\),显然,如果我们将 \(B_{ij}\) 展开写成矩阵形式,就会发现它是一个对称矩阵。现在我们的目的是证明或者证伪结论:对于任意的位置改变有 \(\delta^2\mathcal{W}>0\)。}
	
	\par{\kaishu 单独对式(3.3.12)处理,不能得到任何有效的结论,因为不同方向的位置变分会相互影响(你也可以说相互耦合)。为了方便处理,我们将矩阵 \(\mathrm{B} = \left(B_{ij}\right)\) 对角化:}
	
	\vspace{-3mm}
	
	\begin{equation*}
		\mathcal{B} = \mathrm{S}^{T}\mathrm{B}\,\mathrm{S} = \mathrm{diag}\left(b_1,\, b_2,\,\cdots,\, b_{n}\right)\qquad\left(i,j= 1,\,\cdots,\, n\right)
	\end{equation*}
	
	\vspace{2mm}
	
	\par{\kaishu 根据线性代数知识,\(b_i\) 是矩阵 \(\mathcal{B}\) 的特征值,由于对角化过程中有:\(\mathrm{S}^{T}\mathrm{S} = \mathbbm{1}\),我们将得到:}
	
	\vspace{-3mm}
	
	\begin{equation}
		\delta^2\mathcal{W} = b_{i}\left(\delta\tilde{r}_{a}^i\right)^2\qquad\left(i =1,\,2,\,\cdots,\, n\right)
	\end{equation}
	
	\vspace{2mm}
	
	\par{\kaishu 其中:\(\delta\tilde{r}_{a}^i\) 对应的是第 \(i\) 个特征值的特征向量,这些向量可以理解为:\textcolor{blue}{我们将多个导体的扰动简化成了“简正模式”,此时它们的“简正扰动”相互独立。}现在我们只需要关注每一个导体的独立扰动即可,判断稳定的状态就是——任意的扰动会造成能量的增加。这意味着 \(\delta^2\mathcal{W}>0\),也就是所有的特征值均大于0:}
	
	\vspace{-4mm}
	
	\begin{equation*}
		b_{i}>0\qquad\left(i = 1,\,2,\,\cdots,\, n\right)
	\end{equation*}
	
	\vspace{2mm}
	
	\par{\kaishu 但是,\(\varphi_{a}\) 不一定能满足:$\frac{q_a}{2}\,\frac{\partial^2\varphi_a}{\partial r_{a}^i \,\partial r_{a}^i}>0$。不妨假设 \(q_{a}\) 为正电荷,\(\varphi_{a}\) 表示:除 \(q_{a}\) 的其他带电导体在 \(\textbf{r}_a\) 处产生的电势。\(\varphi_{a}\) 满足泊松方程,而我们此时忽略 \(q_a\),\(\varphi_{a}\) 满足的是拉普拉斯方程:}
	
	\begin{equation}
		\nabla^2\varphi_{a} = \frac{\partial^2\varphi_{a}}{\partial r_{a}^i\,\partial r_a^i} = 0
	\end{equation}
	
	\vspace{1mm}
	
	\par{\kaishu 式(3.3.14)中的 \(i\) 角标仍满足爱因斯坦求和规则。将上式与 \(b_{i}\) 对比,我们将得到:}
	
	\begin{equation}
		0\equiv\frac{q_{a}}{2}\,\nabla^2\varphi_{a} = \frac{\partial^2\varphi_{a}}{\partial r_{a}^i\,\partial r_a^i} = \sum_{i = 1}^{n}b_{i} = \tr\mathcal{B}
	\end{equation}
	
	\vspace{2mm}
	
	\par{\kaishu \(\tr\mathcal{B}\) 表示矩阵 \(\mathcal{B}\) 的迹(trace),根据线性代数知识,在矩阵对角化过程中,矩阵的迹是不变量,我们有:\(\tr\mathrm{B} = \tr\mathcal{B} = 0\)。我们不难发现:\(\sum_{i = 1}^{n}b_{i} = 0\) 与任意的 \(b_{i}>0\) 矛盾,所以 \(\delta^2\mathcal{W}>0\) 不总是成立的。体系只可能存在“鞍点”类型的驻点,不存在局域的极小值点。\textcolor{blue}{如果这个体系是一维的,上述结论也成立,因为此时 \(b_1 \equiv 0\)。}}
	
	\vspace{1mm}
	
	\par{\kaishu 因此只有静电相互作用的电荷体系,不可能存在稳定的状态;我们想要构建任何稳定的静电体系,就必须添加其他的约束力。\textcolor{blue}{在静电学中,我们总是假定存在非静电的约束力。}}
	
	\subsection{\kaishu 基本的电介质静电学}
	
	\par{\kaishu 我们前面已经较为详细的讨论了静电体系中的导体,下面将讨论基本的电介质静电体系。此处我们先讨论电介质的边界条件,以及晶体的部分介电性质(作为补充内容出现)。在下一节,我们将解决给定边界条件泊松方程问题(不再是形式解)。}
	
	\subsubsection{\kaishu 电介质的边界条件}
	
	\par{\kaishu 电介质中的泊松方程我们已经在3.1节静电方程中推导出,我们研究均匀的介质,\(\varepsilon\) 为常数,介质中泊松方程可以写成:}
	
	\vspace{-4mm}
	
	\begin{equation*}
		\nabla^2\varphi = -\frac{\rho}{\varepsilon}
	\end{equation*}
	
	\vspace{2mm}
	
	\par{\kaishu 这里我们可以从麦克斯韦方程组的边界条件,推出静电体系中电介质满足的边界条件。假定存在两种介质接触,介电常数分别为:\(\varepsilon_{1}\)、\(\varepsilon_{2}\),边界处电荷面密度为 \(\sigma\),边界条件写成:}
	
	\begin{equation}
		\begin{cases}
			\textbf{n}\cdot\left(\textbf{D}_{1}-\textbf{D}_{2}\right) = \sigma\\[3pt]
			\textbf{n}\times\left(\textbf{E}_{1}-\textbf{E}_{2}\right) = 0
		\end{cases}
	\end{equation}
	
	\vspace{2mm}
	
	\par{\kaishu 其中的下标表示介质内的物理量。我们将式(3.4.1)改写成电势的形式:}
	
	\vspace{-2mm}
	
	\begin{equation}
		\varepsilon_{1}\,\frac{\partial\varphi_{1}}{\partial n}-\varepsilon_{2}\,\frac{\partial\varphi_{2}}{\partial n} = -\sigma
	\end{equation}
	
	\vspace{2mm}
	
	\par{\kaishu 式(3.4.2)很容易就可以得到,只需要将 \(\textbf{D} = -\varepsilon\nabla\varphi = -\varepsilon\,\frac{\partial\varphi}{\partial n}\) 代入即可。我们现在也可以将 \(\textbf{E} = -\nabla\varphi\) 代入式(3.4.1)中,\(\textbf{n}\) 表示法向单位向量,先作出示意图如下:}
	
	\begin{center}
		\includegraphics[width=0.6\linewidth]{C:/Users/16965/OneDrive/Pictures/Electrondynamics/Electrodynamics fig 3-4-1.pdf}
		\captionof{figure}{\text{\kaishu 1、2介质交界面示意图,\(\textbf{E}_{n}\) 表示电场法向分量,\(\textbf{E}_{t}\) 表示电场切面内分量}}\label{fig:3-4-1}
	\end{center}
	
	\par{\kaishu 我们根据图3.4.1理解,我们现在希望 \(\textbf{n}\times\left(\textbf{E}_{1}-\textbf{E}_{2}\right) = 0\),将电场 \(\textbf{E}\) 分解为法向分量和介质接触部分切面的分量,可以计算得到:}
	
	\vspace{-3mm}
	
	\begin{equation*}
		\textbf{n}\times\left(\textbf{E}_{1}-\textbf{E}_{2}\right) = \hat{e}\cdot\left(E_{1t}-E_{2t}\right) = 0
	\end{equation*}
	
	\vspace{2mm}
	
	\par{\kaishu 我们得到 \(E_{1t} = E_{2t}\),电场的切向分量相等,用电势表示之,有:}
	
	\vspace{-2mm}
	
	\begin{equation}
		\frac{\partial\varphi_{1}}{\partial n_{\parallel}} = \frac{\partial\varphi_{2}}{\partial n_{\parallel}} \qquad\Rightarrow\qquad \left.\left(\varphi_{1}-\varphi_{2}\right)\right|_{S} = \text{const}
	\end{equation}
	
	\vspace{2mm}
	
	\par{\kaishu 式(3.4.3)就是我们想要得到的边界条件,\(\text{const}\) 是与接触位置无关的常数。下面我们只需要求出常数即可。根据电势的定义,有:}
	
	\vspace{-3mm}
	
	\begin{equation}
		\varphi_{2}-\varphi_{1} = \int_{2}^{1}\textbf{E}\cdot\mathrm{d}\vec{l}\quad\xrightarrow[]{h\to 0}\,\left(\frac{E_{2n}+E_{1n}}{2}\right)h
	\end{equation}
	
	\vspace{2mm}
	
	\par{\kaishu 其中:\(h\) 为1,2两个点之间的距离,\(E_{1n}\)、\(E_{2n}\) 为界面两端的垂直电场分量。我们知道:界面上的电场不可能无穷大,所以,在式(3.4.4)描述的情形中,电场的不连续性\textbf{不会影响积分结果趋近于0},电势 \(\varphi\) 是连续的。}
	
	\par{\kaishu 综合上述讨论,我们可以得到,在两种不同介质边界交界处,电势满足的边界条件为:}
	
	\vspace{-6mm}
	
	\begin{gather}
		\varepsilon_{1}\,\frac{\partial\varphi_{1}}{\partial n}-\varepsilon_{2}\,\frac{\partial\varphi_{2}}{\partial n} = -\sigma\\[5pt]
		\left.\varphi_{1}\right|_{\text{surface}} = \left.\varphi_{2}\right|_{\text{surface}}
	\end{gather}
	
	\vspace{1mm}
	
	\par{\kaishu 除点电荷外,连续带电体的问题均可以使用边界条件解方程。\textcolor{blue}{事实上,我们可以把导体视作 \(\varepsilon\to\infty\) 的电介质,其物理意义是:静电情况下,导体对电场的影响与 \(\varepsilon\to\infty\) 的电介质产生的影响相同。实际上,由于 \(\textbf{P} = \varepsilon_{0}\chi\,\textbf{E}\),\(\textbf{P}\) 是有限量,此时又有:\(\chi\to\infty\),我们得到:\(\textbf{E}\to0\),这恰好说明:静电体系内,理想导体内部电场为0。}}
	
	\subsubsection{\kaishu 补充:各向同性介电体的电致伸缩}
	
	\par{\kaishu 对于电场内的固体介电体,不能简单对其引入压强概念。因为我们已经说明(见附录2.C、2.D),此时作用在物体上的力沿物体表面变化,而且是各向异性的(即使我们研究的物体有各向同性)。如果我们要确定这种介电体的形变(称为\textbf{电致伸缩}),必须依靠弹性理论。
	
	如果我们只对物体总的体积变化感兴趣,则情况可以大为简化。我们假定物体的形状不变,物体的形变可以被视为各方向均匀压缩或伸展。
	
	假定我们要研究的介质处于大气中(\(\varepsilon \sim \varepsilon_{0}\)),用 \(P\) 标记物体表面产生的均匀压强。设物体的总亥姆霍兹自由能为 \(F\),根据热力学关系式,有:}
	
	\vspace{-3mm}
	
	\begin{equation}
		P = -\left(\frac{\partial F}{\partial V}\right)_{T}
	\end{equation}
	
	\vspace{2mm}
	
	\par{\kaishu 在 \(\mathrm{d}F\) 的微分式中必须加入一项 \(-P \,\mathrm{d}V\),在均匀的外电场中,我们可以写出:}
	
	\vspace{-2mm}
	
	\begin{equation}
		\mathrm{d}F = -S \,\mathrm{d}T-P \,\mathrm{d}V-\textbf{P}\cdot\mathrm{d}\textbf{E}
	\end{equation}
	
	\vspace{2mm}
	
	\par{\kaishu 其中:\(S\) 表示系统的熵,\(T\) 为热力学温度。引入物体的吉布斯自由能:}
	
	\vspace{-2mm}
	
	\begin{equation}
		G = F+PV
	\end{equation}
	
	\vspace{2mm}
	
	\par{\kaishu 然后我们对式(3.4.9)两侧取微分,有:}
	
	\vspace{-2mm}
	
	\begin{equation}
		\mathrm{d}G = -S \,\mathrm{d}T+V \,\mathrm{d}P-\textbf{P}\cdot\mathrm{d}\textbf{E}
	\end{equation}
	
	\vspace{2mm}
	
	\par{\kaishu 我们给出前提条件:热力学量在外电场内的改变量很小,在微小增量对应的近似条件下:两种自由能的微小变化(对于 \(G\) 给定 \(T\) 和 \(P\);对于 \(F\) 给定 \(T\) 和 \(V\))彼此相等。因此,在均匀外场的条件下,我们可以将体系的吉布斯自由能写成如下形式:}
	
	\vspace{-2mm}
	
	\begin{equation}
		G(P,T) = G(P_{0},T_{0})-\frac{1}{2}\,\textbf{P}\cdot\textbf{E}
	\end{equation}
	
	\vspace{2mm}
	
	\par{\kaishu 其中:\(G(P_{0},T_{0})\) 表示给定 \(P = P_{0}\) 和 \(T = T_{0}\),且不存在电场时物体的总吉布斯自由能。}
	
	\vspace{1mm}
	
	\par{\kaishu 我们可以写出一般的本构关系:\(P_{i} = V\varepsilon_{ij}E_{j}\),\(V\) 是介电体的体积,\(\varepsilon\) 此时描述的是介电体单位体积内的介电性质。代入式(3.4.11)中,得到:}
	
	\vspace{-2mm}
	
	\begin{equation}
		G(P,T) = G(P_{0},T_{0})-\frac{1}{2}V\varepsilon_{ij}E_{i}E_{j}
	\end{equation}
	
	\vspace{2mm}
	
	\par{\kaishu 如果我们假定介电常数 \(\varepsilon_{r}\simeq1\),我们可以将上面的情况简化为 \(\textbf{P} = V\varepsilon\,\textbf{E}\),式(3.4.12)简化为:}
	
	\vspace{-2mm}
	
	\begin{equation}
		G(P,T) = G(P_{0},T_{0})-\frac{1}{2}V\varepsilon |\textbf{E}|^{2}
	\end{equation}
	
	\vspace{2mm}
	
	\par{\kaishu 系统电场内,介电体的体积变化记为 \(V-V_{0}\),我们可以通过对吉布斯自由能求偏微分得到:}
	
	\begin{equation}
		V-V_{0} = -\frac{1}{2}\left[\frac{\partial(\varepsilon V)}{\partial P}\right]_{T}|\textbf{E}|^{2}
	\end{equation}
	
	\vspace{5mm}
	
	\par{\kaishu 与导体不同(导体只能拉伸),介电体的体积变化既可以是正的,又可以是负的。类似地,还可以计算得到:等温状态下,外电场中的介电体所吸收的热量 \(Q\)(压强也不变)。}
	
	\begin{equation}
		Q = \frac{T}{2}\left[\frac{\partial(\varepsilon V)}{\partial T}\right]_{P}|\textbf{E}|^{2}
	\end{equation}
	
	\vspace{2mm}
	
	\par{\kaishu \(Q\) 取正值对应吸热过程,反之是放热过程。\textcolor{blue}{如果我们考虑的是绝热过程的介电体,加上外界电场后,物体温度会产生变化 \(\Delta T = -\frac{Q}{C_{P}}\),\(C_{P}\) 为等压热容。}}
	
	\subsubsection{\kaishu 补充:晶体的介电性质}
	
	\par{\kaishu 下面我们探讨的是各向异性的晶体,\(\textbf{D}\) 和 \(\textbf{E}\) 之间存在更复杂的关系(将其简化为线性关系),下面我们写出较为普遍的形式:}
	
	\vspace{-5mm}
	
	\begin{equation}
		D_{i} = D_{0i}+\varepsilon_{ij}\, E_{j}
	\end{equation}
	
	\vspace{2mm}
	
	\par{\kaishu 其中:\(\textbf{D}_{0}\) 是常矢量,分量为 \(D_{0i}\),全部 \(\varepsilon_{ij}\) 分量构成一个二阶张量——\textbf{介电张量}。但是式(3.4.16) 中的自由项 \(\textbf{D}_{0}\) 并不是在任何晶体中都存在;实际上,晶体的晶胞需要满足高对称性,不允许常矢量随意出现。\(\textbf{D}\) 和 \(\textbf{E}\) 关系为:}
	
	\vspace{-5mm}
	
	\begin{equation}
		D_{i} = \varepsilon_{ij}E_{j}\qquad\left(\varepsilon_{ij} = \varepsilon_{ji}\right)
	\end{equation}
	
	\vspace{3mm}
	
	\par{\kaishu 张量 \(\varepsilon_{ij}\) 的对称性可以这样证明:\textcolor{blue}{亥姆霍兹自由能 \(F\) 满足 \(\widetilde{F} = F_{0}(T,\,\rho)-\frac{1}{2}\varepsilon E^{2}\),有:}}
	
	\vspace{-2mm}
	
	\begin{equation*}
		\textcolor{blue}{-\frac{\partial^{2}\widetilde{F}}{\partial E_{j}\,\partial E_{i}} = \frac{\partial D}{\partial E_{j}} = \varepsilon_{ij}}
	\end{equation*}
	
	\vspace{2mm}
	
	\par{\kaishu \textcolor{blue}{由于 \(\frac{\partial^{2}\widetilde{F}}{\partial E_{j}\,\partial E_{i}} = \frac{\partial^{2}\widetilde{F}}{\partial E_{i}\,\partial E_{j}}\),我们自然得到 \(\varepsilon_{ij} = \varepsilon_{ji}\)。}此时的量 \(\widetilde{F}\) 为:}
	
	\begin{equation}
		\widetilde{F} = F_{0}-\frac{1}{2}\varepsilon_{ij}E_{i}E_{j}
	\end{equation}
	
	\vspace{2mm}
	
	\par{\kaishu 此时我们可以写出亥姆霍兹自由能:}
	
	\begin{equation}
		F = \widetilde{F}+E_{i}D_{i} = F_{0}+\frac{1}{2}\varepsilon_{ij}^{-1}D_{i}D_{k}
	\end{equation}
	
	\begin{center}
		\begin{mdframed}[shadow = false]
			
		\par{\textbf{补充:}式(3.4.19)的简要解释
		
		我们从单位体积的内能开始讨论:设 \(U\)、\(S\)、\(\rho\) 分别为物体单位体积的内能、熵和质量。在给定体积内,不存在电场时内能的热力学微分关系式为:}
		
		\vspace{-3mm}
		
		\begin{equation*}
			\mathrm{d}U = T \,\mathrm{d}S+\zeta\,\mathrm{d}\rho
		\end{equation*}
		
		\vspace{2mm}
		
		\par{\kaishu 其中:\(\zeta\) 是物质单位质量的化学势,\textcolor{blue}{在热力学中,常用单位体积粒子数 \(N\) 代替质量密度 \(\rho\),\(\rho = Nm\),\(m\) 是单粒子的质量。单位质量的化学势与单个粒子的化学势 \(\mu\) 相差一个因子 \(\mu = \zeta m\)。}}
			
		\end{mdframed}
	\end{center}
	
	\begin{center}
		\begin{mdframed}[shadow = false]
			
		\par{\kaishu 如果介电体内存在电场,则上式会添加一项电场项,变成:}
		
		\vspace{-3mm}
		
		\begin{equation*}
			\mathrm{d}U = T \,\mathrm{d}S+\zeta\,\mathrm{d}\rho+\textbf{E}\cdot\mathrm{d}\textbf{D}
		\end{equation*}
		
		\vspace{1mm}
		
		\par{\kaishu 对于介电体单位体积的亥姆霍兹自由能:\(F = U-TS\),相应得到:}
		
		\vspace{-3mm}
		
		\begin{equation*}
			\mathrm{d}F = -S \,\mathrm{d}T+\zeta\,\mathrm{d}\rho+\textbf{E}\cdot\mathrm{d}\textbf{D}
		\end{equation*}
		
		\vspace{1mm}
		
		\par{\kaishu 这两个关系式构成了介电体热力学基础。\(F\) 求偏微分时保持温度不变,避免使用熵函数。如果我们想要独立变量表示为 \(\textbf{E}\) 的分量,引入新的热力学函数:}
		
		\vspace{-3mm}
		
		\begin{equation*}
			\widetilde{U} = U-\textbf{E}\cdot\textbf{D}\qquad\widetilde{F} = F-\textbf{E}\cdot\textbf{D}
		\end{equation*}
		
		\vspace{1mm}
		
		\par{\kaishu 因此我们能得到式(3.4.19)的结论。}
		
		\vspace{1mm}
			
		\end{mdframed}
	\end{center}
	
	\par{\kaishu 与其他二阶对称张量相同,我们可以适当地选择坐标轴,将张量 \(\varepsilon_{ij}\) 转化为对角形式。因此在一般情况下,张量 \(\varepsilon_{ij}\) 由三个独立量 \(\varepsilon^{(1)}\)、\(\varepsilon^{(2)}\)、\(\varepsilon^{(3)}\) 决定。又因为在各向同性物体中 \(\varepsilon>1\);此处这三个量也大于1。\textcolor{blue}{根据晶体对称性的不同,张量 \(\varepsilon_{ij}\) 的主值数目也可以小于3个。}}
	
	\vspace{1mm}
	
	\par{\kaishu 在三斜晶系、单斜晶系和正交晶系的晶体中,这三个主值各不相同。这种晶体称为\textbf{双轴晶体}。此时,在三斜晶系内,张量 \(\varepsilon_{ij}\) 的主轴方向和晶体学方向均不是单值关系;在单斜晶系内,有一个主轴方向是预先确定的,即:它必须和二次对称轴重合或者垂直于晶体的对称平面;在正交晶系内,张量 \(\varepsilon_{ij}\) 的三个主轴都是在晶体学上确定的。}
	
	\begin{center}
		\includegraphics[width=0.95\linewidth]{C:/Users/16965/OneDrive/Pictures/Electrondynamics/Electrodynamics fig 3-4-2.pdf}
		\captionof{figure}{\text{\kaishu 部分晶系示意图,从左至右分别为三斜晶系、单斜晶系、正交晶系}}\label{fig:3-4-2}
	\end{center}
	
	\par{\kaishu 在四方晶系、三方晶系和六方晶系的晶体内,三个主值中有两个相等于是总共只有两个独立量,这样的晶体称为\textbf{单轴晶体}。这时主轴中有一个和四次、三次或六次晶体学对称轴重合,而其他两个主轴方向可以任意选择。}
	
	\vspace{1mm}
	
	\par{\kaishu 最后,在立方晶系晶体内,张量 \(\varepsilon_{ij}\) 的三个主值全部相同,而主轴方向任意。这意味着张量 \(\varepsilon_{ij}\) 的形式为 \(\varepsilon\delta_{ij}\),也就是由一个标量 \(\varepsilon\) 来决定。换句话说,就介电性质来说立方对称晶体和各向同性物体没有差别。}
	
	\begin{center}
		\includegraphics[width=0.7\linewidth]{C:/Users/16965/OneDrive/Pictures/Electrondynamics/Electrodynamics fig 3-4-3.pdf}
		\captionof{figure}{\text{\kaishu 剩余晶系示意图,第一行为四方晶系和六方晶系;第二行为三方晶系和立方晶系}}\label{fig:3-4-3}
	\end{center}
	
	\par{\kaishu 现在我们讨论式(3.4.16)中 \(\textbf{D}_{0}\ne0\) 的情况,这一项的存在表明:介电体即使在没有外加电场时也会自发地极化;这种晶体称为\textbf{热释电晶体}。但是,这种自发极化的量取值总是非常小(与分子场比较),这一情况是由于大的 \(\textbf{D}_{0}\) 会导致晶体内存在强电场,这导致体系在能量上不可能与热力学平衡相对应。\(\textbf{D}_{0}\) 较小的同时也保证:我们可以合理地将 \(\textbf{D}\) 展开为 \(\textbf{E}\) 的幂级数,展开式的领头项和次领头项就是式(3.4.16)。下面考虑关系式:}
	
	\vspace{-2mm}
	
	\begin{equation}
		-\frac{\partial\widetilde{F}}{\partial E_{i}} = D_{i} = D_{0i}+\varepsilon_{ij}E_{j}
	\end{equation}
	
	\vspace{2mm}
	
	\par{\kaishu 对式(3.4.20)进行积分,我们可以求得热释电晶体的热力学虽,由此:}
	
	\begin{equation}
		\widetilde{F} = F_{0}-\frac{1}{2}\varepsilon_{ij}E_{i}E_{j}-E_{i}D_{0i}
	\end{equation}
	
	\vspace{2mm}
	
	\par{\kaishu 此时单位体积亥姆霍兹自由能为:}
	
	\vspace{-5mm}
	
	\begin{align}
		F &= \widetilde{F}+E_{i}D_{i} = F_{0}+\frac{1}{2}\varepsilon_{ij}E_{i}E_{j}\nonumber\\[5pt]
		&= F_{0}+\frac{1}{2}\varepsilon_{ij}^{-1}\left(D_{i}-D_{0i}\right)\left(D_{j}-D_{0j}\right)
	\end{align}
	
	\vspace{3mm}
	
	\par{\kaishu \textcolor{blue}{我们在这些公式的推导中忽略了晶体压电效应,上述公式适用于物体全部体积内电场均匀的情况(忽略物体内的应力)。事实上,如果没有外界电场,热释电晶体的总自由能不仅依赖于自身的体积,还依赖于自身的形状。}
	
	上面的讨论中我们需要知道:并非所有的晶体对称性可能出现热释电现象,只有存在一个方向对所有的对称变换都保持不变(其中也要求不能变为反方向)的晶体,才是热释电晶体;恒定矢量 \(\textbf{D}_{0}\) 就在这个方向上。由一个轴和通过该轴的对称平面组成的对称群才满足这一条件;特
	别是,具有对称中心的晶体显然不可能是热释电晶体。我们列举存在热释电性质的晶体类型如下:
	
	\vspace{1mm}
	
	三斜晶系:\(\boldsymbol{C}_{1}\),单斜晶系:\(\boldsymbol{C}_{s}\)、\(\boldsymbol{C}_{2}\),正交晶系:\(\boldsymbol{C}_{2v}\),四方晶系:\(\boldsymbol{C}_{4}\)、\(\boldsymbol{C}_{4v}\),三方晶系:\(\boldsymbol{C}_{3}\)、\(\boldsymbol{C}_{3v}\),六方晶系:\(\boldsymbol{C}_{6}\)、\(\boldsymbol{C}_{6v}\)
	
	\vspace{1mm}
	
	在立方晶系中,一般不存在热释电晶体,\(\boldsymbol{C}_{1}\) 类晶体中,热释电矢量 \(\textbf{D}_{0}\) 的方向与晶体学确定的方向无关;在 \(\boldsymbol{C}_{s}\) 类晶体中,\(\textbf{D}_{0}\) 的方向必须处于对称平面内。在其他8种晶类中,\(\textbf{D}_{0}\) 的方向和对称轴的方向重合。
	
	在通常条件下,热释电晶体没有总电偶极矩,虽然晶体中的极化并不等于零。问题在于,自发极化的介电体内有不为零的电场强度 \(\textbf{E}\)。实际上,晶体样品通常都具有某种很小但不为零的电导率,出于这一原因,电场的存在将引起电流,这种电流一直维持到物体面上生成的自由电荷使晶体样品内的电场消失为止。从空气中淀积在样品表面的离子也起类似的作用。实验上,加热晶体使晶体的自发极化量值发生变化,且这些变化显露出来时,就可观察到晶体的热释电性质。}
	
	\subsubsection{\kaishu 补充:压电体}
	
\end{document}